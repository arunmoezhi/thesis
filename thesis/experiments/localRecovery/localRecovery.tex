%\begin{limitscope}
%%%%% local recovery experiments macros - begin
\newcommand{\retries}{\emph {retries}}
\newcommand{\seekTime}{\emph {seek-time}}
\newcommand{\modifyTime}{\emph {modify-time}}
\newcommand{\seekLength}{\emph {seek-length}}
\newcommand{\estLatencyImpact}{\emph {latency-impact}}
\newcommand{\throughput}{\emph {throughput}}
\newcommand{\action}{modify}
%%%%% local recovery experiments macros - end

In this section we evaluate our local recovery technique. 

To show that our local recovery algorithm is sufficiently general, we implemented it for three different concurrent internal BSTs, namely those based on:
\begin{enumerate*}[label=(\roman*)]
\item the lock-free BST by Howley and Jones~\cite{HowJon:2012:SPAA}, denoted by \HJBST{},
\item the lock-based BST by Ramachandran and Mittal~\cite{RamMit:2015:PPoPP}, denoted by \CASTLE{} and
\item the RCU (Read-Copy-Update) lock-based BST by Arbel and Attiya~\cite{ArbAtt:2014:PODC}, denoted by \CITRUS{}.
\end{enumerate*}
These implementations were chosen so that we covered both lock-free and lock-based approaches. We choose two lock-based implementations as one is based on locking edges~\cite{RamMit:2015:PPoPP} and the other is based on locking nodes using RCU framework~\cite{ArbAtt:2014:PODC}.

\subsection*{Experimental Setup}
As local recovery is useful for high contention cases, we conducted our experiments on a single Intel Xeon Phi SE10P Coprocessor \footnote{http://www.intel.com/content/www/us/en/processors/xeon/xeon-phi-detail.html}
having 61 1.1 GHz cores with 4 hardware threads per core and 8GB of GDDR5 memory. With this machine we were able to experiment with up to 244 threads.

\subsection*{Key distribution}
Usually uniform key distribution (where all keys have same frequency of occurrence) have been used to evaluate concurrent BSTs. But in many of the real world workloads, keys have skewed distribution~\cite{ClaSha+:2009:SIAM} where some keys are more popular than others. Zipfian distribution, a type of power-law distribution simulates this behaviour~\cite{BreCao+:1999:INFOCOM,FalJag:1992:VLDB,GraSun+:1994:SIGMOD}. It is characterized by a parameter $\alpha$ which usually lies between 0.5 and 1~\cite{BreCao+:1999:INFOCOM,AdaHub:2002:GLOTTMET}. In our experiments we used both uniform and Zipfian distributions to evaluate the local recovery algorithm.

\Tabref{localRecovery-perf-numbers} summarizes the performance gap (with respect to system throughput) between the base algorithm and its extension using local recovery for uniform and Zipfian distributions. 
\begin{table}
\centering
\caption[Effect of local recovery on system throughput]{Effect of local recovery on system throughput. Positive number indicates a gain while a negative number indicates a drop.}
\label{tab:localRecovery-perf-numbers}
\renewcommand{\arraystretch}{1.25}
\small
\begin{tabular}{|c|c|c|}
\hline
{\bf Algorithm}  & {\bf \begin{tabular}[c]{@{}c@{}}Uniform\\ (max\%, min\%)\end{tabular}} & {\bf \begin{tabular}[c]{@{}c@{}}Zipfian\\ (max\%, min\%)\end{tabular}} \\ \hline
{\bf \CASTLE{}} & (14, -15)                                                            & (31, -17)                                                            \\ \hline
{\bf \CITRUS{}} & (-1, -58)                                                            & (-2, -8)                                                             \\ \hline
{\bf \HJBST{}}  & (-4, -29)                                                            & (70, -36)                                                           \\ \hline
\end{tabular}
\end{table}

To better understand the effect of local recovery, we also measure several metrics defined in \tabref{metrics-definition}.
\begin{table}
\centering
\caption{Performance metrics used to evaluate the effect of local recovery.}
\label{tab:metrics-definition}
\begin{tabular}{|l|>{}m{0.7\linewidth}|}
\hline
{\bf Metric}					& {\bf Description} 		\\ \hline
{\retries} 				& { fraction of restarts for an operation}      					\\ \hline
{\seekTime}  					& {total time an operation spends on tree traversal including all restarts as well as stack processing time (in $\mu$s)}  \\ \hline
{\modifyTime}  					& {total time an operation spends on \action{} phase (in $\mu$s)}      					\\ \hline
{\seekLength} 				& {number of nodes traversed in the tree by an operation}      					\\ \hline
{\estLatencyImpact}		& {a rough approximation of the amount of clock cycles devoted to each L1 cache miss (a measure of cache performance)~\cite{JefRei:2013:Book}}      					\\ \hline
\end{tabular}                
\end{table}
\Figsref{localRecovery-throughput-uniform}{metrics-uniform} show the impact of local recovery on throughput and various metrics for uniform distribution. We see that the performance gain is marginal and, in many cases, is actually slightly worse due to the overhead of stack maintenance. This is not surprising because, for small trees, even though contention is higher, seek time is small to begin with and any benefit of local recovery is nullified by additional overhead of stack maintenance. For larger trees, even though seek time is larger, contention is low as key accesses are spread evenly.
% Style to select only points from #1 to #2 (inclusive)
\pgfplotsset
{
	select coords between index/.style 2 args=
	{
    x filter/.code=
		{
        \ifnum\coordindex<#1\def\pgfmathresult{}\fi
        \ifnum\coordindex>#2\def\pgfmathresult{}\fi
    }
	}
}
\begin{figure}
\centering
\nextwithlateexternal% < added
\begin{tikzpicture}
	\begin{groupplot}[group style={group size= 4 by 3},height=5.25cm,width=4.5cm,max space between ticks=20,minor tick num=1,tick label style={font=\scriptsize}]
		\nextgroupplot[title={\scriptsize 2K keys},ylabel={\scriptsize Read-Dominated},y label style={at={(0.15,0.5)}},xtick={1,61,122,183,244},mark size=1.5pt]
				\addplot[black, densely dashed,mark=square] [select coords between index={0}{8}] 		table[x=NumOfThreads,y=citrus-MOPS,col sep=space]	 	{Data/localRecovery/mic/threadSweep-uniform.csv}; 	\label{plots:CITRUS:mopsU}
				\addplot[red,densely dashed,mark=triangle*] [select coords between index={0}{8}] 		table[x=NumOfThreads,y=howley-MOPS,col sep=space]	 	{Data/localRecovery/mic/threadSweep-uniform.csv}; 	\label{plots:HOWLEY:mopsU}
				%\addplot[blue,densely dashed,mark=asterisk][select coords between index={0}{8}] 		table[x=NumOfThreads,y=icdcn-MOPS,col sep=space] 	 	{Data/localRecovery/mic/threadSweep-uniform.csv};  	\label{plots:ICDCN:mopsU}
				\addplot[green, densely dashed,mark=o] 		 	[select coords between index={0}{8}] 		table[x=NumOfThreads,y=castle-MOPS,col sep=space]	 	{Data/localRecovery/mic/threadSweep-uniform.csv}; 	\label{plots:CASTLE:mopsU}
				\addplot[black, semithick,mark=square] 	  	[select coords between index={0}{8}] 		table[x=NumOfThreads,y=citrusL-MOPS,col sep=space]	{Data/localRecovery/mic/threadSweep-uniform.csv}; 	\label{plots:CITRUSL:mopsU}
				\addplot[red,semithick,mark=triangle*] 			[select coords between index={0}{8}] 		table[x=NumOfThreads,y=howleyL-MOPS,col sep=space]	{Data/localRecovery/mic/threadSweep-uniform.csv}; 	\label{plots:HOWLEYL:mopsU}
				%\addplot[blue,semithick,mark=asterisk] 		[select coords between index={0}{8}] 		table[x=NumOfThreads,y=icdcnL-MOPS,col sep=space] 	{Data/localRecovery/mic/threadSweep-uniform.csv};  	\label{plots:ICDCNL:mopsU}
				\addplot[green, semithick,mark=o] 					[select coords between index={0}{8}] 		table[x=NumOfThreads,y=castleL-MOPS,col sep=space]	{Data/localRecovery/mic/threadSweep-uniform.csv}; 	\label{plots:CASTLEL:mopsU}
				\coordinate (top) at (rel axis cs:0,1);% coordinate at top of the first plot	
		\nextgroupplot[title={\scriptsize 20K keys},xtick={1,61,122,183,244},mark size=1.5pt]	
				\addplot[black, densely dashed,mark=square] [select coords between index={9}{17}] 	table[x=NumOfThreads,y=citrus-MOPS,col sep=space] 	{Data/localRecovery/mic/threadSweep-uniform.csv};  
				\addplot[red,densely dashed,mark=triangle*] [select coords between index={9}{17}] 	table[x=NumOfThreads,y=howley-MOPS,col sep=space] 	{Data/localRecovery/mic/threadSweep-uniform.csv};  
				%\addplot[blue,densely dashed,mark=asterisk][select coords between index={9}{17}] 	table[x=NumOfThreads,y=icdcn-MOPS,col sep=space]  	{Data/localRecovery/mic/threadSweep-uniform.csv};  
				\addplot[green, densely dashed,mark=o] 			[select coords between index={9}{17}] 	table[x=NumOfThreads,y=castle-MOPS,col sep=space] 	{Data/localRecovery/mic/threadSweep-uniform.csv}; 
				\addplot[black, semithick,mark=square] 			[select coords between index={9}{17}] 	table[x=NumOfThreads,y=citrusL-MOPS,col sep=space] 	{Data/localRecovery/mic/threadSweep-uniform.csv};
				\addplot[red,semithick,mark=triangle*] 			[select coords between index={9}{17}] 	table[x=NumOfThreads,y=howleyL-MOPS,col sep=space] 	{Data/localRecovery/mic/threadSweep-uniform.csv};
				%\addplot[blue,semithick,mark=asterisk] 		[select coords between index={9}{17}] 	table[x=NumOfThreads,y=icdcnL-MOPS,col sep=space]  	{Data/localRecovery/mic/threadSweep-uniform.csv}; 
				\addplot[green, semithick,mark=o] 					[select coords between index={9}{17}] 	table[x=NumOfThreads,y=castleL-MOPS,col sep=space] 	{Data/localRecovery/mic/threadSweep-uniform.csv};
		\nextgroupplot[title={\scriptsize 200K keys},xtick={1,61,122,183,244},mark size=1.5pt]
				\addplot[black, densely dashed,mark=square] [select coords between index={18}{26}] 	table[x=NumOfThreads,y=citrus-MOPS,col sep=space] 	{Data/localRecovery/mic/threadSweep-uniform.csv};
				\addplot[red,densely dashed,mark=triangle*] [select coords between index={18}{26}] 	table[x=NumOfThreads,y=howley-MOPS,col sep=space] 	{Data/localRecovery/mic/threadSweep-uniform.csv};
				%\addplot[blue,densely dashed,mark=asterisk][select coords between index={18}{26}] 	table[x=NumOfThreads,y=icdcn-MOPS,col sep=space]  	{Data/localRecovery/mic/threadSweep-uniform.csv};
				\addplot[green, densely dashed,mark=o] 			[select coords between index={18}{26}] 	table[x=NumOfThreads,y=castle-MOPS,col sep=space] 	{Data/localRecovery/mic/threadSweep-uniform.csv};
				\addplot[black, semithick,mark=square] 			[select coords between index={18}{26}] 	table[x=NumOfThreads,y=citrusL-MOPS,col sep=space] 	{Data/localRecovery/mic/threadSweep-uniform.csv};
				\addplot[red,semithick,mark=triangle*] 			[select coords between index={18}{26}] 	table[x=NumOfThreads,y=howleyL-MOPS,col sep=space] 	{Data/localRecovery/mic/threadSweep-uniform.csv};
				%\addplot[blue,semithick,mark=asterisk] 		[select coords between index={18}{26}] 	table[x=NumOfThreads,y=icdcnL-MOPS,col sep=space]  	{Data/localRecovery/mic/threadSweep-uniform.csv}; 
				\addplot[green, semithick,mark=o] 					[select coords between index={18}{26}] 	table[x=NumOfThreads,y=castleL-MOPS,col sep=space] 	{Data/localRecovery/mic/threadSweep-uniform.csv};
		\nextgroupplot[title={\scriptsize 2M keys},xtick={1,61,122,183,244},mark size=1.5pt]	
				\addplot[black, densely dashed,mark=square] [select coords between index={27}{35}] 	table[x=NumOfThreads,y=citrus-MOPS,col sep=space] 	{Data/localRecovery/mic/threadSweep-uniform.csv}; 
				\addplot[red,densely dashed,mark=triangle*] [select coords between index={27}{35}] 	table[x=NumOfThreads,y=howley-MOPS,col sep=space] 	{Data/localRecovery/mic/threadSweep-uniform.csv};
				%\addplot[blue,densely dashed,mark=asterisk][select coords between index={27}{35}] 	table[x=NumOfThreads,y=icdcn-MOPS,col sep=space]  	{Data/localRecovery/mic/threadSweep-uniform.csv};
				\addplot[green, densely dashed,mark=o] 			[select coords between index={27}{35}] 	table[x=NumOfThreads,y=castle-MOPS,col sep=space] 	{Data/localRecovery/mic/threadSweep-uniform.csv};
				\addplot[black, semithick,mark=square] 			[select coords between index={27}{35}] 	table[x=NumOfThreads,y=citrusL-MOPS,col sep=space] 	{Data/localRecovery/mic/threadSweep-uniform.csv};
				\addplot[red,semithick,mark=triangle*] 			[select coords between index={27}{35}] 	table[x=NumOfThreads,y=howleyL-MOPS,col sep=space] 	{Data/localRecovery/mic/threadSweep-uniform.csv};
				%\addplot[blue,semithick,mark=asterisk] 		[select coords between index={27}{35}] 	table[x=NumOfThreads,y=icdcnL-MOPS,col sep=space]  	{Data/localRecovery/mic/threadSweep-uniform.csv}; 
				\addplot[green, semithick,mark=o] 					[select coords between index={27}{35}] 	table[x=NumOfThreads,y=castleL-MOPS,col sep=space] 	{Data/localRecovery/mic/threadSweep-uniform.csv};
		\nextgroupplot[ylabel={\scriptsize Mixed},y label style={at={(0.15,0.5)}},xtick={1,61,122,183,244},mark size=1.5pt]		
				\addplot[black, densely dashed,mark=square] [select coords between index={36}{44}] 	table[x=NumOfThreads,y=citrus-MOPS,col sep=space] 	{Data/localRecovery/mic/threadSweep-uniform.csv}; 
				\addplot[red,densely dashed,mark=triangle*] [select coords between index={36}{44}] 	table[x=NumOfThreads,y=howley-MOPS,col sep=space] 	{Data/localRecovery/mic/threadSweep-uniform.csv};
				%\addplot[blue,densely dashed,mark=asterisk][select coords between index={36}{44}] 	table[x=NumOfThreads,y=icdcn-MOPS,col sep=space]  	{Data/localRecovery/mic/threadSweep-uniform.csv};
				\addplot[green, densely dashed,mark=o] 			[select coords between index={36}{44}] 	table[x=NumOfThreads,y=castle-MOPS,col sep=space] 	{Data/localRecovery/mic/threadSweep-uniform.csv};
				\addplot[black, semithick,mark=square] 			[select coords between index={36}{44}] 	table[x=NumOfThreads,y=citrusL-MOPS,col sep=space] 	{Data/localRecovery/mic/threadSweep-uniform.csv};
				\addplot[red,semithick,mark=triangle*] 			[select coords between index={36}{44}] 	table[x=NumOfThreads,y=howleyL-MOPS,col sep=space] 	{Data/localRecovery/mic/threadSweep-uniform.csv};
				%\addplot[blue,semithick,mark=asterisk] 		[select coords between index={36}{44}] 	table[x=NumOfThreads,y=icdcnL-MOPS,col sep=space]  	{Data/localRecovery/mic/threadSweep-uniform.csv}; 
				\addplot[green, semithick,mark=o] 					[select coords between index={36}{44}] 	table[x=NumOfThreads,y=castleL-MOPS,col sep=space] 	{Data/localRecovery/mic/threadSweep-uniform.csv};
		\nextgroupplot[xtick={1,61,122,183,244},mark size=1.5pt]		
				\addplot[black, densely dashed,mark=square] [select coords between index={45}{53}] 	table[x=NumOfThreads,y=citrus-MOPS,col sep=space] 	{Data/localRecovery/mic/threadSweep-uniform.csv}; 
				\addplot[red,densely dashed,mark=triangle*] [select coords between index={45}{53}] 	table[x=NumOfThreads,y=howley-MOPS,col sep=space] 	{Data/localRecovery/mic/threadSweep-uniform.csv};
				%\addplot[blue,densely dashed,mark=asterisk][select coords between index={45}{53}] 	table[x=NumOfThreads,y=icdcn-MOPS,col sep=space]  	{Data/localRecovery/mic/threadSweep-uniform.csv};
				\addplot[green, densely dashed,mark=o] 			[select coords between index={45}{53}] 	table[x=NumOfThreads,y=castle-MOPS,col sep=space] 	{Data/localRecovery/mic/threadSweep-uniform.csv};
				\addplot[black, semithick,mark=square] 			[select coords between index={45}{53}] 	table[x=NumOfThreads,y=citrusL-MOPS,col sep=space] 	{Data/localRecovery/mic/threadSweep-uniform.csv};
				\addplot[red,semithick,mark=triangle*] 			[select coords between index={45}{53}] 	table[x=NumOfThreads,y=howleyL-MOPS,col sep=space] 	{Data/localRecovery/mic/threadSweep-uniform.csv};
				%\addplot[blue,semithick,mark=asterisk] 		[select coords between index={45}{53}] 	table[x=NumOfThreads,y=icdcnL-MOPS,col sep=space]  	{Data/localRecovery/mic/threadSweep-uniform.csv}; 
				\addplot[green, semithick,mark=o] 					[select coords between index={45}{53}] 	table[x=NumOfThreads,y=castleL-MOPS,col sep=space] 	{Data/localRecovery/mic/threadSweep-uniform.csv};
		\nextgroupplot[xtick={1,61,122,183,244},mark size=1.5pt]		
				\addplot[black, densely dashed,mark=square] [select coords between index={54}{62}] 	table[x=NumOfThreads,y=citrus-MOPS,col sep=space] 	{Data/localRecovery/mic/threadSweep-uniform.csv}; 
				\addplot[red,densely dashed,mark=triangle*] [select coords between index={54}{62}] 	table[x=NumOfThreads,y=howley-MOPS,col sep=space] 	{Data/localRecovery/mic/threadSweep-uniform.csv};
				%\addplot[blue,densely dashed,mark=asterisk][select coords between index={54}{62}] 	table[x=NumOfThreads,y=icdcn-MOPS,col sep=space]  	{Data/localRecovery/mic/threadSweep-uniform.csv};
				\addplot[green, densely dashed,mark=o] 			[select coords between index={54}{62}] 	table[x=NumOfThreads,y=castle-MOPS,col sep=space] 	{Data/localRecovery/mic/threadSweep-uniform.csv};	
				\addplot[black, semithick,mark=square] 			[select coords between index={54}{62}] 	table[x=NumOfThreads,y=citrusL-MOPS,col sep=space] 	{Data/localRecovery/mic/threadSweep-uniform.csv};
				\addplot[red,semithick,mark=triangle*] 			[select coords between index={54}{62}] 	table[x=NumOfThreads,y=howleyL-MOPS,col sep=space] 	{Data/localRecovery/mic/threadSweep-uniform.csv};
				%\addplot[blue,semithick,mark=asterisk] 		[select coords between index={54}{62}] 	table[x=NumOfThreads,y=icdcnL-MOPS,col sep=space]  	{Data/localRecovery/mic/threadSweep-uniform.csv}; 
				\addplot[green, semithick,mark=o] 					[select coords between index={54}{62}] 	table[x=NumOfThreads,y=castleL-MOPS,col sep=space] 	{Data/localRecovery/mic/threadSweep-uniform.csv};
		\nextgroupplot[xtick={1,61,122,183,244},mark size=1.5pt]		
				\addplot[black, densely dashed,mark=square] [select coords between index={63}{71}] 	table[x=NumOfThreads,y=citrus-MOPS,col sep=space] 	{Data/localRecovery/mic/threadSweep-uniform.csv}; 
				\addplot[red,densely dashed,mark=triangle*] [select coords between index={63}{71}] 	table[x=NumOfThreads,y=howley-MOPS,col sep=space] 	{Data/localRecovery/mic/threadSweep-uniform.csv};
				%\addplot[blue,densely dashed,mark=asterisk][select coords between index={63}{71}] 	table[x=NumOfThreads,y=icdcn-MOPS,col sep=space]  	{Data/localRecovery/mic/threadSweep-uniform.csv};
				\addplot[green, densely dashed,mark=o] 			[select coords between index={63}{71}] 	table[x=NumOfThreads,y=castle-MOPS,col sep=space] 	{Data/localRecovery/mic/threadSweep-uniform.csv};
				\addplot[black, semithick,mark=square] 			[select coords between index={63}{71}] 	table[x=NumOfThreads,y=citrusL-MOPS,col sep=space] 	{Data/localRecovery/mic/threadSweep-uniform.csv};
				\addplot[red,semithick,mark=triangle*] 			[select coords between index={63}{71}] 	table[x=NumOfThreads,y=howleyL-MOPS,col sep=space] 	{Data/localRecovery/mic/threadSweep-uniform.csv};
				%\addplot[blue,semithick,mark=asterisk] 		[select coords between index={63}{71}] 	table[x=NumOfThreads,y=icdcnL-MOPS,col sep=space]  	{Data/localRecovery/mic/threadSweep-uniform.csv}; 
				\addplot[green, semithick,mark=o] 					[select coords between index={63}{71}] 	table[x=NumOfThreads,y=castleL-MOPS,col sep=space] 	{Data/localRecovery/mic/threadSweep-uniform.csv};				
		\nextgroupplot[xlabel={\scriptsize Number of Threads},ylabel={\scriptsize Write-Dominated},y label style={at={(0.15,0.5)}},xtick={1,61,122,183,	244},mark size=1.5pt]
				\addplot[black, densely dashed,mark=square] [select coords between index={72}{80}] 	table[x=NumOfThreads,y=citrus-MOPS,col sep=space] 	{Data/localRecovery/mic/threadSweep-uniform.csv};    
				\addplot[red,densely dashed,mark=triangle*] [select coords between index={72}{80}] 	table[x=NumOfThreads,y=howley-MOPS,col sep=space] 	{Data/localRecovery/mic/threadSweep-uniform.csv};    
				%\addplot[blue,densely dashed,mark=asterisk][select coords between index={72}{80}] 	table[x=NumOfThreads,y=icdcn-MOPS,col sep=space]  	{Data/localRecovery/mic/threadSweep-uniform.csv};     
				\addplot[green, densely dashed,mark=o] 			[select coords between index={72}{80}] 	table[x=NumOfThreads,y=castle-MOPS,col sep=space] 	{Data/localRecovery/mic/threadSweep-uniform.csv};	   
				\addplot[black, semithick,mark=square] 			[select coords between index={72}{80}] 	table[x=NumOfThreads,y=citrusL-MOPS,col sep=space] 	{Data/localRecovery/mic/threadSweep-uniform.csv}; 
				\addplot[red,semithick,mark=triangle*] 			[select coords between index={72}{80}] 	table[x=NumOfThreads,y=howleyL-MOPS,col sep=space] 	{Data/localRecovery/mic/threadSweep-uniform.csv};   
				%\addplot[blue,semithick,mark=asterisk] 		[select coords between index={72}{80}] 	table[x=NumOfThreads,y=icdcnL-MOPS,col sep=space]  	{Data/localRecovery/mic/threadSweep-uniform.csv};    
				\addplot[green, semithick,mark=o] 					[select coords between index={72}{80}] 	table[x=NumOfThreads,y=castleL-MOPS,col sep=space] 	{Data/localRecovery/mic/threadSweep-uniform.csv};	
		\nextgroupplot[xlabel={\scriptsize Number of Threads},xtick={1,61,122,183,244},mark size=1.5pt]	
				\addplot[black, densely dashed,mark=square] [select coords between index={81}{89}] 	table[x=NumOfThreads,y=citrus-MOPS,col sep=space] 	{Data/localRecovery/mic/threadSweep-uniform.csv}; 
				\addplot[red,densely dashed,mark=triangle*] [select coords between index={81}{89}] 	table[x=NumOfThreads,y=howley-MOPS,col sep=space] 	{Data/localRecovery/mic/threadSweep-uniform.csv};
				%\addplot[blue,densely dashed,mark=asterisk][select coords between index={81}{89}] 	table[x=NumOfThreads,y=icdcn-MOPS,col sep=space]  	{Data/localRecovery/mic/threadSweep-uniform.csv};
				\addplot[green, densely dashed,mark=o] 			[select coords between index={81}{89}] 	table[x=NumOfThreads,y=castle-MOPS,col sep=space] 	{Data/localRecovery/mic/threadSweep-uniform.csv};	
				\addplot[black, semithick,mark=square] 			[select coords between index={81}{89}] 	table[x=NumOfThreads,y=citrusL-MOPS,col sep=space] 	{Data/localRecovery/mic/threadSweep-uniform.csv};
				\addplot[red,semithick,mark=triangle*] 			[select coords between index={81}{89}] 	table[x=NumOfThreads,y=howleyL-MOPS,col sep=space] 	{Data/localRecovery/mic/threadSweep-uniform.csv};
				%\addplot[blue,semithick,mark=asterisk] 		[select coords between index={81}{89}] 	table[x=NumOfThreads,y=icdcnL-MOPS,col sep=space]  	{Data/localRecovery/mic/threadSweep-uniform.csv}; 
				\addplot[green, semithick,mark=o] 					[select coords between index={81}{89}] 	table[x=NumOfThreads,y=castleL-MOPS,col sep=space] 	{Data/localRecovery/mic/threadSweep-uniform.csv};
		\nextgroupplot[xlabel={\scriptsize Number of Threads},xtick={1,61,122,183,244},mark size=1.5pt]	
				\addplot[black, densely dashed,mark=square] [select coords between index={90}{98}] 	table[x=NumOfThreads,y=citrus-MOPS,col sep=space] 	{Data/localRecovery/mic/threadSweep-uniform.csv}; 
				\addplot[red,densely dashed,mark=triangle*] [select coords between index={90}{98}] 	table[x=NumOfThreads,y=howley-MOPS,col sep=space] 	{Data/localRecovery/mic/threadSweep-uniform.csv};
				%\addplot[blue,densely dashed,mark=asterisk][select coords between index={90}{98}] 	table[x=NumOfThreads,y=icdcn-MOPS,col sep=space]  	{Data/localRecovery/mic/threadSweep-uniform.csv};
				\addplot[green, densely dashed,mark=o] 			[select coords between index={90}{98}] 	table[x=NumOfThreads,y=castle-MOPS,col sep=space] 	{Data/localRecovery/mic/threadSweep-uniform.csv};	
				\addplot[black, semithick,mark=square] 			[select coords between index={90}{98}] 	table[x=NumOfThreads,y=citrusL-MOPS,col sep=space] 	{Data/localRecovery/mic/threadSweep-uniform.csv};
				\addplot[red,semithick,mark=triangle*] 			[select coords between index={90}{98}] 	table[x=NumOfThreads,y=howleyL-MOPS,col sep=space] 	{Data/localRecovery/mic/threadSweep-uniform.csv};
				%\addplot[blue,semithick,mark=asterisk] 		[select coords between index={90}{98}] 	table[x=NumOfThreads,y=icdcnL-MOPS,col sep=space]  	{Data/localRecovery/mic/threadSweep-uniform.csv}; 
				\addplot[green, semithick,mark=o] 					[select coords between index={90}{98}] 	table[x=NumOfThreads,y=castleL-MOPS,col sep=space] 	{Data/localRecovery/mic/threadSweep-uniform.csv};
		\nextgroupplot[xlabel={\scriptsize Number of Threads},xtick={1,61,122,183,244},mark size=1.5pt]	
				\addplot[black, densely dashed,mark=square] [select coords between index={99}{107}]	table[x=NumOfThreads,y=citrus-MOPS,col sep=space] 	{Data/localRecovery/mic/threadSweep-uniform.csv}; 
				\addplot[red,densely dashed,mark=triangle*] [select coords between index={99}{107}]	table[x=NumOfThreads,y=howley-MOPS,col sep=space] 	{Data/localRecovery/mic/threadSweep-uniform.csv};
				%\addplot[blue,densely dashed,mark=asterisk][select coords between index={99}{107}]	table[x=NumOfThreads,y=icdcn-MOPS,col sep=space]  	{Data/localRecovery/mic/threadSweep-uniform.csv};
				\addplot[green, densely dashed,mark=o] 			[select coords between index={99}{107}]	table[x=NumOfThreads,y=castle-MOPS,col sep=space] 	{Data/localRecovery/mic/threadSweep-uniform.csv};		
				\addplot[black, semithick,mark=square] 			[select coords between index={99}{107}]	table[x=NumOfThreads,y=citrusL-MOPS,col sep=space] 	{Data/localRecovery/mic/threadSweep-uniform.csv};
				\addplot[red,semithick,mark=triangle*] 			[select coords between index={99}{107}]	table[x=NumOfThreads,y=howleyL-MOPS,col sep=space] 	{Data/localRecovery/mic/threadSweep-uniform.csv};
				%\addplot[blue,semithick,mark=asterisk] 		[select coords between index={99}{107}]	table[x=NumOfThreads,y=icdcnL-MOPS,col sep=space]  	{Data/localRecovery/mic/threadSweep-uniform.csv}; 
				\addplot[green, semithick,mark=o] 					[select coords between index={99}{107}]	table[x=NumOfThreads,y=castleL-MOPS,col sep=space] 	{Data/localRecovery/mic/threadSweep-uniform.csv};				
				\coordinate (bot) at (rel axis cs:1,0);% coordinate at bottom of the last plot
\end{groupplot}
	\path (top-|current bounding box.west)-- node[anchor=south,rotate=90] {\scriptsize system throughput} (bot-|current bounding box.west);
	\path (top|-current bounding box.north)-- coordinate(legendpos) (bot|-current bounding box.north);
	\matrix[matrix of nodes, anchor=south, draw, inner sep=0.2em, draw] at ([yshift=1ex]legendpos)
  {
    \ref{plots:HOWLEY:mopsU}& 	\HJBST{} 				& [5pt]
    %\ref{plots:ICDCN:mopsU}& 	\ICDCN{} 				& [5pt]
    \ref{plots:CITRUS:mopsU}& 	\CITRUS{} 			& [5pt]
		\ref{plots:CASTLE:mopsU}& 	\CASTLE{} 			\\
		\ref{plots:HOWLEYL:mopsU}& 	\HJBST{} (LR)		& [5pt]
    %\ref{plots:ICDCNL:mopsU}& 	\ICDCN{} (LR)		& [5pt]
    \ref{plots:CITRUSL:mopsU}& 	\CITRUS{} (LR)	& [5pt]
		\ref{plots:CASTLEL:mopsU}& 	\CASTLE{}	(LR)	\\
	};
\end{tikzpicture}
\caption[Impact of local recovery on throughput - uniform distribution]{Comparison of throughput of different concurrent BST implementations with (solid lines) and without (dotted lines) local recovery for uniform distribution. Each row represents a workload type. Each column represents a key space range. Higher is better.}
\label{fig:localRecovery-throughput-uniform}
\end{figure}
% Style to select only points from #1 to #2 (inclusive)
\pgfplotsset{
  select row/.style={
    x filter/.code={\ifnum\coordindex=#1\else\def\pgfmathresult{}\fi}
  },
}

\definecolor{colorbrewer1}{RGB}{166,206,227}
\definecolor{colorbrewer2}{RGB}{31,120,180}
\definecolor{colorbrewer3}{RGB}{178,223,138}
\definecolor{colorbrewer4}{RGB}{51,160,44}
\definecolor{colorbrewer5}{RGB}{251,154,153}
\definecolor{colorbrewer6}{RGB}{227,26,28}
\definecolor{colorbrewer7}{RGB}{253,191,111}
\definecolor{colorbrewer8}{RGB}{255,127,0}
\definecolor{colorbrewer9}{RGB}{202,178,214}
\definecolor{colorbrewer10}{RGB}{106,61,154}
\definecolor{colorbrewer11}{RGB}{222,179,155}
\definecolor{colorbrewer12}{RGB}{177,89,40}
			
\pgfplotscreateplotcyclelist{paired}{
{draw=colorbrewer1 ,pattern color=colorbrewer1 ,pattern=horizontal lines},
{draw=colorbrewer2 ,pattern color=colorbrewer2 ,pattern=horizontal lines},
{draw=colorbrewer3 ,pattern color=colorbrewer3 ,pattern=vertical lines},
{draw=colorbrewer4 ,pattern color=colorbrewer4 ,pattern=vertical lines},
{draw=colorbrewer5 ,pattern color=colorbrewer5 ,pattern=north east lines},
{draw=colorbrewer6 ,pattern color=colorbrewer6 ,pattern=north east lines},
{draw=colorbrewer7 ,pattern color=colorbrewer7 ,pattern=north west lines},
{draw=colorbrewer8 ,pattern color=colorbrewer8 ,pattern=north west lines},
{draw=colorbrewer9 ,pattern color=colorbrewer9 ,pattern=grid},
{draw=colorbrewer10,pattern color=colorbrewer10,pattern=grid},
{draw=colorbrewer11,pattern color=colorbrewer11,pattern=crosshatch},
{draw=colorbrewer12,pattern color=colorbrewer12,pattern=crosshatch}}
			
\def\twinplot#1#2{
  \pgfplotsforeachungrouped \row in {0,...,5}{
      \edef\justplotit{
        \noexpand\addplot+[bar shift=(-0.5)*\pgfplotbarwidth]
          table [x=metric, select row=\row, y=#1] {#2};}
      \justplotit
			\edef\justplotit{
        \noexpand\addplot+[bar shift=(0.5)*\pgfplotbarwidth]
          table [x=metric, select row=\row, y=#1L] {#2};}
      \justplotit
  }
}

\begin{figure}
\centering
\nextwithlateexternal% < added
\begin{tikzpicture}
\begin{groupplot}
	[
    group style={group size= 3 by 6,xticklabels at=edge bottom},
    height=3.5cm,
    width=5cm,
    ybar=1pt,
    tick label style={font=\scriptsize},
    x tick label style={rotate=45,anchor=east},
    symbolic x coords={retries,seek-time,modify-time,seek-length,throughput},
    ylabel style={align=center},
		cycle list name=paired,
    max space between ticks=15,
		ymin=0
  ]
	\pgfkeys{/pgf/bar width=5.5pt}
	
	\nextgroupplot[title=\CASTLE,ylabel={\scriptsize 2K}] \twinplot{castle}{Data/localRecovery/mic/scaled/uniform/2k-90-9-1.csv}
	%\nextgroupplot[title=\ICDCN] 						\twinplot{icdcn}{Data/localRecovery/mic/scaled/uniform/2k-90-9-1.csv}
	\nextgroupplot[title=\CITRUS] 						\twinplot{citrus}{Data/localRecovery/mic/scaled/uniform/2k-90-9-1.csv}
	\nextgroupplot[title=\HJBST] 							\twinplot{howley}{Data/localRecovery/mic/scaled/uniform/2k-90-9-1.csv}
	\coordinate (mtop0) at (rel axis cs:0,1);% coordinate at top of the first plot
		
	\nextgroupplot[ylabel={\scriptsize 200K}]							\twinplot{castle}{Data/localRecovery/mic/scaled/uniform/200k-90-9-1.csv}
	%\nextgroupplot														\twinplot{icdcn}{Data/localRecovery/mic/scaled/uniform/200k-90-9-1.csv}
	\nextgroupplot														\twinplot{citrus}{Data/localRecovery/mic/scaled/uniform/200k-90-9-1.csv}
	\nextgroupplot														\twinplot{howley}{Data/localRecovery/mic/scaled/uniform/200k-90-9-1.csv}
	\coordinate (mbot0) at (rel axis cs:1,0);% coordinate at bottom of the last plot
	
	\nextgroupplot[ylabel={\scriptsize 2K}] 							\twinplot{castle}{Data/localRecovery/mic/scaled/uniform/2k-70-20-10.csv}
	%\nextgroupplot														\twinplot{icdcn}{Data/localRecovery/mic/scaled/uniform/2k-70-20-10.csv}
	\nextgroupplot						  							\twinplot{citrus}{Data/localRecovery/mic/scaled/uniform/2k-70-20-10.csv}
	\nextgroupplot						  							\twinplot{howley}{Data/localRecovery/mic/scaled/uniform/2k-70-20-10.csv}
	\coordinate (mtop) at (rel axis cs:0,1);% coordinate at top of the first plot
		
	\nextgroupplot[ylabel={\scriptsize 200K}]							\twinplot{castle}{Data/localRecovery/mic/scaled/uniform/200k-70-20-10.csv}
	%\nextgroupplot														\twinplot{icdcn}{Data/localRecovery/mic/scaled/uniform/200k-70-20-10.csv}
	\nextgroupplot														\twinplot{citrus}{Data/localRecovery/mic/scaled/uniform/200k-70-20-10.csv}
	\nextgroupplot														\twinplot{howley}{Data/localRecovery/mic/scaled/uniform/200k-70-20-10.csv}
	\coordinate (mbot) at (rel axis cs:1,0);% coordinate at bottom of the last plot
	
	\nextgroupplot[ylabel={\scriptsize 2K}]								\twinplot{castle}{Data/localRecovery/mic/scaled/uniform/2k-0-50-50.csv}
	%\nextgroupplot														\twinplot{icdcn}{Data/localRecovery/mic/scaled/uniform/2k-0-50-50.csv}
	\nextgroupplot														\twinplot{citrus}{Data/localRecovery/mic/scaled/uniform/2k-0-50-50.csv}
	\nextgroupplot														\twinplot{howley}{Data/localRecovery/mic/scaled/uniform/2k-0-50-50.csv}
	\coordinate (wtop) at (rel axis cs:0,1);% coordinate at top of the first plot
	
	\nextgroupplot[ylabel={\scriptsize 200K}]							\twinplot{castle}{Data/localRecovery/mic/scaled/uniform/200k-0-50-50.csv}
	%\nextgroupplot														\twinplot{icdcn}{Data/localRecovery/mic/scaled/uniform/200k-0-50-50.csv}
	\nextgroupplot														\twinplot{citrus}{Data/localRecovery/mic/scaled/uniform/200k-0-50-50.csv}
	\nextgroupplot														\twinplot{howley}{Data/localRecovery/mic/scaled/uniform/200k-0-50-50.csv}
	\coordinate (wbot) at (rel axis cs:1,0);% coordinate at bottom of the last plot	
	
\end{groupplot}
\path (mtop0-|current bounding box.west)-- node[anchor=south,rotate=90,yshift=-0.9cm] {\scriptsize Read-Dominated} (mbot0-|current bounding box.west);
\path (mtop-|current bounding box.west)-- node[anchor=south,rotate=90,yshift=-0.9cm] {\scriptsize Mixed} (mbot-|current bounding box.west);
\path (wtop-|current bounding box.west)-- node[anchor=south,rotate=90,yshift=-0.9cm] {\scriptsize Write-Dominated} (wbot-|current bounding box.west);
\end{tikzpicture}
\caption[Impact of local recovery on various metrics - uniform distribution]{Impact of local recovery on various metrics for two key space ranges (2K and 200K) with uniform key distribution.
}
\label{fig:metrics-uniform}
\end{figure}

\Figsref{localRecovery-throughput-zipf}{metrics-zipf} show the impact of local recovery on throughput and various metrics for Zipfian distribution.  In general, Zipfian distribution causes more contention than uniform distribution. So even for small trees for which seek times are smaller, we still see performance gains for mixed and write-dominated workloads. As the graphs show,  we see a consistent increase in throughput (up to \localRecoveryMaximumgap{}) using local recovery as we move from low contention scenarios (read-dominated workload and large tree size)  to high contention scenarios (write-dominated workload and small tree size). Specifically, we see up to 70\% and 31\% improvements for \HJBST{} and \CASTLE{} respectively. Except \CITRUS{}, the other two implementations scale fairly well up to 244 threads.
% Style to select only points from #1 to #2 (inclusive)
\pgfplotsset
{
	select coords between index/.style 2 args=
	{
    x filter/.code=
		{
        \ifnum\coordindex<#1\def\pgfmathresult{}\fi
        \ifnum\coordindex>#2\def\pgfmathresult{}\fi
    }
	}
}
\begin{figure}
\centering
\nextwithlateexternal% < added
\begin{tikzpicture}
	\begin{groupplot}[group style={group size= 4 by 3},height=5.25cm,width=4.5cm,max space between ticks=20,minor tick num=1,tick label style={font=\scriptsize}]
		\nextgroupplot[title={\scriptsize 2K keys},ylabel={\scriptsize Read-Dominated},y label style={at={(0.15,0.5)}},xtick={1,61,122,183,244},mark size=1.5pt]
				\addplot[black, densely dashed,mark=square] [select coords between index={0}{8}] 		table[x=NumOfThreads,y=citrus-MOPS,col sep=space]	 	{Data/localRecovery/mic/threadSweep-zipf.csv}; 	\label{plots:CITRUS:mopsZ}
				\addplot[red,densely dashed,mark=triangle*] [select coords between index={0}{8}] 		table[x=NumOfThreads,y=howley-MOPS,col sep=space]	 	{Data/localRecovery/mic/threadSweep-zipf.csv}; 	\label{plots:HOWLEY:mopsZ}
				%\addplot[blue,densely dashed,mark=asterisk][select coords between index={0}{8}] 		table[x=NumOfThreads,y=icdcn-MOPS,col sep=space] 	 	{Data/localRecovery/mic/threadSweep-zipf.csv};  \label{plots:ICDCN:mopsZ}
				\addplot[green, densely dashed,mark=o] 		 	[select coords between index={0}{8}] 		table[x=NumOfThreads,y=castle-MOPS,col sep=space]	 	{Data/localRecovery/mic/threadSweep-zipf.csv}; 	\label{plots:CASTLE:mopsZ}
				\addplot[black, semithick,mark=square] 	  	[select coords between index={0}{8}] 		table[x=NumOfThreads,y=citrusL-MOPS,col sep=space]	{Data/localRecovery/mic/threadSweep-zipf.csv}; 	\label{plots:CITRUSL:mopsZ}
				\addplot[red,semithick,mark=triangle*] 			[select coords between index={0}{8}] 		table[x=NumOfThreads,y=howleyL-MOPS,col sep=space]	{Data/localRecovery/mic/threadSweep-zipf.csv}; 	\label{plots:HOWLEYL:mopsZ}
				%\addplot[blue,semithick,mark=asterisk] 		[select coords between index={0}{8}] 		table[x=NumOfThreads,y=icdcnL-MOPS,col sep=space] 	{Data/localRecovery/mic/threadSweep-zipf.csv};  \label{plots:ICDCNL:mopsZ}
				\addplot[green, semithick,mark=o] 					[select coords between index={0}{8}] 		table[x=NumOfThreads,y=castleL-MOPS,col sep=space]	{Data/localRecovery/mic/threadSweep-zipf.csv}; 	\label{plots:CASTLEL:mopsZ}
				\coordinate (top) at (rel axis cs:0,1);% coordinate at top of the first plot	
		\nextgroupplot[title={\scriptsize 20K keys},xtick={1,61,122,183,244},mark size=1.5pt]	
				\addplot[black, densely dashed,mark=square] [select coords between index={9}{17}] 	table[x=NumOfThreads,y=citrus-MOPS,col sep=space] 	{Data/localRecovery/mic/threadSweep-zipf.csv};  
				\addplot[red,densely dashed,mark=triangle*] [select coords between index={9}{17}] 	table[x=NumOfThreads,y=howley-MOPS,col sep=space] 	{Data/localRecovery/mic/threadSweep-zipf.csv};  
				%\addplot[blue,densely dashed,mark=asterisk][select coords between index={9}{17}] 	table[x=NumOfThreads,y=icdcn-MOPS,col sep=space]  	{Data/localRecovery/mic/threadSweep-zipf.csv};  
				\addplot[green, densely dashed,mark=o] 			[select coords between index={9}{17}] 	table[x=NumOfThreads,y=castle-MOPS,col sep=space] 	{Data/localRecovery/mic/threadSweep-zipf.csv}; 
				\addplot[black, semithick,mark=square] 			[select coords between index={9}{17}] 	table[x=NumOfThreads,y=citrusL-MOPS,col sep=space] 	{Data/localRecovery/mic/threadSweep-zipf.csv};
				\addplot[red,semithick,mark=triangle*] 			[select coords between index={9}{17}] 	table[x=NumOfThreads,y=howleyL-MOPS,col sep=space] 	{Data/localRecovery/mic/threadSweep-zipf.csv};
				%\addplot[blue,semithick,mark=asterisk] 		[select coords between index={9}{17}] 	table[x=NumOfThreads,y=icdcnL-MOPS,col sep=space]  	{Data/localRecovery/mic/threadSweep-zipf.csv}; 
				\addplot[green, semithick,mark=o] 					[select coords between index={9}{17}] 	table[x=NumOfThreads,y=castleL-MOPS,col sep=space] 	{Data/localRecovery/mic/threadSweep-zipf.csv};
		\nextgroupplot[title={\scriptsize 200K keys},xtick={1,61,122,183,244},mark size=1.5pt]
				\addplot[black, densely dashed,mark=square] [select coords between index={18}{26}] 	table[x=NumOfThreads,y=citrus-MOPS,col sep=space] 	{Data/localRecovery/mic/threadSweep-zipf.csv};
				\addplot[red,densely dashed,mark=triangle*] [select coords between index={18}{26}] 	table[x=NumOfThreads,y=howley-MOPS,col sep=space] 	{Data/localRecovery/mic/threadSweep-zipf.csv};
				%\addplot[blue,densely dashed,mark=asterisk][select coords between index={18}{26}] 	table[x=NumOfThreads,y=icdcn-MOPS,col sep=space]  	{Data/localRecovery/mic/threadSweep-zipf.csv};
				\addplot[green, densely dashed,mark=o] 			[select coords between index={18}{26}] 	table[x=NumOfThreads,y=castle-MOPS,col sep=space] 	{Data/localRecovery/mic/threadSweep-zipf.csv};
				\addplot[black, semithick,mark=square] 			[select coords between index={18}{26}] 	table[x=NumOfThreads,y=citrusL-MOPS,col sep=space] 	{Data/localRecovery/mic/threadSweep-zipf.csv};
				\addplot[red,semithick,mark=triangle*] 			[select coords between index={18}{26}] 	table[x=NumOfThreads,y=howleyL-MOPS,col sep=space] 	{Data/localRecovery/mic/threadSweep-zipf.csv};
				%\addplot[blue,semithick,mark=asterisk] 		[select coords between index={18}{26}] 	table[x=NumOfThreads,y=icdcnL-MOPS,col sep=space]  	{Data/localRecovery/mic/threadSweep-zipf.csv}; 
				\addplot[green, semithick,mark=o] 					[select coords between index={18}{26}] 	table[x=NumOfThreads,y=castleL-MOPS,col sep=space] 	{Data/localRecovery/mic/threadSweep-zipf.csv};
		\nextgroupplot[title={\scriptsize 2M keys},xtick={1,61,122,183,244},mark size=1.5pt]	
				\addplot[black, densely dashed,mark=square] [select coords between index={27}{35}] 	table[x=NumOfThreads,y=citrus-MOPS,col sep=space] 	{Data/localRecovery/mic/threadSweep-zipf.csv}; 
				\addplot[red,densely dashed,mark=triangle*] [select coords between index={27}{35}] 	table[x=NumOfThreads,y=howley-MOPS,col sep=space] 	{Data/localRecovery/mic/threadSweep-zipf.csv};
				%\addplot[blue,densely dashed,mark=asterisk][select coords between index={27}{35}] 	table[x=NumOfThreads,y=icdcn-MOPS,col sep=space]  	{Data/localRecovery/mic/threadSweep-zipf.csv};
				\addplot[green, densely dashed,mark=o] 			[select coords between index={27}{35}] 	table[x=NumOfThreads,y=castle-MOPS,col sep=space] 	{Data/localRecovery/mic/threadSweep-zipf.csv};
				\addplot[black, semithick,mark=square] 			[select coords between index={27}{35}] 	table[x=NumOfThreads,y=citrusL-MOPS,col sep=space] 	{Data/localRecovery/mic/threadSweep-zipf.csv};
				\addplot[red,semithick,mark=triangle*] 			[select coords between index={27}{35}] 	table[x=NumOfThreads,y=howleyL-MOPS,col sep=space] 	{Data/localRecovery/mic/threadSweep-zipf.csv};
				%\addplot[blue,semithick,mark=asterisk] 		[select coords between index={27}{35}] 	table[x=NumOfThreads,y=icdcnL-MOPS,col sep=space]  	{Data/localRecovery/mic/threadSweep-zipf.csv}; 
				\addplot[green, semithick,mark=o] 					[select coords between index={27}{35}] 	table[x=NumOfThreads,y=castleL-MOPS,col sep=space] 	{Data/localRecovery/mic/threadSweep-zipf.csv};
		\nextgroupplot[ylabel={\scriptsize Mixed},y label style={at={(0.15,0.5)}},xtick={1,61,122,183,244},mark size=1.5pt]		
				\addplot[black, densely dashed,mark=square] [select coords between index={36}{44}] 	table[x=NumOfThreads,y=citrus-MOPS,col sep=space] 	{Data/localRecovery/mic/threadSweep-zipf.csv}; 
				\addplot[red,densely dashed,mark=triangle*] [select coords between index={36}{44}] 	table[x=NumOfThreads,y=howley-MOPS,col sep=space] 	{Data/localRecovery/mic/threadSweep-zipf.csv};
				%\addplot[blue,densely dashed,mark=asterisk][select coords between index={36}{44}] 	table[x=NumOfThreads,y=icdcn-MOPS,col sep=space]  	{Data/localRecovery/mic/threadSweep-zipf.csv};
				\addplot[green, densely dashed,mark=o] 			[select coords between index={36}{44}] 	table[x=NumOfThreads,y=castle-MOPS,col sep=space] 	{Data/localRecovery/mic/threadSweep-zipf.csv};
				\addplot[black, semithick,mark=square] 			[select coords between index={36}{44}] 	table[x=NumOfThreads,y=citrusL-MOPS,col sep=space] 	{Data/localRecovery/mic/threadSweep-zipf.csv};
				\addplot[red,semithick,mark=triangle*] 			[select coords between index={36}{44}] 	table[x=NumOfThreads,y=howleyL-MOPS,col sep=space] 	{Data/localRecovery/mic/threadSweep-zipf.csv};
				%\addplot[blue,semithick,mark=asterisk] 		[select coords between index={36}{44}] 	table[x=NumOfThreads,y=icdcnL-MOPS,col sep=space]  	{Data/localRecovery/mic/threadSweep-zipf.csv}; 
				\addplot[green, semithick,mark=o] 					[select coords between index={36}{44}] 	table[x=NumOfThreads,y=castleL-MOPS,col sep=space] 	{Data/localRecovery/mic/threadSweep-zipf.csv};
		\nextgroupplot[xtick={1,61,122,183,244},mark size=1.5pt]		
				\addplot[black, densely dashed,mark=square] [select coords between index={45}{53}] 	table[x=NumOfThreads,y=citrus-MOPS,col sep=space] 	{Data/localRecovery/mic/threadSweep-zipf.csv}; 
				\addplot[red,densely dashed,mark=triangle*] [select coords between index={45}{53}] 	table[x=NumOfThreads,y=howley-MOPS,col sep=space] 	{Data/localRecovery/mic/threadSweep-zipf.csv};
				%\addplot[blue,densely dashed,mark=asterisk][select coords between index={45}{53}] 	table[x=NumOfThreads,y=icdcn-MOPS,col sep=space]  	{Data/localRecovery/mic/threadSweep-zipf.csv};
				\addplot[green, densely dashed,mark=o] 			[select coords between index={45}{53}] 	table[x=NumOfThreads,y=castle-MOPS,col sep=space] 	{Data/localRecovery/mic/threadSweep-zipf.csv};
				\addplot[black, semithick,mark=square] 			[select coords between index={45}{53}] 	table[x=NumOfThreads,y=citrusL-MOPS,col sep=space] 	{Data/localRecovery/mic/threadSweep-zipf.csv};
				\addplot[red,semithick,mark=triangle*] 			[select coords between index={45}{53}] 	table[x=NumOfThreads,y=howleyL-MOPS,col sep=space] 	{Data/localRecovery/mic/threadSweep-zipf.csv};
				%\addplot[blue,semithick,mark=asterisk] 		[select coords between index={45}{53}] 	table[x=NumOfThreads,y=icdcnL-MOPS,col sep=space]  	{Data/localRecovery/mic/threadSweep-zipf.csv}; 
				\addplot[green, semithick,mark=o] 					[select coords between index={45}{53}] 	table[x=NumOfThreads,y=castleL-MOPS,col sep=space] 	{Data/localRecovery/mic/threadSweep-zipf.csv};
		\nextgroupplot[xtick={1,61,122,183,244},mark size=1.5pt]		
				\addplot[black, densely dashed,mark=square] [select coords between index={54}{62}] 	table[x=NumOfThreads,y=citrus-MOPS,col sep=space] 	{Data/localRecovery/mic/threadSweep-zipf.csv}; 
				\addplot[red,densely dashed,mark=triangle*] [select coords between index={54}{62}] 	table[x=NumOfThreads,y=howley-MOPS,col sep=space] 	{Data/localRecovery/mic/threadSweep-zipf.csv};
				%\addplot[blue,densely dashed,mark=asterisk][select coords between index={54}{62}] 	table[x=NumOfThreads,y=icdcn-MOPS,col sep=space]  	{Data/localRecovery/mic/threadSweep-zipf.csv};
				\addplot[green, densely dashed,mark=o] 			[select coords between index={54}{62}] 	table[x=NumOfThreads,y=castle-MOPS,col sep=space] 	{Data/localRecovery/mic/threadSweep-zipf.csv};	
				\addplot[black, semithick,mark=square] 			[select coords between index={54}{62}] 	table[x=NumOfThreads,y=citrusL-MOPS,col sep=space] 	{Data/localRecovery/mic/threadSweep-zipf.csv};
				\addplot[red,semithick,mark=triangle*] 			[select coords between index={54}{62}] 	table[x=NumOfThreads,y=howleyL-MOPS,col sep=space] 	{Data/localRecovery/mic/threadSweep-zipf.csv};
				%\addplot[blue,semithick,mark=asterisk] 		[select coords between index={54}{62}] 	table[x=NumOfThreads,y=icdcnL-MOPS,col sep=space]  	{Data/localRecovery/mic/threadSweep-zipf.csv}; 
				\addplot[green, semithick,mark=o] 					[select coords between index={54}{62}] 	table[x=NumOfThreads,y=castleL-MOPS,col sep=space] 	{Data/localRecovery/mic/threadSweep-zipf.csv};
		\nextgroupplot[xtick={1,61,122,183,244},mark size=1.5pt]		
				\addplot[black, densely dashed,mark=square] [select coords between index={63}{71}] 	table[x=NumOfThreads,y=citrus-MOPS,col sep=space] 	{Data/localRecovery/mic/threadSweep-zipf.csv}; 
				\addplot[red,densely dashed,mark=triangle*] [select coords between index={63}{71}] 	table[x=NumOfThreads,y=howley-MOPS,col sep=space] 	{Data/localRecovery/mic/threadSweep-zipf.csv};
				%\addplot[blue,densely dashed,mark=asterisk][select coords between index={63}{71}] 	table[x=NumOfThreads,y=icdcn-MOPS,col sep=space]  	{Data/localRecovery/mic/threadSweep-zipf.csv};
				\addplot[green, densely dashed,mark=o] 			[select coords between index={63}{71}] 	table[x=NumOfThreads,y=castle-MOPS,col sep=space] 	{Data/localRecovery/mic/threadSweep-zipf.csv};
				\addplot[black, semithick,mark=square] 			[select coords between index={63}{71}] 	table[x=NumOfThreads,y=citrusL-MOPS,col sep=space] 	{Data/localRecovery/mic/threadSweep-zipf.csv};
				\addplot[red,semithick,mark=triangle*] 			[select coords between index={63}{71}] 	table[x=NumOfThreads,y=howleyL-MOPS,col sep=space] 	{Data/localRecovery/mic/threadSweep-zipf.csv};
				%\addplot[blue,semithick,mark=asterisk] 		[select coords between index={63}{71}] 	table[x=NumOfThreads,y=icdcnL-MOPS,col sep=space]  	{Data/localRecovery/mic/threadSweep-zipf.csv}; 
				\addplot[green, semithick,mark=o] 					[select coords between index={63}{71}] 	table[x=NumOfThreads,y=castleL-MOPS,col sep=space] 	{Data/localRecovery/mic/threadSweep-zipf.csv};				
		\nextgroupplot[xlabel={\scriptsize Number of Threads},ylabel={\scriptsize Write-Dominated},y label style={at={(0.15,0.5)}},xtick={1,61,122,183,	244},mark size=1.5pt]
				\addplot[black, densely dashed,mark=square] [select coords between index={72}{80}] 	table[x=NumOfThreads,y=citrus-MOPS,col sep=space] 	{Data/localRecovery/mic/threadSweep-zipf.csv};    
				\addplot[red,densely dashed,mark=triangle*] [select coords between index={72}{80}] 	table[x=NumOfThreads,y=howley-MOPS,col sep=space] 	{Data/localRecovery/mic/threadSweep-zipf.csv};    
				%\addplot[blue,densely dashed,mark=asterisk][select coords between index={72}{80}] 	table[x=NumOfThreads,y=icdcn-MOPS,col sep=space]  	{Data/localRecovery/mic/threadSweep-zipf.csv};     
				\addplot[green, densely dashed,mark=o] 			[select coords between index={72}{80}] 	table[x=NumOfThreads,y=castle-MOPS,col sep=space] 	{Data/localRecovery/mic/threadSweep-zipf.csv};	   
				\addplot[black, semithick,mark=square] 			[select coords between index={72}{80}] 	table[x=NumOfThreads,y=citrusL-MOPS,col sep=space] 	{Data/localRecovery/mic/threadSweep-zipf.csv}; 
				\addplot[red,semithick,mark=triangle*] 			[select coords between index={72}{80}] 	table[x=NumOfThreads,y=howleyL-MOPS,col sep=space] 	{Data/localRecovery/mic/threadSweep-zipf.csv};   
				%\addplot[blue,semithick,mark=asterisk] 		[select coords between index={72}{80}] 	table[x=NumOfThreads,y=icdcnL-MOPS,col sep=space]  	{Data/localRecovery/mic/threadSweep-zipf.csv};    
				\addplot[green, semithick,mark=o] 					[select coords between index={72}{80}] 	table[x=NumOfThreads,y=castleL-MOPS,col sep=space] 	{Data/localRecovery/mic/threadSweep-zipf.csv};	
		\nextgroupplot[xlabel={\scriptsize Number of Threads},xtick={1,61,122,183,244},mark size=1.5pt]	
				\addplot[black, densely dashed,mark=square] [select coords between index={81}{89}] 	table[x=NumOfThreads,y=citrus-MOPS,col sep=space] 	{Data/localRecovery/mic/threadSweep-zipf.csv}; 
				\addplot[red,densely dashed,mark=triangle*] [select coords between index={81}{89}] 	table[x=NumOfThreads,y=howley-MOPS,col sep=space] 	{Data/localRecovery/mic/threadSweep-zipf.csv};
				%\addplot[blue,densely dashed,mark=asterisk][select coords between index={81}{89}] 	table[x=NumOfThreads,y=icdcn-MOPS,col sep=space]  	{Data/localRecovery/mic/threadSweep-zipf.csv};
				\addplot[green, densely dashed,mark=o] 			[select coords between index={81}{89}] 	table[x=NumOfThreads,y=castle-MOPS,col sep=space] 	{Data/localRecovery/mic/threadSweep-zipf.csv};	
				\addplot[black, semithick,mark=square] 			[select coords between index={81}{89}] 	table[x=NumOfThreads,y=citrusL-MOPS,col sep=space] 	{Data/localRecovery/mic/threadSweep-zipf.csv};
				\addplot[red,semithick,mark=triangle*] 			[select coords between index={81}{89}] 	table[x=NumOfThreads,y=howleyL-MOPS,col sep=space] 	{Data/localRecovery/mic/threadSweep-zipf.csv};
				%\addplot[blue,semithick,mark=asterisk] 		[select coords between index={81}{89}] 	table[x=NumOfThreads,y=icdcnL-MOPS,col sep=space]  	{Data/localRecovery/mic/threadSweep-zipf.csv}; 
				\addplot[green, semithick,mark=o] 					[select coords between index={81}{89}] 	table[x=NumOfThreads,y=castleL-MOPS,col sep=space] 	{Data/localRecovery/mic/threadSweep-zipf.csv};
		\nextgroupplot[xlabel={\scriptsize Number of Threads},xtick={1,61,122,183,244},mark size=1.5pt]	
				\addplot[black, densely dashed,mark=square] [select coords between index={90}{98}] 	table[x=NumOfThreads,y=citrus-MOPS,col sep=space] 	{Data/localRecovery/mic/threadSweep-zipf.csv}; 
				\addplot[red,densely dashed,mark=triangle*] [select coords between index={90}{98}] 	table[x=NumOfThreads,y=howley-MOPS,col sep=space] 	{Data/localRecovery/mic/threadSweep-zipf.csv};
				%\addplot[blue,densely dashed,mark=asterisk][select coords between index={90}{98}] 	table[x=NumOfThreads,y=icdcn-MOPS,col sep=space]  	{Data/localRecovery/mic/threadSweep-zipf.csv};
				\addplot[green, densely dashed,mark=o] 			[select coords between index={90}{98}] 	table[x=NumOfThreads,y=castle-MOPS,col sep=space] 	{Data/localRecovery/mic/threadSweep-zipf.csv};	
				\addplot[black, semithick,mark=square] 			[select coords between index={90}{98}] 	table[x=NumOfThreads,y=citrusL-MOPS,col sep=space] 	{Data/localRecovery/mic/threadSweep-zipf.csv};
				\addplot[red,semithick,mark=triangle*] 			[select coords between index={90}{98}] 	table[x=NumOfThreads,y=howleyL-MOPS,col sep=space] 	{Data/localRecovery/mic/threadSweep-zipf.csv};
				%\addplot[blue,semithick,mark=asterisk] 		[select coords between index={90}{98}] 	table[x=NumOfThreads,y=icdcnL-MOPS,col sep=space]  	{Data/localRecovery/mic/threadSweep-zipf.csv}; 
				\addplot[green, semithick,mark=o] 					[select coords between index={90}{98}] 	table[x=NumOfThreads,y=castleL-MOPS,col sep=space] 	{Data/localRecovery/mic/threadSweep-zipf.csv};
		\nextgroupplot[xlabel={\scriptsize Number of Threads},xtick={1,61,122,183,244},mark size=1.5pt]	
				\addplot[black, densely dashed,mark=square] [select coords between index={99}{107}]	table[x=NumOfThreads,y=citrus-MOPS,col sep=space] 	{Data/localRecovery/mic/threadSweep-zipf.csv}; 
				\addplot[red,densely dashed,mark=triangle*] [select coords between index={99}{107}]	table[x=NumOfThreads,y=howley-MOPS,col sep=space] 	{Data/localRecovery/mic/threadSweep-zipf.csv};
				%\addplot[blue,densely dashed,mark=asterisk][select coords between index={99}{107}]	table[x=NumOfThreads,y=icdcn-MOPS,col sep=space]  	{Data/localRecovery/mic/threadSweep-zipf.csv};
				\addplot[green, densely dashed,mark=o] 			[select coords between index={99}{107}]	table[x=NumOfThreads,y=castle-MOPS,col sep=space] 	{Data/localRecovery/mic/threadSweep-zipf.csv};		
				\addplot[black, semithick,mark=square] 			[select coords between index={99}{107}]	table[x=NumOfThreads,y=citrusL-MOPS,col sep=space] 	{Data/localRecovery/mic/threadSweep-zipf.csv};
				\addplot[red,semithick,mark=triangle*] 			[select coords between index={99}{107}]	table[x=NumOfThreads,y=howleyL-MOPS,col sep=space] 	{Data/localRecovery/mic/threadSweep-zipf.csv};
				%\addplot[blue,semithick,mark=asterisk] 		[select coords between index={99}{107}]	table[x=NumOfThreads,y=icdcnL-MOPS,col sep=space]  	{Data/localRecovery/mic/threadSweep-zipf.csv}; 
				\addplot[green, semithick,mark=o] 					[select coords between index={99}{107}]	table[x=NumOfThreads,y=castleL-MOPS,col sep=space] 	{Data/localRecovery/mic/threadSweep-zipf.csv};				
				\coordinate (bot) at (rel axis cs:1,0);% coordinate at bottom of the last plot
\end{groupplot}
	\path (top-|current bounding box.west)-- node[anchor=south,rotate=90] {\scriptsize system throughput} (bot-|current bounding box.west);
	\path (top|-current bounding box.north)-- coordinate(legendpos) (bot|-current bounding box.north);
	\matrix[matrix of nodes, anchor=south, draw, inner sep=0.2em, draw] at ([yshift=1ex]legendpos)
  {
    \ref{plots:HOWLEY:mopsZ}&  \HJBST{} 				& [5pt]
    %\ref{plots:ICDCN:mopsZ}&  \ICDCN{} 				& [5pt]
    \ref{plots:CITRUS:mopsZ}&  \CITRUS{} 				& [5pt]
		\ref{plots:CASTLE:mopsZ}&  \CASTLE{} 				\\
		\ref{plots:HOWLEYL:mopsZ}& \HJBST{} (LR)		& [5pt]
    %\ref{plots:ICDCNL:mopsZ}& \ICDCN{} (LR)		& [5pt]
    \ref{plots:CITRUSL:mopsZ}& \CITRUS{} (LR)	& [5pt]
		\ref{plots:CASTLEL:mopsZ}& \CASTLE{}	(LR)	\\
	};
\end{tikzpicture}
\caption[Impact of local recovery on throughput - zipf distribution]{Comparison of throughput of different concurrent BST implementations with (solid lines) and without (dotted lines) local recovery for zipfian distribution with $\alpha$=1. Each row represents a workload type. Each column represents a key space range. Higher is better.}
\label{fig:localRecovery-throughput-zipf}
\end{figure}
% Style to select only points from #1 to #2 (inclusive)
\pgfplotsset{
  select row/.style={
    x filter/.code={\ifnum\coordindex=#1\else\def\pgfmathresult{}\fi}
  },
}

\definecolor{colorbrewer1}{RGB}{166,206,227}
\definecolor{colorbrewer2}{RGB}{31,120,180}
\definecolor{colorbrewer3}{RGB}{178,223,138}
\definecolor{colorbrewer4}{RGB}{51,160,44}
\definecolor{colorbrewer5}{RGB}{251,154,153}
\definecolor{colorbrewer6}{RGB}{227,26,28}
\definecolor{colorbrewer7}{RGB}{253,191,111}
\definecolor{colorbrewer8}{RGB}{255,127,0}
\definecolor{colorbrewer9}{RGB}{202,178,214}
\definecolor{colorbrewer10}{RGB}{106,61,154}
\definecolor{colorbrewer11}{RGB}{222,179,155}
\definecolor{colorbrewer12}{RGB}{177,89,40}
			
\pgfplotscreateplotcyclelist{paired}{
{draw=colorbrewer1 ,pattern color=colorbrewer1 ,pattern=horizontal lines},
{draw=colorbrewer2 ,pattern color=colorbrewer2 ,pattern=horizontal lines},
{draw=colorbrewer3 ,pattern color=colorbrewer3 ,pattern=vertical lines},
{draw=colorbrewer4 ,pattern color=colorbrewer4 ,pattern=vertical lines},
{draw=colorbrewer5 ,pattern color=colorbrewer5 ,pattern=north east lines},
{draw=colorbrewer6 ,pattern color=colorbrewer6 ,pattern=north east lines},
{draw=colorbrewer7 ,pattern color=colorbrewer7 ,pattern=north west lines},
{draw=colorbrewer8 ,pattern color=colorbrewer8 ,pattern=north west lines},
{draw=colorbrewer9 ,pattern color=colorbrewer9 ,pattern=grid},
{draw=colorbrewer10,pattern color=colorbrewer10,pattern=grid},
{draw=colorbrewer11,pattern color=colorbrewer11,pattern=crosshatch},
{draw=colorbrewer12,pattern color=colorbrewer12,pattern=crosshatch}}
			
\def\twinplot#1#2{
  \pgfplotsforeachungrouped \row in {0,...,5}{
      \edef\justplotit{
        \noexpand\addplot+[bar shift=(-0.5)*\pgfplotbarwidth]
          table [x=metric, select row=\row, y=#1] {#2};}
      \justplotit
			\edef\justplotit{
        \noexpand\addplot+[bar shift=(0.5)*\pgfplotbarwidth]
          table [x=metric, select row=\row, y=#1L] {#2};}
      \justplotit
  }
}

\begin{figure}
\centering
\nextwithlateexternal% < added
\begin{tikzpicture}
\begin{groupplot}
	[
    group style={group size= 3 by 6,xticklabels at=edge bottom},
    height=3.5cm,
    width=5cm,
    ybar=1pt,
    tick label style={font=\scriptsize},
    x tick label style={rotate=45,anchor=east},
    symbolic x coords={retries,seek-time,modify-time,seek-length,throughput},
    ylabel style={align=center},
		cycle list name=paired,
    max space between ticks=15,
		ymin=0
  ]
	\pgfkeys{/pgf/bar width=5.5pt}
	
	\nextgroupplot[title=\CASTLE,ylabel={\scriptsize 2K}] \twinplot{castle}{Data/localRecovery/mic/scaled/zipf/2k-90-9-1.csv}
	%\nextgroupplot[title=\ICDCN] 						\twinplot{icdcn}{Data/localRecovery/mic/scaled/zipf/2k-90-9-1.csv}
	\nextgroupplot[title=\CITRUS] 						\twinplot{citrus}{Data/localRecovery/mic/scaled/zipf/2k-90-9-1.csv}
	\nextgroupplot[title=\HJBST] 							\twinplot{howley}{Data/localRecovery/mic/scaled/zipf/2k-90-9-1.csv}
	\coordinate (mtop0) at (rel axis cs:0,1);% coordinate at top of the first plot
		
	\nextgroupplot[ylabel={\scriptsize 200K}]							\twinplot{castle}{Data/localRecovery/mic/scaled/zipf/200k-90-9-1.csv}
	%\nextgroupplot														\twinplot{icdcn}{Data/localRecovery/mic/scaled/zipf/200k-90-9-1.csv}
	\nextgroupplot														\twinplot{citrus}{Data/localRecovery/mic/scaled/zipf/200k-90-9-1.csv}
	\nextgroupplot														\twinplot{howley}{Data/localRecovery/mic/scaled/zipf/200k-90-9-1.csv}
	\coordinate (mbot0) at (rel axis cs:1,0);% coordinate at bottom of the last plot
	
	\nextgroupplot[ylabel={\scriptsize 2K}] 							\twinplot{castle}{Data/localRecovery/mic/scaled/zipf/2k-70-20-10.csv}
	%\nextgroupplot														\twinplot{icdcn}{Data/localRecovery/mic/scaled/zipf/2k-70-20-10.csv}
	\nextgroupplot						  							\twinplot{citrus}{Data/localRecovery/mic/scaled/zipf/2k-70-20-10.csv}
	\nextgroupplot						  							\twinplot{howley}{Data/localRecovery/mic/scaled/zipf/2k-70-20-10.csv}
	\coordinate (mtop) at (rel axis cs:0,1);% coordinate at top of the first plot
		
	\nextgroupplot[ylabel={\scriptsize 200K}]							\twinplot{castle}{Data/localRecovery/mic/scaled/zipf/200k-70-20-10.csv}
	%\nextgroupplot														\twinplot{icdcn}{Data/localRecovery/mic/scaled/zipf/200k-70-20-10.csv}
	\nextgroupplot														\twinplot{citrus}{Data/localRecovery/mic/scaled/zipf/200k-70-20-10.csv}
	\nextgroupplot														\twinplot{howley}{Data/localRecovery/mic/scaled/zipf/200k-70-20-10.csv}
	\coordinate (mbot) at (rel axis cs:1,0);% coordinate at bottom of the last plot
	
	\nextgroupplot[ylabel={\scriptsize 2K}]								\twinplot{castle}{Data/localRecovery/mic/scaled/zipf/2k-0-50-50.csv}
	%\nextgroupplot														\twinplot{icdcn}{Data/localRecovery/mic/scaled/zipf/2k-0-50-50.csv}
	\nextgroupplot														\twinplot{citrus}{Data/localRecovery/mic/scaled/zipf/2k-0-50-50.csv}
	\nextgroupplot														\twinplot{howley}{Data/localRecovery/mic/scaled/zipf/2k-0-50-50.csv}
	\coordinate (wtop) at (rel axis cs:0,1);% coordinate at top of the first plot
	
	\nextgroupplot[ylabel={\scriptsize 200K}]							\twinplot{castle}{Data/localRecovery/mic/scaled/zipf/200k-0-50-50.csv}
	%\nextgroupplot														\twinplot{icdcn}{Data/localRecovery/mic/scaled/zipf/200k-0-50-50.csv}
	\nextgroupplot														\twinplot{citrus}{Data/localRecovery/mic/scaled/zipf/200k-0-50-50.csv}
	\nextgroupplot														\twinplot{howley}{Data/localRecovery/mic/scaled/zipf/200k-0-50-50.csv}
	\coordinate (wbot) at (rel axis cs:1,0);% coordinate at bottom of the last plot	
	
\end{groupplot}
\path (mtop0-|current bounding box.west)-- node[anchor=south,rotate=90,yshift=-0.9cm] {\scriptsize Read-Dominated} (mbot0-|current bounding box.west);
\path (mtop-|current bounding box.west)-- node[anchor=south,rotate=90,yshift=-0.9cm] {\scriptsize Mixed} (mbot-|current bounding box.west);
\path (wtop-|current bounding box.west)-- node[anchor=south,rotate=90,yshift=-0.9cm] {\scriptsize Write-Dominated} (wbot-|current bounding box.west);
\end{tikzpicture}
\caption[Impact of local recovery on various metrics - zipf distribution]{Impact of local recovery on various metrics for two key space ranges (2K and 200K) with zipf key distribution. Each column represents a concurrent BST implementation. To show all the metrics on the same plot, some of the metrics were scaled as follows: \retries{} by $\frac{1}{8}$, \seekLength{} by $\frac{1}{3}$ 
and \throughput{} by 2. For \CITRUS{}, \modifyTime{} was further scaled-down by a factor 20 to make other metrics visible.
}
\label{fig:metrics-zipf}
\end{figure}

As shown in \Figref{metrics-zipf}, for \HJBST{}, all the metrics except \modifyTime{} show improvement.
A modify operation in this algorithm helps any pending operation along its traversal path leading to higher fraction of restarts shown by the metric \retries.
Retries and hence \seekLength{} and \seekTime{} are significantly reduced due to local recovery.

For \CASTLE{}, for smaller trees, we see that all the metrics improved and the throughput increases up to 31\%.
For \CITRUS{}, there was no improvement due to local recovery across all three workloads as this algorithm is optimized for low contention scenarios and does not scale well for high contention scenarios.
Most of its modify operation's time is spent on \action{} phase than in seek phase.
The main reason being a complex delete operation has to wait till all previous readers have completed their traversals.

\begin{comment}
From \figsrefc[ and ]{localRecovery-throughput-stampede-threadSweep-zipf-4Set-Relative}{localRecovery-seekTime-stampede-threadSweep-zipf-4Set-Relative}, we see a clear correlation between seek time and system throughput. As seek time reduces due to local recovery, the system throughput also improves. We see maximum improvement for \HJBST{}. Since \CASTLE{} and \CITRUS{} are lock-based algorithms, no helping is performed during tree traversal.  But in \HJBST{}, during the tree traversal, if a pending operation is seen it is helped and then the current operation is restarted. This results in frequent restarts and hence local recovery improves performance by a larger margin.
\end{comment}
%\end{limitscope}