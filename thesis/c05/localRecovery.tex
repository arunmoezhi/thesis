\begin{limitscope}
\tikzset
{
    treenode/.style = {circle, draw=black, align=center, minimum size=1cm},
    subtree/.style  = {isosceles triangle, draw=black, align=center, minimum height=0.5cm, minimum width=1cm, shape border rotate=90, anchor=north},
    process/.style={rectangle, minimum width=2cm, minimum height=1cm, align=center, text width=2cm, draw},
    connector/.style={circle, minimum size=1cm, align=center, text width=0.5cm, draw},
    arrow/.style={thick, ->, >=stealth}
}
%%%%% localRecovery macros - begin
\newcommand{\remove}[1]{}
\NewDocumentCommand\accesspath{ g g }{\IfNoValueTF{#1}{access-path\xspace}{\IfNoValueTF{#2}{A(#1)\xspace}{A(#1,#2)\xspace}}}
\newcommand{\myterminal}{terminal}
\newcommand{\myanchor}{anchor}
\newcommand{\Myanchor}{Anchor}
\newcommand{\mytarget}{target}
\newcommand{\myadmissible}{admissible}
\newcommand{\mycritical}{critical}
\newcommand{\mysadmissible}{strongly admissible}
\newcommand{\mysafe}{safe}
\newcommand{\myconsistent}{consistent}
\newcommand{\myinconsistent}{inconsistent}
\newcommand{\storedpath}{\Pi}
\newcommand{\prefixpath}[1]{\Pi(#1)}
\newcommand{\injection}{injection}
\newcommand{\myleft}{le\!f\!t}
\newcommand{\myright}{right}
\newcommand{\myparent}{parent}
\newcommand{\InitializeTraversalRecord}{\textsc{InitializeTraversalState}}
\newcommand{\TestForTermination}{\textsc{CanTerminate}}
\newcommand{\FindStartPoint}{\textsc{FindASafeNode}}
\newcommand{\FindAdmissible}{\textsc{ValidatePath}}
\newcommand{\RemoveFromTop}{\textsc{RemoveFromTop}}
\newcommand{\AddToTop}{\textsc{AddToTop}}
\newcommand{\GetTop}{\textsc{GetTop}}
\newcommand{\GetSecondToTop}{\textsc{GetSecondToTop}}
\newcommand{\RemoveUntilCritical}{\textsc{RemoveUntil}}
\newcommand{\RememberCritical}{\textsc{RememberCritical}}
\newcommand{\GetFullEntry}{\textsc{GetFullEntry}}
\newcommand{\IsMarked}{\textsc{IsMarked}}
\newcommand{\IsClean}{\textsc{IsClean}}
\newcommand{\NeedCleanParentNode}{\textsc{NeedCleanParentNode}}
\newcommand{\AddToBottom}{\textsc{AddToBottom}}
\newcommand{\MoveFromTargetToSuccessor}{\textsc{MoveFromTargetToSuccessor}}
\newcommand{\IsEmpty}{\textsc{IsEmpty}}
\newcommand{\Size}{\textsc{Size}}
\newcommand{\GetKey}{\textsc{GetKey}}
\newcommand{\SeekForSuccessor}{\textsc{SeekForSuccessor}}
\newcommand{\NeedSuccessorKey}{\textsc{NeedSuccessorKey}}
\newcommand{\GetChild}{\textsc{GetChild}}
\newcommand{\Move}{\textsc{Move}}
\newcommand{\GetAddress}{\textsc{GetAddress}}
\newcommand{\IsNull}{\textsc{IsNull}}
\newcommand{\PopulateSeekRecord}{\textsc{PopulateSeekRecord}}
\newcommand{\SeekForModify}{\textsc{SeekForModify}}
\newcommand{\SeekForSearch}{\textsc{SeekForSearch}}
\newcommand{\TraverseTree}{\textsc{Traverse}}
\newcommand{\ExamineStack}{\textsc{ExamineStack}}
\newcommand{\numberOfProcesses}{p}
\newcommand{\STELLAR}{\textsc{STELLAR}}
\newcommand{\sNodeOne}{\mathbb{R}}
\newcommand{\sNodeTwo}{\mathbb{S}}
\newcommand{\sKeyOne}{-\infty}
\newcommand{\sKeyTwo}{\infty}
\newcommand{\traversalRecord}{state}
\newcommand{\TraversalRecord}{\textsf{State}}
\newcommand{\opRecord}{opRecord}
\newcommand{\OpRecord}{\textsf{OpRecord}}
\newcommand{\seekRecord}{seekRecord}
\newcommand{\SeekRecord}{\textsf{SeekRecord}}
\newcommand{\maximumgap}{49\%}
\newcommand{\maximumdrop}{10\%}
\newcommand{\dCounters}{DC}
\newcommand{\iCounters}{IC}
\newcommand{\labels}{labels}
\newcommand{\dcounters}{DC}
\newcommand{\icounters}{IC}
\newcommand{\myfigurescaletwo}{0.5}
%%%%% localRecovery macros - end
\section{The Local Recovery Algorithm}
\label{sec:local|recovery}
We first present the main idea behind the algorithm and then provide its pseudo-code with more details.
\section{Overview of the Algorithm}
Every operation in our algorithm uses \emph{seek} function as a subroutine. The seek function traverses the  tree from the root node until it either finds the target key or reaches a non-binary node whose next edge to be followed points to a null node. We refer to the path traversed by the operation during the seek  as the \emph{\accesspath}, and the last node in the \accesspath{} as the \emph{\terminalnode}. The operation then compares the target key with the stored key (the key present in the \terminalnode). Depending on the result of the comparison and the type of the operation, the operation either terminates or moves to the execution phase. In certain cases in which a key may have moved upward along the \accesspath, the seek function may have to restart and traverse the tree again; details about restarting are provided later. We now describe the next steps for each of the type of operation one-by-one. 

\paragraph{Search:} 
A search operation starts by invoking seek operation. It returns \true{} if the stored key matches the target key and \false{} otherwise. 

\paragraph{Insert:}
An insert operation starts by invoking seek operation. It returns \false{} if the target key matches the stored key; otherwise, it moves to the execution phase. In the execution phase, it attempts to insert the key into the tree as a child node of the last node in the \accesspath{} using a \CAS{} instruction. If the instruction succeeds, then the operation returns \true{}; otherwise, it restarts by invoking the seek function again.

\paragraph{Delete:} 
A delete operation starts by invoking seek function. It returns \false{} if the stored key does not match the target key; otherwise, it moves to the execution phase. In the execution phase, it attempts to remove the key stored in the \terminalnode{} of the \accesspath. There are two cases depending on whether the \terminalnode{} is a binary node (has two children) or not (has at most one child). In the first case, the operation is referred to as \emph{complex delete operation}. In the second case, it is referred to as \emph{simple delete operation}. In the case of simple delete, the \terminalnode{} is removed by changing the pointer at the parent node of the \terminalnode. In the 
case of complex delete, the key to be deleted is replaced with the \emph{next largest} key in the tree, which will be stored in the \emph{leftmost node} of the \emph{right subtree} of the \terminalnode.
\section{Details of the Algorithm}
\label{sec:description}

\begin{limitscope}

%% To limit the scope of the macros defined below

%% macros for pseudocode

\newcommand{\child}{child}
\newcommand{\node}{node}
\newcommand{\parent}{parent}

\newcommand{\mainSeekRecord}{seekTargetKey}
\newcommand{\successorSeekRecord}{seekSuccessorKey}


\newcommand{\targetStack}{targetStack}
\newcommand{\successorStack}{successorStack}


\newcommand{\successorStackInUse}{successorStackInUse}
\newcommand{\targetNode}{targetNode}




\newcommand{\key}{key}

\newcommand{\done}{done}
\newcommand{\result}{result}
\newcommand{\status}{status}
\newcommand{\restart}{restart}





\newcommand{\cKey}{key}
\newcommand{\nKey}{key}
\newcommand{\cNode}{current}
\newcommand{\pNode}{parent}
\newcommand{\nMarked}{marked}



\newcommand{\which}{which}
\newcommand{\address}{address}

\newcommand{\anchor}{anchor}

\newcommand{\stack}{stack}
\newcommand{\sTop}{top}
\newcommand{\sBottom}{bottom}
\newcommand{\current}{current}


\newcommand{\admissible}{admissible}
\newcommand{\critical}{critical}
\newcommand{\reference}{re\!f\!erence}

%% \newcommand{\OptReturn}[1][]{\Return #1\;}
\newcommand{\OptReturn}[1][]{}

\newcommand{\injectionPoint}{injectionPoint}



\newcommand{\Search}{\textsc{Search}}
\newcommand{\Insert}{\textsc{Insert}}
\newcommand{\Delete}{\textsc{Delete}}


\newcommand{\Inject}{\textsc{Inject}}



\remove{
\newcommand{\SeekForSuccessor}{\textsc{SeekForSuccessor}}
\newcommand{\NeedSuccessorKey}{\textsc{NeedSuccessorKey}}
\newcommand{\GetChild}{\textsc{GetChild}}
\newcommand{\Move}{\textsc{Move}}
\newcommand{\GetAddress}{\textsc{GetAddress}}
\newcommand{\IsNull}{\textsc{IsNull}}
\newcommand{\PopulateSeekRecord}{\textsc{PopulateSeekRecord}}
}



\newcommand{\mline}[1]{\DontPrintSemicolon #1 \PrintSemicolon}


\newcommand{\LEFT}{\textsf{LEFT}}
\newcommand{\RIGHT}{\textsf{RIGHT}}


\newcommand{\rarrow}{\!\rightarrow\!}
\newcommand{\type}{type}


\newcommand{\SEARCH}{\textsf{SEARCH}}
\newcommand{\INSERT}{\textsf{INSERT}}
\newcommand{\DELETE}{\textsf{DELETE}}

\newcommand{\STOPFOUND}{\textsf{FOUND}}
\newcommand{\STOPNOTFOUND}{\textsf{NOT\_FOUND}}
\newcommand{\DONOTKNOW}{\textsf{CONTINUE}}

\newcommand{\TARGETSTACK}{\textsf{TARGET\_STACK}}
\newcommand{\SUCCESSORSTACK}{\textsf{SUCCESSOR\_STACK}}

%%%%%%%%%%%%%%%%%%%%%%%%%%%%%%%%%%%%%%%%%%%%%%%%%%%%%%%%%%%%%%%%%%%%%%%%%%%%%%%%%%%%

\newcommand{\DefineKeyWords}{
%%
\SetKw{Boolean}{boolean}
\SetKw{Integer}{integer}
\SetKw{LAnd}{~and~}
\SetKw{LOr}{~or~}
\SetKw{LNot}{not}
\SetKw{Struct}{struct}
\SetKw{Null}{null}
\SetKw{True}{true}
\SetKw{False}{false}
\SetKw{Break}{break}
\SetKw{Continue}{continue}
\SetKw{Enum}{enum}
\SetKw{Word}{word}
%%
}

%%%%%%%%%%%%%%%%%%%%%%%%%%%%%%%%%%%%%%%%%%%%%%%%%%%%%%%%%%%%%%%%%%%%%%%%%%%%%%%%%%%%%

%% Data structures used by the local recovery algorithm

%%%%%%%%%%%%%%%%%%%%%%%%%%%%%%%%%%%%%%%%%%%%%%%%%%%%%%%%%%%%%%%%%%%%%%%%%%%%%%%%%%%%%

\begin{algorithm}[tb]
\caption{Data Structures Used} 
\label{algo:local-data|structures}
%%
\DefineKeyWords
%%
\tcp{Used to store information about a node visited during tree traversal}
\DontPrintSemicolon
\Struct StackEntry \{\;
\PrintSemicolon
%%
\label{lin:local-data|structures:begin}
\label{lin:local-stack|entry:begin}
\Indp 
   NodePtr $\node$\;
	 \Enum Direction $\which$\;
   \Integer $\anchor$\;
\Indm 
\}\;
\label{lin:local-stack|entry:end}

\BlankLine

\tcp{Used to store the path from the root node to the current node in the tree}
\DontPrintSemicolon
\Struct \TraversalRecord{} \{\;
\PrintSemicolon
%%
\label{lin:local-traversal|record:begin}
\Indp 
   StackEntry[~] $\stack$\;
	 \Integer $\sTop$\;
	 %% \Integer $\sBottom$\;
\Indm 
\}\;
\label{lin:local-traversal|record:end}

\BlankLine

\tcp{Used to store information about the operation currently in progress}
\DontPrintSemicolon
\Struct \OpRecord{} \{\;
\PrintSemicolon
%%
\label{lin:local-op|record:begin}
\Indp 
   \Enum Type $\type$\;
	 Key $\key$\;
	 %% \TraversalRecord{} $\targetStack$, $\successorStack$\;
	 \TraversalRecord{} $\targetStack$\;
	 %%\TraversalRecord{} $\successorStack$\;
	 %% \Boolean $\successorStackInUse$\;
	 %% NodePtr $\targetNode$\;
	 NodePtr $\injectionPoint$\;
	 \BlankLine
	 \tcp{algorithm-specific fields}
\Indm
\}\;
\label{lin:local-op|record:end}

\BlankLine

\tcp{Used to store the outcome of a tree traversal}
\DontPrintSemicolon
\Struct \SeekRecord \{\;
\PrintSemicolon
%%
\label{lin:local-seek|record:begin}
\Indp 
   %% \TraversalRecord{}Ptr $\traversalRecord$\;
	 \tcp{algorithm-specific fields (\emph{e.g.}, target node and its parent)}
\Indm 
\}\;
\label{lin:local-seek|record:end}

\remove{
   \BlankLine
%%
   \tcp{Local records used when executing an operation}
   \OpRecord{}Ptr $\opRecord$\;
   SeekRecordPtr $\mainSeekRecord$\;
   SeekRecordPtr $\successorSeekRecord$\; 
}
\label{lin:local-data|structures:end}
%%
\end{algorithm}


%%%%%%%%%%%%%%%%%%%%%%%%%%%%%%%%%%%%%%%%%%%%%%%%%%%%%%%%%%%%%%%%%%%%%%%%%%%%%%%%%%%%%

%% Functions used for manipulating traversal stack

%%%%%%%%%%%%%%%%%%%%%%%%%%%%%%%%%%%%%%%%%%%%%%%%%%%%%%%%%%%%%%%%%%%%%%%%%%%%%%%%%%%%%

\begin{algorithm}[tb]
\caption{Functions for Manipulating Traversal Stack} 
\label{algo:local-stack|functions}
%%
\DefineKeyWords
%%
\tcp{Returns the number of elements in the stack}
\DontPrintSemicolon
\Integer \Size( $\traversalRecord$ )\;
\PrintSemicolon
\label{lin:local-stack|begin}
\label{lin:local-size:begin}
\Begin
{
   
   \Return $\traversalRecord \rarrow \sTop + 1$\;
	 \label{lin:local-size:end}
}
%%
\BlankLine
%%
\tcp{Returns the topmost node in the stack}
\DontPrintSemicolon
NodePtr \GetTop( $\traversalRecord$ )\;
\PrintSemicolon
\label{lin:local-get|top:begin}
\Begin
{
   
   $\curly{ \stack, \sTop }$ := $\traversalRecord$\;
	 \label{lin:local-stack|retrieve}
	 \Return $\stack[\sTop] \rarrow \node$\;
	 \label{lin:local-get|top:end}
}
%%
\BlankLine
%%
\tcp{Returns the second topmost node in the stack}
\DontPrintSemicolon
NodePtr \GetSecondToTop( $\traversalRecord$ )\;
\label{lin:local-get|second|to|top:begin}
\PrintSemicolon
\Begin
{
   
   $\curly{ \stack, \sTop }$ := $\traversalRecord$\;
	 \Return $\stack[\sTop-1] \rarrow \node$\;
	 \label{lin:local-get|second|to|top:end}
}
%%
\BlankLine
%%
\tcp{Adds the given node to the stack along with its \myanchor{} node}
\DontPrintSemicolon
\AddToTop(  $\traversalRecord$, $\node$, $\which$ )\;
\PrintSemicolon
\label{lin:local-add|to|top:begin}
\Begin
{
%%
   
   $\curly{ \stack, \sTop }$ := $\traversalRecord$\;
   	
	 \tcp{find the \myanchor{} node}
   \lIf{$\which$ = \RIGHT}
	 {
	    $\anchor$ := $\sTop$
	 }
	 \lElse
	 {
	    $\anchor$ := $\stack[\sTop] \rarrow \anchor$
	 }
   
	 \tcp{push the node into the stack}
   $\stack[\sTop + 1]$ := $\curly{ \node, \which, \anchor }$\;
	 $\traversalRecord \rarrow \sTop$ := $\sTop + 1$\;
	 \OptReturn
	 \label{lin:local-add|to|top:end}
	
%%
}
%%
\BlankLine
%%
\tcp{Removes the topmost node from the stack}
\DontPrintSemicolon
\RemoveFromTop ( $\traversalRecord$ )\;
\PrintSemicolon
\label{lin:local-remove|from|top:begin}
\Begin
{
%%
   
   $\curly{ \stack, \sTop }$ := $\traversalRecord$\;
	
	 \tcp{update the \myanchor{} node of the penultimate entry if needed}
	 $\anchor$ := $\stack[\sTop - 1] \rarrow \anchor$\;
	 \If{$\stack[\sTop] \rarrow \anchor$ $<$ $\stack[\anchor] \rarrow \anchor$}
	 {
	    $\stack[\anchor] \rarrow \anchor$ := $\stack[\sTop] \rarrow \anchor$\;  
	 }
	
	 \tcp{pop the node from the stack}
	 $\traversalRecord \rarrow \sTop$ := $\sTop - 1$\;
	 \OptReturn
	 \label{lin:local-remove|from|top:end}
	 	 
%%
}
%%
\BlankLine
%%
\tcp{Pops the stack until a given entry}
\DontPrintSemicolon
\RemoveUntilCritical( $\traversalRecord$,  $index$ )\;
\PrintSemicolon
\label{lin:local-remove|until|critical:begin}
\Begin
{
  
   $\traversalRecord \rarrow \sTop$ := $index$\;
	 \OptReturn
	 \label{lin:local-remove|until|critical:end}

}
\end{algorithm}

\begin{algorithm}
\caption{Functions for Manipulating Traversal Stack (Continued)} 
\label{algo:local-local-stack|functions|2}
\DefineKeyWords
\tcp{Remember the \mycritical{} node (to avoid locating it again)}
\DontPrintSemicolon
\RememberCritical( $\traversalRecord$,  $\critical$ )\;
\PrintSemicolon
\label{lin:local-remember|critical:begin}
\Begin
{
  
   %% $\curly{ \stack, \sTop }$ := $\traversalRecord$\;  
	 $\curly{ \stack, \sTop }$ := $\traversalRecord$\;
   	
	 $\anchor$ := $\stack[\sTop] \rarrow \anchor$\;
	 \If{$\critical$ $<$ $\stack[\anchor] \rarrow \anchor$}
	 {
	   $\stack[\anchor] \rarrow \anchor$ := $\critical$\;
	 }
	 \OptReturn
	 \label{lin:local-remember|critical:end}
}
%%
\BlankLine
%%
\tcp{Returns a given entry in the stack}
\DontPrintSemicolon
\{ NodePtr, \Enum Direction, \Integer \}  \GetFullEntry( $\traversalRecord$,  $index$ )\;
\PrintSemicolon
\label{lin:local-get|full|entry:begin}
\Begin
{
   
   $\curly{ \stack, \sTop }$ := $\traversalRecord$\;
	 
	
	 \remove{
	
	    \tcp{find the location of the entry in the stack}
			
	    \lIf{$entry$ = $\top$}
	    {
	       $index$ := $\sTop$
	    }
	    \lElse
	    {
	       $index$ := $entry$
	    }
	
	}
	
	\lIf{$index$ = $\top$}
	{
	   \Return $\stack[\sTop]$
	}
	\lElse
	{
	   \Return $\stack[index]$
	}
	

	\label{lin:local-get|full|entry:end}
}
%%
\BlankLine
%%
\tcp{initializes the traversal stack} 
\DontPrintSemicolon
\InitializeTraversalRecord( $\traversalRecord$, $\type$ )\;
\PrintSemicolon
\label{lin:local-initialize|traversal|record:begin}
\Begin
{
%%
  \tcp{initialize the stack using sentinel nodes}
	\tcp{sentinel nodes are never removed from the stack}
	\tcp{a sentinel node is always a safe starting point for the traversal}
	
	\remove{
		
  \uIf{$\type$ = \TARGETSTACK}
	{
		
	   %% $\traversalRecord \rarrow \stack[0]$ := $\ang{\sNodeOne, -1}$\;
     %% $\traversalRecord \rarrow \stack[1]$ := $\ang{\sNodeTwo, 0 }$\;
	   %% $\traversalRecord \rarrow \sTop$ := 1\;
	   %% $\traversalRecord \rarrow \anchor$ := 0\;
	   \tcp{initialize the stack using sentinel nodes}
	   \tcp{sentinel nodes are never removed from the stack}
	   \tcp{a sentinel node is always a safe starting point for the traversal}
	}
	\lElse
	{
	   $\traversalRecord \rarrow \sTop$ := -1
	}
	
	}
	
	\OptReturn
	\label{lin:local-initialize|traversal|record:end}

%%
}
\end{algorithm}


\begin{comment}


%%%%%%%%%%%%%%%%%%%%%%%%%%%%%%%%%%%%%%%%%%%%%%%%%%%%%%%%%%%%%%%%%%%%%%%%%%%%%%%%%%%%%

%% Seek for search operation

%%%%%%%%%%%%%%%%%%%%%%%%%%%%%%%%%%%%%%%%%%%%%%%%%%%%%%%%%%%%%%%%%%%%%%%%%%%%%%%%%%%%%




\begin{algorithm}[tb]
\caption{Seek Function for Target Key (Search Operation)} 
\label{algo:local-seek:search}
%%
\DefineKeyWords
%%
\tcp{Traverses the tree starting from the root until either the key is found or a null pointer is encountered}
%% \tcp{also populates the traversal stack}
\DontPrintSemicolon
\Boolean \TraverseTree( $\opRecord$, $\seekRecord$ )\;
\PrintSemicolon
\label{lin:local-traverse|tree:begin}
\Begin
{
%%
   $\traversalRecord$ := $\opRecord \rarrow \targetStack$\;
	
	 \tcp{initialize the stack and the variables used in the traversal}
	 \InitializeTraversalRecord( $\traversalRecord$, \TARGETSTACK{} )\;
	 \label{lin:local-traverse|tree:initialize}
	 %% \tcp{initialize the variables used in the traversal}
   $\cNode$ := \GetTop( $\traversalRecord$ )\;
	 \label{lin:local-traverse|tree:start}
	 \BlankLine
	 \tcp{traverse the tree (starting from $\cNode$)}
	 \While{\True}
	 {
			\label{lin:local-traverse|tree:while:begin}
	    $\cKey$ := \GetKey( $\cNode$ )\;
			\label{lin:local-traverse|tree:while:first}
		  $\which$ := $\opRecord \rarrow \key < \cKey$ ? \LEFT{} : \RIGHT{}\;
			\label{lin:local-traverse|tree:select}
		  \tcp{read the next address to de-reference}
		  %% $\ang{ \ast, \ast, \nFlag, \address}$ := $\cNode \rarrow \child[\which]$\;  
			$\reference$ := \GetChild( $\cNode$, $\which$ )\;
			
			\BlankLine
			
		  \lIf{$\opRecord \rarrow \key$ = $\cKey$}
			{ 
			   \Return \True 
				 \label{lin:local-traverse|tree:match}
			}	     
			\lIf{\IsNull( $\reference$ )}
			{ 
			   \Return \False
				 \label{lin:local-traverse|tree:null}
			}	  
				
			\BlankLine
			
		  \tcp{traverse the next edge}
		  %% $\pNode$ := $\cNode$;
			$\address$ := \GetAddress( $\reference$ )\;
			$\cNode$ := $\address$\;
			\tcp{push the next node to be visited into the stack}
			\AddToTop( $\traversalRecord$, $\address$, $\which$ )\;
			\label{lin:local-traverse|tree:stack}
			\label{lin:local-traverse|tree:while:end}
			      
	  }	
		
		\OptReturn[\False]
		\label{lin:local-traverse|tree:end}
%%
}
%%
\BlankLine
%%
\tcp{Checks if the key being searched for has moved up in the path}
\DontPrintSemicolon
\Boolean \ExamineStack( $\opRecord$, $\seekRecord$ )\;
\PrintSemicolon
\label{lin:local-examine|stack:begin}
\Begin
{
%%
  
   $\result$ := \False\;
	 $\traversalRecord$ := $\opRecord \rarrow \targetStack$\;
	 
	 \BlankLine
	
   \tcp{start with the \myanchor{} closest to the topmost node in the stack}
	 $\curly{ \ast, \ast, \critical }$ := \GetFullEntry( $\traversalRecord$, $\top$ )\;
	 \label{lin:local-examine|stack:start}	
			
	 \BlankLine
			
			
	 \While{\True}
	 {
	    \label{lin:local-examine|stack:while:begin}
	    \tcp{retrieve the node and its closest \myanchor{} node from the stack}
	    $\curly{ \node, \ast, \anchor }$ := \GetFullEntry( $\traversalRecord$, $\critical$ )\;
			\tcp{read the attributes of the node}					
		  $\nMarked$ := \IsMarked( $\node$ )\; 
			$\nKey$ := \GetKey( $\node$ )\;
					
			\uIf{$\opRecord \rarrow \key$ = $\nKey$}
			{  
			   \label{lin:local-examine|stack:while:found:begin}
			   \tcp{the key stored in the node matches the one being searched for}
			   $\result$ := \True\;
				 \Break\;
				 \label{lin:local-examine|stack:while:found:end}
			} \uElseIf{($\opRecord \rarrow \key$ $<$ $\nKey$) \LOr \LNot($\nMarked$)}
			{
			   \label{lin:local-examine|stack:while:not|found:begin}
			   \tcp{the target key did not exist continuously in the tree}
			   \Break\;
				 \label{lin:local-examine|stack:while:not|found:end}
			} \Else(\tcp*[h]{examine the preceding \myanchor{} node})
			{			
			   \label{lin:local-examine|stack:while:continue:begin}
			   %% \tcp{examine the preceding \myanchor{} node}
			   $\critical$ := $\anchor$\;
				 \label{lin:local-examine|stack:while:continue:end}
			}
		  \label{lin:local-examine|stack:while:end} 
   }
	
	 %% \BlankLine
	 
	 %% \tcp{return the outcome}
	 %% \PopulateSeekRecord( $\seekRecord$, $\traversalRecord$ )\;
	 \Return $\result$\;
	 \label{lin:local-examine|stack:end}	
%%
}
%%
\BlankLine
%%
\tcp{Looks for a given key in the tree (invoked by a search operation)}
\DontPrintSemicolon
\Boolean \SeekForSearch( $\opRecord$, $\seekRecord$ )\;
\PrintSemicolon
\label{lin:local-seek|search:begin}
\Begin
{
%%
   
   
   \tcp{traverse the tree from top to down}
	 
	 $\result$ := \TraverseTree( $\opRecord$, $\seekRecord$ )\;
	 \label{lin:local-seek|search:traverse|tree}	
	 \If{\LNot($\result$)}	
	 {
	     \tcp{check if the key has moved up in the path}
	     $\result$ := \ExamineStack( $\opRecord$, $\seekRecord$ )\;
			 \label{lin:local-seek|search:examine|stack}
	 }
	
	 \tcp{return the outcome}
	 %% \tcp{return the outcome}
	 \PopulateSeekRecord( $\seekRecord$, $\traversalRecord$ )\;
	 \Return $\result$\;
   \label{lin:local-seek|search:end}
%%
}
%%
\end{algorithm}

\end{comment}

%%%%%%%%%%%%%%%%%%%%%%%%%%%%%%%%%%%%%%%%%%%%%%%%%%%%%%%%%%%%%%%%%%%%%%%%%%%%%%%%%%%%%

%% Functions used to achieve local recovery

%%%%%%%%%%%%%%%%%%%%%%%%%%%%%%%%%%%%%%%%%%%%%%%%%%%%%%%%%%%%%%%%%%%%%%%%%%%%%%%%%%%%%

\begin{algorithm}[tb]
\caption{Functions used to Achieve Local Recovery} 
\label{algo:local-local:recovery}
%%
\DefineKeyWords
%%
\tcp{Determines if the last node in the path is \mysafe}
\DontPrintSemicolon
\Boolean \FindAdmissible( $\opRecord$, $\traversalRecord$ )\;
\PrintSemicolon
\label{lin:local-test|safety:begin}
\Begin
{
%% 
   
   \tcp{examine the \myanchor{} nodes in the path one-by-one starting from the closest one}
	
	 $\curly{ \ast, \ast, \critical }$ := \GetFullEntry( $\traversalRecord$, $\top$ )\;
	 \While{\True}
	 {
	    \label{lin:local-test|safety:while:begin}
	    \tcp{retrieve the node and its \myanchor{} from the stack}
	    $\curly{ \node, \ast, \anchor }$ := \GetFullEntry( $\traversalRecord$, $\critical$ )\;
		  \tcp{read the attributes of the node}
			$\nMarked$ := \IsMarked( $\node$ )\; 
			$\nKey$ := \GetKey( $\node$ )\;
			
			\BlankLine
			
			\uIf(\tcp*[f]{the \myanchor{} node is still \myconsistent{}}){$\opRecord \rarrow \key$ $>$ $\nKey$}
		  {
			   \label{lin:local-test|safety:while:consistent:begin}
				
				 \uIf(\tcp*[f]{the last node is \mysafe{}}){\LNot($\nMarked$)}
				 {  
				    \RememberCritical( $\traversalRecord$, $\critical$ )\;
				    \Return \True\;
				 }
				 \Else(\tcp*[f]{the \myanchor{} node is \myinadmissible{}. discard the suffix and return})
				 {
				    %\uIf{\IsGreen(~)} 
						%{
						%   \tcp{examine the previous \myanchor{} node}
				    %   $\critical$ := $\anchor$\;
						%}
						%\Else
						%{
						   \RemoveUntilCritical( $\traversalRecord$, $\critical$ )\;
							 \Return \False\;
						%}
						
				 }
				 \label{lin:local-test|safety:while:consistent:end}
				
			}
			\Else(\tcp*[f]{the \myanchor{} node is \mynonconsistent{}. discard the suffix and return})
			{
			    \label{lin:local-test|safety:while:nonconsistent:begin}
			    \RemoveUntilCritical( $\traversalRecord$, $\critical$ )\;
					\Return \False\;
					\label{lin:local-test|safety:while:nonconsistent:end}
			}
				
			
			\label{lin:local-test|safety:while:end} 
	 }
	 
	 \OptReturn[\False]
	 \label{lin:local-test|safety:end}
	  
%%
}
%%
\BlankLine
%%
\tcp{Find a suitable node in the path from where to restart}
\DontPrintSemicolon
\Boolean \FindStartPoint( $\opRecord$, $\traversalRecord$ )\;
\PrintSemicolon
\label{lin:local-find|start|point:begin}
\Begin
{
%%
   
	 \While{\True}
   {
	    \label{lin:local-find|start|point:while:begin}
	    \tcp{backtrack until an unmarked node}
			$\cNode$ := \GetTop( $\traversalRecord$ )\;
			\label{lin:local-find|start|point:while:backtrack:begin}
    		
	    \While{\IsMarked( $\cNode$ )}
			{
			  
			   \RemoveFromTop( $\traversalRecord$ )\;
				 $\cNode$ := \GetTop( $\traversalRecord$ )\;
         
			}
			
			\BlankLine

      \tcp{check if the algorithm needs a clean parent node}
			\If{\NeedCleanParentNode( $\opRecord$, $\cNode$ )}
			{ 
			   \label{lin:local-find|start|point:while:clean:begin}
				 $\pNode$ := \GetSecondToTop( $\traversalRecord$ )\; 
				 \If{\LNot(\IsClean( $\pNode$ ))}
				 {
				    \tcp{need to backtrack even further}
						
											
				    \RemoveFromTop( $\traversalRecord$ )\;
						\Continue\;
						\label{lin:local-find|start|point:while:clean:end}
						\label{lin:local-find|start|point:while:backtrack:end}
				 }
			}
			
			
			\BlankLine
			
			\tcp{check if the last node in the path is a suitable restart point}
		
	    $\result$ := \FindAdmissible( $\opRecord$, $\traversalRecord$ )\;
	    \label{lin:local-find|start|point:while:test|safety}
			
			\lIf{$\result$}
			{
			   \Return \True
			}
			   
			\tcp{the path has been truncated and its last node is \myinadmissible{}}
		  $\status$ := \ExamineTop( $\opRecord$ )\;
			
			\lIf{$\status$ $\in$ \{ \STOPFOUND{}, \STOPNOTFOUND{} \}}
			{
			   \Return \False
			}
			
			\label{lin:local-find|start|point:while:end}
	 }
	
	 \OptReturn[\False]
	 \label{lin:local-find|start|point:end}
%%
}
\end{algorithm}
\begin{algorithm}[tb]
\caption{Functions used to Achieve Local Recovery  (Continued)} 
\label{algo:local-local:recovery|2}
\DefineKeyWords
\tcp{Invoked after a node in the path was deemed to be not \mysafe{}. In this case, the path would have been truncated such that its last node is \myinadmissible{}}
\tcp{Returns one of the following three values: \STOPFOUND{}, \STOPNOTFOUND{} or \DONOTKNOW{}}
\DontPrintSemicolon
\Enum Outcome \ExamineTop( $\opRecord$ )\;
\PrintSemicolon
\label{lin:local-examine|top:begin}
\Begin
{
   $\traversalRecord$ := $\opRecord \rarrow \targetStack$\;
	 \tcp{retrieve the topmost node from the stack which must be \myinadmissible{}}
	 $\cNode$ := \GetTop( $\traversalRecord$ )\;
	 $\cKey$ := \GetKey( $\cNode$ )\;
	 \BlankLine
	 \uIf{$\opRecord \rarrow \key$ $>$ $\cKey$} 
	 {   
	     \tcp{the last node is \myconsistent{}}
	     %% \tcp{can only happen for a non-green algorithm}
			 \Return \DONOTKNOW\;
			 \label{lin:local-examine|top:continue|1}
	 } 
	 \uElseIf{$\opRecord \rarrow \key$ $<$ $\cKey$}
	 {  
	    \label{lin:local-examine|top:inconsistent}
	    \tcp{the last node is \myinconsistent{}}
	    \lIf{$\opRecord \rarrow \type$ = \INSERT}
			{
			   \Return \DONOTKNOW
				 \label{lin:local-examine|top:continue|2}
			}
			\lElse
			{
			   \Return \STOPNOTFOUND
			}
	
	 }
	 \Else(\tcp*[h]{the last node contains the matching key})
	 {
	    \label{lin:local-examine|top:matching}
	    
	    \Return \STOPFOUND\;
	 }
	
	 \OptReturn[\DONOTKNOW]
	 \label{lin:local-examine|top:end}
}
%%
\end{algorithm}






%%%%%%%%%%%%%%%%%%%%%%%%%%%%%%%%%%%%%%%%%%%%%%%%%%%%%%%%%%%%%%%%%%%%%%%%%%%%%%%%%%%%%

%% Seek for target key

%%%%%%%%%%%%%%%%%%%%%%%%%%%%%%%%%%%%%%%%%%%%%%%%%%%%%%%%%%%%%%%%%%%%%%%%%%%%%%%%%%%%%




\begin{algorithm}[tb]
\caption{Seek Function for Target Key} 
\label{algo:local-seek}
%%
\DefineKeyWords
%%
\tcp{Looks for a given key in the tree (invoked by every operation)}
\DontPrintSemicolon
\Boolean \SeekForTarget( $\opRecord$, $\seekRecord$ )\;
\PrintSemicolon
\label{lin:local-seek:begin}
\Begin
{
%%
   
   $\traversalRecord$ := $\opRecord \rarrow \targetStack$\;
	 $\status$ := \DONOTKNOW\;
	 
	 \BlankLine
	
	 
	 \While{$\status$ = \DONOTKNOW}
	 { 
	    \label{lin:local-seek:while:begin}
	    \tcp{find a suitable restart point in the path}
      $\result$ := \FindStartPoint( $\opRecord$, $\traversalRecord$ )\;
	    \label{lin:local-seek:while:find|start|point} 
			
			\If(\tcp*[f]{examine the last node in the path}){\LNot($\result$)} 
			{
			   $\status$ := \ExamineTop( $\opRecord$ )\;
				 \Continue\;
				    
			}
			
	    \tcp{traverse the tree starting from the topmost node in the stack}
      $\cNode$ := \GetTop( $\traversalRecord$ )\;
			\label{lin:local-seek:while:traversal:begin}
	    \While{\True}
	    {
			  \label{lin:local-seek:while:traversal:first}
	       $\cKey$ := \GetKey( $\cNode$ )\;
		     $\which$ := $\opRecord \rarrow \key < \cKey$ ? \LEFT{} : \RIGHT{}\;
				 \label{lin:local-seek:while:traversal:select}
		    
			   \tcp{read the next address to de-reference}
		     $\reference$ := \GetChild( $\cNode$, $\which$ )\;
			   		
		     \BlankLine
				
		     \If{($\opRecord \rarrow \key$ = $\cKey$) \LOr \IsNull( $\reference$ )}
			   {
				    \label{lin:local-seek:while:traversal:stop:begin}
						
						\lIf{\IsNull( $\reference$ )}
						{
						   $\cKey$ := \GetKey( $\cNode$ )
						}
						
			      \tcp{either stop or backtrack \& restart }
												
						\uIf{$\opRecord \rarrow \key$ $\not=$ $\cKey$}
				    {
						   \tcp{if an insert operation, store the injection point}
							 \If{$\opRecord \rarrow \type$ = \INSERT}
							 {
							    $\opRecord \rarrow \injectionPoint$ :=  \GetAddress( $\reference$ )\;
									\label{lin:local-seek:while:traversal:store|injection}
							 }
							
							 \tcp{test if the terminal node is \mysafe} 		
						   $\result$ := \FindAdmissible( $\opRecord$, $\traversalRecord$ )\;
							 \label{lin:local-seek:while:traversal:test|safety}
							
					     \uIf(\tcp*[f]{terminal node is a \mysafe{} node}){$\result$}
						   {  
						      $\status$ := \STOPNOTFOUND\;
									\label{lin:local-seek:while:traversal:safe}
						   }
						   \Else(\tcp*[f]{examine the last node in the path})
							 {
							    $\status$ := \ExamineTop( $\opRecord$ )\;
									\label{lin:local-seek:while:traversal:not|safe}
						   }
						} 
						\Else(\tcp*[f]{terminal node contains the matching key})
						{
						   $\status$ := \STOPFOUND\;
							 \label{lin:local-seek:while:traversal:match}
						}
											
					  
						\Break; \tcp*[f]{terminate the current traversal}
						\label{lin:local-seek:while:traversal:stop:end}
						 
			   }
			  				
				 \BlankLine
					
			   $\address$ := \GetAddress( $\reference$ ); \tcp*[f]{traverse the next edge}
				
				 
				 \If{$\opRecord \rarrow \type$ $\in$ \{ \INSERT{}, \DELETE{} \}}
				 {
				    
				    $\restart$ := \Move( $\cNode$, $\address$, $\which$ )\;
				    \label{lin:local-seek:while:traversal:move}
				    \If(\tcp*[f]{the algorithm wants to restart the traversal}){$\restart$}
				    {
						   \Break\;
				    }
				 }
				 
				
				 
				
				 \AddToTop( $\traversalRecord$, $\address$, $\which$ ); \tcp*[f]{push the node visited into the stack}
			   \label{lin:local-seek:while:traversal:push}
			   \label{lin:local-seek:while:traversal:end}
	    } %% inner while loop
			
			
			
	    \label{lin:local-seek:while:end}	
	 }	 %% outer while loop
		
	 \BlankLine
		
	 \tcp{return the outcome}

	 \PopulateSeekRecord( $\seekRecord$, $\opRecord$ )\;
	 \label{lin:local-seek:populate}
	 \Return ($\status$ = \STOPFOUND \  ? \  \True : \False)\;
   \label{lin:local-seek:end}
%%
}
\end{algorithm}





\remove{



%%%%%%%%%%%%%%%%%%%%%%%%%%%%%%%%%%%%%%%%%%%%%%%%%%%%%%%%%%%%%%%%%%%%%%%%%%%%%%%%%%%%%

%% Seek for successor key

%%%%%%%%%%%%%%%%%%%%%%%%%%%%%%%%%%%%%%%%%%%%%%%%%%%%%%%%%%%%%%%%%%%%%%%%%%%%%%%%%%%%%



\begin{algorithm}[tb]
\caption{Seek Function for Successor Key} 
\label{algo:local-seek:successor}
%%
\DefineKeyWords
%%
\tcp{Looks for the next largest key with respect to a given key (invoked by a complex delete operation)}
\DontPrintSemicolon
\Boolean \SeekForSuccessor( $\opRecord$, $\seekRecord$ )\;
\PrintSemicolon
\label{lin:local-seek|successor:begin}
\Begin
{
%%
   
   \tcp{the stack used in locating the successor key is initialized before this function is invoked}	
	 $\traversalRecord$ := $\opRecord \rarrow \successorStack$\;
	 \While{\True}
	 {
	
	    \label{lin:local-seek|successor:while:begin}
		  \tcp{backtrack until either an unmarked node or the stack becomes empty}
			
	    
			\While{(\Size( $\traversalRecord$ ) $>$ 1)}      
			{
			    \label{lin:local-seek|successor:while:backtrack:begin}
			    $\cNode$ := \GetTop( $\traversalRecord$ )\;
					\lIf{\LNot(\IsMarked( $\cNode$ ))}
					{
					   \Break
					} 
					\lElse
					{
					   \RemoveFromTop( $\traversalRecord$ )
					}
					\label{lin:local-seek|successor:while:backtrack:end}
			}
	
	    \BlankLine
			
			\tcp{backtrack further if a clean parent is needed but the parent is not clean}
			\If{(\Size( $\traversalRecord$ ) $\geq$ 2)}
			{
			   \label{lin:local-seek|successor:while:clean:begin}
			   \If{\NeedCleanParentNode( $\opRecord$, $\cNode$ )}
		     {
				    
			      \tcp{the parent node should be a clean node}
				    $\pNode$ := \GetSecondToTop( $\traversalRecord$ )\;
				    \If{\LNot(\IsClean( $\pNode$ ))}
				    {
				       \RemoveFromTop( $\traversalRecord$ )\;
					     \Continue\;
							 \label{lin:local-seek|successor:while:clean:end}
				    }
			   }
				
			}
			
			\BlankLine
			
	    \tcp{check if the successor key is still needed}
	    $\reference$ := \NeedSuccessorKey( $\opRecord$ )\;
			\label{lin:local-seek|successor:while:need|successor}
			\If{\IsNull( $\reference$ )}{ 
			   \tcp{successor key no longer required}
			   \Return \false\;
			}
			
			\BlankLine
			
			$\cNode$ := \GetTop( $\traversalRecord$ )\;
			\uIf{(\Size( $\traversalRecord$ ) = 1)}
			{
			   \label{lin:local-seek|successor:while:traversal:if:begin}
			   %% $\address$ := \GetAddress( $\reference$ )\;
				 \tcp{visit the node pointed to by the reference returned by \NeedSuccessorKey{} function}
			   $\which$ := \RIGHT\;
				 \label{lin:local-seek|successor:while:traversal:if:end}
			}
			\Else
			{
			   \label{lin:local-seek|successor:while:traversal:else:begin}
			   \tcp{follow the left child node of the top node, if it exists}
		 	   $\reference$ := \GetChild( $\cNode$, \LEFT{} )\;
			   %% \lIf{\IsNull( $\reference$ )}{ \Break }
				 %% $\address$ := \GetAddress( $\reference$ )\;
			   $\which$ := \LEFT\;
				 \label{lin:local-seek|successor:while:traversal:else:end}
			}
			
			\Repeat{\True}
	    {
			   \label{lin:local-seek|successor:while:traversal:begin}
			   \tcp{stop if reference is null}
				 \lIf{\IsNull( $\reference$ )}{ \Break }
				
				 \tcp{obtain the address of the node}
				 $\address$ := \GetAddress( $\reference$ )\;	
				
				 \BlankLine
				
				 \tcp{traverse the edge}
				 $\restart$ := \Move( $\cNode$, $\address$, $\which$ )\;
				 \label{lin:local-seek|successor:while:traversal:move}
			   \If{$\restart$}
			   {
			      \tcp{the algorithm wants to restart the traversal}
				    \Break\;
						\label{lin:local-seek|successor:while:traversal:restart}
			   }  
				 
			   \tcp{push the node visited into the stack}
				 \AddToTop( $\traversalRecord$, $\address$, $\which$ )\;
				 \label{lin:local-seek|successor:while:traversal:stack}
				 \label{lin:local-seek|successor:while:traversal:advance:begin}
			   $\cNode$ := $\address$\;
				 \tcp{determine the next node to be visited}
			   $\reference$ := \GetChild( $\cNode$, \LEFT{} )\;
			   $\which$ := \LEFT{}\;
				 \label{lin:local-seek|successor:while:traversal:advance:end}
				 \label{lin:local-seek|successor:while:traversal:end}
			}
			\label{lin:local-seek|successor:while:end}
	 }
	
	
	 \BlankLine
	
	 \tcp{return the outcome}
	 \PopulateSeekRecord( $\seekRecord$, $\opRecord$ )\;
	 \Return \True;
	 \label{lin:local-seek|successor:end}
%%
}
%%
\end{algorithm}

}


\end{limitscope}

A pseudo-code of the local recovery algorithm is given in \pseudosref{local-data|structures}{local-seek:modify}. The pseudo-code only shows the seek phase of an algorithm and not its execution phase since the execution phase is algorithm-specific. We have also moved the pseudo-code for local recovery when looking for a successor key to the appendix due to lack of space.


The local recovery algorithm assumes that the original algorithm supports the following functions:
\begin{enumerate*}[label=(\alph*)]
%%
\item \GetKey(~), \IsMarked(~) and \GetChild(~) returns the various attributes of a tree node,
\item \IsNull(~) returns true if a reference is null and false otherwise,
\item \GetAddress(~) returns the node address stored in a reference, if non-null,
\item \Move(~) enables the original algorithm to move along an edge, which may invoke helping and restarting of the traversal as in~\cite{HowJon:2012:SPAA},
\item \NeedCleanParentNode(~) returns true if the operation needs the parent node to be clean and have no operation in progress (needed for a delete operation since it needs to modify a child pointer at the parent node), and
\item \PopulateSeekRecord(~) copies the relevant information from the traversal state required by the algorithm into a seek record.
\end{enumerate*}

\begin{comment}
\NeedSuccessorKey(~) evaluates if the successor key is still needed for the target key and returns a reference which is null if no successor key is needed and an address of the terminal node's right child otherwise
\end{comment}

\Pseudoref{local-data|structures} shows the data structures used by the local recovery algorithm. Note that all the data structures shown in \Pseudoref{local-data|structures} are \emph{local} to a process not shared among processes. A process uses three main data structures, namely \TraversalRecord{}, \OpRecord{} and \SeekRecord{}. A \TraversalRecord{} (\linesref{local-traversal|record:begin}{local-traversal|record:end}) is essentially a stack used to store the nodes visited during tree traversal when looking for a key (target or successor). Note that the traversal stack satisfies the last-in-first-out (LIFO) semantics but our algorithm sometimes uses it in a non-traditional way by accessing entries in the middle of the stack. One way to implement such an ``augmented'' stack is to use an auto-resizing vector provided as part of C++ STL library or Java package. Each entry in a traversal stack (\linesref{local-stack|entry:begin}{local-stack|entry:end}) stores the address of the node, the location of its closest \myanchor{} node (within the stack's vector) and whether the node is a left or right child of its parent. An \OpRecord{} (\linesref{local-op|record:begin}{local-op|record:end}) stores information about the operation such as type and key as well two stacks: one used when looking for the target key (all operations) and one used when looking for the successor key (only complex delete operations). Finally, a \SeekRecord{} (\linesref{local-seek|record:begin}{local-seek|record:end}) is used to return the outcome of a tree traversal to the original algorithm. Its fields are algorithm-specific. For example, for \CASTLE{}, \SeekRecord{} contains three fields: 
\begin{enumerate*}[label=(\alph*)]
\item two addresses, namely those of the target node and its parent, and
\item the contents of the injection point where an insert operation needs to attach the new node. 
\end{enumerate*}

\Pseudoref{local-stack|functions} shows the functions used to manipulate a traversal stack. The function \Size{} (\linesref{local-size:begin}{local-size:end}) returns the number of entries in the stack. The functions \GetTop{} (\linesref{local-get|top:begin}{local-get|top:end}) and \GetSecondToTop{} (\linesref{local-get|second|to|top:begin}{local-get|second|to|top:end}) return the address of the node stored in the topmost entry and the entry below it, respectively. The function \AddToTop{} (\linesref{local-add|to|top:begin}{local-add|to|top:end}) adds an entry to the top of the stack while \RemoveFromTop{} (\linesref{local-remove|from|top:begin}{local-remove|from|top:end}) removes an entry from the top of the stack. The function \RemoveUntilCritical{} (\linesref{local-remove|until|critical:begin}{local-remove|until|critical:end}) removes the entries from the top of the stack until a given point. The function \RememberCritical{} (\linesref{local-remember|critical:begin}{local-remember|critical:end}) updates the \myanchor{} field of the \myanchor{} node of the topmost entry in the stack. The function \GetFullEntry{} (\linesref{local-get|full|entry:begin}{local-get|full|entry:end} returns all the three fields of a given entry in the stack (may not be the topmost entry). The function \InitializeTraversalRecord{} (\linesref{local-initialize|traversal|record:begin}{local-initialize|traversal|record:end}) initializes a traversal stack. The stack for target key %is initialized using sentinel nodes while the stack for successor key is initialized as empty.

\Pseudosref[ \& ] {local-seek:search}{local-seek:search:2} shows the functions used to find the target key by a search operation. The function \SeekForSearch{} (\linesref{local-seek|search:begin}{local-seek|search:end}) first traverses the tree starting from the root node (\lineref{local-seek|search:traverse|tree}). If the traversal fails to locate the key, then the key may have moved up the tree. To address this possibility, the function examines the traversal stack to determine whether or not that is the case (\lineref{local-seek|search:examine|stack}). The function \TraverseTree{} (\linesref{local-traverse|tree:begin}{local-traverse|tree:end}) first initializes the traversal stack (\lineref{local-traverse|tree:initialize}) and then, starting from the topmost node in the stack (\lineref{local-traverse|tree:start}), follows either the left or the right child pointer (\lineref{local-traverse|tree:select}) until it either finds the key (\lineref{local-traverse|tree:match}) or encounters a null pointer (\lineref{local-traverse|tree:null}). It also populates the traversal stack as it moves (\lineref{local-traverse|tree:stack}). The function \ExamineStack (\linesref{local-examine|stack:begin}{local-examine|stack:end}) examines the \myanchor{} nodes stored in the stack in the reverse order in which they were visited, starting from the \myanchor{} node closest to the topmost node in the traversal stack (\lineref{local-examine|stack:start}). If the \myanchor{} node's key matches the target key, then the function returns true (\linesref{local-examine|stack:while:found:begin}{local-examine|stack:while:found:end}). If the \myanchor{} node is no longer \myconsistent{} or is unmarked, then the function returns false (\linesref{local-examine|stack:while:not|found:begin}{local-examine|stack:while:not|found:end}). Otherwise, the function backtracks and examines the preceding \myanchor{} node in the stack (\linesref{local-examine|stack:while:continue:begin}{local-examine|stack:while:continue:end}).

\Pseudosref{local-local:recovery}{local-local:recovery:2} $\&$ ~\ref{algo:local-seek:modify} show the functions used to find the target key by a modify (insert or delete) operation. The function \SeekForModify{} (\linesref{local-seek|modify:begin}{local-seek|modify:end}) first backtracks to a \mysafe{} node in the stack (\lineref{local-seek|modify:while:find|start|point}). Initially, the starting point is typically a sentinel node which is a \mysafe{} node. The function then traverses the tree from top to down by following either the left or the right child pointer (\lineref{local-seek|modify:while:traversal:select}) until it either finds the key or encounters a null pointer (\linesref{local-seek|modify:while:traversal:stop:begin}{local-seek|modify:while:traversal:stop:end}). In case the terminal node's key is greater than the target key, the function checks whether the path stored in the traversal stack is still valid (\lineref{local-seek|modify:while:traversal:find|admissible}). If not, the traversal is restarted. As the traversal moves down the tree, the function also populates the traversal stack (\linesref{local-seek|modify:while:traversal:move:begin}{local-seek|modify:while:traversal:move:end}). The function \FindAdmissible{} (\linesref{local-find|admissible:begin}{local-find|admissible:end}) checks whether or not the path stored in the stack is still valid. To that end, it examines  the \myanchor{} nodes in the stack in the reverse order in which they were visited, starting from the \myanchor{} node closest to the topmost node in the traversal stack. There are three possible cases. First, the \myanchor{} node is still consistent (\linesref{local-find|admissible:while:consistent:begin}{local-find|admissible:while:consistent:end}). In this case, the path is deemed to be valid if the \myanchor{} node is unmarked; otherwise, the function moves to the preceding \myanchor{} node. Second, the \myanchor{} node is no longer consistent (\linesref{local-find|admissible:while:not|consistent:begin}{local-find|admissible:while:not|consistent:end}). In this case, the path is deemed to be invalid. However, if the operation is a delete operation, then it can be deduced that the key did not exist in the tree  continuously and the function returns indicating that the key was not found (thereby causing the operation to terminate). Finally, the \myanchor{} node's key matches the target key (\linesref{local-find|admissible:while:match:begin}{local-find|admissible:while:match:end}). In this case, if the \myanchor{} node is marked and the operation is a delete operation, then the path is deemed to be invalid (and further backtracking is required). This is because the key may be in the process of moving up the tree. Otherwise, the function returns indicating that the key was found. The function \FindStartPoint{} (\linesref{local-find|start|point:begin}{local-find|start|point:end}) finds a \mysafe{} node on the path stored in the stack from which the operation can restart its traversal. To that end, it backtracks to an unmarked node with a clean parent if required (\linesref{local-find|start|point:while:backtrack:begin}{local-find|start|point:while:backtrack:end}). It then checks whether or not the remaining path in the stack is still valid (\lineref{local-find|start|point:while:find|admissible}). If not, it repeats the above-mentioned steps.

\begin{limitscope}

%% To limit the scope of the macros defined below

%% macros for pseudocode

\newcommand{\child}{child}
\newcommand{\node}{node}
\newcommand{\parent}{parent}

\newcommand{\mainSeekRecord}{seekTargetKey}
\newcommand{\successorSeekRecord}{seekSuccessorKey}


\newcommand{\targetStack}{targetStack}
\newcommand{\successorStack}{successorStack}


\newcommand{\successorStackInUse}{successorStackInUse}
\newcommand{\targetNode}{targetNode}




\newcommand{\key}{key}

\newcommand{\done}{done}
\newcommand{\result}{result}
\newcommand{\status}{status}
\newcommand{\restart}{restart}





\newcommand{\cKey}{key}
\newcommand{\nKey}{key}
\newcommand{\cNode}{current}
\newcommand{\pNode}{parent}
\newcommand{\nMarked}{marked}



\newcommand{\which}{which}
\newcommand{\address}{address}

\newcommand{\anchor}{anchor}

\newcommand{\stack}{stack}
\newcommand{\sTop}{top}
\newcommand{\sBottom}{bottom}
\newcommand{\current}{current}

\remove{
\newcommand{\traversalRecord}{state}
\newcommand{\TraversalRecord}{State}
\newcommand{\opRecord}{opRecord}
\newcommand{\OpRecord}{OpRecord}
\newcommand{\seekRecord}{seekRecord}
\newcommand{\SeekRecord}{SeekRecord}
}

\newcommand{\admissible}{admissible}
\newcommand{\critical}{critical}
\newcommand{\reference}{re\!f\!erence}

%% \newcommand{\OptReturn}[1][]{\Return #1\;}
\newcommand{\OptReturn}[1][]{}

\newcommand{\injectionPoint}{injectionPoint}



\newcommand{\Search}{\textsc{Search}}
\newcommand{\Insert}{\textsc{Insert}}
\newcommand{\Delete}{\textsc{Delete}}
\newcommand{\Seek}{\textsc{Seek}}

\newcommand{\Inject}{\textsc{Inject}}


%%
\newcommand{\WFSeekForSearchBOSize}{\textsc{WFSeekForSearchBasedOnSize}}
\newcommand{\WFSeekForSearchBOHeight}{\textsc{WFSeekForSearchBasedOnHeight}}
%%
\newcommand{\WFTraverseTreeBOCount}{\textsc{TraverseBasedOnCount}}
\newcommand{\WFTraverseTreeBOTimeStamp}{\textsc{TraverseBasedOnTimeStamp}}
%%
\remove{
\newcommand{\SeekForSuccessor}{\textsc{SeekForSuccessor}}
\newcommand{\NeedSuccessorKey}{\textsc{NeedSuccessorKey}}
\newcommand{\GetChild}{\textsc{GetChild}}
\newcommand{\Move}{\textsc{Move}}
\newcommand{\GetAddress}{\textsc{GetAddress}}
\newcommand{\IsNull}{\textsc{IsNull}}
\newcommand{\PopulateSeekRecord}{\textsc{PopulateSeekRecord}}
}



\newcommand{\mline}[1]{\DontPrintSemicolon #1 \PrintSemicolon}


\newcommand{\LEFT}{\textsf{LEFT}}
\newcommand{\RIGHT}{\textsf{RIGHT}}


\newcommand{\rarrow}{\!\rightarrow\!}
\newcommand{\type}{type}
\newcommand{\limit}{limit}


\newcommand{\SEARCH}{\textsf{SEARCH}}
\newcommand{\INSERT}{\textsf{INSERT}}
\newcommand{\DELETE}{\textsf{DELETE}}

\newcommand{\STOPFOUND}{\textsf{FOUND}}
\newcommand{\STOPNOTFOUND}{\textsf{NOT\_FOUND}}
\newcommand{\ADMISSIBLE}{\textsf{SAFE}}
\newcommand{\INADMISSIBLE}{\textsf{NOT\_SAFE}}

\newcommand{\TARGETSTACK}{\textsf{TARGET\_STACK}}
\newcommand{\SUCCESSORSTACK}{\textsf{SUCCESSOR\_STACK}}

%%%%%%%%%%%%%%%%%%%%%%%%%%%%%%%%%%%%%%%%%%%%%%%%%%%%%%%%%%%%%%%%%%%%%%%%%%%%%%%%%%%%

\newcommand{\DefineKeyWords}{
%%
\SetKw{Boolean}{boolean}
\SetKw{Integer}{integer}
\SetKw{LAnd}{~and~}
\SetKw{LOr}{~or~}
\SetKw{LNot}{not}
\SetKw{Struct}{struct}
\SetKw{Null}{null}
\SetKw{True}{true}
\SetKw{False}{false}
\SetKw{Break}{break}
\SetKw{Continue}{continue}
\SetKw{Enum}{enum}
\SetKw{Word}{word}
%%
}

%%%%%%%%%%%%%%%%%%%%%%%%%%%%%%%%%%%%%%%%%%%%%%%%%%%%%%%%%%%%%%%%%%%%%%%%%%%%%%%%%%%%%

%% Seek for successor key

%%%%%%%%%%%%%%%%%%%%%%%%%%%%%%%%%%%%%%%%%%%%%%%%%%%%%%%%%%%%%%%%%%%%%%%%%%%%%%%%%%%%%



\begin{algorithm}[tbh]
\caption{Seek Function for Successor Key} 
\label{algo:seek:successor}
%%
\DefineKeyWords
%%
\tcp{Looks for the next largest key with respect to a given key}
\DontPrintSemicolon
\Boolean \SeekForSuccessor( $\opRecord$, $\seekRecord$ )\;
\PrintSemicolon
\label{lin:local-seek|successor:begin}
\Begin
{
%%
   
   \tcp{the stack used in locating the successor key is initialized}
	 $\traversalRecord$ := $\opRecord \rarrow \successorStack$\;
	 \While{\True}
	 {
	
	    \label{lin:local-seek|successor:while:begin}
		  \tcp{backtrack until either an unmarked node or the stack becomes empty}
			
	    
			\While{(\Size( $\traversalRecord$ ) $>$ 1)}      
			{
			    \label{lin:local-seek|successor:while:backtrack:begin}
			    $\cNode$ := \GetTop( $\traversalRecord$ )\;
					\lIf{\LNot(\IsMarked( $\cNode$ ))}
					{
					   \Break
					} 
					\lElse
					{
					   \RemoveFromTop( $\traversalRecord$ )
					}
					\label{lin:local-seek|successor:while:backtrack:end}
			}
	
	    \BlankLine
			
			\tcp{backtrack further if a clean parent is needed but the parent is not clean}
			\If{(\Size( $\traversalRecord$ ) $\geq$ 2)}
			{
			   \label{lin:local-seek|successor:while:clean:begin}
			   \If{\NeedCleanParentNode( $\opRecord$, $\cNode$ )}
		     {
				    
			      \tcp{the parent node should be a clean node}
				    $\pNode$ := \GetSecondToTop( $\traversalRecord$ )\;
				    \If{\LNot(\IsClean( $\pNode$ ))}
				    {
				       \RemoveFromTop( $\traversalRecord$ )\;
					     \Continue\;
							 \label{lin:local-seek|successor:while:clean:end}
				    }
			   }
				
			}
			
			%\BlankLine
			
	    \tcp{check if the successor key is still needed}
	    $\reference$ := \NeedSuccessorKey( $\opRecord$ )\;
			\label{lin:local-seek|successor:while:need|successor}
			\If(\tcp*[f]{successor key no longer required}){\IsNull( $\reference$ )}{ 
			   \Return \false\;
			}
					
			$\cNode$ := \GetTop( $\traversalRecord$ )\;
			\uIf{(\Size( $\traversalRecord$ ) = 1)}
			{
			   \label{lin:local-seek|successor:while:traversal:if:begin}
			   %% $\address$ := \GetAddress( $\reference$ )\;
				 \tcp{visit the node pointed to by the reference returned by \NeedSuccessorKey{} function}
			   $\which$ := \RIGHT\;
				 \label{lin:local-seek|successor:while:traversal:if:end}
			}
			\Else(\tcp*[f]{follow the left child node of the top node, if it exists})
			{
			   \label{lin:local-seek|successor:while:traversal:else:begin}
		 	   $\reference$ := \GetChild( $\cNode$, \LEFT{} )\;
			   %% \lIf{\IsNull( $\reference$ )}{ \Break }
				 %% $\address$ := \GetAddress( $\reference$ )\;
			   $\which$ := \LEFT\;
				 \label{lin:local-seek|successor:while:traversal:else:end}
			}
			%\BlankLine
			\Repeat(\tcp*[f]{stop if reference is null}){\True}
	    {
			   \label{lin:local-seek|successor:while:traversal:begin}
				 \lIf{\IsNull( $\reference$ )}{ \Break }
				
				 \tcp{obtain the address of the node}
				 $\address$ := \GetAddress( $\reference$ )\;	
				
				 \tcp{traverse the edge}
				 $\restart$ := \Move( $\cNode$, $\address$, $\which$ )\;
				 \label{lin:local-seek|successor:while:traversal:move}
			   \If(\tcp*[f]{the algorithm wants to restart the traversal}){$\restart$}
			   {
				    \Break\;
						\label{lin:local-seek|successor:while:traversal:restart}
			   }  
				 
			   \tcp{push the node visited into the stack}
				 \AddToTop( $\traversalRecord$, $\address$, $\which$ )\;
				 \label{lin:local-seek|successor:while:traversal:stack}
				 \label{lin:local-seek|successor:while:traversal:advance:begin}
			   $\cNode$ := $\address$\;
				 \tcp{determine the next node to be visited}
			   $\reference$ := \GetChild( $\cNode$, \LEFT{} )\;
			   $\which$ := \LEFT{}\;
				 \label{lin:local-seek|successor:while:traversal:advance:end}
				 \label{lin:local-seek|successor:while:traversal:end}
			}
			\label{lin:local-seek|successor:while:end}
	 }
	
	 \tcp{return the outcome}
	 \PopulateSeekRecord( $\seekRecord$, $\opRecord$ )\;
	 \Return \True;
	 \label{lin:local-seek|successor:end}
%%
}
%%
\end{algorithm}
\end{limitscope}

\Pseudoref{seek:successor} shows the function \SeekForSuccessor{} used to locate the successor key by a complex delete operation (\linesref{local-seek|successor:begin}{local-seek|successor:end}). The function first backtracks to an unmarked node with a clean parent if required (\linesref{local-seek|successor:while:backtrack:begin}{local-seek|successor:while:clean:end}). It then checks whether or not the successor key is still needed by invoking \NeedSuccessorKey{} function (\lineref{local-seek|successor:while:need|successor}). The function \NeedSuccessorKey{} returns a reference, which is null if the successor key is no longer needed and contains the address of the target node's right child otherwise. This address is used as a traversal point if the stack only contains a single entry (the node whose key needs to be replaced). If the successor key is still needed, then the function repeatedly follows the left child pointer until it encounters a null pointer (\linesref{local-seek|successor:while:traversal:begin}{local-seek|successor:while:traversal:end}). While moving down the tree, the function also populates the traversal stack (\lineref{local-seek|successor:while:traversal:stack}).


\end{limitscope}