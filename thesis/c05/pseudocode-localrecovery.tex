\begin{limitscope}

%% macros for pseudocode
\newcommand{\child}{child}
\newcommand{\node}{node}
\newcommand{\parent}{parent}
\newcommand{\mainSeekRecord}{seekTargetKey}
\newcommand{\successorSeekRecord}{seekSuccessorKey}
\newcommand{\targetStack}{targetStack}
\newcommand{\successorStack}{successorStack}
\newcommand{\successorStackInUse}{successorStackInUse}
\newcommand{\targetNode}{targetNode}
\newcommand{\key}{key}
\newcommand{\done}{done}
\newcommand{\result}{result}
\newcommand{\status}{status}
\newcommand{\restart}{restart}
\newcommand{\cKey}{key}
\newcommand{\nKey}{key}
\newcommand{\cNode}{current}
\newcommand{\pNode}{parent}
\newcommand{\nMarked}{marked}
\newcommand{\which}{which}
\newcommand{\address}{address}
\newcommand{\anchor}{anchor}
\newcommand{\stack}{stack}
\newcommand{\sTop}{top}
\newcommand{\sBottom}{bottom}
\newcommand{\current}{current}
\remove{
\newcommand{\traversalRecord}{state}
\newcommand{\TraversalRecord}{State}
\newcommand{\opRecord}{opRecord}
\newcommand{\OpRecord}{OpRecord}
\newcommand{\seekRecord}{seekRecord}
\newcommand{\SeekRecord}{SeekRecord}
}
\newcommand{\admissible}{admissible}
\newcommand{\critical}{critical}
\newcommand{\reference}{re\!f\!erence}
\newcommand{\OptReturn}[1][]{}
\newcommand{\injectionPoint}{injectionPoint}
\newcommand{\Search}{\textsc{Search}}
\newcommand{\Insert}{\textsc{Insert}}
\newcommand{\Delete}{\textsc{Delete}}
\newcommand{\Seek}{\textsc{Seek}}
\newcommand{\Inject}{\textsc{Inject}}
%%
\newcommand{\WFSeekForSearchBOSize}{\textsc{WFSeekForSearchBasedOnSize}}
\newcommand{\WFSeekForSearchBOHeight}{\textsc{WFSeekForSearchBasedOnHeight}}
%%
\newcommand{\WFTraverseTreeBOCount}{\textsc{TraverseBasedOnCount}}
\newcommand{\WFTraverseTreeBOTimeStamp}{\textsc{TraverseBasedOnTimeStamp}}
%%
\remove{
\newcommand{\SeekForSuccessor}{\textsc{SeekForSuccessor}}
\newcommand{\NeedSuccessorKey}{\textsc{NeedSuccessorKey}}
\newcommand{\GetChild}{\textsc{GetChild}}
\newcommand{\Move}{\textsc{Move}}
\newcommand{\GetAddress}{\textsc{GetAddress}}
\newcommand{\IsNull}{\textsc{IsNull}}
\newcommand{\PopulateSeekRecord}{\textsc{PopulateSeekRecord}}
}
\newcommand{\mline}[1]{\DontPrintSemicolon #1 \PrintSemicolon}
\newcommand{\LEFT}{\textsf{LEFT}}
\newcommand{\RIGHT}{\textsf{RIGHT}}
\newcommand{\rarrow}{\!\rightarrow\!}
\newcommand{\type}{type}
\newcommand{\limit}{limit}
\newcommand{\SEARCH}{\textsf{SEARCH}}
\newcommand{\INSERT}{\textsf{INSERT}}
\newcommand{\DELETE}{\textsf{DELETE}}
\newcommand{\STOPFOUND}{\textsf{FOUND}}
\newcommand{\STOPNOTFOUND}{\textsf{NOT\_FOUND}}
\newcommand{\ADMISSIBLE}{\textsf{SAFE}}
\newcommand{\INADMISSIBLE}{\textsf{NOT\_SAFE}}
\newcommand{\TARGETSTACK}{\textsf{TARGET\_STACK}}
\newcommand{\SUCCESSORSTACK}{\textsf{SUCCESSOR\_STACK}}
%%%%%%%%%%%%%%%%%%%%%%%%%%%%%%%%%%%%%%%%%%%%%%%%%%%%%%%%%%%%%%%%%%%%%%%%%%%%%%%%%%%%
\newcommand{\DefineKeyWords}{
%%
\SetKw{Boolean}{boolean}
\SetKw{Integer}{integer}
\SetKw{LAnd}{~and~}
\SetKw{LOr}{~or~}
\SetKw{LNot}{not}
\SetKw{Struct}{struct}
\SetKw{Null}{null}
\SetKw{True}{true}
\SetKw{False}{false}
\SetKw{Break}{break}
\SetKw{Continue}{continue}
\SetKw{Enum}{enum}
\SetKw{Word}{word}
%%
}
%%%%%%%%%%%%%%%%%%%%%%%%%%%%%%%%%%%%%%%%%%%%%%%%%%%%%%%%%%%%%%%%%%%%%%%%%%%%%%%%%%%%%

%% Data structures used by the local recovery algorithm

%%%%%%%%%%%%%%%%%%%%%%%%%%%%%%%%%%%%%%%%%%%%%%%%%%%%%%%%%%%%%%%%%%%%%%%%%%%%%%%%%%%%%

\begin{algorithm}[tb]
\caption{Data Structures Used} 
\label{algo:local-data|structures}
%%
\DefineKeyWords
%%
\tcp{Used to store information about a node visited during tree traversal}
\DontPrintSemicolon
\Struct StackEntry \{\;
\PrintSemicolon
%%
\label{lin:local-data|structures:begin}
\label{lin:local-stack|entry:begin}
\Indp 
   NodePtr $\node$\;
	 \Enum Direction $\which$\;
   \Integer $\anchor$\;
\Indm 
\}\;
\label{lin:local-stack|entry:end}

\BlankLine

\tcp{Used to store the path from the root node to the current node in the tree}
\DontPrintSemicolon
\Struct \TraversalRecord{} \{\;
\PrintSemicolon
%%
\label{lin:local-traversal|record:begin}
\Indp 
   StackEntry[~] $\stack$\;
	 \Integer $\sTop$\;
	 %% \Integer $\sBottom$\;
\Indm 
\}\;
\label{lin:local-traversal|record:end}

\BlankLine

\tcp{Used to store information about the operation currently in progress}
\DontPrintSemicolon
\Struct \OpRecord{} \{\;
\PrintSemicolon
%%
\label{lin:local-op|record:begin}
\Indp 
   \Enum Type $\type$\;
	 Key $\key$\;
	 \TraversalRecord{} $\targetStack$, $\successorStack$\;
	 %%\TraversalRecord{} $\successorStack$\;
	 %% \Boolean $\successorStackInUse$\;
	 %% NodePtr $\targetNode$\;
	 NodePtr $\injectionPoint$\;
	 \BlankLine
	 \tcp{algorithm-specific fields}
\Indm
\}\;
\label{lin:local-op|record:end}

\BlankLine

\tcp{Used to store the outcome of a tree traversal}
\DontPrintSemicolon
\Struct \SeekRecord \{\;
\PrintSemicolon
%%
\label{lin:local-seek|record:begin}
\Indp 
   %% \TraversalRecord{}Ptr $\traversalRecord$\;
	 \tcp{algorithm-specific fields (\emph{e.g.}, target node and its parent)}
\Indm 
\}\;
\label{lin:local-seek|record:end}

\remove{
   \BlankLine
%%
   \tcp{Local records used when executing an operation}
   \OpRecord{}Ptr $\opRecord$\;
   SeekRecordPtr $\mainSeekRecord$\;
   SeekRecordPtr $\successorSeekRecord$\; 
}
\label{lin:local-data|structures:end}
%%
\end{algorithm}


%%%%%%%%%%%%%%%%%%%%%%%%%%%%%%%%%%%%%%%%%%%%%%%%%%%%%%%%%%%%%%%%%%%%%%%%%%%%%%%%%%%%%

%% Functions used for manipulating traversal stack

%%%%%%%%%%%%%%%%%%%%%%%%%%%%%%%%%%%%%%%%%%%%%%%%%%%%%%%%%%%%%%%%%%%%%%%%%%%%%%%%%%%%%

\begin{algorithm}[tb]
\caption{Functions for Manipulating Traversal Stack} 
\label{algo:local-stack|functions}
%%
\DefineKeyWords
%%
\tcp{Returns the number of elements in the stack}
\DontPrintSemicolon
\Integer \Size( $\traversalRecord$ )\;
\PrintSemicolon
\label{lin:local-stack|begin}
\label{lin:local-size:begin}
\Begin
{
   
   \Return $\traversalRecord \rarrow \sTop + 1$\;
	 \label{lin:local-size:end}
}
%%
\BlankLine
%%
\tcp{Returns the topmost node in the stack}
\DontPrintSemicolon
NodePtr \GetTop( $\traversalRecord$ )\;
\PrintSemicolon
\label{lin:local-get|top:begin}
\Begin
{
   
   $\curly{ \stack, \sTop }$ := $\traversalRecord$\;
	 \label{lin:local-stack|retrieve}
	 \Return $\stack[\sTop] \rarrow \node$\;
	 \label{lin:local-get|top:end}
}
%%
\BlankLine
%%
\tcp{Returns the second topmost node in the stack}
\DontPrintSemicolon
NodePtr \GetSecondToTop( $\traversalRecord$ )\;
\label{lin:local-get|second|to|top:begin}
\PrintSemicolon
\Begin
{
   
   $\curly{ \stack, \sTop }$ := $\traversalRecord$\;
	 \Return $\stack[\sTop-1] \rarrow \node$\;
	 \label{lin:local-get|second|to|top:end}
}
%%
\BlankLine
%%
\tcp{Adds the given node to the stack along with its \myanchor{} node}
\DontPrintSemicolon
\AddToTop(  $\traversalRecord$, $\node$, $\which$ )\;
\PrintSemicolon
\label{lin:local-add|to|top:begin}
\Begin
{
%%
   
   $\curly{ \stack, \sTop }$ := $\traversalRecord$\;
   	
	 \tcp{find the \myanchor{} node}
   \lIf{$\which$ = \RIGHT}
	 {
	    $\anchor$ := $\sTop$
	 }
	 \lElse
	 {
	    $\anchor$ := $\stack[\sTop] \rarrow \anchor$
	 }
   
	 \tcp{push the node into the stack}
   $\stack[\sTop + 1]$ := $\curly{ \node, \which, \anchor }$\;
	 $\traversalRecord \rarrow \sTop$ := $\sTop + 1$\;
	 \OptReturn
	 \label{lin:local-add|to|top:end}
	
%%
}
%%
\BlankLine
%%
\tcp{Removes the topmost node from the stack}
\DontPrintSemicolon
\RemoveFromTop ( $\traversalRecord$ )\;
\PrintSemicolon
\label{lin:local-remove|from|top:begin}
\Begin
{
%%
   
   $\curly{ \stack, \sTop }$ := $\traversalRecord$\;
	
	 \tcp{update the \myanchor{} node of the penultimate entry if needed}
	 $\anchor$ := $\stack[\sTop - 1] \rarrow \anchor$\;
	 \If{$\stack[\sTop] \rarrow \anchor$ $<$ $\stack[\anchor] \rarrow \anchor$}
	 {
	    $\stack[\anchor] \rarrow \anchor$ := $\stack[\sTop] \rarrow \anchor$\;  
	 }
	
	 \tcp{pop the node from the stack}
	 $\traversalRecord \rarrow \sTop$ := $\sTop - 1$\;
	 \OptReturn
	 \label{lin:local-remove|from|top:end}
	 	 
%%
}
%%
\BlankLine
%%
\tcp{Pops the stack until a given entry}
\DontPrintSemicolon
\RemoveUntilCritical( $\traversalRecord$,  $index$ )\;
\PrintSemicolon
\label{lin:local-remove|until|critical:begin}
\Begin
{
  
   $\traversalRecord \rarrow \sTop$ := $index$\;
	 \OptReturn
	 \label{lin:local-remove|until|critical:end}

}
\end{algorithm}

\begin{algorithm}
\caption{Functions for Manipulating Traversal Stack (Continued)} 
\label{algo:local-stack|functions|2}
\DefineKeyWords
%%
%\BlankLine
%%
\tcp{Remember the \mycritical{} node (to avoid locating it again)}
\DontPrintSemicolon
\RememberCritical( $\traversalRecord$,  $\critical$ )\;
\PrintSemicolon
\label{lin:local-remember|critical:begin}
\Begin
{
  
   %% $\curly{ \stack, \sTop }$ := $\traversalRecord$\;  
	 $\curly{ \stack, \sTop }$ := $\traversalRecord$\;
   	
	 $\anchor$ := $\stack[\sTop] \rarrow \anchor$\;
	 \If{$\critical$ $<$ $\stack[\anchor] \rarrow \anchor$}
	 {
	   $\stack[\anchor] \rarrow \anchor$ := $\critical$\;
	 }
	 \OptReturn
	 \label{lin:local-remember|critical:end}
}
%%
\BlankLine
%%
\tcp{Returns a given entry in the stack}
\DontPrintSemicolon
\{ NodePtr, \Enum Direction, \Integer \} \\
\qquad\qquad\qquad\qquad \GetFullEntry( $\traversalRecord$,  $index$ )\;
\PrintSemicolon
\label{lin:local-get|full|entry:begin}
\Begin
{
   
   $\curly{ \stack, \sTop }$ := $\traversalRecord$\;
	 
	
	 \remove{
	
	    \tcp{find the location of the entry in the stack}
			
	    \lIf{$entry$ = $\top$}
	    {
	       $index$ := $\sTop$
	    }
	    \lElse
	    {
	       $index$ := $entry$
	    }
	
	}
	
	\lIf{$index$ = $\top$}
	{
	   \Return $\stack[\sTop]$
	}
	\lElse
	{
	   \Return $\stack[index]$
	}
	

	\label{lin:local-get|full|entry:end}
}
%%
\BlankLine
%%
\tcp{initializes the traversal stack} 
\DontPrintSemicolon
\InitializeTraversalRecord( $\traversalRecord$, $\type$ )\;
\PrintSemicolon
\label{lin:local-initialize|traversal|record:begin}
\Begin
{
%%
  
  \uIf{$\type$ = \TARGETSTACK}
	{
		
	   %% $\traversalRecord \rarrow \stack[0]$ := $\ang{\sNodeOne, -1}$\;
     %% $\traversalRecord \rarrow \stack[1]$ := $\ang{\sNodeTwo, 0 }$\;
	   %% $\traversalRecord \rarrow \sTop$ := 1\;
	   %% $\traversalRecord \rarrow \anchor$ := 0\;
	   \tcp{initialize the stack using sentinel nodes}
	   \tcp{sentinel nodes are never removed from the stack}
	   \tcp{a sentinel node is always a safe starting point for the traversal}
	}
	\lElse
	{
	   $\traversalRecord \rarrow \sTop$ := -1
	}
	
	\OptReturn
	\label{lin:local-initialize|traversal|record:end}

%%
}
\end{algorithm}


\begin{comment}

\begin{algorithm}[tb]
\caption{Functions for Manipulating Traversal Stack (Continued)} 
\label{algo:local-stack|functions|2}
%%
\DefineKeyWords

\tcp{initializes the traversal stack} 
\DontPrintSemicolon
\InitializeTraversalRecord( $\traversalRecord$, $\type$ )\;
\PrintSemicolon
\Begin
{
%%
 
  \uIf{$\type$ = \TARGETSTACK}
	{
		
	   %% $\traversalRecord \rarrow \stack[0]$ := $\ang{\sNodeOne, -1}$\;
     %% $\traversalRecord \rarrow \stack[1]$ := $\ang{\sNodeTwo, 0 }$\;
	   %% $\traversalRecord \rarrow \sTop$ := 1\;
	   %% $\traversalRecord \rarrow \anchor$ := 0\;
	   \tcp{initialize the stack using sentinel nodes}
	   \tcp{the sentinel nodes are never removed from the stack}
	   \tcp{a sentinel node is always a safe starting point for the traversal}
	}
	\lElse{ $\traversalRecord \rarrow \sTop$ := -1 }
	
	\OptReturn

%%
}

\end{algorithm}

\end{comment}


%%%%%%%%%%%%%%%%%%%%%%%%%%%%%%%%%%%%%%%%%%%%%%%%%%%%%%%%%%%%%%%%%%%%%%%%%%%%%%%%%%%%%

%% Seek for search operation

%%%%%%%%%%%%%%%%%%%%%%%%%%%%%%%%%%%%%%%%%%%%%%%%%%%%%%%%%%%%%%%%%%%%%%%%%%%%%%%%%%%%%




\begin{algorithm}[tb]
\caption{Seek Function for Target Key (Search Operation)} 
\label{algo:local-seek:search}
%%
\DefineKeyWords
%%
\tcp{Traverses the tree starting from the root until either the key is found or a null pointer is encountered}
%% \tcp{also populates the traversal stack}
\DontPrintSemicolon
\Boolean \TraverseTree( $\opRecord$, $\seekRecord$ )\;
\PrintSemicolon
\label{lin:local-traverse|tree:begin}
\Begin
{
%%
   $\traversalRecord$ := $\opRecord \rarrow \targetStack$\;
	
	 \tcp{initialize the stack and the variables used in the traversal}
	 \InitializeTraversalRecord( $\traversalRecord$, \TARGETSTACK{} )\;
	 \label{lin:local-traverse|tree:initialize}
	 %% \tcp{initialize the variables used in the traversal}
   $\cNode$ := \GetTop( $\traversalRecord$ )\;
	 \label{lin:local-traverse|tree:start}
	 \BlankLine
	 \tcp{traverse the tree (starting from $\cNode$)}
	 \While{\True}
	 {
			\label{lin:local-traverse|tree:while:begin}
	    $\cKey$ := \GetKey( $\cNode$ )\;
			\label{lin:local-traverse|tree:while:first}
		  $\which$ := $\opRecord \rarrow \key < \cKey$ ? \LEFT{} : \RIGHT{}\;
			\label{lin:local-traverse|tree:select}
		  \tcp{read the next address to de-reference}
		  %% $\ang{ \ast, \ast, \nFlag, \address}$ := $\cNode \rarrow \child[\which]$\;  
			$\reference$ := \GetChild( $\cNode$, $\which$ )\;
			
			\BlankLine
			
		  \lIf{$\opRecord \rarrow \key$ = $\cKey$}
			{ 
			   \Return \True 
				 \label{lin:local-traverse|tree:match}
			}	     
			\lIf{\IsNull( $\reference$ )}
			{ 
			   \Return \False
				 \label{lin:local-traverse|tree:null}
			}	  
				
			\BlankLine
			
		  \tcp{traverse the next edge}
		  %% $\pNode$ := $\cNode$;
			$\address$ := \GetAddress( $\reference$ )\;
			$\cNode$ := $\address$\;
			\tcp{push the next node to be visited into the stack}
			\AddToTop( $\traversalRecord$, $\address$, $\which$ )\;
			\label{lin:local-traverse|tree:stack}
			\label{lin:local-traverse|tree:while:end}
			      
	  }	
		
		\OptReturn[\False]
		\label{lin:local-traverse|tree:end}
%%
}
%%
\BlankLine
%%
\tcp{Checks if the key being searched for has moved up in the path stored in the stack}
\DontPrintSemicolon
\Boolean \ExamineStack( $\opRecord$, $\seekRecord$ )\;
\PrintSemicolon
\label{lin:local-examine|stack:begin}
\Begin
{
%%
  
   $\result$ := \False\;
	 $\traversalRecord$ := $\opRecord \rarrow \targetStack$\;
	 
	 \BlankLine
	
   \tcp{start with the \myanchor{} closest to the topmost node in the stack}
	 $\curly{ \ast, \ast, \critical }$ := \GetFullEntry( $\traversalRecord$, $\top$ )\;
	 \label{lin:local-examine|stack:start}	
			
	 \BlankLine
			
			
	 \While{\True}
	 {
	    \label{lin:local-examine|stack:while:begin}
	    \tcp{retrieve the node and its closest \myanchor{} node from the stack}
	    $\curly{ \node, \ast, \anchor }$ := \GetFullEntry( $\traversalRecord$, $\critical$ )\;
			\tcp{read the attributes of the node}					
		  $\nMarked$ := \IsMarked( $\node$ )\; 
			$\nKey$ := \GetKey( $\node$ )\;
					
			\uIf{$\opRecord \rarrow \key$ = $\nKey$}
			{  
			   \label{lin:local-examine|stack:while:found:begin}
			   \tcp{the key stored in the node matches the one being searched for}
			   $\result$ := \True\;
				 \Break\;
				 \label{lin:local-examine|stack:while:found:end}
			} \uElseIf{($\opRecord \rarrow \key$ $<$ $\nKey$) \LOr \LNot($\nMarked$)}
			{
			   \label{lin:local-examine|stack:while:not|found:begin}
			   \tcp{the target key did not exist continuously in the tree}
			   \Break\;
				 \label{lin:local-examine|stack:while:not|found:end}
			} \Else(\tcp*[h]{examine the preceding \myanchor{} node})
			{			
			   \label{lin:local-examine|stack:while:continue:begin}
			   %% \tcp{examine the preceding \myanchor{} node}
			   $\critical$ := $\anchor$\;
				 \label{lin:local-examine|stack:while:continue:end}
			}
		  \label{lin:local-examine|stack:while:end} 
   }
	
	 %% \BlankLine
	 
	 %% \tcp{return the outcome}
	 %% \PopulateSeekRecord( $\seekRecord$, $\traversalRecord$ )\;
	 \Return $\result$\;
	 \label{lin:local-examine|stack:end}	
%%
}
\end{algorithm}

\begin{algorithm}[tb]
\caption{Seek Function for Target Key (Search Operation) (continued)} 
\label{algo:local-seek:search:2}
%%
\DefineKeyWords
%%
%\BlankLine
%%
\tcp{Looks for a given key in the tree (invoked by a search operation)}
\DontPrintSemicolon
\Boolean \SeekForSearch( $\opRecord$, $\seekRecord$ )\;
\PrintSemicolon
\label{lin:local-seek|search:begin}
\Begin
{
%%
   
   
   \tcp{traverse the tree from top to down}
	 
	 $\result$ := \TraverseTree( $\opRecord$, $\seekRecord$ )\;
	 \label{lin:local-seek|search:traverse|tree}	
	 \If{\LNot($\result$)}	
	 {
	     \tcp{check if the key has moved up in the path stored in the stack}
	     $\result$ := \ExamineStack( $\opRecord$, $\seekRecord$ )\;
			 \label{lin:local-seek|search:examine|stack}
	 }
	
	 \tcp{return the outcome}
	 %% \tcp{return the outcome}
	 \PopulateSeekRecord( $\seekRecord$, $\traversalRecord$ )\;
	 \Return $\result$\;
   \label{lin:local-seek|search:end}
%%
}
%%
\end{algorithm}


%%%%%%%%%%%%%%%%%%%%%%%%%%%%%%%%%%%%%%%%%%%%%%%%%%%%%%%%%%%%%%%%%%%%%%%%%%%%%%%%%%%%%

%% Functions used to achieve local recovery

%%%%%%%%%%%%%%%%%%%%%%%%%%%%%%%%%%%%%%%%%%%%%%%%%%%%%%%%%%%%%%%%%%%%%%%%%%%%%%%%%%%%%

\begin{algorithm}[tb]
%\caption{Functions used to Achieve Local Recovery} 
\caption{validate path}
\label{algo:local-local:recovery}
%%
\DefineKeyWords
%%
\tcp{Determines if the path stored in the stack is still valid}
\tcp{Returns one of the following four values:}
\tcp{\{ \ADMISSIBLE, \INADMISSIBLE,  \STOPFOUND, \STOPNOTFOUND \}}
\tcp{May backtrack along the path under certain situations}
\DontPrintSemicolon
\Enum Outcome \FindAdmissible( $\opRecord$, $\traversalRecord$ )\;
\PrintSemicolon
\label{lin:local-find|admissible:begin}
\Begin
{
%% 
   
   \tcp{check if any of the \myanchor{} nodes in the stack has become \myinconsistent{} starting with the one immediately preceding the topmost node in the stack}
	 %%  \tcp{start with the \myanchor{} node of the topmost node in the stack}
	 $\curly{ \ast, \ast, \critical }$ := \GetFullEntry( $\traversalRecord$, $\top$ )\;
	 	
	 \While{\True}
	 {
	    \label{lin:local-find|admissible:while:begin}
	    \tcp{retrieve the node and its \myanchor{} from the stack}
	    $\curly{ \node, \ast, \anchor }$ := \GetFullEntry( $\traversalRecord$, $\critical$ )\;
		  \tcp{read the attributes of the node}
			$\nMarked$ := \IsMarked( $\node$ )\; 
			$\nKey$ := \GetKey( $\node$ )\;
			
			\BlankLine
			
			\uIf{$\opRecord \rarrow \key$ $>$ $\nKey$}
		  {
			   \label{lin:local-find|admissible:while:consistent:begin}
			   \tcp{the \myanchor{} node is still \myconsistent{}}
				
				 \uIf{\LNot($\nMarked$)}
				 { 
				   
				    \tcp{the \accesspath is still valid}
				    \RememberCritical( $\traversalRecord$, $\critical$ )\;
				    \Return \ADMISSIBLE\;
				 }
				 \Else(\tcp*[h]{need to check the previous \myanchor{} node})
				 {
				    $\critical$ := $\anchor$\;
						\label{lin:local-find|admissible:while:continue}
				 }
				 \label{lin:local-find|admissible:while:consistent:end}
				
			} \uElseIf{$\opRecord \rarrow \key$ $<$ $\nKey$}
			{
			   \label{lin:local-find|admissible:while:not|consistent:begin}
			   \tcp{the \myanchor{} node is no longer consistent}
			   \uIf{$\opRecord \rarrow \type$ = \DELETE}
				 { 
				   
				    \tcp{the target key did not exist continuously in the tree}
				    \Return \STOPNOTFOUND\;
				 }
				 \Else(\tcp*[h]{the path is not valid})
				 {
			       \RemoveUntilCritical( $\traversalRecord$, $\critical$ )\;
						 \label{lin:local-find|admissible:consistent}
						 \Return \INADMISSIBLE\;
				 } 
				 \label{lin:local-find|admissible:while:not|consistent:end}
			} \Else(\tcp*[h]{the two keys match})
			{
			   \label{lin:local-find|admissible:while:match:begin}
			   %% \tcp{the key stored in the \myanchor{} node matches the one being searched for}
				
						
				 \RemoveUntilCritical( $\traversalRecord$, $\critical$ )\;
				 	
				 
				 \tcp{stop the traversal}
				 \Return \STOPFOUND\;
				 \label{lin:local-find|admissible:while:match:end}
				
				 \remove{
			   \uIf{($\opRecord \rarrow \type$ = \INSERT) \LOr \LNot($\nMarked$)}
				 { 
				    \tcp{stop the traversal}
				    \Return \STOPFOUND\;
						\label{lin:local-find|admissible:while:match:stop}
				 } 
				 \Else
				 {   
				    \tcp{the \myanchor{} node is marked; the key may have moved up}
				    \Return \INADMISSIBLE\;
						\label{lin:local-find|admissible:while:match:continue}
						\label{lin:local-find|admissible:while:match:end}
				 }
				 }
				
				
			}
			\label{lin:local-find|admissible:while:end} 
	 }
	 
	 \OptReturn[\INADMISSIBLE]
	 \label{lin:local-find|admissible:end}
	  
%%
}
\end{algorithm}

\begin{algorithm}[tb]
\caption{find a safe node}
\label{algo:local-local:recovery:2}
%%
\DefineKeyWords
%%
\BlankLine
%%
\tcp{Backtracks along the path stored in the stack until a suitable restart point is found}
\DontPrintSemicolon
\Enum Outcome \FindStartPoint( $\opRecord$, $\traversalRecord$ )\;
\PrintSemicolon
\label{lin:local-find|start|point:begin}
\Begin
{
%%
   
	 \While{\True}
   {
	    \label{lin:local-find|start|point:while:begin}
	    \tcp{backtrack until an unmarked node}
			$\cNode$ := \GetTop( $\traversalRecord$ )\;
			\label{lin:local-find|start|point:while:backtrack:begin}
    		
	    \While{\IsMarked( $\cNode$ )}
			{
			  
			   \RemoveFromTop( $\traversalRecord$ )\;
				 $\cNode$ := \GetTop( $\traversalRecord$ )\;
         
			}
			
			\BlankLine

      \tcp{check if the algorithm needs a clean parent node}
			\If{\NeedCleanParentNode( $\opRecord$, $\cNode$ )}
			{ 
			   \label{lin:local-find|start|point:while:clean:begin}
				 $\pNode$ := \GetSecondToTop( $\traversalRecord$ )\; 
				 \If{\LNot(\IsClean( $\pNode$ ))}
				 {
				    \tcp{need to backtrack even further}
						
											
				    \RemoveFromTop( $\traversalRecord$ )\;
						\Continue\;
						\label{lin:local-find|start|point:while:clean:end}
						\label{lin:local-find|start|point:while:backtrack:end}
				 }
			}
			
			
			\BlankLine
			
			\tcp{check if the topmost node in the stack is a suitable restart point}
		
	    $\result$ := \FindAdmissible( $\opRecord$, $\traversalRecord$ )\;
	    \label{lin:local-find|start|point:while:find|admissible}
		  \lIf{ $\result$ $\neq$ \INADMISSIBLE }
			{
			   \Return $\result$
			}
			
			\label{lin:local-find|start|point:while:end}
	 }
	
	 \OptReturn[\ADMISSIBLE]
	 \label{lin:local-find|start|point:end}
%%
}
%%
\end{algorithm}






%%%%%%%%%%%%%%%%%%%%%%%%%%%%%%%%%%%%%%%%%%%%%%%%%%%%%%%%%%%%%%%%%%%%%%%%%%%%%%%%%%%%%

%% Seek for target key

%%%%%%%%%%%%%%%%%%%%%%%%%%%%%%%%%%%%%%%%%%%%%%%%%%%%%%%%%%%%%%%%%%%%%%%%%%%%%%%%%%%%%




\begin{algorithm}[tb]
\caption{Seek Function for Target Key (Modify Operation)} 
\label{algo:local-seek:modify}
%%
\DefineKeyWords
%%
\tcp{Looks for a given key in the tree (invoked by insert/delete operation)}
\DontPrintSemicolon
\Boolean \SeekForModify( $\opRecord$, $\seekRecord$ )\;
\PrintSemicolon
\label{lin:local-seek|modify:begin}
\Begin
{
%%
   
   $\traversalRecord$ := $\opRecord \rarrow \targetStack$\;
	 $\result$ := \INADMISSIBLE\;
	 
	 \BlankLine
	
	 \While{$\result$ = \INADMISSIBLE}
	 { 
	    \label{lin:local-seek|modify:while:begin}
	    \tcp{backtrack to a suitable restart point in the path stored in the stack}
      $\result$ := \FindStartPoint( $\opRecord$, $\traversalRecord$ )\;
	    \label{lin:local-seek|modify:while:find|start|point}
	    \lIf{$\result$ $\in$ $\{ \STOPFOUND, \STOPNOTFOUND \}$}{ \Break }
			
			\BlankLine
			
	    \tcp{traverse the tree starting from the topmost node in the stack}
      $\cNode$ := \GetTop( $\traversalRecord$ )\;
			\label{lin:local-seek|modify:while:traversal:begin}
	    \While{\True}
	    {
			  
	       $\cKey$ := \GetKey( $\cNode$ )\;
		     $\which$ := $\opRecord \rarrow \key < \cKey$ ? \LEFT{} : \RIGHT{}\;
				 \label{lin:local-seek|modify:while:traversal:select}
		     \tcp{read the next address to de-reference}
		     $\reference$ := \GetChild( $\cNode$, $\which$ )\;
			   		
		     \BlankLine
				
		     \If{($\opRecord \rarrow \key$ = $\cKey$) \LOr \IsNull( $\reference$ )}
			   {
				    \label{lin:local-seek|modify:while:traversal:stop:begin}
			      \tcp{either stop or backtrack \& restart }
			      \uIf{$\opRecord \rarrow \key$ $<$ $\cKey$}
				    {
						   \tcp{check if the path traversed is still valid}
				       $\result$ := \FindAdmissible( $\opRecord$, $\traversalRecord$ )\;
							 \label{lin:local-seek|modify:while:traversal:find|admissible}
							 
							 
							
				    } \Else(\tcp*[h]{determine what value to return})
				    {
						  
							 
							 \label{lin:local-seek|modify:while:traversal:stop}
							 %% \tcp{determine what value to return}
							
							 $\result$ := \STOPFOUND\;
							 
						   \If{$\opRecord \rarrow \key$ $\neq$ $\cKey$}
							 {
							    \tcp{remember the address read}
									$\opRecord \rarrow \injectionPoint$ := \\
									\qquad\qquad \GetAddress( $\reference$ )\;
							    $\result$ := \STOPNOTFOUND\;
							 }
							
						} 
				   
					  \tcp{terminate the current traversal}
						\Break\;
						\label{lin:local-seek|modify:while:traversal:stop:end}
						 
			   }
			  				
				 \BlankLine
					
				 \tcp{traverse the next edge}
			   $\address$ := \GetAddress( $\reference$ )\;
				 \label{lin:local-seek|modify:while:traversal:move:begin}
				 $\restart$ := \Move( $\cNode$, $\address$, $\which$ )\;
				
				 \If{$\restart$}
				 {
				    \tcp{the algorithm wants to restart the traversal}
						\Break\;
				 }  
				 
				
				 \tcp{push the node visited into the stack}
				
				 \AddToTop( $\traversalRecord$, $\address$, $\which$ )\;
				 \label{lin:local-seek|modify:while:traversal:move:end}
			   \label{lin:local-seek|modify:while:traversal:end}
	    } %% inner while loop
			
			
			
	    \label{lin:local-seek|modify:while:end}	
	 }	 %% outer while loop
		
	 \BlankLine
		
	 \tcp{return the outcome}

	 \PopulateSeekRecord( $\seekRecord$, $\opRecord$ )\;
	 \label{lin:local-seek|modify:populate}
	 \Return ($\result$ = \STOPFOUND \  ? \  \True : \False)\;
   \label{lin:local-seek|modify:end}
%%
}
\end{algorithm}





\remove{



%%%%%%%%%%%%%%%%%%%%%%%%%%%%%%%%%%%%%%%%%%%%%%%%%%%%%%%%%%%%%%%%%%%%%%%%%%%%%%%%%%%%%

%% Seek for successor key

%%%%%%%%%%%%%%%%%%%%%%%%%%%%%%%%%%%%%%%%%%%%%%%%%%%%%%%%%%%%%%%%%%%%%%%%%%%%%%%%%%%%%



\begin{algorithm}[tb]
\caption{Seek Function for Successor Key} 
\label{algo:local-seek:successor}
%%
\DefineKeyWords
%%
\tcp{Looks for the next largest key with respect to a given key (invoked by a complex delete operation)}
\DontPrintSemicolon
\Boolean \SeekForSuccessor( $\opRecord$, $\seekRecord$ )\;
\PrintSemicolon
\label{lin:local-seek|successor:begin}
\Begin
{
%%
   
   \tcp{the stack used in locating the successor key is initialized before this function is invoked}	
	 $\traversalRecord$ := $\opRecord \rarrow \successorStack$\;
	 \While{\True}
	 {
	
	    \label{lin:local-seek|successor:while:begin}
		  \tcp{backtrack until either an unmarked node or the stack becomes empty}
			
	    
			\While{(\Size( $\traversalRecord$ ) $>$ 1)}      
			{
			    \label{lin:local-seek|successor:while:backtrack:begin}
			    $\cNode$ := \GetTop( $\traversalRecord$ )\;
					\lIf{\LNot(\IsMarked( $\cNode$ ))}
					{
					   \Break
					} 
					\lElse
					{
					   \RemoveFromTop( $\traversalRecord$ )
					}
					\label{lin:local-seek|successor:while:backtrack:end}
			}
	
	    \BlankLine
			
			\tcp{backtrack further if a clean parent is needed but the parent is not clean}
			\If{(\Size( $\traversalRecord$ ) $\geq$ 2)}
			{
			   \label{lin:local-seek|successor:while:clean:begin}
			   \If{\NeedCleanParentNode( $\opRecord$, $\cNode$ )}
		     {
				    
			      \tcp{the parent node should be a clean node}
				    $\pNode$ := \GetSecondToTop( $\traversalRecord$ )\;
				    \If{\LNot(\IsClean( $\pNode$ ))}
				    {
				       \RemoveFromTop( $\traversalRecord$ )\;
					     \Continue\;
							 \label{lin:local-seek|successor:while:clean:end}
				    }
			   }
				
			}
			
			\BlankLine
			
	    \tcp{check if the successor key is still needed}
	    $\reference$ := \NeedSuccessorKey( $\opRecord$ )\;
			\label{lin:local-seek|successor:while:need|successor}
			\If{\IsNull( $\reference$ )}{ 
			   \tcp{successor key no longer required}
			   \Return \false\;
			}
			
			\BlankLine
			
			$\cNode$ := \GetTop( $\traversalRecord$ )\;
			\uIf{(\Size( $\traversalRecord$ ) = 1)}
			{
			   \label{lin:local-seek|successor:while:traversal:if:begin}
			   %% $\address$ := \GetAddress( $\reference$ )\;
				 \tcp{visit the node pointed to by the reference returned by \NeedSuccessorKey{} function}
			   $\which$ := \RIGHT\;
				 \label{lin:local-seek|successor:while:traversal:if:end}
			}
			\Else
			{
			   \label{lin:local-seek|successor:while:traversal:else:begin}
			   \tcp{follow the left child node of the top node, if it exists}
		 	   $\reference$ := \GetChild( $\cNode$, \LEFT{} )\;
			   %% \lIf{\IsNull( $\reference$ )}{ \Break }
				 %% $\address$ := \GetAddress( $\reference$ )\;
			   $\which$ := \LEFT\;
				 \label{lin:local-seek|successor:while:traversal:else:end}
			}
			
			\Repeat{\True}
	    {
			   \label{lin:local-seek|successor:while:traversal:begin}
			   \tcp{stop if reference is null}
				 \lIf{\IsNull( $\reference$ )}{ \Break }
				
				 \tcp{obtain the address of the node}
				 $\address$ := \GetAddress( $\reference$ )\;	
				
				 \BlankLine
				
				 \tcp{traverse the edge}
				 $\restart$ := \Move( $\cNode$, $\address$, $\which$ )\;
				 \label{lin:local-seek|successor:while:traversal:move}
			   \If{$\restart$}
			   {
			      \tcp{the algorithm wants to restart the traversal}
				    \Break\;
						\label{lin:local-seek|successor:while:traversal:restart}
			   }  
				 
			   \tcp{push the node visited into the stack}
				 \AddToTop( $\traversalRecord$, $\address$, $\which$ )\;
				 \label{lin:local-seek|successor:while:traversal:stack}
				 \label{lin:local-seek|successor:while:traversal:advance:begin}
			   $\cNode$ := $\address$\;
				 \tcp{determine the next node to be visited}
			   $\reference$ := \GetChild( $\cNode$, \LEFT{} )\;
			   $\which$ := \LEFT{}\;
				 \label{lin:local-seek|successor:while:traversal:advance:end}
				 \label{lin:local-seek|successor:while:traversal:end}
			}
			\label{lin:local-seek|successor:while:end}
	 }
	
	
	 \BlankLine
	
	 \tcp{return the outcome}
	 \PopulateSeekRecord( $\seekRecord$, $\opRecord$ )\;
	 \Return \True;
	 \label{lin:local-seek|successor:end}
%%
}
%%
\end{algorithm}

}


\end{limitscope}