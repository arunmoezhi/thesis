\begin{limitscope}

%% To limit the scope of the macros defined below

%% macros for pseudocode

\newcommand{\child}{child}
\newcommand{\node}{node}
\newcommand{\parent}{parent}

\newcommand{\mainSeekRecord}{seekTargetKey}
\newcommand{\successorSeekRecord}{seekSuccessorKey}


\newcommand{\targetStack}{targetStack}
\newcommand{\successorStack}{successorStack}


\newcommand{\successorStackInUse}{successorStackInUse}
\newcommand{\targetNode}{targetNode}




\newcommand{\key}{key}

\newcommand{\done}{done}
\newcommand{\result}{result}
\newcommand{\status}{status}
\newcommand{\restart}{restart}





\newcommand{\cKey}{key}
\newcommand{\nKey}{key}
\newcommand{\cNode}{current}
\newcommand{\pNode}{parent}
\newcommand{\nMarked}{marked}



\newcommand{\which}{which}
\newcommand{\address}{address}

\newcommand{\anchor}{anchor}

\newcommand{\stack}{stack}
\newcommand{\sTop}{top}
\newcommand{\sBottom}{bottom}
\newcommand{\current}{current}

\remove{
\newcommand{\traversalRecord}{state}
\newcommand{\TraversalRecord}{State}
\newcommand{\opRecord}{opRecord}
\newcommand{\OpRecord}{OpRecord}
\newcommand{\seekRecord}{seekRecord}
\newcommand{\SeekRecord}{SeekRecord}
}

\newcommand{\admissible}{admissible}
\newcommand{\critical}{critical}
\newcommand{\reference}{re\!f\!erence}

%% \newcommand{\OptReturn}[1][]{\Return #1\;}
\newcommand{\OptReturn}[1][]{}

\newcommand{\injectionPoint}{injectionPoint}



\newcommand{\Search}{\textsc{Search}}
\newcommand{\Insert}{\textsc{Insert}}
\newcommand{\Delete}{\textsc{Delete}}
\newcommand{\Seek}{\textsc{Seek}}

\newcommand{\Inject}{\textsc{Inject}}


%%
\newcommand{\WFSeekForSearchBOSize}{\textsc{WFSeekForSearchBasedOnSize}}
\newcommand{\WFSeekForSearchBOHeight}{\textsc{WFSeekForSearchBasedOnHeight}}
%%
\newcommand{\WFTraverseTreeBOCount}{\textsc{TraverseBasedOnCount}}
\newcommand{\WFTraverseTreeBOTimeStamp}{\textsc{TraverseBasedOnTimeStamp}}
%%
\remove{
\newcommand{\SeekForSuccessor}{\textsc{SeekForSuccessor}}
\newcommand{\NeedSuccessorKey}{\textsc{NeedSuccessorKey}}
\newcommand{\GetChild}{\textsc{GetChild}}
\newcommand{\Move}{\textsc{Move}}
\newcommand{\GetAddress}{\textsc{GetAddress}}
\newcommand{\IsNull}{\textsc{IsNull}}
\newcommand{\PopulateSeekRecord}{\textsc{PopulateSeekRecord}}
}



\newcommand{\mline}[1]{\DontPrintSemicolon #1 \PrintSemicolon}


\newcommand{\LEFT}{\textsf{LEFT}}
\newcommand{\RIGHT}{\textsf{RIGHT}}


\newcommand{\rarrow}{\!\rightarrow\!}
\newcommand{\type}{type}
\newcommand{\limit}{limit}


\newcommand{\SEARCH}{\textsf{SEARCH}}
\newcommand{\INSERT}{\textsf{INSERT}}
\newcommand{\DELETE}{\textsf{DELETE}}

\newcommand{\STOPFOUND}{\textsf{FOUND}}
\newcommand{\STOPNOTFOUND}{\textsf{NOT\_FOUND}}
\newcommand{\ADMISSIBLE}{\textsf{SAFE}}
\newcommand{\INADMISSIBLE}{\textsf{NOT\_SAFE}}

\newcommand{\TARGETSTACK}{\textsf{TARGET\_STACK}}
\newcommand{\SUCCESSORSTACK}{\textsf{SUCCESSOR\_STACK}}

%%%%%%%%%%%%%%%%%%%%%%%%%%%%%%%%%%%%%%%%%%%%%%%%%%%%%%%%%%%%%%%%%%%%%%%%%%%%%%%%%%%%

\newcommand{\DefineKeyWords}{
%%
\SetKw{Boolean}{boolean}
\SetKw{Integer}{integer}
\SetKw{LAnd}{~and~}
\SetKw{LOr}{~or~}
\SetKw{LNot}{not}
\SetKw{Struct}{struct}
\SetKw{Null}{null}
\SetKw{True}{true}
\SetKw{False}{false}
\SetKw{Break}{break}
\SetKw{Continue}{continue}
\SetKw{Enum}{enum}
\SetKw{Word}{word}
%%
}

%%%%%%%%%%%%%%%%%%%%%%%%%%%%%%%%%%%%%%%%%%%%%%%%%%%%%%%%%%%%%%%%%%%%%%%%%%%%%%%%%%%%%

%% Seek for successor key

%%%%%%%%%%%%%%%%%%%%%%%%%%%%%%%%%%%%%%%%%%%%%%%%%%%%%%%%%%%%%%%%%%%%%%%%%%%%%%%%%%%%%



\begin{algorithm}[tbh]
\caption{Seek Function for Successor Key} 
\label{algo:seek:successor}
%%
\DefineKeyWords
%%
\tcp{Looks for the next largest key with respect to a given key}
\DontPrintSemicolon
\Boolean \SeekForSuccessor( $\opRecord$, $\seekRecord$ )\;
\PrintSemicolon
\label{lin:local-seek|successor:begin}
\Begin
{
%%
   
   \tcp{the stack used in locating the successor key is initialized before this function is invoked}	
	 $\traversalRecord$ := $\opRecord \rarrow \successorStack$\;
	 \While{\True}
	 {
	
	    \label{lin:local-seek|successor:while:begin}
		  \tcp{backtrack until either an unmarked node or the stack becomes empty}
			
	    
			\While{(\Size( $\traversalRecord$ ) $>$ 1)}      
			{
			    \label{lin:local-seek|successor:while:backtrack:begin}
			    $\cNode$ := \GetTop( $\traversalRecord$ )\;
					\lIf{\LNot(\IsMarked( $\cNode$ ))}
					{
					   \Break
					} 
					\lElse
					{
					   \RemoveFromTop( $\traversalRecord$ )
					}
					\label{lin:local-seek|successor:while:backtrack:end}
			}
	
	    \BlankLine
			
			\tcp{backtrack further if a clean parent is needed but the parent is not clean}
			\If{(\Size( $\traversalRecord$ ) $\geq$ 2)}
			{
			   \label{lin:local-seek|successor:while:clean:begin}
			   \If{\NeedCleanParentNode( $\opRecord$, $\cNode$ )}
		     {
				    
			      \tcp{the parent node should be a clean node}
				    $\pNode$ := \GetSecondToTop( $\traversalRecord$ )\;
				    \If{\LNot(\IsClean( $\pNode$ ))}
				    {
				       \RemoveFromTop( $\traversalRecord$ )\;
					     \Continue\;
							 \label{lin:local-seek|successor:while:clean:end}
				    }
			   }
				
			}
			
			\BlankLine
			
	    \tcp{check if the successor key is still needed}
	    $\reference$ := \NeedSuccessorKey( $\opRecord$ )\;
			\label{lin:local-seek|successor:while:need|successor}
			\If(\tcp*[f]{successor key no longer required}){\IsNull( $\reference$ )}{ 
			   \Return \false\;
			}
					
			$\cNode$ := \GetTop( $\traversalRecord$ )\;
			\uIf{(\Size( $\traversalRecord$ ) = 1)}
			{
			   \label{lin:local-seek|successor:while:traversal:if:begin}
			   %% $\address$ := \GetAddress( $\reference$ )\;
				 \tcp{visit the node pointed to by the reference returned by \NeedSuccessorKey{} function}
			   $\which$ := \RIGHT\;
				 \label{lin:local-seek|successor:while:traversal:if:end}
			}
			\Else(\tcp*[f]{follow the left child node of the top node, if it exists})
			{
			   \label{lin:local-seek|successor:while:traversal:else:begin}
		 	   $\reference$ := \GetChild( $\cNode$, \LEFT{} )\;
			   %% \lIf{\IsNull( $\reference$ )}{ \Break }
				 %% $\address$ := \GetAddress( $\reference$ )\;
			   $\which$ := \LEFT\;
				 \label{lin:local-seek|successor:while:traversal:else:end}
			}
			\BlankLine
			\Repeat(\tcp*[f]{stop if reference is null}){\True}
	    {
			   \label{lin:local-seek|successor:while:traversal:begin}
				 \lIf{\IsNull( $\reference$ )}{ \Break }
				
				 \tcp{obtain the address of the node}
				 $\address$ := \GetAddress( $\reference$ )\;	
				
				 \tcp{traverse the edge}
				 $\restart$ := \Move( $\cNode$, $\address$, $\which$ )\;
				 \label{lin:local-seek|successor:while:traversal:move}
			   \If(\tcp*[f]{the algorithm wants to restart the traversal}){$\restart$}
			   {
				    \Break\;
						\label{lin:local-seek|successor:while:traversal:restart}
			   }  
				 
			   \tcp{push the node visited into the stack}
				 \AddToTop( $\traversalRecord$, $\address$, $\which$ )\;
				 \label{lin:local-seek|successor:while:traversal:stack}
				 \label{lin:local-seek|successor:while:traversal:advance:begin}
			   $\cNode$ := $\address$\;
				 \tcp{determine the next node to be visited}
			   $\reference$ := \GetChild( $\cNode$, \LEFT{} )\;
			   $\which$ := \LEFT{}\;
				 \label{lin:local-seek|successor:while:traversal:advance:end}
				 \label{lin:local-seek|successor:while:traversal:end}
			}
			\label{lin:local-seek|successor:while:end}
	 }
	
	 \tcp{return the outcome}
	 \PopulateSeekRecord( $\seekRecord$, $\opRecord$ )\;
	 \Return \True;
	 \label{lin:local-seek|successor:end}
%%
}
%%
\end{algorithm}
\end{limitscope}