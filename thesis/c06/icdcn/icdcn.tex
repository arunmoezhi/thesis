In this section we evaluate \ICDCN{} against three other implementations of a concurrent BST, namely those based on:
%%
\begin{enumerate}[label=(\roman*)]
\item the lock-free internal BST by Howley and Jones~\cite{HowJon:2012:SPAA}, denoted by \HJBST{},
\item the lock-free external BST by Natarajan and Mittal~\cite{NatMit:2014:PPoPP}, denoted by \NMBST{}, and 
\item the RCU-based internal BST by Arbel and Attiya~\cite{ArbAtt:2014:PODC}, denoted by \CITRUS{}.
\end{enumerate}

% Style to select only points from #1 to #2 (inclusive)
\pgfplotsset
{
	select coords between index/.style 2 args=
	{
    x filter/.code=
		{
        \ifnum\coordindex<#1\def\pgfmathresult{}\fi
        \ifnum\coordindex>#2\def\pgfmathresult{}\fi
    }
	}
}
\begin{figure}[htp]
\centering
\begin{tikzpicture}[scale=0.9, transform shape]
	\begin{groupplot}[group style={group size= 3 by 3},height=3cm,width=4cm,xmode=log,log basis x={2},max space between ticks=20,minor tick num=1,tick label style={font=\tiny},xlabel style={font=\tiny},ylabel style={font=\tiny}, title style={font=\tiny}]
		\nextgroupplot[title=Small,ylabel={Read-Dominated},xtick=data]
				%\addplot[brown, semithick,mark=square,mark size=1] 	[select coords between index={0}{3}] table[x=keyspace,y=Mops-citrus,col sep=space]{Data/snb32/keySweep.csv};  \label{plots:CITRUS:ika}
				\addplot[red,semithick,mark=triangle,mark size=1] 	[select coords between index={0}{3}] table[x=keyspace,y=Mops-howley,col sep=space]{Data/snb32/keySweep.csv};  \label{plots:HJ-BST:ika}
				\addplot[blue,semithick,mark=asterisk,mark size=1] 	[select coords between index={0}{3}] table[x=keyspace,y=Mops-ppop,col sep=space]{Data/snb32/keySweep.csv};  	\label{plots:NM-BST:ika}
				\addplot[green, semithick,mark=o,mark size=1] 			[select coords between index={0}{3}] table[x=keyspace,y=Mops-icdcn,col sep=space]{Data/snb32/keySweep.csv};  \label{plots:icdcn:ika}
				\coordinate (top) at (rel axis cs:0,1);% coordinate at top of the first plot
		\nextgroupplot[title=Medium,xtick=data]
				%\addplot[brown, semithick,mark=square,mark size=1] 	[select coords between index={4}{7}] table[x=keyspace,y=Mops-citrus,col sep=space]{Data/snb32/keySweep.csv}; 
				\addplot[red,semithick,mark=triangle,mark size=1] 	[select coords between index={4}{7}] table[x=keyspace,y=Mops-howley,col sep=space]{Data/snb32/keySweep.csv}; 
				\addplot[blue,semithick,mark=asterisk,mark size=1] 	[select coords between index={4}{7}] table[x=keyspace,y=Mops-ppop,col sep=space]{Data/snb32/keySweep.csv};   
				\addplot[green, semithick,mark=o,mark size=1] 			[select coords between index={4}{7}] table[x=keyspace,y=Mops-icdcn,col sep=space]{Data/snb32/keySweep.csv};
		\nextgroupplot[title=Large,xtick=data]
				%\addplot[brown, semithick,mark=square,mark size=1] 	[select coords between index={8}{11}] table[x=keyspace,y=Mops-citrus,col sep=space]{Data/snb32/keySweep.csv};
				\addplot[red,semithick,mark=triangle,mark size=1] 	[select coords between index={8}{11}] table[x=keyspace,y=Mops-howley,col sep=space]{Data/snb32/keySweep.csv};
				\addplot[blue,semithick,mark=asterisk,mark size=1] 	[select coords between index={8}{11}] table[x=keyspace,y=Mops-ppop,col sep=space]{Data/snb32/keySweep.csv};  
				\addplot[green, semithick,mark=o,mark size=1] 			[select coords between index={8}{11}] table[x=keyspace,y=Mops-icdcn,col sep=space]{Data/snb32/keySweep.csv};
		\nextgroupplot[ylabel={Mixed},xtick=data]
				%\addplot[brown, semithick,mark=square,mark size=1] 	[select coords between index={12}{15}] table[x=keyspace,y=Mops-citrus,col sep=space]{Data/snb32/keySweep.csv};
				\addplot[red,semithick,mark=triangle,mark size=1] 	[select coords between index={12}{15}] table[x=keyspace,y=Mops-howley,col sep=space]{Data/snb32/keySweep.csv};
				\addplot[blue,semithick,mark=asterisk,mark size=1] 	[select coords between index={12}{15}] table[x=keyspace,y=Mops-ppop,col sep=space]{Data/snb32/keySweep.csv};  
				\addplot[green, semithick,mark=o,mark size=1] 			[select coords between index={12}{15}] table[x=keyspace,y=Mops-icdcn,col sep=space]{Data/snb32/keySweep.csv};
		\nextgroupplot[xtick=data]
				%\addplot[brown, semithick,mark=square,mark size=1] 	[select coords between index={16}{19}] table[x=keyspace,y=Mops-citrus,col sep=space]{Data/snb32/keySweep.csv};
				\addplot[red,semithick,mark=triangle,mark size=1] 	[select coords between index={16}{19}] table[x=keyspace,y=Mops-howley,col sep=space]{Data/snb32/keySweep.csv};
				\addplot[blue,semithick,mark=asterisk,mark size=1] 	[select coords between index={16}{19}] table[x=keyspace,y=Mops-ppop,col sep=space]{Data/snb32/keySweep.csv};  
				\addplot[green, semithick,mark=o,mark size=1] 			[select coords between index={16}{19}] table[x=keyspace,y=Mops-icdcn,col sep=space]{Data/snb32/keySweep.csv};
		\nextgroupplot[xtick=data]
				%\addplot[brown, semithick,mark=square,mark size=1] 	[select coords between index={20}{23}] table[x=keyspace,y=Mops-citrus,col sep=space]{Data/snb32/keySweep.csv};
				\addplot[red,semithick,mark=triangle,mark size=1] 	[select coords between index={20}{23}] table[x=keyspace,y=Mops-howley,col sep=space]{Data/snb32/keySweep.csv};
				\addplot[blue,semithick,mark=asterisk,mark size=1] 	[select coords between index={20}{23}] table[x=keyspace,y=Mops-ppop,col sep=space]{Data/snb32/keySweep.csv};  
				\addplot[green, semithick,mark=o,mark size=1] 			[select coords between index={20}{23}] table[x=keyspace,y=Mops-icdcn,col sep=space]{Data/snb32/keySweep.csv};
		\nextgroupplot[xlabel={Key Space Size},ylabel={Write-Dominated},xtick=data]
				%\addplot[brown, semithick,mark=square,mark size=1] 	[select coords between index={24}{27}] table[x=keyspace,y=Mops-citrus,col sep=space]{Data/snb32/keySweep.csv}; 
				\addplot[red,semithick,mark=triangle,mark size=1] 	[select coords between index={24}{27}] table[x=keyspace,y=Mops-howley,col sep=space]{Data/snb32/keySweep.csv};
				\addplot[blue,semithick,mark=asterisk,mark size=1] 	[select coords between index={24}{27}] table[x=keyspace,y=Mops-ppop,col sep=space]{Data/snb32/keySweep.csv};  
				\addplot[green, semithick,mark=o,mark size=1] 			[select coords between index={24}{27}] table[x=keyspace,y=Mops-icdcn,col sep=space]{Data/snb32/keySweep.csv};	
		\nextgroupplot[xlabel={Key Space Size},xtick=data]
				%\addplot[brown, semithick,mark=square,mark size=1] 	[select coords between index={28}{31}] table[x=keyspace,y=Mops-citrus,col sep=space]{Data/snb32/keySweep.csv};
				\addplot[red,semithick,mark=triangle,mark size=1] 	[select coords between index={28}{31}] table[x=keyspace,y=Mops-howley,col sep=space]{Data/snb32/keySweep.csv};
				\addplot[blue,semithick,mark=asterisk,mark size=1] 	[select coords between index={28}{31}] table[x=keyspace,y=Mops-ppop,col sep=space]{Data/snb32/keySweep.csv};  
				\addplot[green, semithick,mark=o,mark size=1] 			[select coords between index={28}{31}] table[x=keyspace,y=Mops-icdcn,col sep=space]{Data/snb32/keySweep.csv};
		\nextgroupplot[xlabel={Key Space Size},xtick=data]
				%\addplot[brown, semithick,mark=square,mark size=1] 	[select coords between index={32}{35}] table[x=keyspace,y=Mops-citrus,col sep=space]{Data/snb32/keySweep.csv}; 
				\addplot[red,semithick,mark=triangle,mark size=1] 	[select coords between index={32}{35}] table[x=keyspace,y=Mops-howley,col sep=space]{Data/snb32/keySweep.csv};
				\addplot[blue,semithick,mark=asterisk,mark size=1] 	[select coords between index={32}{35}] table[x=keyspace,y=Mops-ppop,col sep=space]{Data/snb32/keySweep.csv};  
				\addplot[green, semithick,mark=o,mark size=1] 			[select coords between index={32}{35}] table[x=keyspace,y=Mops-icdcn,col sep=space]{Data/snb32/keySweep.csv};						
				\coordinate (bot) at (rel axis cs:1,0);% coordinate at bottom of the last plot
	\end{groupplot}
	\path (top-|current bounding box.west)-- node[anchor=south,rotate=90] {\tiny System throughput (million operations/second)} (bot-|current bounding box.west);
	\path (top|-current bounding box.north)-- coordinate(legendpos) (bot|-current bounding box.north);
	\matrix[matrix of nodes, anchor=south, draw, inner sep=0.2em, draw] at ([yshift=1ex]legendpos)
  {
    \ref{plots:HJ-BST:ika}& \tiny \HJBST{} & [5pt]
    \ref{plots:NM-BST:ika}& \tiny \NMBST{} & [5pt]
    %\ref{plots:CITRUS:ika}& \tiny \CITRUS{} & [5pt]
		\ref{plots:icdcn:ika}& \tiny \ICDCN{} \\
	};
\end{tikzpicture}
%\caption[\ICDCN{} - Comparison of throughput of different algorithms - key sweep]{Comparison of system throughput of different algorithms at 32 threads. Each row represents a workload type. Each column represents a range of key space range. Higher the ratio, better the performance of the algorithm.}
\label{fig:icdcn-keySweep-absolute}
\end{figure}
% Style to select only points from #1 to #2 (inclusive)
\pgfplotsset
{
	select coords between index/.style 2 args=
	{
    x filter/.code=
		{
        \ifnum\coordindex<#1\def\pgfmathresult{}\fi
        \ifnum\coordindex>#2\def\pgfmathresult{}\fi
    }
	}
}
\begin{figure}
\centering
\nextwithlateexternal% < added
\begin{tikzpicture}
	\begin{groupplot}[group style={group size= 3 by 3},height=5.5cm,width=5.5cm,max space between ticks=20,minor tick num=1,tick label style={font=\footnotesize}]
		\nextgroupplot[title=20K keys,ylabel={Read-Dominated},xtick=data]
				%\addplot[brown, semithick,mark=square] 	[select coords between index={6}{11}] table[x=threads,y=Mops-citrus,col sep=space]{Data/snb32/threadSweep.csv};  \label{plots:CITRUS:it}
				\addplot[red,semithick,mark=triangle] 	[select coords between index={6}{11}] table[x=threads,y=Mops-howley,col sep=space]{Data/snb32/threadSweep.csv};  \label{plots:HJ-BST:it}
				\addplot[blue,semithick,mark=asterisk] 	[select coords between index={6}{11}] table[x=threads,y=Mops-ppop,col sep=space]{Data/snb32/threadSweep.csv};  \label{plots:NM-BST:it}
				\addplot[green, semithick,mark=o] 			[select coords between index={6}{11}] table[x=threads,y=Mops-icdcn,col sep=space]{Data/snb32/threadSweep.csv};  \label{plots:icdcn:it}
				\coordinate (top) at (rel axis cs:0,1);% coordinate at top of the first plot
		\nextgroupplot[title=200K keys,xtick=data]
				%\addplot[brown, semithick,mark=square] 	[select coords between index={18}{23}] table[x=threads,y=Mops-citrus,col sep=space]{Data/snb32/threadSweep.csv};  
				\addplot[red,semithick,mark=triangle] 	[select coords between index={18}{23}] table[x=threads,y=Mops-howley,col sep=space]{Data/snb32/threadSweep.csv};  
				\addplot[blue,semithick,mark=asterisk] 	[select coords between index={18}{23}] table[x=threads,y=Mops-ppop,col sep=space]{Data/snb32/threadSweep.csv};  
				\addplot[green, semithick,mark=o] 			[select coords between index={18}{23}] table[x=threads,y=Mops-icdcn,col sep=space]{Data/snb32/threadSweep.csv}; 
		\nextgroupplot[title=2M keys,xtick=data]
				%\addplot[brown, semithick,mark=square] 	[select coords between index={30}{35}] table[x=threads,y=Mops-citrus,col sep=space]{Data/snb32/threadSweep.csv};
				\addplot[red,semithick,mark=triangle] 	[select coords between index={30}{35}] table[x=threads,y=Mops-howley,col sep=space]{Data/snb32/threadSweep.csv};
				\addplot[blue,semithick,mark=asterisk] 	[select coords between index={30}{35}] table[x=threads,y=Mops-ppop,col sep=space]{Data/snb32/threadSweep.csv};
				\addplot[green, semithick,mark=o] 			[select coords between index={30}{35}] table[x=threads,y=Mops-icdcn,col sep=space]{Data/snb32/threadSweep.csv};
		\nextgroupplot[ylabel={Mixed},xtick=data]
				%\addplot[brown, semithick,mark=square] 	[select coords between index={42}{47}] table[x=threads,y=Mops-citrus,col sep=space]{Data/snb32/threadSweep.csv}; 
				\addplot[red,semithick,mark=triangle] 	[select coords between index={42}{47}] table[x=threads,y=Mops-howley,col sep=space]{Data/snb32/threadSweep.csv};
				\addplot[blue,semithick,mark=asterisk] 	[select coords between index={42}{47}] table[x=threads,y=Mops-ppop,col sep=space]{Data/snb32/threadSweep.csv};
				\addplot[green, semithick,mark=o] 			[select coords between index={42}{47}] table[x=threads,y=Mops-icdcn,col sep=space]{Data/snb32/threadSweep.csv};
		\nextgroupplot[xtick=data]
				%\addplot[brown, semithick,mark=square] 	[select coords between index={54}{59}] table[x=threads,y=Mops-citrus,col sep=space]{Data/snb32/threadSweep.csv}; 
				\addplot[red,semithick,mark=triangle] 	[select coords between index={54}{59}] table[x=threads,y=Mops-howley,col sep=space]{Data/snb32/threadSweep.csv};
				\addplot[blue,semithick,mark=asterisk] 	[select coords between index={54}{59}] table[x=threads,y=Mops-ppop,col sep=space]{Data/snb32/threadSweep.csv};
				\addplot[green, semithick,mark=o] 			[select coords between index={54}{59}] table[x=threads,y=Mops-icdcn,col sep=space]{Data/snb32/threadSweep.csv};
		\nextgroupplot[xtick=data]
				%\addplot[brown, semithick,mark=square] 	[select coords between index={66}{71}] table[x=threads,y=Mops-citrus,col sep=space]{Data/snb32/threadSweep.csv}; 
				\addplot[red,semithick,mark=triangle] 	[select coords between index={66}{71}] table[x=threads,y=Mops-howley,col sep=space]{Data/snb32/threadSweep.csv};
				\addplot[blue,semithick,mark=asterisk] 	[select coords between index={66}{71}] table[x=threads,y=Mops-ppop,col sep=space]{Data/snb32/threadSweep.csv};
				\addplot[green, semithick,mark=o] 			[select coords between index={66}{71}] table[x=threads,y=Mops-icdcn,col sep=space]{Data/snb32/threadSweep.csv};
		\nextgroupplot[xlabel={Number of Threads},ylabel={Write-Dominated},xtick=data]
				%\addplot[brown, semithick,mark=square] 	[select coords between index={78}{83}] table[x=threads,y=Mops-citrus,col sep=space]{Data/snb32/threadSweep.csv}; 
				\addplot[red,semithick,mark=triangle] 	[select coords between index={78}{83}] table[x=threads,y=Mops-howley,col sep=space]{Data/snb32/threadSweep.csv};
				\addplot[blue,semithick,mark=asterisk] 	[select coords between index={78}{83}] table[x=threads,y=Mops-ppop,col sep=space]{Data/snb32/threadSweep.csv};
				\addplot[green, semithick,mark=o] 			[select coords between index={78}{83}] table[x=threads,y=Mops-icdcn,col sep=space]{Data/snb32/threadSweep.csv};	
		\nextgroupplot[xlabel={Number of Threads},xtick=data]
				%\addplot[brown, semithick,mark=square] 	[select coords between index={90}{95}] table[x=threads,y=Mops-citrus,col sep=space]{Data/snb32/threadSweep.csv}; 
				\addplot[red,semithick,mark=triangle] 	[select coords between index={90}{95}] table[x=threads,y=Mops-howley,col sep=space]{Data/snb32/threadSweep.csv};
				\addplot[blue,semithick,mark=asterisk] 	[select coords between index={90}{95}] table[x=threads,y=Mops-ppop,col sep=space]{Data/snb32/threadSweep.csv};
				\addplot[green, semithick,mark=o] 			[select coords between index={90}{95}] table[x=threads,y=Mops-icdcn,col sep=space]{Data/snb32/threadSweep.csv};	
		\nextgroupplot[xlabel={Number of Threads},xtick=data]
				%\addplot[brown, semithick,mark=square] 	[select coords between index={102}{107}] table[x=threads,y=Mops-citrus,col sep=space]{Data/snb32/threadSweep.csv}; 
				\addplot[red,semithick,mark=triangle] 	[select coords between index={102}{107}] table[x=threads,y=Mops-howley,col sep=space]{Data/snb32/threadSweep.csv};
				\addplot[blue,semithick,mark=asterisk] 	[select coords between index={102}{107}] table[x=threads,y=Mops-ppop,col sep=space]{Data/snb32/threadSweep.csv};
				\addplot[green, semithick,mark=o] 			[select coords between index={102}{107}] table[x=threads,y=Mops-icdcn,col sep=space]{Data/snb32/threadSweep.csv};						
				\coordinate (bot) at (rel axis cs:1,0);% coordinate at bottom of the last plot
\end{groupplot}
	\path (top-|current bounding box.west)-- node[anchor=south,rotate=90] {\small System throughput (million operations/second)} (bot-|current bounding box.west);
	%\node[right,rotate=90] at (-1.4,-7.2){\small ~Throughput (million ops/sec)};
	\path (top|-current bounding box.north)-- coordinate(legendpos) (bot|-current bounding box.north);
	\matrix[matrix of nodes, anchor=south, draw, inner sep=0.2em, draw] at ([xshift=-5ex,yshift=1ex]legendpos)
  {
    \ref{plots:HJ-BST:it}& \HJBST{} & [5pt]
    \ref{plots:NM-BST:it}& \NMBST{} & [5pt]
    %\ref{plots:CITRUS:it}& \CITRUS{} & [5pt]
		\ref{plots:icdcn:it}& \ICDCN{} \\
	};
\end{tikzpicture}
\caption[\ICDCN{} - Comparison of throughput of different algorithms - thread sweep]{Comparison of system throughput of different algorithms. Each row represents a workload type and each column represents a key space size. Higher the throughput, better the performance of the algorithm.}
\label{fig:icdcn-threadSweep}
\end{figure}

The results of our experiments are shown in \figref{icdcn-keySweep-absolute} and \figref{icdcn-threadSweep}. In \figref{icdcn-keySweep-absolute}, each row represents a specific workload (read-dominated, mixed or write-dominated) and each column represents a specific key space size; \textit{small} (8Ki to 64Ki), \textit{medium} (128Ki to 1Mi) and \textit{large} (2Mi to 16Mi). \figref{icdcn-threadSweep} shows the scaling with respect to the number of threads for key space size of 2\textsuperscript{19} (512Ki). We do not show the numbers for \CITRUS{} in the graphs as it had the worst performance among all implementations (slower by a factor of four in some cases). This is not surprising as \CITRUS{} is optimized for read operations (\emph{e.g.}, 98\% reads \& 2\% updates)~\cite{ArbAtt:2014:PODC}.


As the graphs show, \ICDCN{} achieved nearly same or higher throughput than the other two implementations for medium and large key space sizes (except for medium key space size with write-dominated workload). Specifically, at 32 threads and for a read-dominated workload, \ICDCN{} had \icdcnMaximumgap{} and 15\% higher throughput than the next best performer for key space sizes of 512Ki and 1Mi, respectively. Also, at 32 threads and for a mixed workload, \ICDCN{} had 22\% and 15\% higher throughput than the next best performer for key space sizes of 1Mi and 2Mi, respectively. Overall, \ICDCN{} outperformed the next best implementation by as much as \icdcnMaximumgap{}; it outperformed \HJBST{} by as much as 38\% (mixed) and \NMBST{} by as much as 30\%(read-dominated). For large key space sizes, the overhead of traversing the tree appears to dominate the overhead of actually modifying the operation's window, and the gap between various implementations becomes smaller.

\begin{table}[htp]
\centering
\caption{Comparison of different concurrent algorithms in the absence of contention.}
\label{tab:castle-comparison}
\renewcommand{\arraystretch}{1.25}
\small
\begin{tabular}{|c|c|c|c|c|}
\hline
\multirow{2}{*}{\textbf{Algorithm}} & 
\multicolumn{2}{c|}{\textbf{\begin{tabular}[c]{@{}c@{}}Number of  Objects \\  Allocated \end{tabular}}} & 
\multicolumn{2}{c|}{\textbf{\begin{tabular}[c]{@{}c@{}}Number of Synchronization \\ Primitives Executed \end{tabular}}} \\ \cline{2-5} 
%%
 & \textbf{Insert} & \textbf{Delete} & \textbf{Insert} & \textbf{Delete} \\ \hline
%%
\multirow{2}{*}{\HJBST} & \multirow{2}{*}{2} & \multirow{2}{*}{1} & \multirow{2}{*}{3} & simple: 4 \\ 
 &  &  &  & complex: 9 \\ \hline
%%
\NMBST{} & 2 & 0 & 1 & 3 \\ \hline
%%
\multirow{2}{*}{\CASTLE} & \multirow{2}{*}{1} & \multirow{2}{*}{0} & \multirow{2}{*}{1} & simple: 3 \\ 
 &  &  &  & complex: 4 \\ \hline
\end{tabular}
\end{table}
There are several reasons why \ICDCN{} outperformed the other two implementations in many cases. First, as \tabref{icdcn-comparison} shows, our algorithm allocates fewer objects than the two other algorithms on average considering the fact that the fraction of insert operations will generally be larger than the fraction of delete operations in any realistic workload. Further, we observed in our experiments that the number of simple delete operations outnumbered the number of complex delete operations by two to one, and our algorithm does not allocate any object for a simple delete operation. Second, again as \tabref{icdcn-comparison} shows, our algorithm executes the same number of atomic instructions as in~\cite{NatMit:2014:PPoPP} for insert operations;  and, in all the cases,  executes same or fewer atomic instructions than in~\cite{HowJon:2012:SPAA}. This is important since an atomic instruction is more expensive to execute than a simple read or write instruction. Third, we observed in our experiments that \ICDCN{} had a smaller memory footprint than the other two implementations (by almost a factor of two) since it uses internal representation and allocates fewer objects. As a result, it was likely able to benefit from caching to a larger degree than \HJBST{} and \NMBST{}.