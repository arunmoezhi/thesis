\begin{limitscope}

%% To limit the scope of the macros defined below

%% macros for pseudocode
\newcommand{\leftChild}{le\!f\!t}
\newcommand{\rightChild}{right}
\newcommand{\child}{child}
\newcommand{\canReplace}{readyToReplace}
\newcommand{\markAndKey}{mKey}

\newcommand{\node}{node}
\newcommand{\parent}{parent}

\newcommand{\terminalEdge}{lastEdge}
\newcommand{\targetEdge}{targetEdge}
\newcommand{\parentTargetEdge}{pTargetEdge}
\newcommand{\successorEdge}{successorEdge}
\newcommand{\parentSuccessorEdge}{pSuccessorEdge}
\newcommand{\injectionEdge}{injectionEdge}
\newcommand{\penultimateEdge}{pLastEdge}

\newcommand{\targetKey}{targetKey}
\newcommand{\currentKey}{currentKey}

\newcommand{\newNode}{newNode}
\newcommand{\reference}{re\!f\!erence}
\newcommand{\state}{state}

\newcommand{\StateRecord}{StateRecord}
\newcommand{\AnchorRecord}{AnchorRecord}

\newcommand{\mline}[1]{\DontPrintSemicolon #1 \PrintSemicolon}

\newcommand{\prev}{prev}
\newcommand{\curr}{curr}

\newcommand{\prevSeekRecord}{pSeekRecord}
\newcommand{\prevAnchorRecord}{pAnchorRecord}
%% \newcommand{\currSeekRecord}{cSeekRecord}
\newcommand{\anchorRecord}{anchorRecord}

\newcommand{\oldContents}{oldValue}
\newcommand{\newContents}{newValue}

\newcommand{\INJECTION}{\textsf{INJECTION}}
\newcommand{\DISCOVERY}{\textsf{DISCOVERY}}
\newcommand{\CLEANUP}{\textsf{CLEANUP}}
\newcommand{\FINISHED}{\textsf{FINISHED}}

\newcommand{\DELETEFLAG}{\textsf{DELETE\_FLAG}}
\newcommand{\PROMOTEFLAG}{\textsf{PROMOTE\_FLAG}}
\newcommand{\INTENTFLAG}{\textsf{INTENT\_FLAG}}
\newcommand{\flag}{f\!lag}

\newcommand{\COMPLEX}{\textsf{COMPLEX}}
\newcommand{\SIMPLE}{\textsf{SIMPLE}}

\newcommand{\LEFT}{\textsf{LEFT}}
\newcommand{\RIGHT}{\textsf{RIGHT}}

\newcommand{\targetSeekRecord}{targetRecord}
\newcommand{\successorSeekRecord}{successorRecord}

\newcommand{\dFlag}{d}
\newcommand{\iFlag}{i}
\newcommand{\pFlag}{p}
\newcommand{\nFlag}{n}
\newcommand{\mFlag}{m}
\newcommand{\lNFlag}{lN}
\newcommand{\rNFlag}{rN}

\newcommand{\rarrow}{\!\rightarrow\!}


%%%%%%%%%%%%%%%%%%%%%%%%%%%%%%%%%%%%%%%%%%%%%%%%%%%%%%%%%%%%%%%%%%%%%%%%%%%%%%%%%%%%

\newcommand{\DefineKeyWords}{
%%
\SetKw{Boolean}{Boolean}
\SetKw{LAnd}{~and~}
\SetKw{LOr}{~or~}
\SetKw{LNot}{not}
\SetKw{Struct}{struct}
\SetKw{Null}{null}
\SetKw{True}{true}
\SetKw{False}{false}
\SetKw{Break}{break}
\SetKw{Continue}{continue}
\SetKw{Enum}{enum}
%%
}

%%%%%%%%%%%%%%%%%%%%%%%%%%%%%%%%%%%%%%%%%%%%%%%%%%%%%%%%%%%%%%%%%%%%%%%%%%%%%%%%%%%%%

%% DATA STRUCTURES


\begin{algorithm}[!thb]
%%
\DefineKeyWords
%%

%% define data structures used in the algorithm

\DontPrintSemicolon
\Struct Node \{\;
\label{ln:icdcn-node|begin}
\PrintSemicolon
\Indp 
   $\{ \Boolean, \text{Key} \}$ $\markAndKey$\;
   $\{ \Boolean, \Boolean, \Boolean, \Boolean, \text{NodePtr} \}$ $\child[2]$\;
   \Boolean $\canReplace$\;
\Indm
\}\;
\label{ln:node|end}
%%
\BlankLine

\DontPrintSemicolon
\Struct Edge \{\;
\label{ln:edge|begin}
\PrintSemicolon
\Indp 
   %% NodePtr $\parent$\;
   %% NodePtr $\child$\;
	 NodePtr $\parent$, $\child$\;
   \Enum $which$ \{ \LEFT{}, \RIGHT{} \}\;
\Indm
\}\;
\label{ln:edge|end}
%%
\BlankLine

\DontPrintSemicolon
\Struct SeekRecord \{\;
\PrintSemicolon
\Indp 
%%
   %% Edge $\terminalEdge$\;
%%
   %% Edge $\penultimateEdge$\;
%%
   %% Edge $\injectionEdge$\;
   Edge $\terminalEdge$, $\penultimateEdge$, $\injectionEdge$\;
\Indm
\}\;
%%
\BlankLine


\BlankLine
\DontPrintSemicolon
\Struct \AnchorRecord{} \{\;
\PrintSemicolon
\Indp 
   NodePtr $\node$\;
   Key $key$\;
\Indm
\}\;
%%

\BlankLine
\DontPrintSemicolon
\Struct \StateRecord{} \{\;
\PrintSemicolon
\Indp 
%%
   %% int $depth$\;
   %% Edge $\targetEdge$\;
	 %% Edge $\parentTargetEdge$\;
	 Edge $\targetEdge$, $\parentTargetEdge$\;
%%
   %% Key $\targetKey$\;
	 %% Key $\currentKey$\;
	 Key $\targetKey$, $\currentKey$\;
   \Enum $mode$ \{ \INJECTION{}, \DISCOVERY{}, \CLEANUP{} \}\;
   \Enum $type$ \{ \SIMPLE{}, \COMPLEX{} \} \;
%%
   \tcp{the next field stores pointer to a seek record; it is used for finding the successor if the delete operation is complex}
   SeekRecordPtr $\successorSeekRecord$\; 
\Indm
\}\;
%%
\BlankLine
\tcp{object to store information about the tree traversal when looking for a given key (used by the seek function)}
SeekRecordPtr $\targetSeekRecord$ := new seek record\;
\tcp{object to store information about process' own delete operation}
\StateRecord{Ptr} $myState$ := new state\;


\caption{Data Structures Used}
\label{algo:icdcn-data|structures}
\end{algorithm}



\begin{algorithm}[!thb]
%%
\DefineKeyWords


%% SEEK


%%
%% traverses the tree from the root node to a leaf node looking for a given key
%%
\DontPrintSemicolon
\Seek( $key$, $seekRecord$ )\;
\PrintSemicolon
\Begin
{
   $\prevAnchorRecord$ := $\curly{ \snodetwo{}, \skey{1} }$\;
   \While{\True}
   {
	    \tcp{initialize all variables used in traversal}
		  $\penultimateEdge$ := $\curly{ \snodeone, \snodetwo, \RIGHT }$; \qquad
			$\terminalEdge$ := $\curly{ \snodetwo, \snodethree, \RIGHT }$\;
			$\curr$ := $\snodethree$; \qquad
			$\anchorRecord$ := $\curly{ \snodetwo{}, \skey{1} }$\;
			\BlankLine
			\While{\True}
			{
			    \tcp{read the key stored in the current node}
			    $\ang{ \ast, cKey }$ := $\curr \rarrow \markAndKey$\;
				  \tcp{find the next edge to follow}
					$which$ := $key < cKey$ ? \LEFT : \RIGHT\;
				  $\ang{ \nFlag, \ast, \dFlag, \pFlag, next }$ := $\curr \rarrow \child[which]$\;
					\tcp{check for the completion of the traversal}
				  \If{$key = cKey$ \LOr $\nFlag$}
				  {
				     \tcp{either key found or no next edge to follow; stop the traversal}
						 $seekRecord \rarrow \penultimateEdge$ := $\penultimateEdge$\;
						 $seekRecord \rarrow \terminalEdge$ := $\terminalEdge$\;
						 $seekRecord \rarrow \injectionEdge$ := $\curly{ \curr, next, which }$\;
						 \BlankLine
						 \uIf(\tcp*[h]{keys match}){$key = cKey$}
						 { 
						    \Return\;
						 }
						 \lElse { \Break }
				  }
				  \BlankLine			   
				  \If{$which$ = \RIGHT}
				  {
				     \tcp{next edge to be traversed is a right edge; keep track of the current node and its key}
						 $\anchorRecord$ := $\ang{ \curr, cKey }$\;
				  }	
				  \BlankLine
				  \tcp{traverse the next edge}
					$\penultimateEdge$ := $\terminalEdge$; \qquad
					$\terminalEdge$ := $\curly{ \curr, next, which }$; \qquad
				  $\curr$ := $next$\; 
		  }
		  \tcp{key was not found; check if can stop}
		  $\ang{ \ast, \ast, \dFlag, \pFlag, \ast }$ := $\anchorRecord.\node \rarrow \child[\RIGHT]$\;			
			\uIf{\LNot($\dFlag$) \LAnd \LNot($\pFlag$)}
			{
			   \tcp{anchor node is still part of the tree; check if anchor node's key has changed}
				 $\ang{ \ast, aKey }$ := $\anchorRecord.\node \rarrow \markAndKey$\;
				 \lIf{$\anchorRecord.key$ = $aKey$}
			   {  
				    \Return
				 } 
			}	
			\Else
			{ 
			   \tcp{check if the anchor record (the node and its key) matches that of the previous traversal}
			   \If{$\prevAnchorRecord = \anchorRecord$}
			   {
				    \tcp{return the results of the previous traversal}
					  $seekRecord$ := $\prevSeekRecord$\;
				    \Return\;
		     }
			}
			\tcp{store the results of the traversal and restart}
			$\prevSeekRecord$ := $seekRecord$; \qquad
			$\prevAnchorRecord$ := $\anchorRecord$;					
   }
} 
%% End of seek function
\caption{Seek Function}
\label{algo:icdcn-seek}
\end{algorithm}


%% SEARCH
\begin{algorithm}[!thb]
%%
\DefineKeyWords
\DontPrintSemicolon
\Boolean \Search( $key$ )\;
\PrintSemicolon
\Begin
{
   \Seek( $key$, $mySeekRecord$ )\;
	 \BlankLine
	 %% $\node$ := $mySeekRecord \rarrow \node$\;
   $\node$ := $mySeekRecord \rarrow \terminalEdge.\child$\;
   $\ang{ \ast , nKey }$ := $\node \rarrow \markAndKey$\;
	 \BlankLine
   \lIf{nKey = key}{\Return \True}
   \lElse{\Return \False}
}
\caption{Search Operation}
\label{algo:icdcn-search}
\end{algorithm}

%% INSERT
\begin{algorithm}[tp]
%%
\DefineKeyWords
\DontPrintSemicolon
\Boolean \Insert( $key$ )\;
\PrintSemicolon
\Begin
{
   \While{\True}
	 {
      \Seek( $key$, $\targetSeekRecord$ )\;
			\BlankLine
			$\targetEdge$ := $\targetSeekRecord \rarrow \terminalEdge$\;
			$\node$ := $\targetEdge.\child$\;
			$\ang{ \ast , nKey }$ := $\node \rarrow \markAndKey$\; 
			\lIf{$key = nKey$}{\Return \False}
			\BlankLine
			\tcp{create a new node and initialize its fields}
			$\newNode$ := create a new node\;
			$\newNode \rarrow \markAndKey$ := $\ang{ 0_m, key }$\;
			$\newNode \rarrow \child[\LEFT]$ := $\ang{ 1_n, 0_i, 0_d, 0_p, \Null }$\;
			$\newNode \rarrow \child[\RIGHT]$ := $\ang{ 1_n, 0_i, 0_d, 0_p, \Null }$\;
			$\newNode \rarrow \canReplace$ := \False\;
			\BlankLine
			$which$ := $\targetSeekRecord \rarrow \injectionEdge.which$\;
			$address$ := $\targetSeekRecord \rarrow \injectionEdge.\child$\;
			$result$ := \CAS($\node \rarrow \child[which]$, $\ang{ 1_n, 0_i, 0_d, 0_p, address }$, $\ang{ 0_n, 0_i, 0_d, 0_p, \newNode }$)\;
			\lIf{$result$}{\Return \True}
			\BlankLine	
			\tcp{help if needed}
		  $\ang{ \ast, \ast, \dFlag, \pFlag, \ast }$ := $\node \rarrow \child[which]$\;
			\lIf{$\dFlag$}
			{
			   \HelpTargetNode( $\targetEdge$ )
			} 
			\lElseIf{$p$}
			{  
			   \HelpSuccessorNode( $\targetEdge$ )
			}
	}
}
\caption{Insert Operation}
\label{algo:icdcn-insert}
\end{algorithm}

%% DELETE
\begin{algorithm}[!thb]
\DefineKeyWords
\DontPrintSemicolon
\Boolean \Delete( $key$ )\;
\PrintSemicolon
\Begin
{
   \tcp{initialize the state record}
 	 $myState \rarrow \targetKey$ := $key$; $\qquad$
	 $myState \rarrow \currentKey$ := $key$\;
	 $myState \rarrow mode$ := \INJECTION\;
	 \BlankLine
   \While{\True}
	 {
      \Seek( $myState \rarrow \currentKey$, $\targetSeekRecord$ )\;
			$\targetEdge$ := $\targetSeekRecord \rarrow \terminalEdge$; $\qquad$
			$\parentTargetEdge$ := $\targetSeekRecord \rarrow \penultimateEdge$\;
			$\ang{ \ast , nKey }$ := $\targetEdge.\child \rarrow \markAndKey$\; 
			\BlankLine
			\If{$myState \rarrow \currentKey \neq nKey$}
			{
			   \tcp{the key does not exist in the tree}
			   \lIf{$myState \rarrow mode$ = \INJECTION}{\Return \False}
				 \lElse{\Return \True}
			}
 	    \BlankLine		   	
			\tcp{perform appropriate action depending on the mode}
	    \If{$myState \rarrow mode$ = \INJECTION}
		  {
				 \tcp{store a reference to the target edge}
   	     $myState \rarrow \targetEdge$ := $\targetEdge$\;
   	     $myState \rarrow \parentTargetEdge$ := $\parentTargetEdge$\;
				 \tcp{attempt to inject the operation at the node}
				 %% $result$ := \Inject( $myState$ )\;
				 \Inject( $myState$ )\;					 								 
			}
			\BlankLine
			\tcp{mode would have changed if injection was successful}
				 
			\If{$myState \rarrow mode \neq$ \INJECTION}
			{
				 \tcp{check if the target node found by the seek function matches the one stored in the state record}			
			   %%\If{$\left(\text{\parbox[c]{1.75in}{$myState \rarrow \targetEdge.\child$ $\neq$  \\ \mbox{}\hfill$\targetEdge.\child$}}\right)$}
				 \lIf{$myState \rarrow \targetEdge.\child$ $\neq$ $\targetEdge.\child$}
				 {
				    \Return \True
				 }						
				 \tcp{update the target edge information using the most recent seek}
				 $myState \rarrow \targetEdge$ := $\targetEdge$\; 			 				
		  }				
			\BlankLine							
			\If{$myState \rarrow mode$ = \DISCOVERY}
			{
				 \tcp{complex delete operation; locate the successor node and mark its child edges with promote flag} 
			   \FindAndMarkSuccessor( $myState$ )\;			 
			}			
			\If{$myState \rarrow mode$ = \DISCOVERY}
			{
				 \tcp{complex delete operation; promote the successor node's key and remove the successor node}
		     \RemoveSuccessor( $myState$ )\;						
			}			
			\BlankLine				
			\If{$myState \rarrow mode$ = \CLEANUP}
			{
			   \tcp{either remove the target node (simple delete) or replace it with a new node with all fields unmarked  (complex delete)}
			   $result$ := \Cleanup( $myState$ )\;
				 \lIf{$result$}{\Return \True}
				 \Else{
				    $\ang{ \ast, nKey }$ := $\targetEdge.\child \rarrow \markAndKey$\;
						$myState \rarrow \currentKey$ := $nKey$\;
				 }					
		  }
	 }	
   %% \Return\;
}
\caption{Delete Operation}
\label{algo:icdcn-delete}
\end{algorithm}

%% INJECT
\begin{algorithm}[!thb]
%%
\DefineKeyWords
\DontPrintSemicolon
\Inject( $\state$ )\;
\PrintSemicolon
\Begin
{
   $\targetEdge$ := $\state \rarrow \targetEdge$\;
	 \tcp{try to set the intent flag on the target edge}
	 \tcp{retrieve attributes of the target edge}
	 $\parent$ := $\targetEdge.\parent$\;
	 $\node$ := $\targetEdge.\child$\;
	 $which$ := $\targetEdge.which$\;
	 \BlankLine
	 \mline{$result$ := \CAS( \parbox[t]{2.075in}{$\parent \rarrow \child[which]$, \\ $\ang{ 0_n, 0_i, 0_d, 0_p, \node }$,  $\ang{ 0_n, 1_i, 0_d, 0_p, \node }$ );}\;}
	 \If{\LNot($result$)}
	 {
	    \tcp{unable to set the intent flag; help if needed}
			$\ang{ \ast, \iFlag, \dFlag, \pFlag, address }$ := $\parent \rarrow \child[which]$\;
			\lIf{$\iFlag$}
			{
			   \HelpTargetNode( $\targetEdge$ )
			} 
			\uElseIf{$\dFlag$}
			{
			   \HelpTargetNode( $\state \rarrow \parentTargetEdge$ )\;
			} 
			\ElseIf{$\pFlag$}
			{
			   \HelpSuccessorNode( $\state \rarrow \parentTargetEdge$ )\;
			}

      \Return;					
	 }

   \BlankLine
	 \tcp{mark the left edge for deletion}

	 $result$ := \MarkChildEdge( $\state$, \LEFT{} )\;
	 
	 \lIf{\LNot($result$)}
	 {
	    \Return
	 } 
	 \tcp{mark the right edge for deletion; cannot fail}
	 \MarkChildEdge( $\state$, \RIGHT{} )\;
	   
	 \BlankLine
	 \tcp{initialize the type and mode of the operation}
	 \InitializeTypeAndUpdateMode( $\state$ );	
}

\caption{Injecting a Deletion Operation}
\label{algo:inject}
\end{algorithm}









%% FINDANDMARKSUCCESSOR


\begin{algorithm}[!thb]
%%
\DefineKeyWords

\DontPrintSemicolon
\FindAndMarkSuccessor( $\state$ )\;
\PrintSemicolon
\Begin
{
   \tcp{retrieve the addresses from the state record}
   $\node := \state \rarrow \targetEdge.\child$\;
	 $seekRecord$ := $\state \rarrow \successorSeekRecord$\; 
   
	 \BlankLine
   \While{\True}
	 {
	 
	    \tcp{read the mark flag of the key in the target node}  
	    $\ang{ \mFlag, \ast}$ := $\node \rarrow \markAndKey$\; 
	    

	  	\tcp{find the node with the smallest key in the right subtree}
	    $result$ := \FindSmallest( $\state$ )\;
			
						
			\BlankLine
			\If{$\mFlag$ \LOr \LNot($result$)} 
			{
			   \tcp{successor node had already been selected \emph{before} the traversal or the right subtree is empty}
				 \Break\;
			}
			
				
			\tcp{retrieve the information from the seek record}
			$\successorEdge$ := $seekRecord \rarrow \terminalEdge$\;
			%% $\ang{ \nFlag, \ast, \ast, \ast, \leftChild}$ :=  $\successorEdge.\child \rarrow \child[\LEFT]$\;
			%% \lIf{\LNot($\nFlag$)}{ \Continue }
			$\leftChild$ := $seekRecord \rarrow \injectionEdge.\child$\;
			
			\BlankLine
			\tcp{read the mark flag of the key under deletion}
      $\ang{ \mFlag, \ast}$ := $\node \rarrow \markAndKey$\;
			
			\If(\tcp*[h]{successor node has already been selected}){$\mFlag$}
			{
			   %% \tcp{successor node has already been selected}
			   %% \lIf{$p$}{ \Break }
				 %% \lElse{ \Continue }
				 \Continue\;
				 
			}
			
			
		


          
			\tcp{try to set the promote flag on the left edge}
			\mline{$result$ := \CAS( \parbox[t]{1.875in}{$\successorEdge.\child \rarrow \child[\LEFT]$, \\ 
			                                             $\ang{ 1_n, 0_i, 0_d, 0_p, \leftChild }$, \\ $\ang{ 1_n, 0_i, 0_d, 1_p, \node }$ );}\;}
			
			\lIf{$result$}{\Break}
			
			\BlankLine
			\tcp{attempt to mark the edge failed; recover from the failure and retry if needed}
			%% $\ang{ n, \ast, d, p, \ast }$ := $\successorEdge.\child \rarrow \child[\LEFT]$\;
			$\ang{ \nFlag, \ast, \dFlag, \ast, \ast }$ := $\successorEdge.\child \rarrow \child[\LEFT]$\;
			
			
      %% \lIf{$p$}
      %% {
      %%    \Break
      %% }   

      %% \If{\LNot($n$)}
			%% { 
			%%   \tcp{the node found has since gained a left child}
			%%   \Continue\;
			%% }

			\If{$\nFlag$ \LAnd $\dFlag$}
			{
			    \tcp{the node found is undergoing deletion; need to help}
					
								
					%% \mline{\HelpTargetNode( \parbox[t]{1.5in}{$\successorEdge$, \\ $\state \rarrow depth + 1$ );}\;}
		      \HelpTargetNode( $\successorEdge$ )\;
       } 
	 }	
   \BlankLine
   \tcp{update the operation mode}
	 \UpdateMode( $\state$ );
}

\caption{Locating the Successor Node}
\label{algo:findandmark}
\end{algorithm}




%% REMOVESUCCESSOR
\begin{algorithm}[!thb]
%%
\DefineKeyWords
\DontPrintSemicolon
\RemoveSuccessor( $\state$ )\;
\PrintSemicolon
\Begin
{
   \tcp{retrieve addresses from the state record}
   $\node$ := $\state \rarrow \targetEdge.\child$\;
   $seekRecord$ := $\state \rarrow \successorSeekRecord$\;
   \tcp{extract information about the successor node}
	 %% \tcp{assumes that the state's seek record contains valid information}
   $\successorEdge$ := $seekRecord \rarrow \terminalEdge$\;
	 \BlankLine
	 \tcp{ascertain that seek record for successor node contains valid information}
	 $\ang{ \ast, \ast, \ast, \pFlag, address }$ := $\successorEdge.\child \rarrow \child[\LEFT]$\;
	 \If{\LNot($\pFlag$) \LOr ($address$ $\neq$ $\node$)}
	 {
	    $\node \rarrow \canReplace$ := \True\;
			\UpdateMode( $\state$ )\;
	    \Return\;
	 }
   \BlankLine
   \tcp{mark the right edge for promotion if unmarked}
   \MarkChildEdge( $\state$, \RIGHT{} )\; 
   \BlankLine
   \tcp{promote the key}
   $\node \rarrow \markAndKey$ := $\ang{ 1_m, \successorEdge.\child \rarrow \markAndKey }$\;
   \While{\True}
   {
      \tcp{check if the successor is the right child of the target node itself}
	    \uIf{$\successorEdge.\parent$ = $\node$}
	    {
	       \tcp{need to modify the right edge of target node whose delete flag is set}
				 $dFlag$ := 1; \qquad
			   $which$ := \RIGHT\;
	    }
	    \Else
	    {
			   $dFlag$ := 0; \qquad
			   $which$ := \LEFT\;
	    }
      $\ang{ \ast, \iFlag, \ast, \ast, \ast }$ := $\successorEdge.\parent \rarrow \child[which]$\;			
      \BlankLine			
	    $\ang{ \nFlag, \ast, \ast, \ast, \rightChild }$ := $\successorEdge.\child \rarrow \child[\RIGHT]$\;	
	    $\oldContents$ := $\ang{ 0_n, \iFlag, dFlag, 0_p, \successorEdge.\child }$\;	    
			\uIf(\tcp*[f]{only set the null flag; do not change the address}){$\nFlag$}
	    {				
				 %\mline{\parbox[t]{1.75in}{$\newContents$ := \\ \mbox{} \qquad $\ang{ 1_n, 0_i, dFlag, 0_p,  \successorEdge.\child }$;}}
				 $\newContents$ := $\ang{ 1_n, 0_i, dFlag, 0_p,  \successorEdge.\child }$\;
	    }
	    \Else(\tcp*[f]{switch the pointer to point to the grand child})
	    {	 				 
				 $\newContents$ := $\ang{ 0_n, 0_i, dFlag, 0_p, \rightChild }$ \;		 
	    }	 
      \remove{ \lIf{$result$}{\Break} }			
			%\mline{$result$ := \CAS( \parbox[t]{1.77in}{$\successorEdge.\parent \rarrow \child[which]$, \\ $\oldContents$, $\newContents$ );}\;}
			$result$ := \CAS($\successorEdge.\parent \rarrow \child[which]$,$\oldContents$, $\newContents$)\;
			\lIf{$result$ \LOr $dFlag$}{ \Break }
	    %\BlankLine			
			$\ang{ \ast, \ast, \dFlag, \ast, \ast }$ := $\successorEdge.\parent \rarrow \child[which]$\;
			$\penultimateEdge$ := $seekRecord \rarrow \penultimateEdge$\;
			\If{$\dFlag$ \LAnd ($\penultimateEdge.\parent$ $\neq$ \Null)}
			{
			   %% \mline{\HelpTargetNode( \parbox[t]{1.25in}{$\penultimateEdge$, \\ $\state \rarrow depth + 1$ );}\;}
				 \HelpTargetNode( $\penultimateEdge$ )\;
			}			
      \BlankLine			
 	    $result$ := \FindSmallest( $\state$ )\;
			$\terminalEdge$ := $seekRecord \rarrow \terminalEdge$\;
	    %\If{$\left(\text{\parbox[c]{1.875in}{\LNot($result$) \LOr \\ $\terminalEdge.\child$ $\neq$ $\successorEdge.\child$}}\right)$}
			\If{\LNot($result$) \LOr $\terminalEdge.\child$ $\neq$ $\successorEdge.\child$}
			{
			   \Break;
				 \tcp*[f]{the successor node has already been removed}
			} 
			\lElse
			{
			   $\successorEdge$ := $seekRecord \rarrow \terminalEdge$
			}
   }
   \BlankLine
	 $\node \rarrow \canReplace$ := \True\;
   \UpdateMode( $\state$ )\;	
}
\caption{Removing the Successor Node}
\label{algo:remove}
\end{algorithm}



%% CLEANUP

\begin{algorithm}[!thb]
%%
\DefineKeyWords



\DontPrintSemicolon
\Boolean \Cleanup( $\state$ )\;
\PrintSemicolon
\Begin
{
   %% \tcp{retrieve the addresses from the state record}
   %% $\node$ := $\state \rarrow \node$\;
	 %% $\parent$ := $\state \rarrow \parent$\;
	
	 %% \BlankLine
	
	 %% \tcp{determine which edge of the parent needs to be switched} 
	 %% $\ang{ \ast, pKey }$ := $\parent \rarrow \markAndKey$\;
	 %% $\ang{ \ast, nKey }$ := $\node \rarrow \markAndKey$\;
	 %% $pWhich$ := $nKey < pKey$ ? \LEFT : \RIGHT\;
	 $\ang{\parent, \node, pWhich}$ := $\state \rarrow \targetEdge$\;
	 
	
	 \BlankLine
	 
	 \uIf{$\state \rarrow type$ = \COMPLEX}
	 {
	  	   
	    \tcp{replace the node with a new copy in which all fields are unmarked} 
			$\ang{ \ast, nKey }$ := $\node \rarrow \markAndKey$\;
			$newNode \rarrow \markAndKey$ := $\ang{ 0_m, nKey }$\;		
			\BlankLine
			\tcp{initialize left and right child pointers}
  		$\ang{ \ast, \ast, \ast, \ast, \leftChild }$ := $\node \rarrow \child[\LEFT]$\;
			$\newNode \rarrow \child[\LEFT]$  := $\ang{ 0_n, 0_i, 0_d, 0_p, \leftChild }$\;
			$\ang{ \nFlag, \ast, \ast, \ast, \rightChild }$ := $\node \rarrow \child[\RIGHT]$\;
			\uIf{$\nFlag$}
			{
			  $\newNode \rarrow \child[\RIGHT]$  := $\ang{ 1_n, 0_i, 0_d, 0_p, \Null }$\;
			}
			\lElse
			{
			  $\newNode \rarrow \child[\RIGHT]$  := $\ang{ 0_n, 0_i, 0_d, 0_p, \rightChild }$
			}
			\BlankLine
			\tcp{initialize the arguments of \CAS{} instruction}
			$\oldContents$ := $\ang{ 0_n, 1_i, 0_d, 0_p, \node }$\;
			$\newContents$ := $\ang{ 0_n, 0_i, 0_d, 0_p, \newNode }$\;
			
			%% \tcp{switch the edge at the parent}
			%% \mline{$result$ := \CAS( \parbox[t]{1.875in}{$\parent \rarrow \child[pWhich]$, \\ $\ang{ 0_n, 1_i, 0_d, 0_p, \node }$, $\ang{ 0_n, 0_i, 0_d, 0_p, \newNode }$ );}\;}
			 
	
	 }
	 \Else(\tcp*[h]{remove the node})
	 {
	   			
	    %% \tcp{remove the node}
			
			\tcp{determine to which grand child will the edge at the parent be switched}
			\uIf{$\node \rarrow \child[\LEFT]$ = $\ang{ 1_n, \ast, \ast, \ast, \ast }$}
			{
		     $nWhich$ := \RIGHT\;
			}
			\lElse{$nWhich$ := \LEFT}
			
			\BlankLine
			\tcp{initialize the arguments of the \CAS{} instruction}
			$\oldContents$ := $\ang{ 0_n, 1_i, 0_d, 0_p, \node }$\;
			$\ang{ \nFlag, \ast, \ast, \ast, address }$ := $\node \rarrow \child[nWhich]$\; 
  		\uIf(\tcp*[h]{set the null flag only}){$\nFlag$}
			{
			   $\newContents$ := $\ang{ 1_n, 0_i, 0_d, 0_p, \node }$\;
			   %% \tcp{set the null flag only; do not change the address}
			   %% \mline{$result$ := \CAS( \parbox[t]{1.25in}{$\parent \rarrow \child[pWhich]$, \\ $\ang{ 0_n, 1_i, 0_d, 0_p, \node }$, \\ $\ang{ 1_n, 0_i, 0_d, 0_p, \node }$ );}\;}
			}
			\Else(\tcp*[h]{change the pointer to the grand child})
			{
			   $\newContents$ := $\ang{ 0_n, 0_i, 0_d, 0_p, address }$ \;
				 %% \mline{$result$ := \CAS( \parbox[t]{1.25in}{$\parent \rarrow \child[pWhich]$, \\ $\ang{ 0_n, 1_i, 0_d, 0_p, \node }$, \\ $\ang{ 0_n, 0_i, 0_d, 0_p, address }$ );}\;}
			}
			
			
			
	 }
	
	  \BlankLine
		\mline{$result$ := \CAS( \parbox[t]{1.75in}{$\parent \rarrow \child[pWhich]$, \\ $\oldContents$, $\newContents$ );}\;}
		\Return $result$\;
		



}

\caption{Cleaning Up the Tree}
\label{algo:cleanup}
\end{algorithm}





\begin{algorithm}[!thb]
%%
\DefineKeyWords
\DontPrintSemicolon
\Boolean \MarkChildEdge( $\state$, $which$ )\;
\PrintSemicolon
\Begin
{

   \uIf{$\state \rarrow mode$ = \INJECTION}
	 {
	    $edge$ := $\state \rarrow \targetEdge$\; 
	    $\flag$ := \DELETEFLAG\;
	 }
	 \Else
	 {
	    $edge$ := $( \state \rarrow \successorSeekRecord ) \rarrow \terminalEdge$\; 
	    $\flag$ := \PROMOTEFLAG\;
	 }
	 
	 
   $\node$ := $edge.\child$\;
	
   \BlankLine
  
	 \While{\True}
	 {
	    $\ang{\nFlag, \iFlag, \dFlag, \pFlag, address}$ := $\node \rarrow \child[which]$\;
			
			\uIf{$\iFlag$}
			{
			   $helpeeEdge$ := $\curly{ \node, address, which }$\;
				 %% \HelpTargetNode( $helpeeEdge$, $\state \rarrow depth + 1$ )\;
				 \HelpTargetNode( $helpeeEdge$ )\;
				 \Continue\;
			}
			\uElseIf{$\dFlag$}
			{
			   \uIf{$\flag$ = \PROMOTEFLAG}
				 {
				    %% \HelpTargetNode( $edge$, $\state \rarrow depth + 1$  )\;
						\HelpTargetNode( $edge$ )\;
						\Return \False\;
				 } 
				 \lElse
				 {
				    \Return \True
				 }
			}
			\ElseIf{$\pFlag$}
			{
			   \uIf{$\flag$ = \DELETEFLAG}
				 {
				    %% \HelpSuccessorNode( $edge$, $\state \rarrow depth + 1$  )\;
						\HelpSuccessorNode( $edge$ )\;
						\Return \False\;
				 } 
				 \lElse
				 {
				    \Return \True
				 }
			}
			
			$\oldContents$ := $\ang{ \nFlag, 0_i, 0_d, 0_p, address }$\;
			$\newContents$ := $\oldContents \: | \: \flag$\;
			\mline{$result$ := \CAS( \parbox[t]{1.5in}{$\node \rarrow \child[which]$, $\oldContents$, \\ $\newContents$ );}\;}
			
			\lIf{$result$}{ \Break }
			
			
	 }

   \Return \True\;
}
\caption{Mark Child Edge}
\label{algo:markChildEdge}


%\remove{

\end{algorithm}

%% FINDSMALLEST

\begin{algorithm}[!thb]
%%
\DefineKeyWords
%}

\BlankLine

\DontPrintSemicolon
\Boolean \FindSmallest( $\state$ )\;
\PrintSemicolon
\Begin
{
   \tcp{find the node with the smallest key in the subtree rooted at the right child of the target node}
	 $\node$ := $\state \rarrow \targetEdge.\child$\;
	 $seekRecord$ := $\state \rarrow seekRecord$\;
	 $\ang{ \nFlag, \ast, \ast, \ast, \rightChild }$ := $\node \rarrow \child[\RIGHT]$\;
	 \If(\tcp*[h]{the right subtree is empty}){$\nFlag$}
	 {
	    %% \tcp{the right subtree is empty}
			\Return \False\;
	 }
	
	 \BlankLine	
		
	 \tcp{initialize the variables used in the traversal}
	 
	
	 %% $\ang{ \ast, \ast, \ast, \ast, \rightChild }$ := $\node \rarrow \child[RIGHT]$\;
	 $\terminalEdge$ := $\ang{ \node, \rightChild, \RIGHT }$\;
	 $\penultimateEdge$ := $\ang{ \node, \rightChild, \RIGHT }$\;
		 
	 %% \BlankLine
	 	
	 \While{\True}
	 {
	    $\curr$ := $\terminalEdge.\child$\;
      $\ang{ \nFlag, \ast, \ast, \ast, \leftChild }$ := $\curr \rarrow \child[\LEFT]$\;			
			\If(\tcp*[h]{reached the node with the smallest key}){$\nFlag$}	
			{
			   $\injectionEdge$ := $\ang{\curr, \leftChild, \LEFT}$\;
			   \Break\;
			}				
			\BlankLine			
			\tcp{traverse the next edge}			
			$\penultimateEdge$ := $\terminalEdge$\;
	    $\terminalEdge$ := $\ang{ \curr, \leftChild, \LEFT }$\;			
	 }	
	 \BlankLine
	 \tcp{initialize seek record and return}
	 $seekRecord \rarrow \terminalEdge$ := $\terminalEdge$\;
	 $seekRecord \rarrow \penultimateEdge$ := $\penultimateEdge$\;
   $seekRecord \rarrow \injectionEdge$ := $\injectionEdge$\;
	 \Return \True\;	
}
\caption{Find Smallest}
\label{algo:findSmallest}
\end{algorithm}


\begin{algorithm}[!thb]
%%
\DefineKeyWords


\DontPrintSemicolon
\InitializeTypeAndUpdateMode( $\state$ )\;
\PrintSemicolon
\Begin
{

   \tcp{retrieve the target node's address from the state record}
   $\node$ := $\state \rarrow \targetEdge.\child$\;
	 
	
	 \BlankLine
	 %% $\canReplace$ := $\node \rarrow \canReplace$\;
	 $\ang{ \lNFlag, \ast, \ast, \ast, \ast }$ := $\node \rarrow \child[\LEFT]$\;
	 $\ang{ \rNFlag, \ast, \ast, \ast, \ast }$ := $\node \rarrow \child[\RIGHT]$\;
	
	 \uIf{$\lNFlag$ \LOr $\rNFlag$}
	 {
	    \tcp{one of the child pointers is null}
	    $\ang{\mFlag, \ast }$ := $\node \rarrow \markAndKey$\;
	    \lIf{$\mFlag$}
	    {
	      $\state \rarrow type$ := \COMPLEX
	      %% $\node \rarrow \canReplace$ := \True\;
	    }
	    \lElse
	    {
	      $\state \rarrow type$ := \SIMPLE
	     }
	 }
	 \Else(\tcp*[h]{both child pointers are non-null})
	 {
	    %% \tcp{both child pointers are non-null}
	    $\state \rarrow type$ := \COMPLEX\;
	 }
	
	 \UpdateMode( $\state$ )\;
	
	 %% \Return\;

}

\remove{

\end{algorithm}


%% UPDATEMODE

\begin{algorithm}[!thb]
%%
\DefineKeyWords

}

\BlankLine

\DontPrintSemicolon
\UpdateMode( $\state$ )\;
\PrintSemicolon
\Begin
{
	
	 \tcp{update the operation mode}

	 \BlankLine
	 \uIf(\tcp*[h]{simple delete}){$\state \rarrow type$ = \SIMPLE}
	 {
	    %% \tcp{simple delete}	
			$\state \rarrow mode$ := \CLEANUP\;
	 }
	 \Else(\tcp*[h]{complex delete})
	 {
	  	%% \tcp{complex delete}	

      $\node$ := $\state \rarrow \targetEdge.\child$\;
			\uIf{$\node \rarrow \canReplace$}
			{
			   $\state \rarrow mode$ := \CLEANUP\;
			}
			\lElse{$\state \rarrow mode$ := \DISCOVERY}
	 }
	
	 %% \Return\;
}

\caption{Helper Routines}
\label{algo:helper|2}
\end{algorithm}

%% HELP

\begin{algorithm}[!thb]
%%
\DefineKeyWords




\DontPrintSemicolon
%% \HelpTargetNode( $helpeeEdge$, $depth$ )\;
\HelpTargetNode( $helpeeEdge$ )\;
\PrintSemicolon
\Begin
{
   %% \lIf{$depth$ = number of processes}{ \Return }
	 %% \BlankLine		
	 \tcp{intent flag must be set on the edge}
	 \tcp{obtain new state record and initialize it}
	 $\state \rarrow \targetEdge$ := $helpeeEdge$\;
	 %% $\state \rarrow depth$ := $depth$\;
	 $\state \rarrow mode$ := \INJECTION\;
	 \BlankLine	
	 \tcp{mark the left and right edges if unmarked}
	 $result$ := \MarkChildEdge( $\state$, \LEFT{} )\;
	 \lIf{\LNot($result$)}{ 
	    %% \tcp{promote flag must have been set on the left edge}
			%% \HelpSuccessorNode( $helpeeEdge$, $depth + 1$ )\;
	    \Return
	 }
	 \MarkChildEdge( $\state$, \RIGHT{} )\;
	 
	 \InitializeTypeAndUpdateMode( $\state$ )\;
	
			
	 
	 \BlankLine
	
	 \tcp{perform the remaining steps of a delete operation}
   \If{$\state \rarrow mode$ = \DISCOVERY}
	 {
			%% \tcp{complex delete operation; locate the successor node and mark its child edges with promote flag}		
	    \FindAndMarkSuccessor( $\state$ )\;
	 						
	 }
			
	 \BlankLine
			
	 \If{$\state \rarrow mode$ = \DISCOVERY}
	 {
						
			%% \tcp{complex delete operation; promote the successor node's key and remove the successor node}
	    \RemoveSuccessor( $\state$ )\;
		   						
	 }
				
	 \BlankLine	
				
	 \lIf{$\state \rarrow mode$ = \CLEANUP}
	 {
	    %% \tcp{either remove the target node (simple delete) or replace it with a new node with unmarked edges (complex delete)}
	    \Cleanup( $\state$ )
	 }
	
	 %% \Return\;
}

\remove{

\end{algorithm}	
	


\begin{algorithm}[!thb]
%%
\DefineKeyWords

}

\BlankLine

\DontPrintSemicolon
%% \HelpSuccessorNode( $helpeeEdge$, $depth$ )\;
\HelpSuccessorNode( $helpeeEdge$ )\;
\PrintSemicolon
\Begin
{
   %% \lIf{$depth$ = number of processes}{ \Return }
	 %% \BlankLine
   \tcp{retrieve the address of the successor node}
   $\parent$ := $helpeeEdge.\parent$\;
	 $\node$ := $helpeeEdge.\child$\;
	 
	 \tcp{promote flat must be set on the successor node's left edge}
	 \tcp{retrieve the address of the target node}
	 $\ang{ \ast, \ast, \ast, \ast, \leftChild }$ := $\node \rarrow \child[\LEFT]$\;
	 \BlankLine	
	 \tcp{obtain new state record and initialize it}
	 $\state \rarrow \targetEdge$ := $\curly{ \Null, \leftChild, \_ }$\;
	 %% $\state \rarrow depth$ := $depth$\;
	 $\state \rarrow mode$ := \DISCOVERY\;
	 $seekRecord$ := $\state \rarrow \successorSeekRecord$\;
	 \tcp{initialize the seek record in the state record}
	 $seekRecord \rarrow \terminalEdge$ := $helpeeEdge$\;
	 $seekRecord \rarrow \penultimateEdge$ := $\curly{ \Null, \parent, \_ }$\;
   \tcp{promote the successor node's key and remove the successor node}
	 \RemoveSuccessor( $\state$ )\;
	 \tcp{no need to perform the cleanup}
	
	
	 %% \Return\;

}


\caption{Helping Conflicting Delete Operations}
\label{algo:helping}
\end{algorithm}
\end{limitscope}