\documentclass[doublespacing]{utdthesis}
%\documentclass[halfspacing]{utdthesis}
% For one-and-a-half spacing, use: \documentclass[halfspacing]{utdthesis}

%%% Load any desired packages in the space below.
%%% Warning: Do not load packages that change the margins, headers, or footers!
%%%
% Optional: If you want to use Times as your font, load it here.  Note that
% although package "times" should work, it may not be the best choice.  Newer
% LaTeX distributions offer "mathptmx" and "newtxtext,newtxmath" as superior
% replacements.  You should find out which is best for your LaTeX.  (If this
% sounds confusing, you probably shouldn't use Times at all.)
%\usepackage{times}
%
% Optional: If your LaTeX has microtype, use it to improve text quality:
\usepackage{microtype}
%
% Recommended: If your thesis contains math, use the AMS packages:
\usepackage{amsmath,amssymb,amsthm}
%
% Recommended: If your thesis needs to import graphics, use graphicx:
\usepackage{graphicx}
%
% Recommended: If your bibliography contains web page URLs, the url package
% improves their appearance (e.g., better line breaking):
%\usepackage{url}
\usepackage[hidelinks]{hyperref}
%
% Required: To satisfy UTD's formatting requirements for citations, use the
% "natbib" package.  (Use other citation packages at your own risk; not all
% are flexible enough to meet UTD's requirements.)  If you wish to use numeric
% citations, change "authoryear" to "numbers" below.  Use the "chicago" BibTeX
% style, which most closely matches the Turabian formatting required by UTD.
% UTD mandates a blank line between each pair of bibliography entries, so set
% \bibsep as shown below.  Finally, if you are accustomed to using \cite as
% your citation macro, point it to natbib's \citep macro as shown.
%\usepackage[authoryear]{natbib}
\usepackage[numbers,sort]{natbib}
\bibliographystyle{chicago}
\setlength{\bibsep}{12pt plus 1pt minus 1pt}
\let\cite=\citep
%
% Required: If you have any wide tables or figures that need to be typeset
% in landscape, use the rotating package:
\usepackage{rotating}
%
% Optional: If you use hyperref to auto-generate hyperlinks, always load it
% LAST since it modifies everything else.  In addition, only load hyperref if
% you use pdftex or pdflatex to generate PDFs directly.  Do NOT use it if you
% use plain tex or latex to generate a DVI file.  (If you are generating DVI
% files which you then convert to PDF, you should seriously consider switching
% to pdflatex.  The DVI format loses information because it cannot support
% modern PDF document features.  Using pdflatex to generate PDFs directly
% therefore results in PDFs of significantly higher quality.)
%\usepackage{ifpdf}
%\ifpdf
%  \usepackage{hyperref}
%\fi

\usepackage[ruled,vlined,linesnumbered,noresetcount]{algorithm2e}
\SetAlFnt{\footnotesize}
\usepackage[inline]{enumitem}
\usepackage{verbatim}
\usepackage{tikz}
\usepackage{pgfplots}
\usetikzlibrary{shapes.geometric,arrows,fit,matrix,positioning,pgfplots.groupplots,patterns}
\tikzset
{
    treenode/.style = {circle, draw=black, align=center, minimum size=1cm},
    subtree/.style  = {isosceles triangle, draw=black, align=center, minimum height=0.5cm, minimum width=1cm, shape border rotate=90, anchor=north},
    process/.style={rectangle, minimum width=2cm, minimum height=1cm, align=center, text width=2cm, draw},
    connector/.style={circle, minimum size=1cm, align=center, text width=1cm, draw},
    arrow/.style={thick, ->, >=stealth},
    decision/.style ={diamond, draw=black, minimum width=1cm, minimum height=1cm, text badly centered, node distance=3cm, inner sep=0pt}
}

\usepgfplotslibrary{external}
\tikzexternalize
\tikzsetexternalprefix{external_figs/}
\tikzset{external/up to date check=md5}% < added

%%% %http://texwelt.de/wissen/fragen/9476/labels-an-pgfplots/9527 (by Ijon Tichy)
\usepackage{scrlfile}
\usepackage{etoolbox}
\makeatletter
\newif\if@lateexternal
\newcommand*{\nextwithlateexternal}{\@lateexternaltrue}
\renewcommand*{\@lateexternalfalse}{\global\let\if@lateexternal\iffalse}
% Den Systemaufruf von external so ändern, dass er optional doppelt
% stattfindet: Zunächst wie gehabt unmittelbar und zusätzlich nachdem
% die aux-Datei geschlossen (und sogar neu gelesen) wurde.
\patchcmd\tikzexternal@externalizefig@systemcall@@
  {\immediate\write18{\pgf@tempa}}%
  {\immediate\write18{\pgf@tempa}%
    \if@lateexternal
      \begingroup
        \protected@edef\reserved@a{%
          \noexpand\endgroup
          \noexpand\AfterReadingMainAux{%
            \noexpand\immediate\noexpand\write18{%
              \expandafter\detokenize\expandafter{\pgf@tempa}}%
          }%
        }%
      \reserved@a
    \fi
  }%
  {}%
  {\patchFailedError}
% Nun dafür sorgen, dass der Aufruf \nextwithlateexternal nur auf
% den nächsten potentiellen Systemaufruf von external wirkt statt
% auf den nächsten tatsächlichen oder gar alle:
\apptocmd\tikzexternal@externalizefig@systemcall@@
  {\@lateexternalfalse}
  {}
  {\patchFailedError}
\makeatother
%%%

\usepackage{circuitikz}
\usepackage[labelsep=period]{caption}
\usepackage{subcaption}
\usepackage{xparse}
%for tables
\usepackage{multirow}
\usepackage[nocolor]{drawstack}
\usepackage{array}

%
%%% End of packages.

%%% Define all your personal macros here (if you have any).
%
\providecommand{\hyperref}[2][]{#2}

%\newenvironment{exampleclasscode}
% {\parindent=1cm\begin{verse}}
% {\end{verse}}
%
\newtheorem{lemma}{Lemma}
\newtheorem{theorem}{Theorem}
\newtheorem{definition}{Definition}
\newcommand{\preliminaries}{preliminaries}
\newcommand{\CAS}{\textsf{CAS}}
\newcommand{\LL}{\textsf{LL}}
\newcommand{\SC}{\textsf{SC}}
\newcommand{\true}{\textsf{true}}
\newcommand{\false}{\textsf{false}}
\newcommand{\castleMaximumgap}{46\%}
\newcommand{\icdcnMaximumgap}{27\%}
\newcommand{\localRecoveryMaximumgap}{70\%}
\newcommand{\ICDCN}{\textsc{ELFtree}}
\newcommand{\CASTLE}{\textsc{CASTLE}}
\newcommand{\CITRUS}{\textsc{CITRUS}}
\newcommand{\HJBST}{\textsc{LF-IBST}}
\newcommand{\NMBST}{\textsc{LF-EBST}}
%% macros for references - begin
\newcommand{\ang}[1]{\langle #1 \rangle}
\newcommand{\edge}[2]{\ang{{#1}, {#2}}}
\newcommand{\curly}[1]{\{ #1 \}}
\newcommand{\chapterref}[1]{Chapter~\ref{chapter:#1}}
\newcommand{\secref}[2][Section]{{#1}~\ref{sec:#2}}
\newcommand{\Secref}[1]{Section~\ref{sec:#1}}
\newcommand{\figref}[1]{Figure~\ref{fig:#1}}
\newcommand{\Figref}[1]{Figure~\ref{fig:#1}}
\newcommand{\figsref}[2]{figures~\ref{fig:#1}-\ref{fig:#2}}
\newcommand{\Figsref}[2]{Figures~\ref{fig:#1}-\ref{fig:#2}}
\newcommand{\figsrefc}[3][-]{Figures~\ref{fig:#2}{#1}\ref{fig:#3}}
\newcommand{\algoref}[1]{Algorithm~\ref{algo:#1}}
\newcommand{\Algoref}[1]{Algorithm~\ref{algo:#1}}
\newcommand{\algosref}[3][-]{Algorithms~\ref{algo:#2}{#1}\ref{algo:#3}}
\newcommand{\pseudoref}[1]{Pseudo-code~\ref{algo:#1}}
\newcommand{\Pseudoref}[1]{Pseudo-code~\ref{algo:#1}}
\newcommand{\pseudosref}[3][-]{Pseudo-codes~\ref{algo:#2}{#1}\ref{algo:#3}}
\newcommand{\Pseudosref}[3][-]{Pseudo-codes~\ref{algo:#2}{#1}\ref{algo:#3}}
\newcommand{\enumref}[1]{\ref{enum:#1}}
\newcommand{\thmref}[2][Theorem]{{#1}~\ref{thm:#2}}
\newcommand{\Thmref}[2][Theorem]{{#1}~\ref{thm:#2}}
\newcommand{\lemref}[1]{Lemma~\ref{lem:#1}}
\newcommand{\Lemref}[1]{Lemma~\ref{lem:#1}}
\newcommand{\lemsref}[3][ and ]{\mbox{Lemmas~\ref{lem:#2}{#1}\ref{lem:#3}}}
\newcommand{\obsref}[2][Observation]{{#1}~\ref{obs:#2}}
\newcommand{\propref}[1]{Proposition~\ref{prop:#1}}
\renewcommand{\eqref}[1]{(\ref{eq:#1})}
\newcommand{\corref}[1]{Corollary~\ref{cor:#1}}
\newcommand{\tabref}[1]{Table~\ref{tab:#1}}
\newcommand{\Tabref}[1]{Table~\ref{tab:#1}}
\newcommand{\remref}[1]{Remark~\ref{rem:#1}}
\newcommand{\lineref}[1]{line~\ref{lin:#1}}
\newcommand{\linesref}[3][-]{lines~\ref{lin:#2}{#1}\ref{lin:#3}}
\newcommand{\defref}[1]{Definition~\ref{def:#1}}
%% macros for references - end
%%% End of personal macro definitions.


%%% The following definitions MUST come before the document begins.
%
\author{Arunmoezhi Ramachandran}
\title{Concurrent Binary Search Trees: \\ Design and Optimizations}
\thesistype{Dissertation}
\degreefull{Doctor of Philosophy}
\degreeabbr{PhD}
\subject{Computer Science}
\graduationmonth{May}
\graduationyear{2016}
\prevdegrees{BE, MS} % comma-separated list of PREVIOUS degrees

% List committee members in order.  Mark chairpersons with a "*":
\committeemember*{Neeraj Mittal}
\committeemember{Balaji Raghavachari}
\committeemember{Venkatesan Subbarayan}
\committeemember{Rob F. Van Der Wijngaart}
%
%%% End of definitions.

%%% Beginning of actual thesis document.

\begin{document}

\frontmatter

\signaturepage
\copyrightpage{2016} % optional

\begin{dedication} % optional
To my parents and sister
\end{dedication}

\maketitle

\begin{acks}{December 2015}
I would like to thank my PhD advisor, Professor Neeraj Mittal, for guiding me throughout my PhD journey.
I am also grateful to Dr. Rob Van Der Wijngaart who was my mentor during my internship with Intel.
I enjoyed working with him on my first research project.

I am thankful to Professor R. Chandrasekaran who has always been kind and helpful to me.
I also have to thank Professors Balaji Raghavachari and Venkatesan Subbarayan, for their advice and suggestions in general.
\end{acks}

\begin{preface} % may or may not be required, depending on thesis content
% author may insert additional preface text here
\prefacetext
% author may insert additional preface text here
\end{preface}

\begin{abstract}
Over the last decade processor clock speeds have hit a wall. But the demand for performance improvements continues to grow. So there has been a major shift towards multi-core and many-core processors. This motivates the design of concurrent data structures. In such a data structure, multiple processes may need to operate on overlapping regions of the data structure simultaneously.

Designing a concurrent data structure is far more challenging than its sequential counterpart because processes executing concurrently may interleave in many ways. For all such interleavings, a concurrent data structure should manage the contention among the processes in such a way that all operations complete correctly and leave the data structure in a valid state. Concurrent algorithms may be \emph{blocking} or \emph{non-blocking}. Blocking algorithms are usually designed using locks. While a process is holding a lock, it blocks other processes from accessing the portion of the data structure protected by the lock. In a non-blocking algorithm, a suspended process will not prevent other processes from making progress. This is usually achieved by a concept called \emph{helping}, where a process always leaves enough information about its operation, so that even if it gets suspended, another process (which conflicts with the suspended process) can help finish the operation without waiting for the suspended process to resume.

In this work, we present a blocking and a non-blocking algorithm for concurrent manipulation of a binary search tree in an asynchronous shared memory system. Binary search trees are ubiquitous in computer science and are commonly used to implement dictionary abstract data type. We also provide a general optimization technique to improve performance of existing concurrent binary search trees. In this approach processes can recover from failures due to contention more efficiently using \emph{local recovery}. Our approach is sufficiently general in the sense that it can be applied to a variety of concurrent binary search trees based on both blocking and non-blocking approaches. Moreover, we also present several techniques to make search operations on such binary search trees terminate in a finite number of steps. Our techniques ensures that search operations never have to restart due to a failure.

Experiments indicate that our algorithms perform best in most cases. And our optimization technique improves performance of our algorithms and other existing algorithms. 
\end{abstract}

\tableofcontents
\listoffigures % required if you have any figures
\listoftables % required if you have any tables

\mainmatter
\newenvironment{limitscope}{}{}

\chapter{Introduction}
\label{chapter:introduction}
\ifdefined\LONG
\begin{frame}{Introduction}
\begin{itemize}
\item \small CPUs aren't getting faster (memory wall, ILP wall and power wall)
\item \small Shift towards multicore and manycore
\end{itemize}
\begin{center}
\Large {\color{red} Problem}\\
How to keep all the cores \textbf{busy}?
\end{center}
\pause
\begin{center}
\Large {\color{blue} Solutions}\\
Parallel computing \\ %(obvious choice)\\
\pause
Concurrent computing %(a better choice)
\end{center}
\end{frame}

\begin{frame}{Concurrency vs Parallelism}
\begin{center}
\Large {\color{blue}Concurrency} is not {\color{red}parallelism} (it's better!!)
\end{center}
\pause
\begin{columns}
\begin{column}[t]{0.48\textwidth}
\color{red}\rule{\linewidth}{4pt}
Parallel Computing
\setbeamertemplate{itemize/enumerate body begin}{\footnotesize}
\begin{itemize}
\item decades of research done
\item Example - Matrix-Matrix Multiplication
\item \textbf{do} lot of things simultaneously
\item cannot be done on a single CPU
\item \textbf{deterministic} control flow
\item is about \textbf{speedup}
\item \textbf{hard} to debug
\end{itemize}
\end{column}
\begin{column}[t]{0.48\textwidth}
\color{blue}\rule{\linewidth}{4pt}
Concurrent Computing
\setbeamertemplate{itemize/enumerate body begin}{\footnotesize}
\begin{itemize}
\item Relatively new
\item Example - A web crawler, mouse/keyboard
\item \textbf{deal} lot of things simultaneously
\item can be done on a single CPU
\item \textbf{non-deterministic} control flow
\item is about \textbf{hiding latency}
\item \textbf{very hard} to debug
\end{itemize}
\end{column}
\end{columns}
\end{frame}

\else
\begin{frame}{Introduction}
\begin{itemize}
\item \small CPUs aren't getting faster (memory wall, ILP wall and power wall)
\item \small Shift towards multicore and manycore
\end{itemize}
\begin{center}
\Large {\color{red} Problem}\\
How to keep all the cores \textbf{busy}?
\end{center}
\pause
\begin{center}
Concurrent computing
\end{center}
\end{frame}

\begin{frame}{Concurrent computing}
\begin{itemize}
\item Example - A web crawler, mouse/keyboard
\item \textbf{deal} lot of things simultaneously
\item can be done on a single CPU
\item \textbf{non-deterministic} control flow
\item is about \textbf{hiding latency}
\item \textbf{very hard} to debug
\end{itemize}
\end{frame}
\fi

\chapter{\preliminaries}
\label{chapter:\preliminaries}
Processors were originally developed with only one core. So the data structures designed to run on then were sequential in nature.
But as the processing speed of a single processor saturated, there was a shift towards multi-core processors.
These shared-memory multiprocessors concurrently execute multiple threads.
These threads communicate and synchronize through data structures in shared memory.
The efficiency of these data structures are crucial to the performance.
So developing concurrent versions of the sequential data structure became important.
As threads in a multiprocessor can interleave in exponential number of ways, the challenge is to ensure that a concurrent data structure preserves its sequential specification for all possible interleavings.

\section{Designing Concurrent Data Structures}
The following example illustrates the design of a simple concurrent data structure.
Let $x$ be a \emph{shared counter} which can be incremented using a function fetchAndIncrement().
\Figref{sharedCounter} shows various implementations of this counter.
In the sequential version, the \emph{counter} $x$ is first loaded into a register, then the register value is incremented and finally it is stored back into $x$.
Each of these three steps are atomic. But put together they are no longer atomic.
For instance consider this scenario. Let the initial value of $x$ be 0.
Two threads $\alpha$ and $\beta$ read its value and copied it into their local registers.
Then, both of them increment their local registers. Now when they try to write their register value back into the shared counter, there is a race condition.
Without loss of generality let us assume that thread $\alpha$ made the first write. Now $x$ becomes 1.
Then when $\beta$ issues a write, the value of $x$ is overwritten. But the new value of $x$ is still 1.
Ideally, it has to be 2 as two threads have incremented the shared counter.
This is a classic \emph{lost-update} or \emph{write-after-write} problem.

\begin{figure}[htp]
\begin{minipage}{0.45\textwidth}
\SetAlgorithmName{fetchAndIncrement}{}{}
\SetAlFnt{\selectfont \ttfamily}
\renewcommand{\thealgocf}{}
\SetAlgoInsideSkip{} 
\SetAlCapSkip{1.25ex}
\SetAlgoCaptionLayout{centering}
\setlength{\algomargin}{1.5em}
\SetInd{0.5em}{0.625em}
\RestyleAlgo{boxed}
\LinesNumberedHidden
\begin{algorithm}[H]
\label{algo:counter-sequential}
\phantom{acquire(lock)\;}
r1 = x\;
inc(r1)\;
x = r1\;
\phantom{release(lock)\;}
\caption{sequential}
\end{algorithm}
\end{minipage}
%
\begin{minipage}{0.45\textwidth}
\SetAlgorithmName{fetchAndIncrement}{}{}
\SetAlFnt{\selectfont \ttfamily}
\renewcommand{\thealgocf}{}
\SetAlgoInsideSkip{} 
\SetAlCapSkip{1.25ex}
\SetAlgoCaptionLayout{centering}
\setlength{\algomargin}{1.5em}
\SetInd{0.5em}{0.625em}
\RestyleAlgo{boxed}
\LinesNumberedHidden
\begin{algorithm}[H]
\label{algo:counter-lock}
\caption{Using locks}
acquire(lock)\;
r1 = x\;
inc(r1)\;
x = r1\;
release(lock)\;
\end{algorithm}
\end{minipage}


\begin{center}
\begin{minipage}[t]{0.65\textwidth}
\SetAlgorithmName{fetchAndIncrement}{}{}
\SetAlFnt{\selectfont \ttfamily}
\renewcommand{\thealgocf}{}
\SetAlgoInsideSkip{} 
\SetAlCapSkip{1.25ex}
\SetAlgoCaptionLayout{centering}
\setlength{\algomargin}{1.5em}
\SetInd{0.5em}{0.625em}
\RestyleAlgo{boxed}
\LinesNumberedHidden
\begin{algorithm}[H]
\label{algo:counter-CAS}
\caption{using atomic instructions}
\Repeat{(x.compareAndSwap(rOld,rNew))}
{
rOld = x\;
rNew = rOld+1\;
}
\end{algorithm}
\end{minipage}
\end{center}

\caption{Various implementations of a Shared Counter}
\label{fig:sharedCounter}

\end{figure}

A simple way to avoid such race conditions is to wrap them with locks.
Consider the same scenario again. Without loss of generality, let thread $\alpha$ obtain the lock first.
Thread $\beta$ cannot do any operation on the shared counter until thread $\alpha$ releases the lock.
When $\alpha$ releases the lock, the value of $x$ is 1.
Now when $\beta$ obtains the lock it reads the value of $x$ to be 1.
It increments the value of $x$ and releases the lock.
The final value of $x$ is 2 which is as expected.

Another way to solve the race conditions is to use \emph{Read-Modify-Write} (RMW) instructions like \emph{test-and-set}, \emph{fetch-and-add} and \emph{compare-and-swap}.
They essentially read a memory location and write a new value into it atomically.
As shown in \figref{sharedCounter}, A compare-and-swap  instruction takes three arguments: $address$, $old$ and $new$; it compares the contents of a memory location ($address$) to a given value ($old$) and, only if they are the same, modifies the contents of that location to a given new value ($new$).
The same shared counter can be implemented using a compare-and-swap instruction.
For the same scenario, let both $\alpha$ and $\beta$ threads read the value of $x$ into their local register $rOld$.
They compute $rNew$ to be 1 and both of them try to perform a CAS(Compare-And-Swap) operation.
Without loss of generality let $\alpha$ precede $\beta$.
The operation by $\alpha$ would succeed as the the value of $x$(0) is equal to $rOld$(0).
But when $\beta$ tries to perform a CAS operation, the value of $x$ has already changed to 1.
So it would fail, as the value of $x$(1) is not equal to $rOld$(0).
So $\beta$ would retry the operation.
This time it reads the updated value of $x$ into $rOld$ and the CAS operation would succeed leaving the final value of $x$ to be 2.

Each of the two solutions described above handled contention in a different way.
The one using locks are used in developing blocking algorithms and the one using atomic operations are used to develop non-blocking algorithms.

\section{Blocking algorithms}
The granularity of the locks used in blocking algorithms can be \emph{coarse} or \emph{fine}.
Coarse grained locks allows only one thread to operate on a data structure.
But fine grain locks allows multiple threads to operate on different portions of the data structure concurrently.
In general, fine-grained locking approaches provides better performance.
Though blocking algorithm are relatively easier to design than their non-blocking counterpart, they have their own limitations.
They are often prone to deadlocks and priority inversion.
Also they provide weaker progress guarantees.
For instance consider two threads $\alpha$ and $\beta$ trying to obtain lock on a shared counter.
Let $\alpha$ obtain the lock.
Assume that before it increments the counter it gets swapped out of context by the operating system.
Now $\beta$ will not be able to obtain the lock until $\alpha$ releases it.
Though there is no deadlock in this state, the system as a whole is not making any progress.

\section{Non-blocking algorithms}
Non-blocking algorithms uses atomic (Read-Modify-Write) instructions to resolve contention.
The also provide stronger progress guarantees. A non-blocking algorithm is either \emph{lock-free} or \emph{wait-free}.
An algorithm is lock-free if at least one thread is able to make progress over an infinite period of time in a system.
It is wait-free if every thread is able to complete its operations in a finite number of steps over an infinite period of time in a system.
The notion of lock-freedom is at the algorithm level and not at the system level as the atomic operations still uses locks at the hardware level.
In a blocking algorithm a thread owns a lock but in a non-blocking algorithm an operation being performed by a thread owns a lock.

Lock-freedom and wait-freedom are usually achieved using the concept of \emph{helping}.
When a thread performing its operation sees that some other thread is in its way,  then it first \emph{helps} the other thread before continuing its own operation.
To enable helping, every thread, for all its operations on the shared data structure, leaves enough information in the shared memory so that even if it gets swapped out of context by the operating system, some other contending thread might still be able to help finish its operation.
This way the system as a whole is able to make progress even though some threads might be stalled in the middle of its operations.

\section{Linearizability}
Proving correctness of concurrent data structures are as hard as designing them.
For the simple shared counter using locks and atomic operations it is easy to prove the correctness.
But for complex data structures like trees it is much more difficult.
So we need a formal way to define correctness. 

A sequential object has a state and a set of methods which operate on the object making the object move from one valid state to another.
A sequence of method invocations and responses is called a \emph{history}.
Every method has a set of pre-conditions and post-conditions.
Pre-conditions captures the state of the object before the method is invoked and post-conditions captures the state after the method returns.
In a sequential specification of an object, each method is described in isolation.
The interactions among methods are captured by the side-effects on the object state.
So a sequential object needs a meaningful state only between method calls.
Also as methods are described in isolation, new methods can be added without modifying the existing methods.
As the methods are executed one-at-a-time, proving the correctness requires that the any history (a sequence of method invocations and response) respects the sequential specification of the object.

For a concurrent object, multiple methods might be executing concurrently.
Since method calls overlap, an object might never be between method calls and hence we cannot define a valid state.
Also we must consider all the exponential possible interactions among method calls.
Proving the correctness of  a concurrent object requires that some total ordering of any history respects the sequential specification of the object.
%These problems are caused since method calls are not instantaneous. 
%They take finite time and can be modelled as an interval in a time line.

Two well known consistency conditions for shared memory systems are \emph{sequential consistency}~\cite{Lamport:1979} and \emph{linearizability}~\cite{HerWin:1990:TOPLAS}.
Sequential consistency is not composable while linearizability is.
Traditionally software systems as they are built by composing multiple subsystems.
So linearizability is preferred in designing concurrent data structures.

Linearizability requires two properties:
\begin{enumerate*}[label=(\roman*)]
\item the object (or data structure) be sequentially consistent
\item the total ordering which makes it sequentially consistent respect the \emph{real-time ordering} among the operations in the execution.
\end{enumerate*}
Respecting real-time ordering means that if an operation $op_1$ completed before another operation $op_2$, then $op_1$ must be ordered before $op_2$.
In other words, linearizability  requires us to identify a distinct point between a method invocation and response where the method appears to have taken effect instantaneously.
This point is called a \emph{linearization point}.
Now if we order the method calls based on their linearization points then the resulting order should be in the sequential specification of the object.

\ifdefined\LONG
\begin{frame}{Linearizability}
An object has:
\begin{itemize}
\item state
\item a set of methods which operate on the object making the object move from one valid state to another
\end{itemize}
\begin{itemize}
\item A sequence of method invokations and responses is called a \emph{history}
\item Every method has a set of pre-conditions and post-conditions
\item Pre-conditions captures the state of an object before a method is invoked
\item Post-conditions captures the state after a method returns
\end{itemize}
\end{frame}

\begin{frame}{Linearizability}
\begin{itemize}
\item In a sequential specification of an object, each method is described in isolation
\item The interactions among methods are captured by the side-effects on the object state
\item So a sequential object needs a meaningful state only between method calls
\item New methods can be added without modifying the existing methods
\item proving correctness requires that the any history respects the sequential specification of the object
\end{itemize}
\end{frame}

\begin{frame}{Linearizability}
\begin{itemize}
\item For a concurrent object, multiple methods might be executing concurrently
\item Since method calls overlap, an object might never be between method calls
\item exponential possible interactions among method calls
\item Proving the correctness requires that some total ordering of any history respects the sequential specification of the object
\end{itemize}
\end{frame}
\fi
\begin{frame}{Linearizability}
Linearizability requires two properties:
\begin{itemize}
\item the object (or data structure) be sequentially consistent\footnote{\textit{\tiny the result of any execution is the same as if the operations of all the processors were executed in some sequential order, and the operations of each individual processor appear in this sequence in the order specified by its program}}
\item the total ordering which makes it sequentially consistent respect the \emph{real-time ordering} among the operations in the execution
\end{itemize}
\emph{respecting real-time ordering} - if an operation $op_1$ completed before another operation $op_2$, then $op_1$ must be ordered before $op_2$
\end{frame}

\begin{frame}{Linearizability}
\begin{itemize}
\item \emph{linearization point} - a distinct point between a method invokation and response where the method appears to have taken effect instantaneously
\item Order the method calls based on their linearization points
\item Resulting order should be in the sequential specification of the object.
\end{itemize}
\end{frame}

\begin{frame}{Linearizability - Examples}
\begin{figure}[htp]
		\begin{tikzpicture}[scale=0.9, transform shape] 
		\node (x1)  {A};
		\node (x2) [right of=x1,xshift=1cm]{};
		\draw (x1) -- node[above]{read(1)} (x2);
		\node(x3)[right of=x2,xshift=-1cm]{};
		\node(x4) [right of=x3,xshift=1cm]{};
		\draw (x3) -- node[above]{write(2)} (x4);
		
		\node(x5)[right of=x4,xshift=-1cm]{};
		\node(x6) [right of=x5,xshift=1cm]{};
		\draw (x5) -- node[above]{read(2)} (x6);
		\end{tikzpicture}
		\caption{A history of a sequential object}
\end{figure}
\end{frame}

\begin{frame}{Linearizability - Examples}
\begin{figure}[htp]
\begin{tikzpicture}[scale=0.9, transform shape] 
		\node (x1) {A};
		\node (x2) [right of=x1,xshift=1cm]{};
		\draw (x1) -- node[above]{write(0)} (x2);
		\draw (0.5,-0.1) -- (0.5,+0.1);
		
		\node(x3)[right of=x2,xshift=-1cm]{};
		\node(x4) [right of=x3,xshift=1cm]{};
		\draw (x3) -- node[above]{read(1)} (x4);
		\draw (3,-0.1) -- (3,+0.1);
		
		\node(x5)[right of=x4,xshift=-1cm]{};
		\node(x6) [right of=x5,xshift=1cm]{};
		\draw (x5) -- node[above]{write(0)} (x6);
		\draw (5,-0.1) -- (5,+0.1);
		
		\node(B)[below of=x1]{B};
		\node(x7)[below of=x1,xshift=0.5cm]{};
		\node(x8) [right of=x7,xshift=5.5cm]{};
		\draw (x7) -- node[above]{write(1)} (x8);
		\draw (2.5,-1.1) -- (2.5,-0.9);
		
		\node(x9)[right of=x8,xshift=-1cm]{};
		\node(x10) [right of=x9,xshift=1cm]{};
		\draw (x9) -- node[above]{read(0)} (x10);
		\draw (8,-1.1) -- (8,-0.9);
		
		\end{tikzpicture}
		\caption{A history of a concurrent object - linearizable}
\end{figure}
\end{frame}

\begin{frame}{Linearizability - Examples}
\begin{figure}[htp]
\begin{tikzpicture}[scale=0.9, transform shape] 
		\node (x1) {A};
		\node (x2) [right of=x1,xshift=1cm]{};
		\draw (x1) -- node[above]{write(0)} (x2);
		
		\node(x3)[right of=x2,xshift=-1cm]{};
		\node(x4) [right of=x3,xshift=1cm]{};
		\draw (x3) -- node[above]{read(1)} (x4);
		\draw[dashed] (3.9,-1.2) -- (3.9,+0.5);
		
		\node(x5)[right of=x4,xshift=-1cm]{};
		\node(x6) [right of=x5,xshift=1cm]{};
		\draw (x5) -- node[above]{write(0)} (x6);
		
		\node(B)[below of=x1]{B};
		\node(x7)[below of=x1,xshift=0.5cm]{};
		\node(x8) [right of=x7,xshift=5.5cm]{};
		\draw (x7) -- node [above, xshift=-1cm,yshift=-0.1cm] {write(1)} (x8);
		
		\node(x9)[right of=x8,xshift=-1cm]{};
		\node(x10) [right of=x9,xshift=1cm]{};
		\draw (x9) -- node[above]{read(1)} (x10);
		
		\end{tikzpicture}
		\caption{A history of a concurrent object - not linearizable}
\end{figure}
\end{frame}

\Figref{linearizability}(a) shows a sequential execution of reads and writes on a register.
\Figref{linearizability}(b) shows an execution history of a concurrent object.
The linearization points are chosen such that \textit{write(0) $\rightarrow$ write(1) $\rightarrow$ read(1) $\rightarrow$ write(0) $\rightarrow$ read(0)}.
\Figref{linearizability}(c) shows a history which is not linearizable.
Since thread $A$ reads a value 1, the \textit{write()} of thread $B$ must precede the second \textit{write()} of thread $A$.
So the \textit{read()} by thread $B$ cannot read a value 1.

As linearizability is intuitive and composable, most of the concurrent data structure implementations use it as a correctness condition.
Moir and Shavit~\cite{moirSha:2007} provides a detailed literature survey of various concurrent data structure implementations.
For each of the algorithms we present in this dissertation we prove that they are linearizable.

\chapter{System model}
\label{chapter:systemModel}
\section{Binary Search Tree}
We assume that a binary search tree (BST) implements a dictionary abstract data type and supports \emph{search}, \emph{insert} and \emph{delete} operations. For convenience, we refer to the insert and delete operations as \emph{modify} operations. A search operation explores the tree for a given key and returns \true{} if the key is present in the tree and \false{} otherwise. An insert operation adds a given key to the tree if the key is not already present in the tree. Duplicate keys are not allowed in our model. A delete operation removes a key from the tree if the key is indeed present in the tree. In both cases, a modify operation returns \true{} if it changed the set of keys present in the tree (added or removed a key) and \false{} otherwise.

A binary search tree satisfies the following properties:
\begin{enumerate}[label=(\alph*)]
\item the left subtree of a node contains only nodes with keys less than the node's key, 
\item the right subtree of a node contains only nodes with keys greater than or equal to the node's key, and
\item the left and right subtrees of a node are also binary search trees.
\end{enumerate}

\section{Synchronization Primitives}
We assume an asynchronous shared memory system that, in addition to read and write instructions, also supports compare-and-swap (\CAS{}) atomic instruction. A compare-and-swap  instruction takes three arguments: $address$, $old$ and $new$; it compares the contents of a memory location ($address$) to a given value ($old$) and, only if they are the same, modifies the contents of that location to a given new value ($new$). The \CAS{} instruction is commonly available in many modern processors such as Intel~64 and AMD64. 

We also use locks and assume that the following properties hold true about the locks
\begin{enumerate}[label=(\alph*)]
\item safe: it satisfies the mutual exclusion property, \emph{i.e.}, at most one process can hold the lock at any time, and 
\item live: it satisfies the deadlock freedom property, \emph{i.e.}, if the lock is free and one or more processes attempt to acquire the lock, then some process is eventually able to acquire the lock.
\end{enumerate}

\section{Proof of correctness}
To demonstrate the correctness of our algorithm, we use \emph{linearizability}~\cite{HerWin:1990:TOPLAS} for the safety property and \emph{deadlock-freedom}~\cite{HerSha:2012:Book} for the liveness property. Broadly speaking, linearizability requires that an operation should appear to take effect instantaneously at some point during its execution.  Deadlock-freedom requires that some process with a pending operation be able to complete its operation eventually.

\part{Design}

\chapter{Lock based concurrent binary search tree}
\label{chapter:castle}
\newenvironment{limitscope}{}{}
\begin{limitscope}
%%%%% castle macros - begin
\newcommand{\accesspath}{access-path}
\newcommand{\terminalnode}{terminal node}

\newcommand{\true}{\textsf{true}}
\newcommand{\false}{\textsf{false}}

\newcommand{\CAS}{\textsf{CAS}}

\newcommand{\sNodeOne}{\mathbb{R}}
\newcommand{\sNodeTwo}{\mathbb{S}}
\newcommand{\sKeyOne}{\infty_1}
\newcommand{\sKeyTwo}{\infty_2}

\newcommand{\targetnode}{target node}
\newcommand{\anchornode}{anchor node}

\newcommand{\myparent}{parent}
\newcommand{\myleft}{le\!f\!t}
\newcommand{\myright}{right}

\newcommand{\CASTLE}{\textsc{CASTLE}}
\newcommand{\CITRUS}{\textsc{CITRUS}}
\newcommand{\HJBST}{\textsc{LF-IBST}}
\newcommand{\NMBST}{\textsc{LF-EBST}}

\newcommand{\RemoveChild}{\textsc{RemoveChild}}
\newcommand{\LockAll}{\textsc{LockAll}}
\newcommand{\UnlockAll}{\textsc{UnlockAll}}
\newcommand{\ClearFlags}{\textsc{ClearFlags}}
\newcommand{\FindSmallest}{\textsc{FindSmallest}}

\newcommand{\lFlag}{lFlag}
\newcommand{\mFlag}{mFlag}
\newcommand{\nFlag}{nFlag}

%%%%% castle macros - end

\section{The Lock-Based Algorithm}
\label{sec:castle-algorithm}
We first provide an overview of our algorithm. We then describe the algorithm in more detail and also give its pseudo-code. For ease of exposition, we describe our algorithm assuming no memory reclamation, which can be performed using the well-known technique of hazard pointers~\cite{Mic:2004:TPDS}.

\section{Overview of the Algorithm}
Every operation in our algorithm uses \emph{seek} function as a subroutine. The seek function traverses the  tree from the root node until it either finds the target key or reaches a non-binary node whose next edge to be followed points to a null node. We refer to the path traversed by the operation during the seek  as the \emph{\accesspath}, and the last node in the \accesspath{} as the \emph{\terminalnode}. The operation then compares the target key with the stored key (the key present in the \terminalnode). Depending on the result of the comparison and the type of the operation, the operation either terminates or moves to the execution phase. In certain cases in which a key may have moved upward along the \accesspath, the seek function may have to restart and traverse the tree again; details about restarting are provided later. We now describe the next steps for each of the type of operation one-by-one. 

\paragraph{Search:} 
A search operation starts by invoking seek operation. It returns \true{} if the stored key matches the target key and \false{} otherwise. 

\paragraph{Insert:}
An insert operation starts by invoking seek operation. It returns \false{} if the target key matches the stored key; otherwise, it moves to the execution phase. In the execution phase, it attempts to insert the key into the tree as a child node of the last node in the \accesspath{} using a \CAS{} instruction. If the instruction succeeds, then the operation returns \true{}; otherwise, it restarts by invoking the seek function again.

\paragraph{Delete:} 
A delete operation starts by invoking seek function. It returns \false{} if the stored key does not match the target key; otherwise, it moves to the execution phase. In the execution phase, it attempts to remove the key stored in the \terminalnode{} of the \accesspath. There are two cases depending on whether the \terminalnode{} is a binary node (has two children) or not (has at most one child). In the first case, the operation is referred to as \emph{complex delete operation}. In the second case, it is referred to as \emph{simple delete operation}. In the case of simple delete, the \terminalnode{} is removed by changing the pointer at the parent node of the \terminalnode. In the 
case of complex delete, the key to be deleted is replaced with the \emph{next largest} key in the tree, which will be stored in the \emph{leftmost node} of the \emph{right subtree} of the \terminalnode.

\section{Details of the Algorithm}
\label{sec:description}

\begin{limitscope}

%% To limit the scope of the macros defined below

%% macros for pseudocode

\newcommand{\child}{child}
\newcommand{\node}{node}
\newcommand{\parent}{parent}

\newcommand{\mainSeekRecord}{seekTargetKey}
\newcommand{\successorSeekRecord}{seekSuccessorKey}


\newcommand{\targetStack}{targetStack}
\newcommand{\successorStack}{successorStack}


\newcommand{\successorStackInUse}{successorStackInUse}
\newcommand{\targetNode}{targetNode}




\newcommand{\key}{key}

\newcommand{\done}{done}
\newcommand{\result}{result}
\newcommand{\status}{status}
\newcommand{\restart}{restart}





\newcommand{\cKey}{key}
\newcommand{\nKey}{key}
\newcommand{\cNode}{current}
\newcommand{\pNode}{parent}
\newcommand{\nMarked}{marked}



\newcommand{\which}{which}
\newcommand{\address}{address}

\newcommand{\anchor}{anchor}

\newcommand{\stack}{stack}
\newcommand{\sTop}{top}
\newcommand{\sBottom}{bottom}
\newcommand{\current}{current}


\newcommand{\admissible}{admissible}
\newcommand{\critical}{critical}
\newcommand{\reference}{re\!f\!erence}

%% \newcommand{\OptReturn}[1][]{\Return #1\;}
\newcommand{\OptReturn}[1][]{}

\newcommand{\injectionPoint}{injectionPoint}



\newcommand{\Search}{\textsc{Search}}
\newcommand{\Insert}{\textsc{Insert}}
\newcommand{\Delete}{\textsc{Delete}}


\newcommand{\Inject}{\textsc{Inject}}



\remove{
\newcommand{\SeekForSuccessor}{\textsc{SeekForSuccessor}}
\newcommand{\NeedSuccessorKey}{\textsc{NeedSuccessorKey}}
\newcommand{\GetChild}{\textsc{GetChild}}
\newcommand{\Move}{\textsc{Move}}
\newcommand{\GetAddress}{\textsc{GetAddress}}
\newcommand{\IsNull}{\textsc{IsNull}}
\newcommand{\PopulateSeekRecord}{\textsc{PopulateSeekRecord}}
}



\newcommand{\mline}[1]{\DontPrintSemicolon #1 \PrintSemicolon}


\newcommand{\LEFT}{\textsf{LEFT}}
\newcommand{\RIGHT}{\textsf{RIGHT}}


\newcommand{\rarrow}{\!\rightarrow\!}
\newcommand{\type}{type}


\newcommand{\SEARCH}{\textsf{SEARCH}}
\newcommand{\INSERT}{\textsf{INSERT}}
\newcommand{\DELETE}{\textsf{DELETE}}

\newcommand{\STOPFOUND}{\textsf{FOUND}}
\newcommand{\STOPNOTFOUND}{\textsf{NOT\_FOUND}}
\newcommand{\DONOTKNOW}{\textsf{CONTINUE}}

\newcommand{\TARGETSTACK}{\textsf{TARGET\_STACK}}
\newcommand{\SUCCESSORSTACK}{\textsf{SUCCESSOR\_STACK}}

%%%%%%%%%%%%%%%%%%%%%%%%%%%%%%%%%%%%%%%%%%%%%%%%%%%%%%%%%%%%%%%%%%%%%%%%%%%%%%%%%%%%

\newcommand{\DefineKeyWords}{
%%
\SetKw{Boolean}{boolean}
\SetKw{Integer}{integer}
\SetKw{LAnd}{~and~}
\SetKw{LOr}{~or~}
\SetKw{LNot}{not}
\SetKw{Struct}{struct}
\SetKw{Null}{null}
\SetKw{True}{true}
\SetKw{False}{false}
\SetKw{Break}{break}
\SetKw{Continue}{continue}
\SetKw{Enum}{enum}
\SetKw{Word}{word}
%%
}

%%%%%%%%%%%%%%%%%%%%%%%%%%%%%%%%%%%%%%%%%%%%%%%%%%%%%%%%%%%%%%%%%%%%%%%%%%%%%%%%%%%%%

%% Data structures used by the local recovery algorithm

%%%%%%%%%%%%%%%%%%%%%%%%%%%%%%%%%%%%%%%%%%%%%%%%%%%%%%%%%%%%%%%%%%%%%%%%%%%%%%%%%%%%%

\begin{algorithm}[tb]
\caption{Data Structures Used} 
\label{algo:local-data|structures}
%%
\DefineKeyWords
%%
\tcp{Used to store information about a node visited during tree traversal}
\DontPrintSemicolon
\Struct StackEntry \{\;
\PrintSemicolon
%%
\label{lin:local-data|structures:begin}
\label{lin:local-stack|entry:begin}
\Indp 
   NodePtr $\node$\;
	 \Enum Direction $\which$\;
   \Integer $\anchor$\;
\Indm 
\}\;
\label{lin:local-stack|entry:end}

\BlankLine

\tcp{Used to store the path from the root node to the current node in the tree}
\DontPrintSemicolon
\Struct \TraversalRecord{} \{\;
\PrintSemicolon
%%
\label{lin:local-traversal|record:begin}
\Indp 
   StackEntry[~] $\stack$\;
	 \Integer $\sTop$\;
	 %% \Integer $\sBottom$\;
\Indm 
\}\;
\label{lin:local-traversal|record:end}

\BlankLine

\tcp{Used to store information about the operation currently in progress}
\DontPrintSemicolon
\Struct \OpRecord{} \{\;
\PrintSemicolon
%%
\label{lin:local-op|record:begin}
\Indp 
   \Enum Type $\type$\;
	 Key $\key$\;
	 %% \TraversalRecord{} $\targetStack$, $\successorStack$\;
	 \TraversalRecord{} $\targetStack$\;
	 %%\TraversalRecord{} $\successorStack$\;
	 %% \Boolean $\successorStackInUse$\;
	 %% NodePtr $\targetNode$\;
	 NodePtr $\injectionPoint$\;
	 \BlankLine
	 \tcp{algorithm-specific fields}
\Indm
\}\;
\label{lin:local-op|record:end}

\BlankLine

\tcp{Used to store the outcome of a tree traversal}
\DontPrintSemicolon
\Struct \SeekRecord \{\;
\PrintSemicolon
%%
\label{lin:local-seek|record:begin}
\Indp 
   %% \TraversalRecord{}Ptr $\traversalRecord$\;
	 \tcp{algorithm-specific fields (\emph{e.g.}, target node and its parent)}
\Indm 
\}\;
\label{lin:local-seek|record:end}

\remove{
   \BlankLine
%%
   \tcp{Local records used when executing an operation}
   \OpRecord{}Ptr $\opRecord$\;
   SeekRecordPtr $\mainSeekRecord$\;
   SeekRecordPtr $\successorSeekRecord$\; 
}
\label{lin:local-data|structures:end}
%%
\end{algorithm}


%%%%%%%%%%%%%%%%%%%%%%%%%%%%%%%%%%%%%%%%%%%%%%%%%%%%%%%%%%%%%%%%%%%%%%%%%%%%%%%%%%%%%

%% Functions used for manipulating traversal stack

%%%%%%%%%%%%%%%%%%%%%%%%%%%%%%%%%%%%%%%%%%%%%%%%%%%%%%%%%%%%%%%%%%%%%%%%%%%%%%%%%%%%%

\begin{algorithm}[tb]
\caption{Functions for Manipulating Traversal Stack} 
\label{algo:local-stack|functions}
%%
\DefineKeyWords
%%
\tcp{Returns the number of elements in the stack}
\DontPrintSemicolon
\Integer \Size( $\traversalRecord$ )\;
\PrintSemicolon
\label{lin:local-stack|begin}
\label{lin:local-size:begin}
\Begin
{
   
   \Return $\traversalRecord \rarrow \sTop + 1$\;
	 \label{lin:local-size:end}
}
%%
\BlankLine
%%
\tcp{Returns the topmost node in the stack}
\DontPrintSemicolon
NodePtr \GetTop( $\traversalRecord$ )\;
\PrintSemicolon
\label{lin:local-get|top:begin}
\Begin
{
   
   $\curly{ \stack, \sTop }$ := $\traversalRecord$\;
	 \label{lin:local-stack|retrieve}
	 \Return $\stack[\sTop] \rarrow \node$\;
	 \label{lin:local-get|top:end}
}
%%
\BlankLine
%%
\tcp{Returns the second topmost node in the stack}
\DontPrintSemicolon
NodePtr \GetSecondToTop( $\traversalRecord$ )\;
\label{lin:local-get|second|to|top:begin}
\PrintSemicolon
\Begin
{
   
   $\curly{ \stack, \sTop }$ := $\traversalRecord$\;
	 \Return $\stack[\sTop-1] \rarrow \node$\;
	 \label{lin:local-get|second|to|top:end}
}
%%
\BlankLine
%%
\tcp{Adds the given node to the stack along with its \myanchor{} node}
\DontPrintSemicolon
\AddToTop(  $\traversalRecord$, $\node$, $\which$ )\;
\PrintSemicolon
\label{lin:local-add|to|top:begin}
\Begin
{
%%
   
   $\curly{ \stack, \sTop }$ := $\traversalRecord$\;
   	
	 \tcp{find the \myanchor{} node}
   \lIf{$\which$ = \RIGHT}
	 {
	    $\anchor$ := $\sTop$
	 }
	 \lElse
	 {
	    $\anchor$ := $\stack[\sTop] \rarrow \anchor$
	 }
   
	 \tcp{push the node into the stack}
   $\stack[\sTop + 1]$ := $\curly{ \node, \which, \anchor }$\;
	 $\traversalRecord \rarrow \sTop$ := $\sTop + 1$\;
	 \OptReturn
	 \label{lin:local-add|to|top:end}
	
%%
}
%%
\BlankLine
%%
\tcp{Removes the topmost node from the stack}
\DontPrintSemicolon
\RemoveFromTop ( $\traversalRecord$ )\;
\PrintSemicolon
\label{lin:local-remove|from|top:begin}
\Begin
{
%%
   
   $\curly{ \stack, \sTop }$ := $\traversalRecord$\;
	
	 \tcp{update the \myanchor{} node of the penultimate entry if needed}
	 $\anchor$ := $\stack[\sTop - 1] \rarrow \anchor$\;
	 \If{$\stack[\sTop] \rarrow \anchor$ $<$ $\stack[\anchor] \rarrow \anchor$}
	 {
	    $\stack[\anchor] \rarrow \anchor$ := $\stack[\sTop] \rarrow \anchor$\;  
	 }
	
	 \tcp{pop the node from the stack}
	 $\traversalRecord \rarrow \sTop$ := $\sTop - 1$\;
	 \OptReturn
	 \label{lin:local-remove|from|top:end}
	 	 
%%
}
%%
\BlankLine
%%
\tcp{Pops the stack until a given entry}
\DontPrintSemicolon
\RemoveUntilCritical( $\traversalRecord$,  $index$ )\;
\PrintSemicolon
\label{lin:local-remove|until|critical:begin}
\Begin
{
  
   $\traversalRecord \rarrow \sTop$ := $index$\;
	 \OptReturn
	 \label{lin:local-remove|until|critical:end}

}
\end{algorithm}

\begin{algorithm}
\caption{Functions for Manipulating Traversal Stack (Continued)} 
\label{algo:local-local-stack|functions|2}
\DefineKeyWords
\tcp{Remember the \mycritical{} node (to avoid locating it again)}
\DontPrintSemicolon
\RememberCritical( $\traversalRecord$,  $\critical$ )\;
\PrintSemicolon
\label{lin:local-remember|critical:begin}
\Begin
{
  
   %% $\curly{ \stack, \sTop }$ := $\traversalRecord$\;  
	 $\curly{ \stack, \sTop }$ := $\traversalRecord$\;
   	
	 $\anchor$ := $\stack[\sTop] \rarrow \anchor$\;
	 \If{$\critical$ $<$ $\stack[\anchor] \rarrow \anchor$}
	 {
	   $\stack[\anchor] \rarrow \anchor$ := $\critical$\;
	 }
	 \OptReturn
	 \label{lin:local-remember|critical:end}
}
%%
\BlankLine
%%
\tcp{Returns a given entry in the stack}
\DontPrintSemicolon
\{ NodePtr, \Enum Direction, \Integer \}  \GetFullEntry( $\traversalRecord$,  $index$ )\;
\PrintSemicolon
\label{lin:local-get|full|entry:begin}
\Begin
{
   
   $\curly{ \stack, \sTop }$ := $\traversalRecord$\;
	 
	
	 \remove{
	
	    \tcp{find the location of the entry in the stack}
			
	    \lIf{$entry$ = $\top$}
	    {
	       $index$ := $\sTop$
	    }
	    \lElse
	    {
	       $index$ := $entry$
	    }
	
	}
	
	\lIf{$index$ = $\top$}
	{
	   \Return $\stack[\sTop]$
	}
	\lElse
	{
	   \Return $\stack[index]$
	}
	

	\label{lin:local-get|full|entry:end}
}
%%
\BlankLine
%%
\tcp{initializes the traversal stack} 
\DontPrintSemicolon
\InitializeTraversalRecord( $\traversalRecord$, $\type$ )\;
\PrintSemicolon
\label{lin:local-initialize|traversal|record:begin}
\Begin
{
%%
  \tcp{initialize the stack using sentinel nodes}
	\tcp{sentinel nodes are never removed from the stack}
	\tcp{a sentinel node is always a safe starting point for the traversal}
	
	\remove{
		
  \uIf{$\type$ = \TARGETSTACK}
	{
		
	   %% $\traversalRecord \rarrow \stack[0]$ := $\ang{\sNodeOne, -1}$\;
     %% $\traversalRecord \rarrow \stack[1]$ := $\ang{\sNodeTwo, 0 }$\;
	   %% $\traversalRecord \rarrow \sTop$ := 1\;
	   %% $\traversalRecord \rarrow \anchor$ := 0\;
	   \tcp{initialize the stack using sentinel nodes}
	   \tcp{sentinel nodes are never removed from the stack}
	   \tcp{a sentinel node is always a safe starting point for the traversal}
	}
	\lElse
	{
	   $\traversalRecord \rarrow \sTop$ := -1
	}
	
	}
	
	\OptReturn
	\label{lin:local-initialize|traversal|record:end}

%%
}
\end{algorithm}


\begin{comment}


%%%%%%%%%%%%%%%%%%%%%%%%%%%%%%%%%%%%%%%%%%%%%%%%%%%%%%%%%%%%%%%%%%%%%%%%%%%%%%%%%%%%%

%% Seek for search operation

%%%%%%%%%%%%%%%%%%%%%%%%%%%%%%%%%%%%%%%%%%%%%%%%%%%%%%%%%%%%%%%%%%%%%%%%%%%%%%%%%%%%%




\begin{algorithm}[tb]
\caption{Seek Function for Target Key (Search Operation)} 
\label{algo:local-seek:search}
%%
\DefineKeyWords
%%
\tcp{Traverses the tree starting from the root until either the key is found or a null pointer is encountered}
%% \tcp{also populates the traversal stack}
\DontPrintSemicolon
\Boolean \TraverseTree( $\opRecord$, $\seekRecord$ )\;
\PrintSemicolon
\label{lin:local-traverse|tree:begin}
\Begin
{
%%
   $\traversalRecord$ := $\opRecord \rarrow \targetStack$\;
	
	 \tcp{initialize the stack and the variables used in the traversal}
	 \InitializeTraversalRecord( $\traversalRecord$, \TARGETSTACK{} )\;
	 \label{lin:local-traverse|tree:initialize}
	 %% \tcp{initialize the variables used in the traversal}
   $\cNode$ := \GetTop( $\traversalRecord$ )\;
	 \label{lin:local-traverse|tree:start}
	 \BlankLine
	 \tcp{traverse the tree (starting from $\cNode$)}
	 \While{\True}
	 {
			\label{lin:local-traverse|tree:while:begin}
	    $\cKey$ := \GetKey( $\cNode$ )\;
			\label{lin:local-traverse|tree:while:first}
		  $\which$ := $\opRecord \rarrow \key < \cKey$ ? \LEFT{} : \RIGHT{}\;
			\label{lin:local-traverse|tree:select}
		  \tcp{read the next address to de-reference}
		  %% $\ang{ \ast, \ast, \nFlag, \address}$ := $\cNode \rarrow \child[\which]$\;  
			$\reference$ := \GetChild( $\cNode$, $\which$ )\;
			
			\BlankLine
			
		  \lIf{$\opRecord \rarrow \key$ = $\cKey$}
			{ 
			   \Return \True 
				 \label{lin:local-traverse|tree:match}
			}	     
			\lIf{\IsNull( $\reference$ )}
			{ 
			   \Return \False
				 \label{lin:local-traverse|tree:null}
			}	  
				
			\BlankLine
			
		  \tcp{traverse the next edge}
		  %% $\pNode$ := $\cNode$;
			$\address$ := \GetAddress( $\reference$ )\;
			$\cNode$ := $\address$\;
			\tcp{push the next node to be visited into the stack}
			\AddToTop( $\traversalRecord$, $\address$, $\which$ )\;
			\label{lin:local-traverse|tree:stack}
			\label{lin:local-traverse|tree:while:end}
			      
	  }	
		
		\OptReturn[\False]
		\label{lin:local-traverse|tree:end}
%%
}
%%
\BlankLine
%%
\tcp{Checks if the key being searched for has moved up in the path}
\DontPrintSemicolon
\Boolean \ExamineStack( $\opRecord$, $\seekRecord$ )\;
\PrintSemicolon
\label{lin:local-examine|stack:begin}
\Begin
{
%%
  
   $\result$ := \False\;
	 $\traversalRecord$ := $\opRecord \rarrow \targetStack$\;
	 
	 \BlankLine
	
   \tcp{start with the \myanchor{} closest to the topmost node in the stack}
	 $\curly{ \ast, \ast, \critical }$ := \GetFullEntry( $\traversalRecord$, $\top$ )\;
	 \label{lin:local-examine|stack:start}	
			
	 \BlankLine
			
			
	 \While{\True}
	 {
	    \label{lin:local-examine|stack:while:begin}
	    \tcp{retrieve the node and its closest \myanchor{} node from the stack}
	    $\curly{ \node, \ast, \anchor }$ := \GetFullEntry( $\traversalRecord$, $\critical$ )\;
			\tcp{read the attributes of the node}					
		  $\nMarked$ := \IsMarked( $\node$ )\; 
			$\nKey$ := \GetKey( $\node$ )\;
					
			\uIf{$\opRecord \rarrow \key$ = $\nKey$}
			{  
			   \label{lin:local-examine|stack:while:found:begin}
			   \tcp{the key stored in the node matches the one being searched for}
			   $\result$ := \True\;
				 \Break\;
				 \label{lin:local-examine|stack:while:found:end}
			} \uElseIf{($\opRecord \rarrow \key$ $<$ $\nKey$) \LOr \LNot($\nMarked$)}
			{
			   \label{lin:local-examine|stack:while:not|found:begin}
			   \tcp{the target key did not exist continuously in the tree}
			   \Break\;
				 \label{lin:local-examine|stack:while:not|found:end}
			} \Else(\tcp*[h]{examine the preceding \myanchor{} node})
			{			
			   \label{lin:local-examine|stack:while:continue:begin}
			   %% \tcp{examine the preceding \myanchor{} node}
			   $\critical$ := $\anchor$\;
				 \label{lin:local-examine|stack:while:continue:end}
			}
		  \label{lin:local-examine|stack:while:end} 
   }
	
	 %% \BlankLine
	 
	 %% \tcp{return the outcome}
	 %% \PopulateSeekRecord( $\seekRecord$, $\traversalRecord$ )\;
	 \Return $\result$\;
	 \label{lin:local-examine|stack:end}	
%%
}
%%
\BlankLine
%%
\tcp{Looks for a given key in the tree (invoked by a search operation)}
\DontPrintSemicolon
\Boolean \SeekForSearch( $\opRecord$, $\seekRecord$ )\;
\PrintSemicolon
\label{lin:local-seek|search:begin}
\Begin
{
%%
   
   
   \tcp{traverse the tree from top to down}
	 
	 $\result$ := \TraverseTree( $\opRecord$, $\seekRecord$ )\;
	 \label{lin:local-seek|search:traverse|tree}	
	 \If{\LNot($\result$)}	
	 {
	     \tcp{check if the key has moved up in the path}
	     $\result$ := \ExamineStack( $\opRecord$, $\seekRecord$ )\;
			 \label{lin:local-seek|search:examine|stack}
	 }
	
	 \tcp{return the outcome}
	 %% \tcp{return the outcome}
	 \PopulateSeekRecord( $\seekRecord$, $\traversalRecord$ )\;
	 \Return $\result$\;
   \label{lin:local-seek|search:end}
%%
}
%%
\end{algorithm}

\end{comment}

%%%%%%%%%%%%%%%%%%%%%%%%%%%%%%%%%%%%%%%%%%%%%%%%%%%%%%%%%%%%%%%%%%%%%%%%%%%%%%%%%%%%%

%% Functions used to achieve local recovery

%%%%%%%%%%%%%%%%%%%%%%%%%%%%%%%%%%%%%%%%%%%%%%%%%%%%%%%%%%%%%%%%%%%%%%%%%%%%%%%%%%%%%

\begin{algorithm}[tb]
\caption{Functions used to Achieve Local Recovery} 
\label{algo:local-local:recovery}
%%
\DefineKeyWords
%%
\tcp{Determines if the last node in the path is \mysafe}
\DontPrintSemicolon
\Boolean \FindAdmissible( $\opRecord$, $\traversalRecord$ )\;
\PrintSemicolon
\label{lin:local-test|safety:begin}
\Begin
{
%% 
   
   \tcp{examine the \myanchor{} nodes in the path one-by-one starting from the closest one}
	
	 $\curly{ \ast, \ast, \critical }$ := \GetFullEntry( $\traversalRecord$, $\top$ )\;
	 \While{\True}
	 {
	    \label{lin:local-test|safety:while:begin}
	    \tcp{retrieve the node and its \myanchor{} from the stack}
	    $\curly{ \node, \ast, \anchor }$ := \GetFullEntry( $\traversalRecord$, $\critical$ )\;
		  \tcp{read the attributes of the node}
			$\nMarked$ := \IsMarked( $\node$ )\; 
			$\nKey$ := \GetKey( $\node$ )\;
			
			\BlankLine
			
			\uIf(\tcp*[f]{the \myanchor{} node is still \myconsistent{}}){$\opRecord \rarrow \key$ $>$ $\nKey$}
		  {
			   \label{lin:local-test|safety:while:consistent:begin}
				
				 \uIf(\tcp*[f]{the last node is \mysafe{}}){\LNot($\nMarked$)}
				 {  
				    \RememberCritical( $\traversalRecord$, $\critical$ )\;
				    \Return \True\;
				 }
				 \Else(\tcp*[f]{the \myanchor{} node is \myinadmissible{}. discard the suffix and return})
				 {
				    %\uIf{\IsGreen(~)} 
						%{
						%   \tcp{examine the previous \myanchor{} node}
				    %   $\critical$ := $\anchor$\;
						%}
						%\Else
						%{
						   \RemoveUntilCritical( $\traversalRecord$, $\critical$ )\;
							 \Return \False\;
						%}
						
				 }
				 \label{lin:local-test|safety:while:consistent:end}
				
			}
			\Else(\tcp*[f]{the \myanchor{} node is \mynonconsistent{}. discard the suffix and return})
			{
			    \label{lin:local-test|safety:while:nonconsistent:begin}
			    \RemoveUntilCritical( $\traversalRecord$, $\critical$ )\;
					\Return \False\;
					\label{lin:local-test|safety:while:nonconsistent:end}
			}
				
			
			\label{lin:local-test|safety:while:end} 
	 }
	 
	 \OptReturn[\False]
	 \label{lin:local-test|safety:end}
	  
%%
}
%%
\BlankLine
%%
\tcp{Find a suitable node in the path from where to restart}
\DontPrintSemicolon
\Boolean \FindStartPoint( $\opRecord$, $\traversalRecord$ )\;
\PrintSemicolon
\label{lin:local-find|start|point:begin}
\Begin
{
%%
   
	 \While{\True}
   {
	    \label{lin:local-find|start|point:while:begin}
	    \tcp{backtrack until an unmarked node}
			$\cNode$ := \GetTop( $\traversalRecord$ )\;
			\label{lin:local-find|start|point:while:backtrack:begin}
    		
	    \While{\IsMarked( $\cNode$ )}
			{
			  
			   \RemoveFromTop( $\traversalRecord$ )\;
				 $\cNode$ := \GetTop( $\traversalRecord$ )\;
         
			}
			
			\BlankLine

      \tcp{check if the algorithm needs a clean parent node}
			\If{\NeedCleanParentNode( $\opRecord$, $\cNode$ )}
			{ 
			   \label{lin:local-find|start|point:while:clean:begin}
				 $\pNode$ := \GetSecondToTop( $\traversalRecord$ )\; 
				 \If{\LNot(\IsClean( $\pNode$ ))}
				 {
				    \tcp{need to backtrack even further}
						
											
				    \RemoveFromTop( $\traversalRecord$ )\;
						\Continue\;
						\label{lin:local-find|start|point:while:clean:end}
						\label{lin:local-find|start|point:while:backtrack:end}
				 }
			}
			
			
			\BlankLine
			
			\tcp{check if the last node in the path is a suitable restart point}
		
	    $\result$ := \FindAdmissible( $\opRecord$, $\traversalRecord$ )\;
	    \label{lin:local-find|start|point:while:test|safety}
			
			\lIf{$\result$}
			{
			   \Return \True
			}
			   
			\tcp{the path has been truncated and its last node is \myinadmissible{}}
		  $\status$ := \ExamineTop( $\opRecord$ )\;
			
			\lIf{$\status$ $\in$ \{ \STOPFOUND{}, \STOPNOTFOUND{} \}}
			{
			   \Return \False
			}
			
			\label{lin:local-find|start|point:while:end}
	 }
	
	 \OptReturn[\False]
	 \label{lin:local-find|start|point:end}
%%
}
\end{algorithm}
\begin{algorithm}[tb]
\caption{Functions used to Achieve Local Recovery  (Continued)} 
\label{algo:local-local:recovery|2}
\DefineKeyWords
\tcp{Invoked after a node in the path was deemed to be not \mysafe{}. In this case, the path would have been truncated such that its last node is \myinadmissible{}}
\tcp{Returns one of the following three values: \STOPFOUND{}, \STOPNOTFOUND{} or \DONOTKNOW{}}
\DontPrintSemicolon
\Enum Outcome \ExamineTop( $\opRecord$ )\;
\PrintSemicolon
\label{lin:local-examine|top:begin}
\Begin
{
   $\traversalRecord$ := $\opRecord \rarrow \targetStack$\;
	 \tcp{retrieve the topmost node from the stack which must be \myinadmissible{}}
	 $\cNode$ := \GetTop( $\traversalRecord$ )\;
	 $\cKey$ := \GetKey( $\cNode$ )\;
	 \BlankLine
	 \uIf{$\opRecord \rarrow \key$ $>$ $\cKey$} 
	 {   
	     \tcp{the last node is \myconsistent{}}
	     %% \tcp{can only happen for a non-green algorithm}
			 \Return \DONOTKNOW\;
			 \label{lin:local-examine|top:continue|1}
	 } 
	 \uElseIf{$\opRecord \rarrow \key$ $<$ $\cKey$}
	 {  
	    \label{lin:local-examine|top:inconsistent}
	    \tcp{the last node is \myinconsistent{}}
	    \lIf{$\opRecord \rarrow \type$ = \INSERT}
			{
			   \Return \DONOTKNOW
				 \label{lin:local-examine|top:continue|2}
			}
			\lElse
			{
			   \Return \STOPNOTFOUND
			}
	
	 }
	 \Else(\tcp*[h]{the last node contains the matching key})
	 {
	    \label{lin:local-examine|top:matching}
	    
	    \Return \STOPFOUND\;
	 }
	
	 \OptReturn[\DONOTKNOW]
	 \label{lin:local-examine|top:end}
}
%%
\end{algorithm}






%%%%%%%%%%%%%%%%%%%%%%%%%%%%%%%%%%%%%%%%%%%%%%%%%%%%%%%%%%%%%%%%%%%%%%%%%%%%%%%%%%%%%

%% Seek for target key

%%%%%%%%%%%%%%%%%%%%%%%%%%%%%%%%%%%%%%%%%%%%%%%%%%%%%%%%%%%%%%%%%%%%%%%%%%%%%%%%%%%%%




\begin{algorithm}[tb]
\caption{Seek Function for Target Key} 
\label{algo:local-seek}
%%
\DefineKeyWords
%%
\tcp{Looks for a given key in the tree (invoked by every operation)}
\DontPrintSemicolon
\Boolean \SeekForTarget( $\opRecord$, $\seekRecord$ )\;
\PrintSemicolon
\label{lin:local-seek:begin}
\Begin
{
%%
   
   $\traversalRecord$ := $\opRecord \rarrow \targetStack$\;
	 $\status$ := \DONOTKNOW\;
	 
	 \BlankLine
	
	 
	 \While{$\status$ = \DONOTKNOW}
	 { 
	    \label{lin:local-seek:while:begin}
	    \tcp{find a suitable restart point in the path}
      $\result$ := \FindStartPoint( $\opRecord$, $\traversalRecord$ )\;
	    \label{lin:local-seek:while:find|start|point} 
			
			\If(\tcp*[f]{examine the last node in the path}){\LNot($\result$)} 
			{
			   $\status$ := \ExamineTop( $\opRecord$ )\;
				 \Continue\;
				    
			}
			
	    \tcp{traverse the tree starting from the topmost node in the stack}
      $\cNode$ := \GetTop( $\traversalRecord$ )\;
			\label{lin:local-seek:while:traversal:begin}
	    \While{\True}
	    {
			  \label{lin:local-seek:while:traversal:first}
	       $\cKey$ := \GetKey( $\cNode$ )\;
		     $\which$ := $\opRecord \rarrow \key < \cKey$ ? \LEFT{} : \RIGHT{}\;
				 \label{lin:local-seek:while:traversal:select}
		    
			   \tcp{read the next address to de-reference}
		     $\reference$ := \GetChild( $\cNode$, $\which$ )\;
			   		
		     \BlankLine
				
		     \If{($\opRecord \rarrow \key$ = $\cKey$) \LOr \IsNull( $\reference$ )}
			   {
				    \label{lin:local-seek:while:traversal:stop:begin}
						
						\lIf{\IsNull( $\reference$ )}
						{
						   $\cKey$ := \GetKey( $\cNode$ )
						}
						
			      \tcp{either stop or backtrack \& restart }
												
						\uIf{$\opRecord \rarrow \key$ $\not=$ $\cKey$}
				    {
						   \tcp{if an insert operation, store the injection point}
							 \If{$\opRecord \rarrow \type$ = \INSERT}
							 {
							    $\opRecord \rarrow \injectionPoint$ :=  \GetAddress( $\reference$ )\;
									\label{lin:local-seek:while:traversal:store|injection}
							 }
							
							 \tcp{test if the terminal node is \mysafe} 		
						   $\result$ := \FindAdmissible( $\opRecord$, $\traversalRecord$ )\;
							 \label{lin:local-seek:while:traversal:test|safety}
							
					     \uIf(\tcp*[f]{terminal node is a \mysafe{} node}){$\result$}
						   {  
						      $\status$ := \STOPNOTFOUND\;
									\label{lin:local-seek:while:traversal:safe}
						   }
						   \Else(\tcp*[f]{examine the last node in the path})
							 {
							    $\status$ := \ExamineTop( $\opRecord$ )\;
									\label{lin:local-seek:while:traversal:not|safe}
						   }
						} 
						\Else(\tcp*[f]{terminal node contains the matching key})
						{
						   $\status$ := \STOPFOUND\;
							 \label{lin:local-seek:while:traversal:match}
						}
											
					  
						\Break; \tcp*[f]{terminate the current traversal}
						\label{lin:local-seek:while:traversal:stop:end}
						 
			   }
			  				
				 \BlankLine
					
			   $\address$ := \GetAddress( $\reference$ ); \tcp*[f]{traverse the next edge}
				
				 
				 \If{$\opRecord \rarrow \type$ $\in$ \{ \INSERT{}, \DELETE{} \}}
				 {
				    
				    $\restart$ := \Move( $\cNode$, $\address$, $\which$ )\;
				    \label{lin:local-seek:while:traversal:move}
				    \If(\tcp*[f]{the algorithm wants to restart the traversal}){$\restart$}
				    {
						   \Break\;
				    }
				 }
				 
				
				 
				
				 \AddToTop( $\traversalRecord$, $\address$, $\which$ ); \tcp*[f]{push the node visited into the stack}
			   \label{lin:local-seek:while:traversal:push}
			   \label{lin:local-seek:while:traversal:end}
	    } %% inner while loop
			
			
			
	    \label{lin:local-seek:while:end}	
	 }	 %% outer while loop
		
	 \BlankLine
		
	 \tcp{return the outcome}

	 \PopulateSeekRecord( $\seekRecord$, $\opRecord$ )\;
	 \label{lin:local-seek:populate}
	 \Return ($\status$ = \STOPFOUND \  ? \  \True : \False)\;
   \label{lin:local-seek:end}
%%
}
\end{algorithm}





\remove{



%%%%%%%%%%%%%%%%%%%%%%%%%%%%%%%%%%%%%%%%%%%%%%%%%%%%%%%%%%%%%%%%%%%%%%%%%%%%%%%%%%%%%

%% Seek for successor key

%%%%%%%%%%%%%%%%%%%%%%%%%%%%%%%%%%%%%%%%%%%%%%%%%%%%%%%%%%%%%%%%%%%%%%%%%%%%%%%%%%%%%



\begin{algorithm}[tb]
\caption{Seek Function for Successor Key} 
\label{algo:local-seek:successor}
%%
\DefineKeyWords
%%
\tcp{Looks for the next largest key with respect to a given key (invoked by a complex delete operation)}
\DontPrintSemicolon
\Boolean \SeekForSuccessor( $\opRecord$, $\seekRecord$ )\;
\PrintSemicolon
\label{lin:local-seek|successor:begin}
\Begin
{
%%
   
   \tcp{the stack used in locating the successor key is initialized before this function is invoked}	
	 $\traversalRecord$ := $\opRecord \rarrow \successorStack$\;
	 \While{\True}
	 {
	
	    \label{lin:local-seek|successor:while:begin}
		  \tcp{backtrack until either an unmarked node or the stack becomes empty}
			
	    
			\While{(\Size( $\traversalRecord$ ) $>$ 1)}      
			{
			    \label{lin:local-seek|successor:while:backtrack:begin}
			    $\cNode$ := \GetTop( $\traversalRecord$ )\;
					\lIf{\LNot(\IsMarked( $\cNode$ ))}
					{
					   \Break
					} 
					\lElse
					{
					   \RemoveFromTop( $\traversalRecord$ )
					}
					\label{lin:local-seek|successor:while:backtrack:end}
			}
	
	    \BlankLine
			
			\tcp{backtrack further if a clean parent is needed but the parent is not clean}
			\If{(\Size( $\traversalRecord$ ) $\geq$ 2)}
			{
			   \label{lin:local-seek|successor:while:clean:begin}
			   \If{\NeedCleanParentNode( $\opRecord$, $\cNode$ )}
		     {
				    
			      \tcp{the parent node should be a clean node}
				    $\pNode$ := \GetSecondToTop( $\traversalRecord$ )\;
				    \If{\LNot(\IsClean( $\pNode$ ))}
				    {
				       \RemoveFromTop( $\traversalRecord$ )\;
					     \Continue\;
							 \label{lin:local-seek|successor:while:clean:end}
				    }
			   }
				
			}
			
			\BlankLine
			
	    \tcp{check if the successor key is still needed}
	    $\reference$ := \NeedSuccessorKey( $\opRecord$ )\;
			\label{lin:local-seek|successor:while:need|successor}
			\If{\IsNull( $\reference$ )}{ 
			   \tcp{successor key no longer required}
			   \Return \false\;
			}
			
			\BlankLine
			
			$\cNode$ := \GetTop( $\traversalRecord$ )\;
			\uIf{(\Size( $\traversalRecord$ ) = 1)}
			{
			   \label{lin:local-seek|successor:while:traversal:if:begin}
			   %% $\address$ := \GetAddress( $\reference$ )\;
				 \tcp{visit the node pointed to by the reference returned by \NeedSuccessorKey{} function}
			   $\which$ := \RIGHT\;
				 \label{lin:local-seek|successor:while:traversal:if:end}
			}
			\Else
			{
			   \label{lin:local-seek|successor:while:traversal:else:begin}
			   \tcp{follow the left child node of the top node, if it exists}
		 	   $\reference$ := \GetChild( $\cNode$, \LEFT{} )\;
			   %% \lIf{\IsNull( $\reference$ )}{ \Break }
				 %% $\address$ := \GetAddress( $\reference$ )\;
			   $\which$ := \LEFT\;
				 \label{lin:local-seek|successor:while:traversal:else:end}
			}
			
			\Repeat{\True}
	    {
			   \label{lin:local-seek|successor:while:traversal:begin}
			   \tcp{stop if reference is null}
				 \lIf{\IsNull( $\reference$ )}{ \Break }
				
				 \tcp{obtain the address of the node}
				 $\address$ := \GetAddress( $\reference$ )\;	
				
				 \BlankLine
				
				 \tcp{traverse the edge}
				 $\restart$ := \Move( $\cNode$, $\address$, $\which$ )\;
				 \label{lin:local-seek|successor:while:traversal:move}
			   \If{$\restart$}
			   {
			      \tcp{the algorithm wants to restart the traversal}
				    \Break\;
						\label{lin:local-seek|successor:while:traversal:restart}
			   }  
				 
			   \tcp{push the node visited into the stack}
				 \AddToTop( $\traversalRecord$, $\address$, $\which$ )\;
				 \label{lin:local-seek|successor:while:traversal:stack}
				 \label{lin:local-seek|successor:while:traversal:advance:begin}
			   $\cNode$ := $\address$\;
				 \tcp{determine the next node to be visited}
			   $\reference$ := \GetChild( $\cNode$, \LEFT{} )\;
			   $\which$ := \LEFT{}\;
				 \label{lin:local-seek|successor:while:traversal:advance:end}
				 \label{lin:local-seek|successor:while:traversal:end}
			}
			\label{lin:local-seek|successor:while:end}
	 }
	
	
	 \BlankLine
	
	 \tcp{return the outcome}
	 \PopulateSeekRecord( $\seekRecord$, $\opRecord$ )\;
	 \Return \True;
	 \label{lin:local-seek|successor:end}
%%
}
%%
\end{algorithm}

}


\end{limitscope}

A pseudo-code of the local recovery algorithm is given in \pseudosref{local-data|structures}{local-seek:modify}. The pseudo-code only shows the seek phase of an algorithm and not its execution phase since the execution phase is algorithm-specific. We have also moved the pseudo-code for local recovery when looking for a successor key to the appendix due to lack of space.


The local recovery algorithm assumes that the original algorithm supports the following functions:
\begin{enumerate*}[label=(\alph*)]
%%
\item \GetKey(~), \IsMarked(~) and \GetChild(~) returns the various attributes of a tree node,
\item \IsNull(~) returns true if a reference is null and false otherwise,
\item \GetAddress(~) returns the node address stored in a reference, if non-null,
\item \Move(~) enables the original algorithm to move along an edge, which may invoke helping and restarting of the traversal as in~\cite{HowJon:2012:SPAA},
\item \NeedCleanParentNode(~) returns true if the operation needs the parent node to be clean and have no operation in progress (needed for a delete operation since it needs to modify a child pointer at the parent node), and
\item \PopulateSeekRecord(~) copies the relevant information from the traversal state required by the algorithm into a seek record.
\end{enumerate*}

\begin{comment}
\NeedSuccessorKey(~) evaluates if the successor key is still needed for the target key and returns a reference which is null if no successor key is needed and an address of the terminal node's right child otherwise
\end{comment}

\Pseudoref{local-data|structures} shows the data structures used by the local recovery algorithm. Note that all the data structures shown in \Pseudoref{local-data|structures} are \emph{local} to a process not shared among processes. A process uses three main data structures, namely \TraversalRecord{}, \OpRecord{} and \SeekRecord{}. A \TraversalRecord{} (\linesref{local-traversal|record:begin}{local-traversal|record:end}) is essentially a stack used to store the nodes visited during tree traversal when looking for a key (target or successor). Note that the traversal stack satisfies the last-in-first-out (LIFO) semantics but our algorithm sometimes uses it in a non-traditional way by accessing entries in the middle of the stack. One way to implement such an ``augmented'' stack is to use an auto-resizing vector provided as part of C++ STL library or Java package. Each entry in a traversal stack (\linesref{local-stack|entry:begin}{local-stack|entry:end}) stores the address of the node, the location of its closest \myanchor{} node (within the stack's vector) and whether the node is a left or right child of its parent. An \OpRecord{} (\linesref{local-op|record:begin}{local-op|record:end}) stores information about the operation such as type and key as well two stacks: one used when looking for the target key (all operations) and one used when looking for the successor key (only complex delete operations). Finally, a \SeekRecord{} (\linesref{local-seek|record:begin}{local-seek|record:end}) is used to return the outcome of a tree traversal to the original algorithm. Its fields are algorithm-specific. For example, for \CASTLE{}, \SeekRecord{} contains three fields: 
\begin{enumerate*}[label=(\alph*)]
\item two addresses, namely those of the target node and its parent, and
\item the contents of the injection point where an insert operation needs to attach the new node. 
\end{enumerate*}

\Pseudoref{local-stack|functions} shows the functions used to manipulate a traversal stack. The function \Size{} (\linesref{local-size:begin}{local-size:end}) returns the number of entries in the stack. The functions \GetTop{} (\linesref{local-get|top:begin}{local-get|top:end}) and \GetSecondToTop{} (\linesref{local-get|second|to|top:begin}{local-get|second|to|top:end}) return the address of the node stored in the topmost entry and the entry below it, respectively. The function \AddToTop{} (\linesref{local-add|to|top:begin}{local-add|to|top:end}) adds an entry to the top of the stack while \RemoveFromTop{} (\linesref{local-remove|from|top:begin}{local-remove|from|top:end}) removes an entry from the top of the stack. The function \RemoveUntilCritical{} (\linesref{local-remove|until|critical:begin}{local-remove|until|critical:end}) removes the entries from the top of the stack until a given point. The function \RememberCritical{} (\linesref{local-remember|critical:begin}{local-remember|critical:end}) updates the \myanchor{} field of the \myanchor{} node of the topmost entry in the stack. The function \GetFullEntry{} (\linesref{local-get|full|entry:begin}{local-get|full|entry:end} returns all the three fields of a given entry in the stack (may not be the topmost entry). The function \InitializeTraversalRecord{} (\linesref{local-initialize|traversal|record:begin}{local-initialize|traversal|record:end}) initializes a traversal stack. The stack for target key %is initialized using sentinel nodes while the stack for successor key is initialized as empty.

\Pseudosref[ \& ] {local-seek:search}{local-seek:search:2} shows the functions used to find the target key by a search operation. The function \SeekForSearch{} (\linesref{local-seek|search:begin}{local-seek|search:end}) first traverses the tree starting from the root node (\lineref{local-seek|search:traverse|tree}). If the traversal fails to locate the key, then the key may have moved up the tree. To address this possibility, the function examines the traversal stack to determine whether or not that is the case (\lineref{local-seek|search:examine|stack}). The function \TraverseTree{} (\linesref{local-traverse|tree:begin}{local-traverse|tree:end}) first initializes the traversal stack (\lineref{local-traverse|tree:initialize}) and then, starting from the topmost node in the stack (\lineref{local-traverse|tree:start}), follows either the left or the right child pointer (\lineref{local-traverse|tree:select}) until it either finds the key (\lineref{local-traverse|tree:match}) or encounters a null pointer (\lineref{local-traverse|tree:null}). It also populates the traversal stack as it moves (\lineref{local-traverse|tree:stack}). The function \ExamineStack (\linesref{local-examine|stack:begin}{local-examine|stack:end}) examines the \myanchor{} nodes stored in the stack in the reverse order in which they were visited, starting from the \myanchor{} node closest to the topmost node in the traversal stack (\lineref{local-examine|stack:start}). If the \myanchor{} node's key matches the target key, then the function returns true (\linesref{local-examine|stack:while:found:begin}{local-examine|stack:while:found:end}). If the \myanchor{} node is no longer \myconsistent{} or is unmarked, then the function returns false (\linesref{local-examine|stack:while:not|found:begin}{local-examine|stack:while:not|found:end}). Otherwise, the function backtracks and examines the preceding \myanchor{} node in the stack (\linesref{local-examine|stack:while:continue:begin}{local-examine|stack:while:continue:end}).

\Pseudosref{local-local:recovery}{local-local:recovery:2} $\&$ ~\ref{algo:local-seek:modify} show the functions used to find the target key by a modify (insert or delete) operation. The function \SeekForModify{} (\linesref{local-seek|modify:begin}{local-seek|modify:end}) first backtracks to a \mysafe{} node in the stack (\lineref{local-seek|modify:while:find|start|point}). Initially, the starting point is typically a sentinel node which is a \mysafe{} node. The function then traverses the tree from top to down by following either the left or the right child pointer (\lineref{local-seek|modify:while:traversal:select}) until it either finds the key or encounters a null pointer (\linesref{local-seek|modify:while:traversal:stop:begin}{local-seek|modify:while:traversal:stop:end}). In case the terminal node's key is greater than the target key, the function checks whether the path stored in the traversal stack is still valid (\lineref{local-seek|modify:while:traversal:find|admissible}). If not, the traversal is restarted. As the traversal moves down the tree, the function also populates the traversal stack (\linesref{local-seek|modify:while:traversal:move:begin}{local-seek|modify:while:traversal:move:end}). The function \FindAdmissible{} (\linesref{local-find|admissible:begin}{local-find|admissible:end}) checks whether or not the path stored in the stack is still valid. To that end, it examines  the \myanchor{} nodes in the stack in the reverse order in which they were visited, starting from the \myanchor{} node closest to the topmost node in the traversal stack. There are three possible cases. First, the \myanchor{} node is still consistent (\linesref{local-find|admissible:while:consistent:begin}{local-find|admissible:while:consistent:end}). In this case, the path is deemed to be valid if the \myanchor{} node is unmarked; otherwise, the function moves to the preceding \myanchor{} node. Second, the \myanchor{} node is no longer consistent (\linesref{local-find|admissible:while:not|consistent:begin}{local-find|admissible:while:not|consistent:end}). In this case, the path is deemed to be invalid. However, if the operation is a delete operation, then it can be deduced that the key did not exist in the tree  continuously and the function returns indicating that the key was not found (thereby causing the operation to terminate). Finally, the \myanchor{} node's key matches the target key (\linesref{local-find|admissible:while:match:begin}{local-find|admissible:while:match:end}). In this case, if the \myanchor{} node is marked and the operation is a delete operation, then the path is deemed to be invalid (and further backtracking is required). This is because the key may be in the process of moving up the tree. Otherwise, the function returns indicating that the key was found. The function \FindStartPoint{} (\linesref{local-find|start|point:begin}{local-find|start|point:end}) finds a \mysafe{} node on the path stored in the stack from which the operation can restart its traversal. To that end, it backtracks to an unmarked node with a clean parent if required (\linesref{local-find|start|point:while:backtrack:begin}{local-find|start|point:while:backtrack:end}). It then checks whether or not the remaining path in the stack is still valid (\lineref{local-find|start|point:while:find|admissible}). If not, it repeats the above-mentioned steps.

\begin{limitscope}

%% To limit the scope of the macros defined below

%% macros for pseudocode

\newcommand{\child}{child}
\newcommand{\node}{node}
\newcommand{\parent}{parent}

\newcommand{\mainSeekRecord}{seekTargetKey}
\newcommand{\successorSeekRecord}{seekSuccessorKey}


\newcommand{\targetStack}{targetStack}
\newcommand{\successorStack}{successorStack}


\newcommand{\successorStackInUse}{successorStackInUse}
\newcommand{\targetNode}{targetNode}




\newcommand{\key}{key}

\newcommand{\done}{done}
\newcommand{\result}{result}
\newcommand{\status}{status}
\newcommand{\restart}{restart}





\newcommand{\cKey}{key}
\newcommand{\nKey}{key}
\newcommand{\cNode}{current}
\newcommand{\pNode}{parent}
\newcommand{\nMarked}{marked}



\newcommand{\which}{which}
\newcommand{\address}{address}

\newcommand{\anchor}{anchor}

\newcommand{\stack}{stack}
\newcommand{\sTop}{top}
\newcommand{\sBottom}{bottom}
\newcommand{\current}{current}

\remove{
\newcommand{\traversalRecord}{state}
\newcommand{\TraversalRecord}{State}
\newcommand{\opRecord}{opRecord}
\newcommand{\OpRecord}{OpRecord}
\newcommand{\seekRecord}{seekRecord}
\newcommand{\SeekRecord}{SeekRecord}
}

\newcommand{\admissible}{admissible}
\newcommand{\critical}{critical}
\newcommand{\reference}{re\!f\!erence}

%% \newcommand{\OptReturn}[1][]{\Return #1\;}
\newcommand{\OptReturn}[1][]{}

\newcommand{\injectionPoint}{injectionPoint}



\newcommand{\Search}{\textsc{Search}}
\newcommand{\Insert}{\textsc{Insert}}
\newcommand{\Delete}{\textsc{Delete}}
\newcommand{\Seek}{\textsc{Seek}}

\newcommand{\Inject}{\textsc{Inject}}


%%
\newcommand{\WFSeekForSearchBOSize}{\textsc{WFSeekForSearchBasedOnSize}}
\newcommand{\WFSeekForSearchBOHeight}{\textsc{WFSeekForSearchBasedOnHeight}}
%%
\newcommand{\WFTraverseTreeBOCount}{\textsc{TraverseBasedOnCount}}
\newcommand{\WFTraverseTreeBOTimeStamp}{\textsc{TraverseBasedOnTimeStamp}}
%%
\remove{
\newcommand{\SeekForSuccessor}{\textsc{SeekForSuccessor}}
\newcommand{\NeedSuccessorKey}{\textsc{NeedSuccessorKey}}
\newcommand{\GetChild}{\textsc{GetChild}}
\newcommand{\Move}{\textsc{Move}}
\newcommand{\GetAddress}{\textsc{GetAddress}}
\newcommand{\IsNull}{\textsc{IsNull}}
\newcommand{\PopulateSeekRecord}{\textsc{PopulateSeekRecord}}
}



\newcommand{\mline}[1]{\DontPrintSemicolon #1 \PrintSemicolon}


\newcommand{\LEFT}{\textsf{LEFT}}
\newcommand{\RIGHT}{\textsf{RIGHT}}


\newcommand{\rarrow}{\!\rightarrow\!}
\newcommand{\type}{type}
\newcommand{\limit}{limit}


\newcommand{\SEARCH}{\textsf{SEARCH}}
\newcommand{\INSERT}{\textsf{INSERT}}
\newcommand{\DELETE}{\textsf{DELETE}}

\newcommand{\STOPFOUND}{\textsf{FOUND}}
\newcommand{\STOPNOTFOUND}{\textsf{NOT\_FOUND}}
\newcommand{\ADMISSIBLE}{\textsf{SAFE}}
\newcommand{\INADMISSIBLE}{\textsf{NOT\_SAFE}}

\newcommand{\TARGETSTACK}{\textsf{TARGET\_STACK}}
\newcommand{\SUCCESSORSTACK}{\textsf{SUCCESSOR\_STACK}}

%%%%%%%%%%%%%%%%%%%%%%%%%%%%%%%%%%%%%%%%%%%%%%%%%%%%%%%%%%%%%%%%%%%%%%%%%%%%%%%%%%%%

\newcommand{\DefineKeyWords}{
%%
\SetKw{Boolean}{boolean}
\SetKw{Integer}{integer}
\SetKw{LAnd}{~and~}
\SetKw{LOr}{~or~}
\SetKw{LNot}{not}
\SetKw{Struct}{struct}
\SetKw{Null}{null}
\SetKw{True}{true}
\SetKw{False}{false}
\SetKw{Break}{break}
\SetKw{Continue}{continue}
\SetKw{Enum}{enum}
\SetKw{Word}{word}
%%
}

%%%%%%%%%%%%%%%%%%%%%%%%%%%%%%%%%%%%%%%%%%%%%%%%%%%%%%%%%%%%%%%%%%%%%%%%%%%%%%%%%%%%%

%% Seek for successor key

%%%%%%%%%%%%%%%%%%%%%%%%%%%%%%%%%%%%%%%%%%%%%%%%%%%%%%%%%%%%%%%%%%%%%%%%%%%%%%%%%%%%%



\begin{algorithm}[tbh]
\caption{Seek Function for Successor Key} 
\label{algo:seek:successor}
%%
\DefineKeyWords
%%
\tcp{Looks for the next largest key with respect to a given key}
\DontPrintSemicolon
\Boolean \SeekForSuccessor( $\opRecord$, $\seekRecord$ )\;
\PrintSemicolon
\label{lin:local-seek|successor:begin}
\Begin
{
%%
   
   \tcp{the stack used in locating the successor key is initialized}
	 $\traversalRecord$ := $\opRecord \rarrow \successorStack$\;
	 \While{\True}
	 {
	
	    \label{lin:local-seek|successor:while:begin}
		  \tcp{backtrack until either an unmarked node or the stack becomes empty}
			
	    
			\While{(\Size( $\traversalRecord$ ) $>$ 1)}      
			{
			    \label{lin:local-seek|successor:while:backtrack:begin}
			    $\cNode$ := \GetTop( $\traversalRecord$ )\;
					\lIf{\LNot(\IsMarked( $\cNode$ ))}
					{
					   \Break
					} 
					\lElse
					{
					   \RemoveFromTop( $\traversalRecord$ )
					}
					\label{lin:local-seek|successor:while:backtrack:end}
			}
	
	    \BlankLine
			
			\tcp{backtrack further if a clean parent is needed but the parent is not clean}
			\If{(\Size( $\traversalRecord$ ) $\geq$ 2)}
			{
			   \label{lin:local-seek|successor:while:clean:begin}
			   \If{\NeedCleanParentNode( $\opRecord$, $\cNode$ )}
		     {
				    
			      \tcp{the parent node should be a clean node}
				    $\pNode$ := \GetSecondToTop( $\traversalRecord$ )\;
				    \If{\LNot(\IsClean( $\pNode$ ))}
				    {
				       \RemoveFromTop( $\traversalRecord$ )\;
					     \Continue\;
							 \label{lin:local-seek|successor:while:clean:end}
				    }
			   }
				
			}
			
			%\BlankLine
			
	    \tcp{check if the successor key is still needed}
	    $\reference$ := \NeedSuccessorKey( $\opRecord$ )\;
			\label{lin:local-seek|successor:while:need|successor}
			\If(\tcp*[f]{successor key no longer required}){\IsNull( $\reference$ )}{ 
			   \Return \false\;
			}
					
			$\cNode$ := \GetTop( $\traversalRecord$ )\;
			\uIf{(\Size( $\traversalRecord$ ) = 1)}
			{
			   \label{lin:local-seek|successor:while:traversal:if:begin}
			   %% $\address$ := \GetAddress( $\reference$ )\;
				 \tcp{visit the node pointed to by the reference returned by \NeedSuccessorKey{} function}
			   $\which$ := \RIGHT\;
				 \label{lin:local-seek|successor:while:traversal:if:end}
			}
			\Else(\tcp*[f]{follow the left child node of the top node, if it exists})
			{
			   \label{lin:local-seek|successor:while:traversal:else:begin}
		 	   $\reference$ := \GetChild( $\cNode$, \LEFT{} )\;
			   %% \lIf{\IsNull( $\reference$ )}{ \Break }
				 %% $\address$ := \GetAddress( $\reference$ )\;
			   $\which$ := \LEFT\;
				 \label{lin:local-seek|successor:while:traversal:else:end}
			}
			%\BlankLine
			\Repeat(\tcp*[f]{stop if reference is null}){\True}
	    {
			   \label{lin:local-seek|successor:while:traversal:begin}
				 \lIf{\IsNull( $\reference$ )}{ \Break }
				
				 \tcp{obtain the address of the node}
				 $\address$ := \GetAddress( $\reference$ )\;	
				
				 \tcp{traverse the edge}
				 $\restart$ := \Move( $\cNode$, $\address$, $\which$ )\;
				 \label{lin:local-seek|successor:while:traversal:move}
			   \If(\tcp*[f]{the algorithm wants to restart the traversal}){$\restart$}
			   {
				    \Break\;
						\label{lin:local-seek|successor:while:traversal:restart}
			   }  
				 
			   \tcp{push the node visited into the stack}
				 \AddToTop( $\traversalRecord$, $\address$, $\which$ )\;
				 \label{lin:local-seek|successor:while:traversal:stack}
				 \label{lin:local-seek|successor:while:traversal:advance:begin}
			   $\cNode$ := $\address$\;
				 \tcp{determine the next node to be visited}
			   $\reference$ := \GetChild( $\cNode$, \LEFT{} )\;
			   $\which$ := \LEFT{}\;
				 \label{lin:local-seek|successor:while:traversal:advance:end}
				 \label{lin:local-seek|successor:while:traversal:end}
			}
			\label{lin:local-seek|successor:while:end}
	 }
	
	 \tcp{return the outcome}
	 \PopulateSeekRecord( $\seekRecord$, $\opRecord$ )\;
	 \Return \True;
	 \label{lin:local-seek|successor:end}
%%
}
%%
\end{algorithm}
\end{limitscope}

\Pseudoref{seek:successor} shows the function \SeekForSuccessor{} used to locate the successor key by a complex delete operation (\linesref{local-seek|successor:begin}{local-seek|successor:end}). The function first backtracks to an unmarked node with a clean parent if required (\linesref{local-seek|successor:while:backtrack:begin}{local-seek|successor:while:clean:end}). It then checks whether or not the successor key is still needed by invoking \NeedSuccessorKey{} function (\lineref{local-seek|successor:while:need|successor}). The function \NeedSuccessorKey{} returns a reference, which is null if the successor key is no longer needed and contains the address of the target node's right child otherwise. This address is used as a traversal point if the stack only contains a single entry (the node whose key needs to be replaced). If the successor key is still needed, then the function repeatedly follows the left child pointer until it encounters a null pointer (\linesref{local-seek|successor:while:traversal:begin}{local-seek|successor:while:traversal:end}). While moving down the tree, the function also populates the traversal stack (\lineref{local-seek|successor:while:traversal:stack}).

\subsection{Formal Description}

We refer to our algorithm as \CASTLE{} (\underline{C}oncurrent \underline{A}lgorithm for Binary \underline{S}earch \underline{T}ree by \underline{L}ocking \underline{E}dges). 

\begin{limitscope}

%% To limit the scope of the macros defined below

%% macros for pseudocode
\newcommand{\leftChild}{le\!f\!t}
\newcommand{\rightChild}{right}
\newcommand{\child}{child}
\newcommand{\canReplace}{readyToReplace}
\newcommand{\markAndKey}{mKey}

\newcommand{\node}{node}
\newcommand{\parent}{parent}

\newcommand{\terminalEdge}{lastEdge}
\newcommand{\targetEdge}{targetEdge}
\newcommand{\parentTargetEdge}{pTargetEdge}
\newcommand{\successorEdge}{successorEdge}
\newcommand{\parentSuccessorEdge}{pSuccessorEdge}
\newcommand{\injectionEdge}{injectionEdge}
\newcommand{\penultimateEdge}{pLastEdge}

\newcommand{\targetKey}{targetKey}
\newcommand{\currentKey}{currentKey}

\newcommand{\newNode}{newNode}
\newcommand{\reference}{re\!f\!erence}
\newcommand{\state}{state}

\newcommand{\StateRecord}{StateRecord}
\newcommand{\AnchorRecord}{AnchorRecord}

\newcommand{\mline}[1]{\DontPrintSemicolon #1 \PrintSemicolon}

\newcommand{\prev}{prev}
\newcommand{\curr}{curr}

\newcommand{\prevSeekRecord}{pSeekRecord}
\newcommand{\prevAnchorRecord}{pAnchorRecord}
%% \newcommand{\currSeekRecord}{cSeekRecord}
\newcommand{\anchorRecord}{anchorRecord}

\newcommand{\oldContents}{oldValue}
\newcommand{\newContents}{newValue}

\newcommand{\INJECTION}{\textsf{INJECTION}}
\newcommand{\DISCOVERY}{\textsf{DISCOVERY}}
\newcommand{\CLEANUP}{\textsf{CLEANUP}}
\newcommand{\FINISHED}{\textsf{FINISHED}}

\newcommand{\DELETEFLAG}{\textsf{DELETE\_FLAG}}
\newcommand{\PROMOTEFLAG}{\textsf{PROMOTE\_FLAG}}
\newcommand{\INTENTFLAG}{\textsf{INTENT\_FLAG}}
\newcommand{\flag}{f\!lag}

\newcommand{\COMPLEX}{\textsf{COMPLEX}}
\newcommand{\SIMPLE}{\textsf{SIMPLE}}

\newcommand{\LEFT}{\textsf{LEFT}}
\newcommand{\RIGHT}{\textsf{RIGHT}}

\newcommand{\targetSeekRecord}{targetRecord}
\newcommand{\successorSeekRecord}{successorRecord}

\newcommand{\dFlag}{d}
\newcommand{\iFlag}{i}
\newcommand{\pFlag}{p}
\newcommand{\nFlag}{n}
\newcommand{\mFlag}{m}
\newcommand{\lNFlag}{lN}
\newcommand{\rNFlag}{rN}

\newcommand{\rarrow}{\!\rightarrow\!}


%%%%%%%%%%%%%%%%%%%%%%%%%%%%%%%%%%%%%%%%%%%%%%%%%%%%%%%%%%%%%%%%%%%%%%%%%%%%%%%%%%%%

\newcommand{\DefineKeyWords}{
%%
\SetKw{Boolean}{Boolean}
\SetKw{LAnd}{~and~}
\SetKw{LOr}{~or~}
\SetKw{LNot}{not}
\SetKw{Struct}{struct}
\SetKw{Null}{null}
\SetKw{True}{true}
\SetKw{False}{false}
\SetKw{Break}{break}
\SetKw{Continue}{continue}
\SetKw{Enum}{enum}
%%
}

%%%%%%%%%%%%%%%%%%%%%%%%%%%%%%%%%%%%%%%%%%%%%%%%%%%%%%%%%%%%%%%%%%%%%%%%%%%%%%%%%%%%%

%% DATA STRUCTURES


\begin{algorithm}[htp]
%%
\DefineKeyWords
%%

%% define data structures used in the algorithm

\DontPrintSemicolon
\Struct Node \{\;
\label{ln:icdcn-node|begin}
\PrintSemicolon
\Indp 
   $\{ \Boolean, \text{Key} \}$ $\markAndKey$\;
   $\{ \Boolean, \Boolean, \Boolean, \Boolean, \text{NodePtr} \}$ $\child[2]$\;
   \Boolean $\canReplace$\;
\Indm
\}\;
\label{ln:icdcn-node|end}
%%
\BlankLine

\DontPrintSemicolon
\Struct Edge \{\;
\label{ln:icdcn-edge|begin}
\PrintSemicolon
\Indp 
   %% NodePtr $\parent$\;
   %% NodePtr $\child$\;
	 NodePtr $\parent$, $\child$\;
   \Enum $which$ \{ \LEFT{}, \RIGHT{} \}\;
\Indm
\}\;
\label{ln:icdcn-edge|end}
%%
\BlankLine

\DontPrintSemicolon
\Struct SeekRecord \{\;
\PrintSemicolon
\Indp 
%%
   %% Edge $\terminalEdge$\;
%%
   %% Edge $\penultimateEdge$\;
%%
   %% Edge $\injectionEdge$\;
   Edge $\terminalEdge$, $\penultimateEdge$, $\injectionEdge$\;
\Indm
\}\;
%%
\BlankLine


\BlankLine
\DontPrintSemicolon
\Struct \AnchorRecord{} \{\;
\PrintSemicolon
\Indp 
   NodePtr $\node$\;
   Key $key$\;
\Indm
\}\;
%%

\BlankLine
\DontPrintSemicolon
\Struct \StateRecord{} \{\;
\PrintSemicolon
\Indp 
%%
   %% int $depth$\;
   %% Edge $\targetEdge$\;
	 %% Edge $\parentTargetEdge$\;
	 Edge $\targetEdge$, $\parentTargetEdge$\;
%%
   %% Key $\targetKey$\;
	 %% Key $\currentKey$\;
	 Key $\targetKey$, $\currentKey$\;
   \Enum $mode$ \{ \INJECTION{}, \DISCOVERY{}, \CLEANUP{} \}\;
   \Enum $type$ \{ \SIMPLE{}, \COMPLEX{} \} \;
%%
   \tcp{the next field stores pointer to a seek record; it is used for finding the successor if the delete operation is complex}
   SeekRecordPtr $\successorSeekRecord$\; 
\Indm
\}\;
%%
\BlankLine
\tcp{object to store information about the tree traversal when looking for a given key (used by the seek function)}
SeekRecordPtr $\targetSeekRecord$ := new seek record\;
\tcp{object to store information about process' own delete operation}
\StateRecord{Ptr} $myState$ := new state\;


\caption{Data Structures Used}
\label{algo:icdcn-data|structures}
\end{algorithm}



\begin{algorithm}[htp]
%%
\DefineKeyWords


%% SEEK


%%
%% traverses the tree from the root node to a leaf node looking for a given key
%%
\DontPrintSemicolon
\Seek( $key$, $seekRecord$ )\;
\PrintSemicolon
\Begin
{
   $\prevAnchorRecord$ := $\curly{ \snodetwo{}, \skey{1} }$\;
   \While{\True}
   {
	    \tcp{initialize all variables used in traversal}
		  $\penultimateEdge$ := $\curly{ \snodeone, \snodetwo, \RIGHT }$; \qquad
			$\terminalEdge$ := $\curly{ \snodetwo, \snodethree, \RIGHT }$\;
			$\curr$ := $\snodethree$; \qquad
			$\anchorRecord$ := $\curly{ \snodetwo{}, \skey{1} }$\;
			\BlankLine
			\While{\True}
			{
			    \tcp{read the key stored in the current node}
			    $\ang{ \ast, cKey }$ := $\curr \rarrow \markAndKey$\;
				  \tcp{find the next edge to follow}
					$which$ := $key < cKey$ ? \LEFT : \RIGHT\;
				  $\ang{ \nFlag, \ast, \dFlag, \pFlag, next }$ := $\curr \rarrow \child[which]$\;
					\tcp{check for the completion of the traversal}
				  \If{$key = cKey$ \LOr $\nFlag$}
				  {
				     \tcp{either key found or no next edge to follow; stop the traversal}
						 $seekRecord \rarrow \penultimateEdge$ := $\penultimateEdge$\;
						 $seekRecord \rarrow \terminalEdge$ := $\terminalEdge$\;
						 $seekRecord \rarrow \injectionEdge$ := $\curly{ \curr, next, which }$\;
						 \BlankLine
						 \uIf(\tcp*[h]{keys match}){$key = cKey$}
						 { 
						    \Return\;
						 }
						 \lElse { \Break }
				  }
				  \BlankLine			   
				  \If{$which$ = \RIGHT}
				  {
				     \tcp{next edge to be traversed is a right edge; keep track of the current node and its key}
						 $\anchorRecord$ := $\ang{ \curr, cKey }$\;
				  }	
				  \BlankLine
				  \tcp{traverse the next edge}
					$\penultimateEdge$ := $\terminalEdge$; \qquad
					$\terminalEdge$ := $\curly{ \curr, next, which }$; \qquad
				  $\curr$ := $next$\; 
		  }
		  \tcp{key was not found; check if can stop}
		  $\ang{ \ast, \ast, \dFlag, \pFlag, \ast }$ := $\anchorRecord.\node \rarrow \child[\RIGHT]$\;			
			\uIf{\LNot($\dFlag$) \LAnd \LNot($\pFlag$)}
			{
			   \tcp{anchor node is still part of the tree; check if anchor node's key has changed}
				 $\ang{ \ast, aKey }$ := $\anchorRecord.\node \rarrow \markAndKey$\;
				 \lIf{$\anchorRecord.key$ = $aKey$}
			   {  
				    \Return
				 } 
			}	
			\Else
			{ 
			   \tcp{check if the anchor record (the node and its key) matches that of the previous traversal}
			   \If{$\prevAnchorRecord = \anchorRecord$}
			   {
				    \tcp{return the results of the previous traversal}
					  $seekRecord$ := $\prevSeekRecord$\;
				    \Return\;
		     }
			}
			\tcp{store the results of the traversal and restart}
			$\prevSeekRecord$ := $seekRecord$; \qquad
			$\prevAnchorRecord$ := $\anchorRecord$;					
   }
} 
%% End of seek function
\caption{Seek Function}
\label{algo:icdcn-seek}
\end{algorithm}


%% SEARCH
\begin{algorithm}[htp]
%%
\DefineKeyWords
\DontPrintSemicolon
\Boolean \Search( $key$ )\;
\PrintSemicolon
\Begin
{
   \Seek( $key$, $mySeekRecord$ )\;
	 \BlankLine
	 %% $\node$ := $mySeekRecord \rarrow \node$\;
   $\node$ := $mySeekRecord \rarrow \terminalEdge.\child$\;
   $\ang{ \ast , nKey }$ := $\node \rarrow \markAndKey$\;
	 \BlankLine
   \lIf{nKey = key}{\Return \True}
   \lElse{\Return \False}
}
\caption{Search Operation}
\label{algo:icdcn-search}
\end{algorithm}

%% INSERT
\begin{algorithm}[htp]
%%
\DefineKeyWords
\DontPrintSemicolon
\Boolean \Insert( $key$ )\;
\PrintSemicolon
\Begin
{
   \While{\True}
	 {
      \Seek( $key$, $\targetSeekRecord$ )\;
			\BlankLine
			$\targetEdge$ := $\targetSeekRecord \rarrow \terminalEdge$\;
			$\node$ := $\targetEdge.\child$\;
			$\ang{ \ast , nKey }$ := $\node \rarrow \markAndKey$\; 
			\lIf{$key = nKey$}{\Return \False}
			\BlankLine
			\tcp{create a new node and initialize its fields}
			$\newNode$ := create a new node\;
			$\newNode \rarrow \markAndKey$ := $\ang{ 0_m, key }$\;
			$\newNode \rarrow \child[\LEFT]$ := $\ang{ 1_n, 0_i, 0_d, 0_p, \Null }$\;
			$\newNode \rarrow \child[\RIGHT]$ := $\ang{ 1_n, 0_i, 0_d, 0_p, \Null }$\;
			$\newNode \rarrow \canReplace$ := \False\;
			\BlankLine
			$which$ := $\targetSeekRecord \rarrow \injectionEdge.which$\;
			$address$ := $\targetSeekRecord \rarrow \injectionEdge.\child$\;
			$result$ := \CAS($\node \rarrow \child[which]$, $\ang{ 1_n, 0_i, 0_d, 0_p, address }$, $\ang{ 0_n, 0_i, 0_d, 0_p, \newNode }$)\;
			\lIf{$result$}{\Return \True}
			\BlankLine	
			\tcp{help if needed}
		  $\ang{ \ast, \ast, \dFlag, \pFlag, \ast }$ := $\node \rarrow \child[which]$\;
			\lIf{$\dFlag$}
			{
			   \HelpTargetNode( $\targetEdge$ )
			} 
			\lElseIf{$p$}
			{  
			   \HelpSuccessorNode( $\targetEdge$ )
			}
	}
}
\caption{Insert Operation}
\label{algo:icdcn-insert}
\end{algorithm}

%% DELETE
\begin{algorithm}[htp]
\DefineKeyWords
\DontPrintSemicolon
\Boolean \Delete( $key$ )\;
\PrintSemicolon
\Begin
{
   \tcp{initialize the state record}
 	 $myState \rarrow \targetKey$ := $key$; $\qquad$
	 $myState \rarrow \currentKey$ := $key$\;
	 $myState \rarrow mode$ := \INJECTION\;
	 \BlankLine
   \While{\True}
	 {
      \Seek( $myState \rarrow \currentKey$, $\targetSeekRecord$ )\;
			$\targetEdge$ := $\targetSeekRecord \rarrow \terminalEdge$; $\qquad$
			$\parentTargetEdge$ := $\targetSeekRecord \rarrow \penultimateEdge$\;
			$\ang{ \ast , nKey }$ := $\targetEdge.\child \rarrow \markAndKey$\; 
			\BlankLine
			\If{$myState \rarrow \currentKey \neq nKey$}
			{
			   \tcp{the key does not exist in the tree}
			   \lIf{$myState \rarrow mode$ = \INJECTION}{\Return \False}
				 \lElse{\Return \True}
			}
 	    \BlankLine		   	
			\tcp{perform appropriate action depending on the mode}
	    \If{$myState \rarrow mode$ = \INJECTION}
		  {
				 \tcp{store a reference to the target edge}
   	     $myState \rarrow \targetEdge$ := $\targetEdge$\;
   	     $myState \rarrow \parentTargetEdge$ := $\parentTargetEdge$\;
				 \tcp{attempt to inject the operation at the node}
				 %% $result$ := \Inject( $myState$ )\;
				 \Inject( $myState$ )\;					 								 
			}
			\BlankLine
			\tcp{mode would have changed if injection was successful}
				 
			\If{$myState \rarrow mode \neq$ \INJECTION}
			{
				 \tcp{check if the target node found by the seek function matches the one stored in the state record}			
			   %%\If{$\left(\text{\parbox[c]{1.75in}{$myState \rarrow \targetEdge.\child$ $\neq$  \\ \mbox{}\hfill$\targetEdge.\child$}}\right)$}
				 \lIf{$myState \rarrow \targetEdge.\child$ $\neq$ $\targetEdge.\child$}
				 {
				    \Return \True
				 }						
				 \tcp{update the target edge information using the most recent seek}
				 $myState \rarrow \targetEdge$ := $\targetEdge$\; 			 				
		  }				
			\BlankLine							
			\If{$myState \rarrow mode$ = \DISCOVERY}
			{
				 \tcp{complex delete operation; locate the successor node and mark its child edges with promote flag} 
			   \FindAndMarkSuccessor( $myState$ )\;			 
			}			
			\If{$myState \rarrow mode$ = \DISCOVERY}
			{
				 \tcp{complex delete operation; promote the successor node's key and remove the successor node}
		     \RemoveSuccessor( $myState$ )\;						
			}			
			\BlankLine				
			\If{$myState \rarrow mode$ = \CLEANUP}
			{
			   \tcp{either remove the target node (simple delete) or replace it with a new node with all fields unmarked  (complex delete)}
			   $result$ := \Cleanup( $myState$ )\;
				 \lIf{$result$}{\Return \True}
				 \Else{
				    $\ang{ \ast, nKey }$ := $\targetEdge.\child \rarrow \markAndKey$\;
						$myState \rarrow \currentKey$ := $nKey$\;
				 }					
		  }
	 }	
   %% \Return\;
}
\caption{Delete Operation}
\label{algo:icdcn-delete}
\end{algorithm}

%% INJECT
\begin{algorithm}[htp]
%%
\DefineKeyWords
\DontPrintSemicolon
\Inject( $\state$ )\;
\PrintSemicolon
\Begin
{
   $\targetEdge$ := $\state \rarrow \targetEdge$\;
	 \tcp{try to set the intent flag on the target edge}
	 \tcp{retrieve attributes of the target edge}
	 $\parent$ := $\targetEdge.\parent$\;
	 $\node$ := $\targetEdge.\child$\;
	 $which$ := $\targetEdge.which$\;
	 \BlankLine
	 \mline{$result$ := \CAS( \parbox[t]{2.075in}{$\parent \rarrow \child[which]$, \\ $\ang{ 0_n, 0_i, 0_d, 0_p, \node }$,  $\ang{ 0_n, 1_i, 0_d, 0_p, \node }$ );}\;}
	 \If{\LNot($result$)}
	 {
	    \tcp{unable to set the intent flag; help if needed}
			$\ang{ \ast, \iFlag, \dFlag, \pFlag, address }$ := $\parent \rarrow \child[which]$\;
			\lIf{$\iFlag$}
			{
			   \HelpTargetNode( $\targetEdge$ )
			} 
			\uElseIf{$\dFlag$}
			{
			   \HelpTargetNode( $\state \rarrow \parentTargetEdge$ )\;
			} 
			\ElseIf{$\pFlag$}
			{
			   \HelpSuccessorNode( $\state \rarrow \parentTargetEdge$ )\;
			}

      \Return;					
	 }

   \BlankLine
	 \tcp{mark the left edge for deletion}

	 $result$ := \MarkChildEdge( $\state$, \LEFT{} )\;
	 
	 \lIf{\LNot($result$)}
	 {
	    \Return
	 } 
	 \tcp{mark the right edge for deletion; cannot fail}
	 \MarkChildEdge( $\state$, \RIGHT{} )\;
	   
	 \BlankLine
	 \tcp{initialize the type and mode of the operation}
	 \InitializeTypeAndUpdateMode( $\state$ );	
}

\caption{Injecting a Deletion Operation}
\label{algo:icdcn-inject}
\end{algorithm}









%% FINDANDMARKSUCCESSOR


\begin{algorithm}[htp]
%%
\DefineKeyWords

\DontPrintSemicolon
\FindAndMarkSuccessor( $\state$ )\;
\PrintSemicolon
\Begin
{
   \tcp{retrieve the addresses from the state record}
   $\node := \state \rarrow \targetEdge.\child$\;
	 $seekRecord$ := $\state \rarrow \successorSeekRecord$\; 
   
	 \BlankLine
   \While{\True}
	 {
	 
	    \tcp{read the mark flag of the key in the target node}  
	    $\ang{ \mFlag, \ast}$ := $\node \rarrow \markAndKey$\; 
	    

	  	\tcp{find the node with the smallest key in the right subtree}
	    $result$ := \FindSmallest( $\state$ )\;
			
						
			\BlankLine
			\If{$\mFlag$ \LOr \LNot($result$)} 
			{
			   \tcp{successor node had already been selected \emph{before} the traversal or the right subtree is empty}
				 \Break\;
			}
			
				
			\tcp{retrieve the information from the seek record}
			$\successorEdge$ := $seekRecord \rarrow \terminalEdge$\;
			%% $\ang{ \nFlag, \ast, \ast, \ast, \leftChild}$ :=  $\successorEdge.\child \rarrow \child[\LEFT]$\;
			%% \lIf{\LNot($\nFlag$)}{ \Continue }
			$\leftChild$ := $seekRecord \rarrow \injectionEdge.\child$\;
			
			\BlankLine
			\tcp{read the mark flag of the key under deletion}
      $\ang{ \mFlag, \ast}$ := $\node \rarrow \markAndKey$\;
			
			\If(\tcp*[h]{successor node has already been selected}){$\mFlag$}
			{
			   %% \tcp{successor node has already been selected}
			   %% \lIf{$p$}{ \Break }
				 %% \lElse{ \Continue }
				 \Continue\;
				 
			}
			
			
		


          
			\tcp{try to set the promote flag on the left edge}
			\mline{$result$ := \CAS( \parbox[t]{1.875in}{$\successorEdge.\child \rarrow \child[\LEFT]$, \\ 
			                                             $\ang{ 1_n, 0_i, 0_d, 0_p, \leftChild }$, \\ $\ang{ 1_n, 0_i, 0_d, 1_p, \node }$ );}\;}
			
			\lIf{$result$}{\Break}
			
			\BlankLine
			\tcp{attempt to mark the edge failed; recover from the failure and retry if needed}
			%% $\ang{ n, \ast, d, p, \ast }$ := $\successorEdge.\child \rarrow \child[\LEFT]$\;
			$\ang{ \nFlag, \ast, \dFlag, \ast, \ast }$ := $\successorEdge.\child \rarrow \child[\LEFT]$\;
			
			
      %% \lIf{$p$}
      %% {
      %%    \Break
      %% }   

      %% \If{\LNot($n$)}
			%% { 
			%%   \tcp{the node found has since gained a left child}
			%%   \Continue\;
			%% }

			\If{$\nFlag$ \LAnd $\dFlag$}
			{
			    \tcp{the node found is undergoing deletion; need to help}
					
								
					%% \mline{\HelpTargetNode( \parbox[t]{1.5in}{$\successorEdge$, \\ $\state \rarrow depth + 1$ );}\;}
		      \HelpTargetNode( $\successorEdge$ )\;
       } 
	 }	
   \BlankLine
   \tcp{update the operation mode}
	 \UpdateMode( $\state$ );
}

\caption{Locating the Successor Node}
\label{algo:icdcn-findandmark}
\end{algorithm}




%% REMOVESUCCESSOR
\begin{algorithm}[htp]
%%
\DefineKeyWords
\DontPrintSemicolon
\RemoveSuccessor( $\state$ )\;
\PrintSemicolon
\Begin
{
   \tcp{retrieve addresses from the state record}
   $\node$ := $\state \rarrow \targetEdge.\child$\;
   $seekRecord$ := $\state \rarrow \successorSeekRecord$\;
   \tcp{extract information about the successor node}
	 %% \tcp{assumes that the state's seek record contains valid information}
   $\successorEdge$ := $seekRecord \rarrow \terminalEdge$\;
	 \BlankLine
	 \tcp{ascertain that seek record for successor node contains valid information}
	 $\ang{ \ast, \ast, \ast, \pFlag, address }$ := $\successorEdge.\child \rarrow \child[\LEFT]$\;
	 \If{\LNot($\pFlag$) \LOr ($address$ $\neq$ $\node$)}
	 {
	    $\node \rarrow \canReplace$ := \True\;
			\UpdateMode( $\state$ )\;
	    \Return\;
	 }
   \BlankLine
   \tcp{mark the right edge for promotion if unmarked}
   \MarkChildEdge( $\state$, \RIGHT{} )\; 
   \BlankLine
   \tcp{promote the key}
   $\node \rarrow \markAndKey$ := $\ang{ 1_m, \successorEdge.\child \rarrow \markAndKey }$\;
   \While{\True}
   {
      \tcp{check if the successor is the right child of the target node itself}
	    \uIf{$\successorEdge.\parent$ = $\node$}
	    {
	       \tcp{need to modify the right edge of target node whose delete flag is set}
				 $dFlag$ := 1; \qquad
			   $which$ := \RIGHT\;
	    }
	    \Else
	    {
			   $dFlag$ := 0; \qquad
			   $which$ := \LEFT\;
	    }
      $\ang{ \ast, \iFlag, \ast, \ast, \ast }$ := $\successorEdge.\parent \rarrow \child[which]$\;			
      \BlankLine			
	    $\ang{ \nFlag, \ast, \ast, \ast, \rightChild }$ := $\successorEdge.\child \rarrow \child[\RIGHT]$\;	
	    $\oldContents$ := $\ang{ 0_n, \iFlag, dFlag, 0_p, \successorEdge.\child }$\;	    
			\uIf(\tcp*[f]{only set the null flag; do not change the address}){$\nFlag$}
	    {				
				 %\mline{\parbox[t]{1.75in}{$\newContents$ := \\ \mbox{} \qquad $\ang{ 1_n, 0_i, dFlag, 0_p,  \successorEdge.\child }$;}}
				 $\newContents$ := $\ang{ 1_n, 0_i, dFlag, 0_p,  \successorEdge.\child }$\;
	    }
	    \Else(\tcp*[f]{switch the pointer to point to the grand child})
	    {	 				 
				 $\newContents$ := $\ang{ 0_n, 0_i, dFlag, 0_p, \rightChild }$ \;		 
	    }	 
      \remove{ \lIf{$result$}{\Break} }			
			%\mline{$result$ := \CAS( \parbox[t]{1.77in}{$\successorEdge.\parent \rarrow \child[which]$, \\ $\oldContents$, $\newContents$ );}\;}
			$result$ := \CAS($\successorEdge.\parent \rarrow \child[which]$,$\oldContents$, $\newContents$)\;
			\lIf{$result$ \LOr $dFlag$}{ \Break }
	    %\BlankLine			
			$\ang{ \ast, \ast, \dFlag, \ast, \ast }$ := $\successorEdge.\parent \rarrow \child[which]$\;
			$\penultimateEdge$ := $seekRecord \rarrow \penultimateEdge$\;
			\If{$\dFlag$ \LAnd ($\penultimateEdge.\parent$ $\neq$ \Null)}
			{
			   %% \mline{\HelpTargetNode( \parbox[t]{1.25in}{$\penultimateEdge$, \\ $\state \rarrow depth + 1$ );}\;}
				 \HelpTargetNode( $\penultimateEdge$ )\;
			}			
      \BlankLine			
 	    $result$ := \FindSmallest( $\state$ )\;
			$\terminalEdge$ := $seekRecord \rarrow \terminalEdge$\;
	    %\If{$\left(\text{\parbox[c]{1.875in}{\LNot($result$) \LOr \\ $\terminalEdge.\child$ $\neq$ $\successorEdge.\child$}}\right)$}
			\If{\LNot($result$) \LOr $\terminalEdge.\child$ $\neq$ $\successorEdge.\child$}
			{
			   \Break;
				 \tcp*[f]{the successor node has already been removed}
			} 
			\lElse
			{
			   $\successorEdge$ := $seekRecord \rarrow \terminalEdge$
			}
   }
   \BlankLine
	 $\node \rarrow \canReplace$ := \True\;
   \UpdateMode( $\state$ )\;	
}
\caption{Removing the Successor Node}
\label{algo:icdcn-remove}
\end{algorithm}



%% CLEANUP

\begin{algorithm}[htp]
%%
\DefineKeyWords



\DontPrintSemicolon
\Boolean \Cleanup( $\state$ )\;
\PrintSemicolon
\Begin
{
   %% \tcp{retrieve the addresses from the state record}
   %% $\node$ := $\state \rarrow \node$\;
	 %% $\parent$ := $\state \rarrow \parent$\;
	
	 %% \BlankLine
	
	 %% \tcp{determine which edge of the parent needs to be switched} 
	 %% $\ang{ \ast, pKey }$ := $\parent \rarrow \markAndKey$\;
	 %% $\ang{ \ast, nKey }$ := $\node \rarrow \markAndKey$\;
	 %% $pWhich$ := $nKey < pKey$ ? \LEFT : \RIGHT\;
	 $\ang{\parent, \node, pWhich}$ := $\state \rarrow \targetEdge$\;
	 
	
	 \BlankLine
	 
	 \uIf{$\state \rarrow type$ = \COMPLEX}
	 {
	  	   
	    \tcp{replace the node with a new copy in which all fields are unmarked} 
			$\ang{ \ast, nKey }$ := $\node \rarrow \markAndKey$\;
			$newNode \rarrow \markAndKey$ := $\ang{ 0_m, nKey }$\;		
			\BlankLine
			\tcp{initialize left and right child pointers}
  		$\ang{ \ast, \ast, \ast, \ast, \leftChild }$ := $\node \rarrow \child[\LEFT]$\;
			$\newNode \rarrow \child[\LEFT]$  := $\ang{ 0_n, 0_i, 0_d, 0_p, \leftChild }$\;
			$\ang{ \nFlag, \ast, \ast, \ast, \rightChild }$ := $\node \rarrow \child[\RIGHT]$\;
			\uIf{$\nFlag$}
			{
			  $\newNode \rarrow \child[\RIGHT]$  := $\ang{ 1_n, 0_i, 0_d, 0_p, \Null }$\;
			}
			\lElse
			{
			  $\newNode \rarrow \child[\RIGHT]$  := $\ang{ 0_n, 0_i, 0_d, 0_p, \rightChild }$
			}
			\BlankLine
			\tcp{initialize the arguments of \CAS{} instruction}
			$\oldContents$ := $\ang{ 0_n, 1_i, 0_d, 0_p, \node }$\;
			$\newContents$ := $\ang{ 0_n, 0_i, 0_d, 0_p, \newNode }$\;
			
			%% \tcp{switch the edge at the parent}
			%% \mline{$result$ := \CAS( \parbox[t]{1.875in}{$\parent \rarrow \child[pWhich]$, \\ $\ang{ 0_n, 1_i, 0_d, 0_p, \node }$, $\ang{ 0_n, 0_i, 0_d, 0_p, \newNode }$ );}\;}
			 
	
	 }
	 \Else(\tcp*[h]{remove the node})
	 {
	   			
	    %% \tcp{remove the node}
			
			\tcp{determine to which grand child will the edge at the parent be switched}
			\uIf{$\node \rarrow \child[\LEFT]$ = $\ang{ 1_n, \ast, \ast, \ast, \ast }$}
			{
		     $nWhich$ := \RIGHT\;
			}
			\lElse{$nWhich$ := \LEFT}
			
			\BlankLine
			\tcp{initialize the arguments of the \CAS{} instruction}
			$\oldContents$ := $\ang{ 0_n, 1_i, 0_d, 0_p, \node }$\;
			$\ang{ \nFlag, \ast, \ast, \ast, address }$ := $\node \rarrow \child[nWhich]$\; 
  		\uIf(\tcp*[h]{set the null flag only}){$\nFlag$}
			{
			   $\newContents$ := $\ang{ 1_n, 0_i, 0_d, 0_p, \node }$\;
			   %% \tcp{set the null flag only; do not change the address}
			   %% \mline{$result$ := \CAS( \parbox[t]{1.25in}{$\parent \rarrow \child[pWhich]$, \\ $\ang{ 0_n, 1_i, 0_d, 0_p, \node }$, \\ $\ang{ 1_n, 0_i, 0_d, 0_p, \node }$ );}\;}
			}
			\Else(\tcp*[h]{change the pointer to the grand child})
			{
			   $\newContents$ := $\ang{ 0_n, 0_i, 0_d, 0_p, address }$ \;
				 %% \mline{$result$ := \CAS( \parbox[t]{1.25in}{$\parent \rarrow \child[pWhich]$, \\ $\ang{ 0_n, 1_i, 0_d, 0_p, \node }$, \\ $\ang{ 0_n, 0_i, 0_d, 0_p, address }$ );}\;}
			}
			
			
			
	 }
	
	  \BlankLine
		\mline{$result$ := \CAS( \parbox[t]{1.75in}{$\parent \rarrow \child[pWhich]$, \\ $\oldContents$, $\newContents$ );}\;}
		\Return $result$\;
		



}

\caption{Cleaning Up the Tree}
\label{algo:icdcn-cleanup}
\end{algorithm}





\begin{algorithm}[htp]
%%
\DefineKeyWords
\DontPrintSemicolon
\Boolean \MarkChildEdge( $\state$, $which$ )\;
\PrintSemicolon
\Begin
{

   \uIf{$\state \rarrow mode$ = \INJECTION}
	 {
	    $edge$ := $\state \rarrow \targetEdge$\; 
	    $\flag$ := \DELETEFLAG\;
	 }
	 \Else
	 {
	    $edge$ := $( \state \rarrow \successorSeekRecord ) \rarrow \terminalEdge$\; 
	    $\flag$ := \PROMOTEFLAG\;
	 }
	 
	 
   $\node$ := $edge.\child$\;
	
   \BlankLine
  
	 \While{\True}
	 {
	    $\ang{\nFlag, \iFlag, \dFlag, \pFlag, address}$ := $\node \rarrow \child[which]$\;
			
			\uIf{$\iFlag$}
			{
			   $helpeeEdge$ := $\curly{ \node, address, which }$\;
				 %% \HelpTargetNode( $helpeeEdge$, $\state \rarrow depth + 1$ )\;
				 \HelpTargetNode( $helpeeEdge$ )\;
				 \Continue\;
			}
			\uElseIf{$\dFlag$}
			{
			   \uIf{$\flag$ = \PROMOTEFLAG}
				 {
				    %% \HelpTargetNode( $edge$, $\state \rarrow depth + 1$  )\;
						\HelpTargetNode( $edge$ )\;
						\Return \False\;
				 } 
				 \lElse
				 {
				    \Return \True
				 }
			}
			\ElseIf{$\pFlag$}
			{
			   \uIf{$\flag$ = \DELETEFLAG}
				 {
				    %% \HelpSuccessorNode( $edge$, $\state \rarrow depth + 1$  )\;
						\HelpSuccessorNode( $edge$ )\;
						\Return \False\;
				 } 
				 \lElse
				 {
				    \Return \True
				 }
			}
			
			$\oldContents$ := $\ang{ \nFlag, 0_i, 0_d, 0_p, address }$\;
			$\newContents$ := $\oldContents \: | \: \flag$\;
			\mline{$result$ := \CAS( \parbox[t]{1.5in}{$\node \rarrow \child[which]$, $\oldContents$, \\ $\newContents$ );}\;}
			
			\lIf{$result$}{ \Break }
			
			
	 }

   \Return \True\;
}
\caption{Mark Child Edge}
\label{algo:icdcn-markChildEdge}


%\remove{

\end{algorithm}

%% FINDSMALLEST

\begin{algorithm}[htp]
%%
\DefineKeyWords
%}

\BlankLine

\DontPrintSemicolon
\Boolean \FindSmallest( $\state$ )\;
\PrintSemicolon
\Begin
{
   \tcp{find the node with the smallest key in the subtree rooted at the right child of the target node}
	 $\node$ := $\state \rarrow \targetEdge.\child$\;
	 $seekRecord$ := $\state \rarrow seekRecord$\;
	 $\ang{ \nFlag, \ast, \ast, \ast, \rightChild }$ := $\node \rarrow \child[\RIGHT]$\;
	 \If(\tcp*[h]{the right subtree is empty}){$\nFlag$}
	 {
	    %% \tcp{the right subtree is empty}
			\Return \False\;
	 }
	
	 \BlankLine	
		
	 \tcp{initialize the variables used in the traversal}
	 
	
	 %% $\ang{ \ast, \ast, \ast, \ast, \rightChild }$ := $\node \rarrow \child[RIGHT]$\;
	 $\terminalEdge$ := $\ang{ \node, \rightChild, \RIGHT }$\;
	 $\penultimateEdge$ := $\ang{ \node, \rightChild, \RIGHT }$\;
		 
	 %% \BlankLine
	 	
	 \While{\True}
	 {
	    $\curr$ := $\terminalEdge.\child$\;
      $\ang{ \nFlag, \ast, \ast, \ast, \leftChild }$ := $\curr \rarrow \child[\LEFT]$\;			
			\If(\tcp*[h]{reached the node with the smallest key}){$\nFlag$}	
			{
			   $\injectionEdge$ := $\ang{\curr, \leftChild, \LEFT}$\;
			   \Break\;
			}				
			\BlankLine			
			\tcp{traverse the next edge}			
			$\penultimateEdge$ := $\terminalEdge$\;
	    $\terminalEdge$ := $\ang{ \curr, \leftChild, \LEFT }$\;			
	 }	
	 \BlankLine
	 \tcp{initialize seek record and return}
	 $seekRecord \rarrow \terminalEdge$ := $\terminalEdge$\;
	 $seekRecord \rarrow \penultimateEdge$ := $\penultimateEdge$\;
   $seekRecord \rarrow \injectionEdge$ := $\injectionEdge$\;
	 \Return \True\;	
}
\caption{Find Smallest}
\label{algo:icdcn-findSmallest}
\end{algorithm}


\begin{algorithm}[htp]
%%
\DefineKeyWords


\DontPrintSemicolon
\InitializeTypeAndUpdateMode( $\state$ )\;
\PrintSemicolon
\Begin
{

   \tcp{retrieve the target node's address from the state record}
   $\node$ := $\state \rarrow \targetEdge.\child$\;
	 
	
	 \BlankLine
	 %% $\canReplace$ := $\node \rarrow \canReplace$\;
	 $\ang{ \lNFlag, \ast, \ast, \ast, \ast }$ := $\node \rarrow \child[\LEFT]$\;
	 $\ang{ \rNFlag, \ast, \ast, \ast, \ast }$ := $\node \rarrow \child[\RIGHT]$\;
	
	 \uIf{$\lNFlag$ \LOr $\rNFlag$}
	 {
	    \tcp{one of the child pointers is null}
	    $\ang{\mFlag, \ast }$ := $\node \rarrow \markAndKey$\;
	    \lIf{$\mFlag$}
	    {
	      $\state \rarrow type$ := \COMPLEX
	      %% $\node \rarrow \canReplace$ := \True\;
	    }
	    \lElse
	    {
	      $\state \rarrow type$ := \SIMPLE
	     }
	 }
	 \Else(\tcp*[h]{both child pointers are non-null})
	 {
	    %% \tcp{both child pointers are non-null}
	    $\state \rarrow type$ := \COMPLEX\;
	 }
	
	 \UpdateMode( $\state$ )\;
	
	 %% \Return\;

}

\remove{

\end{algorithm}


%% UPDATEMODE

\begin{algorithm}[htp]
%%
\DefineKeyWords

}

\BlankLine

\DontPrintSemicolon
\UpdateMode( $\state$ )\;
\PrintSemicolon
\Begin
{
	
	 \tcp{update the operation mode}

	 \BlankLine
	 \uIf(\tcp*[h]{simple delete}){$\state \rarrow type$ = \SIMPLE}
	 {
	    %% \tcp{simple delete}	
			$\state \rarrow mode$ := \CLEANUP\;
	 }
	 \Else(\tcp*[h]{complex delete})
	 {
	  	%% \tcp{complex delete}	

      $\node$ := $\state \rarrow \targetEdge.\child$\;
			\uIf{$\node \rarrow \canReplace$}
			{
			   $\state \rarrow mode$ := \CLEANUP\;
			}
			\lElse{$\state \rarrow mode$ := \DISCOVERY}
	 }
	
	 %% \Return\;
}

\caption{Helper Routines}
\label{algo:icdcn-helper|2}
\end{algorithm}

%% HELP

\begin{algorithm}[htp]
%%
\DefineKeyWords




\DontPrintSemicolon
%% \HelpTargetNode( $helpeeEdge$, $depth$ )\;
\HelpTargetNode( $helpeeEdge$ )\;
\PrintSemicolon
\Begin
{
   %% \lIf{$depth$ = number of processes}{ \Return }
	 %% \BlankLine		
	 \tcp{intent flag must be set on the edge}
	 \tcp{obtain new state record and initialize it}
	 $\state \rarrow \targetEdge$ := $helpeeEdge$\;
	 %% $\state \rarrow depth$ := $depth$\;
	 $\state \rarrow mode$ := \INJECTION\;
	 \BlankLine	
	 \tcp{mark the left and right edges if unmarked}
	 $result$ := \MarkChildEdge( $\state$, \LEFT{} )\;
	 \lIf{\LNot($result$)}{ 
	    %% \tcp{promote flag must have been set on the left edge}
			%% \HelpSuccessorNode( $helpeeEdge$, $depth + 1$ )\;
	    \Return
	 }
	 \MarkChildEdge( $\state$, \RIGHT{} )\;
	 
	 \InitializeTypeAndUpdateMode( $\state$ )\;
	
			
	 
	 \BlankLine
	
	 \tcp{perform the remaining steps of a delete operation}
   \If{$\state \rarrow mode$ = \DISCOVERY}
	 {
			%% \tcp{complex delete operation; locate the successor node and mark its child edges with promote flag}		
	    \FindAndMarkSuccessor( $\state$ )\;
	 						
	 }
			
	 \BlankLine
			
	 \If{$\state \rarrow mode$ = \DISCOVERY}
	 {
						
			%% \tcp{complex delete operation; promote the successor node's key and remove the successor node}
	    \RemoveSuccessor( $\state$ )\;
		   						
	 }
				
	 \BlankLine	
				
	 \lIf{$\state \rarrow mode$ = \CLEANUP}
	 {
	    %% \tcp{either remove the target node (simple delete) or replace it with a new node with unmarked edges (complex delete)}
	    \Cleanup( $\state$ )
	 }
	
	 %% \Return\;
}

\remove{

\end{algorithm}	
	


\begin{algorithm}[htp]
%%
\DefineKeyWords

}

\BlankLine

\DontPrintSemicolon
%% \HelpSuccessorNode( $helpeeEdge$, $depth$ )\;
\HelpSuccessorNode( $helpeeEdge$ )\;
\PrintSemicolon
\Begin
{
   %% \lIf{$depth$ = number of processes}{ \Return }
	 %% \BlankLine
   \tcp{retrieve the address of the successor node}
   $\parent$ := $helpeeEdge.\parent$\;
	 $\node$ := $helpeeEdge.\child$\;
	 
	 \tcp{promote flat must be set on the successor node's left edge}
	 \tcp{retrieve the address of the target node}
	 $\ang{ \ast, \ast, \ast, \ast, \leftChild }$ := $\node \rarrow \child[\LEFT]$\;
	 \BlankLine	
	 \tcp{obtain new state record and initialize it}
	 $\state \rarrow \targetEdge$ := $\curly{ \Null, \leftChild, \_ }$\;
	 %% $\state \rarrow depth$ := $depth$\;
	 $\state \rarrow mode$ := \DISCOVERY\;
	 $seekRecord$ := $\state \rarrow \successorSeekRecord$\;
	 \tcp{initialize the seek record in the state record}
	 $seekRecord \rarrow \terminalEdge$ := $helpeeEdge$\;
	 $seekRecord \rarrow \penultimateEdge$ := $\curly{ \Null, \parent, \_ }$\;
   \tcp{promote the successor node's key and remove the successor node}
	 \RemoveSuccessor( $\state$ )\;
	 \tcp{no need to perform the cleanup}
	
	
	 %% \Return\;

}


\caption{Helping Conflicting Delete Operations}
\label{algo:icdcn-helping}
\end{algorithm}
\end{limitscope}

A pseudo-code of our algorithm is given in \algosref{castle-data|structures}{helper}.
Different data structures used in our algorithm are shown in \algoref{castle-data|structures}. Besides tree node, we use three additional records:
\begin{enumerate*}[label=(\alph*)]
\item  \emph{seek record:} to store the outcome of a tree traversal both when looking for the target key and the successor key, 
\item \emph{anchor record:} to store information about the \anchornode{} during the seek phase, and
\item  \emph{lock record:} to store information about a tree edge that needs to be locked. 
\end{enumerate*}

The pseudo-code for the seek function is shown in \algoref{castle-seek}. The 
pseudo-codes for search, insert and delete operations are shown  in 
\algoref{castle-search}, \algoref{castle-insert} and \algoref{castle-delete}, respectively. 
\Algoref{lock:unlock} contains the pseudo-code for locking and unlocking a 
set of tree edges, as specified in an array. Finally, \algoref{helper} contains the pseudo-codes for 
three helper functions used by a delete operation, namely:
\begin{enumerate*}[label=(\alph*)]
\item \ClearFlags{}: to clear lock and mark flags from a child field, 
\item \FindSmallest{}: to locate the smallest key in a subtree, and
\item \RemoveChild{}: to remove a given child of a node.
\end{enumerate*}

In the pseudo-code, to improve clarity, we sometimes use subscripts $l$, $m$ and $n$ to denote lock, mark and null flags, respectively.  
\section{Correctness Proof}
It can be shown that our algorithm satisfies linearizability and lock-freedom properties~\cite{HerSha:2012:Book}. Broadly speaking, linearizability requires that an operation should appear to take effect instantaneously at some point during its execution.  Lock-freedom requires that some process should be able to complete its operation in a finite number of its own steps.
It is convenient to treat insert and delete operations that do not change the tree as search operations. 
%%
We call a tree node \emph{active} if it is reachable from the root of the tree. We call a tree node  \emph{passive} if it was active earlier but is not active any more.
It can be verified that, if an insert operation completes successfully, then  its \targetnode{} was active when it performed the successful \CAS{} instruction on the node's child edge.
Likewise, if a delete operation completes successfully, then its \targetnode{} was active when it marked the node's left edge for deletion. Further, for a complex delete, 
the successor node was active when it marked the node's left edge for promotion.



%% We are now ready to prove the correctness of our algorithm.

\subsubsection{All Executions are Linearizable}

We show that an arbitrary execution of our algorithm is linearizable by specifying the \emph{linearization point} of each operation. Note that the linearization point of an operation is the point during its execution at which the operation appeared to have taken effect. Our algorithm supports three types of operations: search, insert and delete. 
%%   
We now specify the linearization point of each operation.
%%
\begin{enumerate}[leftmargin=*, noitemsep]

\item \emph{Insert operation:} The operation is linearized at the point at which it performed the successful \CAS{} instruction that resulted in its target key becoming part of the tree.
		
					
\item \emph{Delete operation:} There are two cases depending on whether the delete operation is simple or complex. If the operation is simple delete, then the operation is linearized at the point at which a successful \CAS{} instruction was performed at the parent of the \targetnode{} that resulted in the \targetnode{} becoming passive. Otherwise, it is linearized at the point at which the original key of the \targetnode{} was replaced with its successor key.
   
 
   
\item \emph{Search operation:} There are two cases depending on whether the \targetnode{} was active when the operation read the key stored in the node. 
If the \targetnode{} was not active, then the operation is linearized at the point at which the \targetnode{} became passive. Otherwise, it is linearized at the 
point at which the read action was performed.
                       

\end{enumerate}


It can be easily verified that, for any execution of the algorithm, the sequence of operations
obtained by ordering operations based on their linearization points is legal, \emph{i.e.}, all operations in the sequence satisfy their 
specification. 
%%
This establishes that \emph{our algorithm generates only linearizable executions}.

\begin{comment}
 
Thus we have:

\begin{theorem}
Every execution of our algorithm is linearizable.
\end{theorem}
	
\end{comment}
							 
\subsubsection{All Executions are Lock-Free}
	
	
We say that the system is in a \emph{quiescent state} if no modify operation 
completes hereafter. We say that the system is in a \emph{potent state} if it 
has one or more pending modify operations. Note that a quiescence is a \emph{
stable} property; once the system is in a quiescent state, it stays in a 
quiescent state. We show that our algorithm is lock-free by proving that 
a potent state is necessarily non-quiescent provided assuming that some 
process with a pending modify operation continues to take steps. 

Assume, by the way of contradiction, that there is an execution of the system 
in which the system eventually reaches a state that is potent as well as 
quiescent. Note that, once the system has reached a quiescent state, it will 
eventually reach a state after which the tree will not undergo any structural 
changes. This is because a modify operation makes at most two structural changes to the 
tree. So, if the tree is undergoing continuous structural changes, then it 
clearly implies that modify operations are continuously completing their 
responses, which contradicts the assumption that the system is in a quiescent 
state. Further, on reaching such a state, the system will reach a state after 
which no new edges in the tree are marked. Again, this is because a modify 
operation marks at most four edges and the set of edges in the tree does not 
change any more. We call such a system state after  which neither
the set of edges nor the set of \emph{marked} edges in the tree change any 
more as a \emph{strongly quiescent state}. Note that, like quiescence, strong 
quiescence is also a stable property. 

From the above discussion, it follows that the system in a quiescent state will eventually reach a state that is strongly quiescent. Consider the search tree in such a strongly quiescent state. It can be verified that no more modify operations can now be injected into the tree, and, moreover,  
all modify operations already injected into the tree are delete operations currently ``stuck'' in either \discovery{} or \cleanup{} mode. 
%%
Now, consider a process, say $p$, that continues to take steps to execute either its own operation or another operation blocking its progress (directly or indirectly) as part of helping. 
%%
Consider the recursive chain of the \emph{helpee} operations that $p$ proceeds to help in order to complete its own operation. Let $\alpha_i$ denote the $i^{th}$ helpee operation in the chain.
It can be shown that:

\begin{lemma}
Let $\mathcal{C}_D$ denote the set of all complex delete operations already injected into the tree that are ''stuck'' in the \discovery{} mode. 
Then,
%%
\begin{enumerate}[leftmargin=*, noitemsep]
%%
\item $\alpha_1 \in \mathcal{C}_D$, and
%%
\item Suppose $p$ is currently helping $\alpha_i$  for some $i \geq 1$ and assume that $\alpha_i \in \mathcal{C}_D$. Let $\alpha_{i+1}$ denote the next operation that $p$ selects to help. Then, 
%%
\begin{enumerate*}[label=(\alph*)]
\item $\alpha_{i+1}$ exists, 
\item $\alpha_{i+1} \in \mathcal{C}_D$, and
\item the target node of $\alpha_{i+1}$ is at strictly larger depth than the target node of $\alpha_i$.
\end{enumerate*}
%%
\end{enumerate}
%%
\end{lemma}


Using the above lemma, we can easily construct a chain of distinct helpee operations whose length exceeds the number of processes---a contradiction. 
%%
This establishes that \emph{our algorithm only generates lock-free executions}.
%%

\begin{comment}
Thus, we have:


\begin{theorem}
Every execution of our algorithm is lock-free.
\end{theorem}

\end{comment}

\end{limitscope}

\chapter{Lock free concurrent binary search tree}
\label{chapter:icdcn}
In this section we evaluate \ICDCN{} against three other implementations of a concurrent BST, namely those based on:
%%
\begin{enumerate}[label=(\roman*)]
\item the lock-free internal BST by Howley and Jones~\cite{HowJon:2012:SPAA}, denoted by \HJBST{},
\item the lock-free external BST by Natarajan and Mittal~\cite{NatMit:2014:PPoPP}, denoted by \NMBST{}, and 
\item the RCU-based internal BST by Arbel and Attiya~\cite{ArbAtt:2014:PODC}, denoted by \CITRUS{}.
\end{enumerate}

% Style to select only points from #1 to #2 (inclusive)
\pgfplotsset
{
	select coords between index/.style 2 args=
	{
    x filter/.code=
		{
        \ifnum\coordindex<#1\def\pgfmathresult{}\fi
        \ifnum\coordindex>#2\def\pgfmathresult{}\fi
    }
	}
}
\begin{figure}[htp]
\centering
\begin{tikzpicture}[scale=0.9, transform shape]
	\begin{groupplot}[group style={group size= 3 by 3},height=3cm,width=4cm,xmode=log,log basis x={2},max space between ticks=20,minor tick num=1,tick label style={font=\tiny},xlabel style={font=\tiny},ylabel style={font=\tiny}, title style={font=\tiny}]
		\nextgroupplot[title=Small,ylabel={Read-Dominated},xtick=data]
				%\addplot[brown, semithick,mark=square,mark size=1] 	[select coords between index={0}{3}] table[x=keyspace,y=Mops-citrus,col sep=space]{Data/snb32/keySweep.csv};  \label{plots:CITRUS:ika}
				\addplot[red,semithick,mark=triangle,mark size=1] 	[select coords between index={0}{3}] table[x=keyspace,y=Mops-howley,col sep=space]{Data/snb32/keySweep.csv};  \label{plots:HJ-BST:ika}
				\addplot[blue,semithick,mark=asterisk,mark size=1] 	[select coords between index={0}{3}] table[x=keyspace,y=Mops-ppop,col sep=space]{Data/snb32/keySweep.csv};  	\label{plots:NM-BST:ika}
				\addplot[green, semithick,mark=o,mark size=1] 			[select coords between index={0}{3}] table[x=keyspace,y=Mops-icdcn,col sep=space]{Data/snb32/keySweep.csv};  \label{plots:icdcn:ika}
				\coordinate (top) at (rel axis cs:0,1);% coordinate at top of the first plot
		\nextgroupplot[title=Medium,xtick=data]
				%\addplot[brown, semithick,mark=square,mark size=1] 	[select coords between index={4}{7}] table[x=keyspace,y=Mops-citrus,col sep=space]{Data/snb32/keySweep.csv}; 
				\addplot[red,semithick,mark=triangle,mark size=1] 	[select coords between index={4}{7}] table[x=keyspace,y=Mops-howley,col sep=space]{Data/snb32/keySweep.csv}; 
				\addplot[blue,semithick,mark=asterisk,mark size=1] 	[select coords between index={4}{7}] table[x=keyspace,y=Mops-ppop,col sep=space]{Data/snb32/keySweep.csv};   
				\addplot[green, semithick,mark=o,mark size=1] 			[select coords between index={4}{7}] table[x=keyspace,y=Mops-icdcn,col sep=space]{Data/snb32/keySweep.csv};
		\nextgroupplot[title=Large,xtick=data]
				%\addplot[brown, semithick,mark=square,mark size=1] 	[select coords between index={8}{11}] table[x=keyspace,y=Mops-citrus,col sep=space]{Data/snb32/keySweep.csv};
				\addplot[red,semithick,mark=triangle,mark size=1] 	[select coords between index={8}{11}] table[x=keyspace,y=Mops-howley,col sep=space]{Data/snb32/keySweep.csv};
				\addplot[blue,semithick,mark=asterisk,mark size=1] 	[select coords between index={8}{11}] table[x=keyspace,y=Mops-ppop,col sep=space]{Data/snb32/keySweep.csv};  
				\addplot[green, semithick,mark=o,mark size=1] 			[select coords between index={8}{11}] table[x=keyspace,y=Mops-icdcn,col sep=space]{Data/snb32/keySweep.csv};
		\nextgroupplot[ylabel={Mixed},xtick=data]
				%\addplot[brown, semithick,mark=square,mark size=1] 	[select coords between index={12}{15}] table[x=keyspace,y=Mops-citrus,col sep=space]{Data/snb32/keySweep.csv};
				\addplot[red,semithick,mark=triangle,mark size=1] 	[select coords between index={12}{15}] table[x=keyspace,y=Mops-howley,col sep=space]{Data/snb32/keySweep.csv};
				\addplot[blue,semithick,mark=asterisk,mark size=1] 	[select coords between index={12}{15}] table[x=keyspace,y=Mops-ppop,col sep=space]{Data/snb32/keySweep.csv};  
				\addplot[green, semithick,mark=o,mark size=1] 			[select coords between index={12}{15}] table[x=keyspace,y=Mops-icdcn,col sep=space]{Data/snb32/keySweep.csv};
		\nextgroupplot[xtick=data]
				%\addplot[brown, semithick,mark=square,mark size=1] 	[select coords between index={16}{19}] table[x=keyspace,y=Mops-citrus,col sep=space]{Data/snb32/keySweep.csv};
				\addplot[red,semithick,mark=triangle,mark size=1] 	[select coords between index={16}{19}] table[x=keyspace,y=Mops-howley,col sep=space]{Data/snb32/keySweep.csv};
				\addplot[blue,semithick,mark=asterisk,mark size=1] 	[select coords between index={16}{19}] table[x=keyspace,y=Mops-ppop,col sep=space]{Data/snb32/keySweep.csv};  
				\addplot[green, semithick,mark=o,mark size=1] 			[select coords between index={16}{19}] table[x=keyspace,y=Mops-icdcn,col sep=space]{Data/snb32/keySweep.csv};
		\nextgroupplot[xtick=data]
				%\addplot[brown, semithick,mark=square,mark size=1] 	[select coords between index={20}{23}] table[x=keyspace,y=Mops-citrus,col sep=space]{Data/snb32/keySweep.csv};
				\addplot[red,semithick,mark=triangle,mark size=1] 	[select coords between index={20}{23}] table[x=keyspace,y=Mops-howley,col sep=space]{Data/snb32/keySweep.csv};
				\addplot[blue,semithick,mark=asterisk,mark size=1] 	[select coords between index={20}{23}] table[x=keyspace,y=Mops-ppop,col sep=space]{Data/snb32/keySweep.csv};  
				\addplot[green, semithick,mark=o,mark size=1] 			[select coords between index={20}{23}] table[x=keyspace,y=Mops-icdcn,col sep=space]{Data/snb32/keySweep.csv};
		\nextgroupplot[xlabel={Key Space Size},ylabel={Write-Dominated},xtick=data]
				%\addplot[brown, semithick,mark=square,mark size=1] 	[select coords between index={24}{27}] table[x=keyspace,y=Mops-citrus,col sep=space]{Data/snb32/keySweep.csv}; 
				\addplot[red,semithick,mark=triangle,mark size=1] 	[select coords between index={24}{27}] table[x=keyspace,y=Mops-howley,col sep=space]{Data/snb32/keySweep.csv};
				\addplot[blue,semithick,mark=asterisk,mark size=1] 	[select coords between index={24}{27}] table[x=keyspace,y=Mops-ppop,col sep=space]{Data/snb32/keySweep.csv};  
				\addplot[green, semithick,mark=o,mark size=1] 			[select coords between index={24}{27}] table[x=keyspace,y=Mops-icdcn,col sep=space]{Data/snb32/keySweep.csv};	
		\nextgroupplot[xlabel={Key Space Size},xtick=data]
				%\addplot[brown, semithick,mark=square,mark size=1] 	[select coords between index={28}{31}] table[x=keyspace,y=Mops-citrus,col sep=space]{Data/snb32/keySweep.csv};
				\addplot[red,semithick,mark=triangle,mark size=1] 	[select coords between index={28}{31}] table[x=keyspace,y=Mops-howley,col sep=space]{Data/snb32/keySweep.csv};
				\addplot[blue,semithick,mark=asterisk,mark size=1] 	[select coords between index={28}{31}] table[x=keyspace,y=Mops-ppop,col sep=space]{Data/snb32/keySweep.csv};  
				\addplot[green, semithick,mark=o,mark size=1] 			[select coords between index={28}{31}] table[x=keyspace,y=Mops-icdcn,col sep=space]{Data/snb32/keySweep.csv};
		\nextgroupplot[xlabel={Key Space Size},xtick=data]
				%\addplot[brown, semithick,mark=square,mark size=1] 	[select coords between index={32}{35}] table[x=keyspace,y=Mops-citrus,col sep=space]{Data/snb32/keySweep.csv}; 
				\addplot[red,semithick,mark=triangle,mark size=1] 	[select coords between index={32}{35}] table[x=keyspace,y=Mops-howley,col sep=space]{Data/snb32/keySweep.csv};
				\addplot[blue,semithick,mark=asterisk,mark size=1] 	[select coords between index={32}{35}] table[x=keyspace,y=Mops-ppop,col sep=space]{Data/snb32/keySweep.csv};  
				\addplot[green, semithick,mark=o,mark size=1] 			[select coords between index={32}{35}] table[x=keyspace,y=Mops-icdcn,col sep=space]{Data/snb32/keySweep.csv};						
				\coordinate (bot) at (rel axis cs:1,0);% coordinate at bottom of the last plot
	\end{groupplot}
	\path (top-|current bounding box.west)-- node[anchor=south,rotate=90] {\tiny System throughput (million operations/second)} (bot-|current bounding box.west);
	\path (top|-current bounding box.north)-- coordinate(legendpos) (bot|-current bounding box.north);
	\matrix[matrix of nodes, anchor=south, draw, inner sep=0.2em, draw] at ([yshift=1ex]legendpos)
  {
    \ref{plots:HJ-BST:ika}& \tiny \HJBST{} & [5pt]
    \ref{plots:NM-BST:ika}& \tiny \NMBST{} & [5pt]
    %\ref{plots:CITRUS:ika}& \tiny \CITRUS{} & [5pt]
		\ref{plots:icdcn:ika}& \tiny \ICDCN{} \\
	};
\end{tikzpicture}
%\caption[\ICDCN{} - Comparison of throughput of different algorithms - key sweep]{Comparison of system throughput of different algorithms at 32 threads. Each row represents a workload type. Each column represents a range of key space range. Higher the ratio, better the performance of the algorithm.}
\label{fig:icdcn-keySweep-absolute}
\end{figure}
% Style to select only points from #1 to #2 (inclusive)
\pgfplotsset
{
	select coords between index/.style 2 args=
	{
    x filter/.code=
		{
        \ifnum\coordindex<#1\def\pgfmathresult{}\fi
        \ifnum\coordindex>#2\def\pgfmathresult{}\fi
    }
	}
}
\begin{figure}
\centering
\nextwithlateexternal% < added
\begin{tikzpicture}
	\begin{groupplot}[group style={group size= 3 by 3},height=5.5cm,width=5.5cm,max space between ticks=20,minor tick num=1,tick label style={font=\footnotesize}]
		\nextgroupplot[title=20K keys,ylabel={Read-Dominated},xtick=data]
				%\addplot[brown, semithick,mark=square] 	[select coords between index={6}{11}] table[x=threads,y=Mops-citrus,col sep=space]{Data/snb32/threadSweep.csv};  \label{plots:CITRUS:it}
				\addplot[red,semithick,mark=triangle] 	[select coords between index={6}{11}] table[x=threads,y=Mops-howley,col sep=space]{Data/snb32/threadSweep.csv};  \label{plots:HJ-BST:it}
				\addplot[blue,semithick,mark=asterisk] 	[select coords between index={6}{11}] table[x=threads,y=Mops-ppop,col sep=space]{Data/snb32/threadSweep.csv};  \label{plots:NM-BST:it}
				\addplot[green, semithick,mark=o] 			[select coords between index={6}{11}] table[x=threads,y=Mops-icdcn,col sep=space]{Data/snb32/threadSweep.csv};  \label{plots:icdcn:it}
				\coordinate (top) at (rel axis cs:0,1);% coordinate at top of the first plot
		\nextgroupplot[title=200K keys,xtick=data]
				%\addplot[brown, semithick,mark=square] 	[select coords between index={18}{23}] table[x=threads,y=Mops-citrus,col sep=space]{Data/snb32/threadSweep.csv};  
				\addplot[red,semithick,mark=triangle] 	[select coords between index={18}{23}] table[x=threads,y=Mops-howley,col sep=space]{Data/snb32/threadSweep.csv};  
				\addplot[blue,semithick,mark=asterisk] 	[select coords between index={18}{23}] table[x=threads,y=Mops-ppop,col sep=space]{Data/snb32/threadSweep.csv};  
				\addplot[green, semithick,mark=o] 			[select coords between index={18}{23}] table[x=threads,y=Mops-icdcn,col sep=space]{Data/snb32/threadSweep.csv}; 
		\nextgroupplot[title=2M keys,xtick=data]
				%\addplot[brown, semithick,mark=square] 	[select coords between index={30}{35}] table[x=threads,y=Mops-citrus,col sep=space]{Data/snb32/threadSweep.csv};
				\addplot[red,semithick,mark=triangle] 	[select coords between index={30}{35}] table[x=threads,y=Mops-howley,col sep=space]{Data/snb32/threadSweep.csv};
				\addplot[blue,semithick,mark=asterisk] 	[select coords between index={30}{35}] table[x=threads,y=Mops-ppop,col sep=space]{Data/snb32/threadSweep.csv};
				\addplot[green, semithick,mark=o] 			[select coords between index={30}{35}] table[x=threads,y=Mops-icdcn,col sep=space]{Data/snb32/threadSweep.csv};
		\nextgroupplot[ylabel={Mixed},xtick=data]
				%\addplot[brown, semithick,mark=square] 	[select coords between index={42}{47}] table[x=threads,y=Mops-citrus,col sep=space]{Data/snb32/threadSweep.csv}; 
				\addplot[red,semithick,mark=triangle] 	[select coords between index={42}{47}] table[x=threads,y=Mops-howley,col sep=space]{Data/snb32/threadSweep.csv};
				\addplot[blue,semithick,mark=asterisk] 	[select coords between index={42}{47}] table[x=threads,y=Mops-ppop,col sep=space]{Data/snb32/threadSweep.csv};
				\addplot[green, semithick,mark=o] 			[select coords between index={42}{47}] table[x=threads,y=Mops-icdcn,col sep=space]{Data/snb32/threadSweep.csv};
		\nextgroupplot[xtick=data]
				%\addplot[brown, semithick,mark=square] 	[select coords between index={54}{59}] table[x=threads,y=Mops-citrus,col sep=space]{Data/snb32/threadSweep.csv}; 
				\addplot[red,semithick,mark=triangle] 	[select coords between index={54}{59}] table[x=threads,y=Mops-howley,col sep=space]{Data/snb32/threadSweep.csv};
				\addplot[blue,semithick,mark=asterisk] 	[select coords between index={54}{59}] table[x=threads,y=Mops-ppop,col sep=space]{Data/snb32/threadSweep.csv};
				\addplot[green, semithick,mark=o] 			[select coords between index={54}{59}] table[x=threads,y=Mops-icdcn,col sep=space]{Data/snb32/threadSweep.csv};
		\nextgroupplot[xtick=data]
				%\addplot[brown, semithick,mark=square] 	[select coords between index={66}{71}] table[x=threads,y=Mops-citrus,col sep=space]{Data/snb32/threadSweep.csv}; 
				\addplot[red,semithick,mark=triangle] 	[select coords between index={66}{71}] table[x=threads,y=Mops-howley,col sep=space]{Data/snb32/threadSweep.csv};
				\addplot[blue,semithick,mark=asterisk] 	[select coords between index={66}{71}] table[x=threads,y=Mops-ppop,col sep=space]{Data/snb32/threadSweep.csv};
				\addplot[green, semithick,mark=o] 			[select coords between index={66}{71}] table[x=threads,y=Mops-icdcn,col sep=space]{Data/snb32/threadSweep.csv};
		\nextgroupplot[xlabel={Number of Threads},ylabel={Write-Dominated},xtick=data]
				%\addplot[brown, semithick,mark=square] 	[select coords between index={78}{83}] table[x=threads,y=Mops-citrus,col sep=space]{Data/snb32/threadSweep.csv}; 
				\addplot[red,semithick,mark=triangle] 	[select coords between index={78}{83}] table[x=threads,y=Mops-howley,col sep=space]{Data/snb32/threadSweep.csv};
				\addplot[blue,semithick,mark=asterisk] 	[select coords between index={78}{83}] table[x=threads,y=Mops-ppop,col sep=space]{Data/snb32/threadSweep.csv};
				\addplot[green, semithick,mark=o] 			[select coords between index={78}{83}] table[x=threads,y=Mops-icdcn,col sep=space]{Data/snb32/threadSweep.csv};	
		\nextgroupplot[xlabel={Number of Threads},xtick=data]
				%\addplot[brown, semithick,mark=square] 	[select coords between index={90}{95}] table[x=threads,y=Mops-citrus,col sep=space]{Data/snb32/threadSweep.csv}; 
				\addplot[red,semithick,mark=triangle] 	[select coords between index={90}{95}] table[x=threads,y=Mops-howley,col sep=space]{Data/snb32/threadSweep.csv};
				\addplot[blue,semithick,mark=asterisk] 	[select coords between index={90}{95}] table[x=threads,y=Mops-ppop,col sep=space]{Data/snb32/threadSweep.csv};
				\addplot[green, semithick,mark=o] 			[select coords between index={90}{95}] table[x=threads,y=Mops-icdcn,col sep=space]{Data/snb32/threadSweep.csv};	
		\nextgroupplot[xlabel={Number of Threads},xtick=data]
				%\addplot[brown, semithick,mark=square] 	[select coords between index={102}{107}] table[x=threads,y=Mops-citrus,col sep=space]{Data/snb32/threadSweep.csv}; 
				\addplot[red,semithick,mark=triangle] 	[select coords between index={102}{107}] table[x=threads,y=Mops-howley,col sep=space]{Data/snb32/threadSweep.csv};
				\addplot[blue,semithick,mark=asterisk] 	[select coords between index={102}{107}] table[x=threads,y=Mops-ppop,col sep=space]{Data/snb32/threadSweep.csv};
				\addplot[green, semithick,mark=o] 			[select coords between index={102}{107}] table[x=threads,y=Mops-icdcn,col sep=space]{Data/snb32/threadSweep.csv};						
				\coordinate (bot) at (rel axis cs:1,0);% coordinate at bottom of the last plot
\end{groupplot}
	\path (top-|current bounding box.west)-- node[anchor=south,rotate=90] {\small System throughput (million operations/second)} (bot-|current bounding box.west);
	%\node[right,rotate=90] at (-1.4,-7.2){\small ~Throughput (million ops/sec)};
	\path (top|-current bounding box.north)-- coordinate(legendpos) (bot|-current bounding box.north);
	\matrix[matrix of nodes, anchor=south, draw, inner sep=0.2em, draw] at ([xshift=-5ex,yshift=1ex]legendpos)
  {
    \ref{plots:HJ-BST:it}& \HJBST{} & [5pt]
    \ref{plots:NM-BST:it}& \NMBST{} & [5pt]
    %\ref{plots:CITRUS:it}& \CITRUS{} & [5pt]
		\ref{plots:icdcn:it}& \ICDCN{} \\
	};
\end{tikzpicture}
\caption[\ICDCN{} - Comparison of throughput of different algorithms - thread sweep]{Comparison of system throughput of different algorithms. Each row represents a workload type and each column represents a key space size. Higher the throughput, better the performance of the algorithm.}
\label{fig:icdcn-threadSweep}
\end{figure}

The results of our experiments are shown in \figref{icdcn-keySweep-absolute} and \figref{icdcn-threadSweep}. In \figref{icdcn-keySweep-absolute}, each row represents a specific workload (read-dominated, mixed or write-dominated) and each column represents a specific key space size; \textit{small} (8Ki to 64Ki), \textit{medium} (128Ki to 1Mi) and \textit{large} (2Mi to 16Mi). \figref{icdcn-threadSweep} shows the scaling with respect to the number of threads for key space size of 2\textsuperscript{19} (512Ki). We do not show the numbers for \CITRUS{} in the graphs as it had the worst performance among all implementations (slower by a factor of four in some cases). This is not surprising as \CITRUS{} is optimized for read operations (\emph{e.g.}, 98\% reads \& 2\% updates)~\cite{ArbAtt:2014:PODC}.


As the graphs show, \ICDCN{} achieved nearly same or higher throughput than the other two implementations for medium and large key space sizes (except for medium key space size with write-dominated workload). Specifically, at 32 threads and for a read-dominated workload, \ICDCN{} had \icdcnMaximumgap{} and 15\% higher throughput than the next best performer for key space sizes of 512Ki and 1Mi, respectively. Also, at 32 threads and for a mixed workload, \ICDCN{} had 22\% and 15\% higher throughput than the next best performer for key space sizes of 1Mi and 2Mi, respectively. Overall, \ICDCN{} outperformed the next best implementation by as much as \icdcnMaximumgap{}; it outperformed \HJBST{} by as much as 38\% (mixed) and \NMBST{} by as much as 30\%(read-dominated). For large key space sizes, the overhead of traversing the tree appears to dominate the overhead of actually modifying the operation's window, and the gap between various implementations becomes smaller.

\begin{table}[htp]
\centering
\caption{Comparison of different concurrent algorithms in the absence of contention.}
\label{tab:castle-comparison}
\renewcommand{\arraystretch}{1.25}
\small
\begin{tabular}{|c|c|c|c|c|}
\hline
\multirow{2}{*}{\textbf{Algorithm}} & 
\multicolumn{2}{c|}{\textbf{\begin{tabular}[c]{@{}c@{}}Number of  Objects \\  Allocated \end{tabular}}} & 
\multicolumn{2}{c|}{\textbf{\begin{tabular}[c]{@{}c@{}}Number of Synchronization \\ Primitives Executed \end{tabular}}} \\ \cline{2-5} 
%%
 & \textbf{Insert} & \textbf{Delete} & \textbf{Insert} & \textbf{Delete} \\ \hline
%%
\multirow{2}{*}{\HJBST} & \multirow{2}{*}{2} & \multirow{2}{*}{1} & \multirow{2}{*}{3} & simple: 4 \\ 
 &  &  &  & complex: 9 \\ \hline
%%
\NMBST{} & 2 & 0 & 1 & 3 \\ \hline
%%
\multirow{2}{*}{\CASTLE} & \multirow{2}{*}{1} & \multirow{2}{*}{0} & \multirow{2}{*}{1} & simple: 3 \\ 
 &  &  &  & complex: 4 \\ \hline
\end{tabular}
\end{table}
There are several reasons why \ICDCN{} outperformed the other two implementations in many cases. First, as \tabref{icdcn-comparison} shows, our algorithm allocates fewer objects than the two other algorithms on average considering the fact that the fraction of insert operations will generally be larger than the fraction of delete operations in any realistic workload. Further, we observed in our experiments that the number of simple delete operations outnumbered the number of complex delete operations by two to one, and our algorithm does not allocate any object for a simple delete operation. Second, again as \tabref{icdcn-comparison} shows, our algorithm executes the same number of atomic instructions as in~\cite{NatMit:2014:PPoPP} for insert operations;  and, in all the cases,  executes same or fewer atomic instructions than in~\cite{HowJon:2012:SPAA}. This is important since an atomic instruction is more expensive to execute than a simple read or write instruction. Third, we observed in our experiments that \ICDCN{} had a smaller memory footprint than the other two implementations (by almost a factor of two) since it uses internal representation and allocates fewer objects. As a result, it was likely able to benefit from caching to a larger degree than \HJBST{} and \NMBST{}.

\part{Optimizations}

\chapter{Local recovery for concurrent binary search trees}
\label{chapter:localRecovery}
\begin{frame}[c]{Local recovery[PPoPP'16 Poster]}
Overview
\begin{itemize}
\item a general technique for local recovery for concurrent BSTs
\item reduces tree traversal cost during failures by restarting closer to an operation�s window
\end{itemize}
Motivation
\begin{itemize}
\item in most concurrent BSTs, execution phase of an operation have constant time complexity
\item seek phase is where an operation may end up spending most of its time (esp for large trees)
\item this technique reduces the seek time
\end{itemize}
\end{frame}

\begin{frame}[c]{Example}
\begin{figure}[htp]
\begin{tikzpicture}[scale=0.5, transform shape] 
	 \newcommand\NODEDX{1.25}
	 \newcommand\NODEDY{1.25}
	 \newcommand\SUBTREEDX{1.5}
	 \newcommand\SUBTREEDY{0.75}
	
   \node (r)	[treenode] 		                at (0, 0)       		                      	{$R$ \\ 100};
   \node (s)	[treenode, fill=black!20] 		at ([shift=({ -\NODEDX, -\NODEDY})]r)     	{$S$ \\ 10};
	 \node (t)	[treenode] 		                at ([shift=({  \NODEDX, -\NODEDY})]s)    		{$T$ \\ 90};
	 \node (u)	[treenode, fill=black!20] 	  at ([shift=({ -\NODEDX, -\NODEDY})]t)     	{$U$ \\ 20};
	 \node (v)	[treenode] 										at ([shift=({  \NODEDX, -\NODEDY})]u)     	{$V$ \\ 80};
	 \node (w)	[treenode] 										at ([shift=({ -\NODEDX, -\NODEDY})]v)     	{$W$ \\ 70};
	 \node (x)	[treenode, fill=black!20] 		at ([shift=({ -\NODEDX, -\NODEDY})]w)     	{$X$ \\ 30};
	 \node (y)	[treenode,fill=red]						at ([shift=({  \NODEDX, -\NODEDY})]x)     	{$Y$ \\ 60};
	 \node (z)	[treenode] 										at ([shift=({ -\NODEDX, -\NODEDY})]y)     	{$Z$ \\ 50};
	 \node (gl) [ground]                      at ([shift=({ -\NODEDX, -\NODEDY})]z)     	{ };
	 \node (gr) [ground]                      at ([shift=({  \NODEDX, -\NODEDY})]z)     	{ };
		
	 \node (sa) [subtree]                     at ([shift=({  \SUBTREEDX, -\SUBTREEDY})]r) {\Large $\alpha$};
	 \node (sb) [subtree]                     at ([shift=({ -\SUBTREEDX, -\SUBTREEDY})]s) {\Large $\beta$};
	 \node (sg) [subtree]                     at ([shift=({  \SUBTREEDX, -\SUBTREEDY})]t) {\Large $\gamma$};
	 \node (sd) [subtree]                     at ([shift=({ -\SUBTREEDX, -\SUBTREEDY})]u) {\Large $\delta$};
	 \node (ss) [subtree]                     at ([shift=({  \SUBTREEDX, -\SUBTREEDY})]v) {\Large $\sigma$};
	 \node (st) [subtree]                     at ([shift=({  \SUBTREEDX, -\SUBTREEDY})]w) {\Large $\tau$};
	 \node (sp) [subtree]                     at ([shift=({ -\SUBTREEDX, -\SUBTREEDY})]x) {\Large $\pi$};
	 \node (sl) [subtree]                     at ([shift=({  \SUBTREEDX, -\SUBTREEDY})]y) {\Large $\lambda$};	
	
	 \node (op) [label={right:{\large $op(50)^{z^{z^z}}$}}] at ([shift=({0.375, 0})]y) {};
	
	 \path[every node/.style={font=\sffamily\small}]
	    (0, 1)  edge[->,very thick]  node {} (r)
		  (r)     edge[->,very thick]  node {} (s)
			(s)     edge[->,very thick]  node {} (t)
			(t)     edge[->,very thick]  node {} (u)
			(u)     edge[->,very thick]  node {} (v)
			(v)     edge[->,very thick]  node {} (w)
			(w)     edge[->,very thick]  node {} (x)
			(x)     edge[->,very thick]  node {} (y)
			(y)     edge[->,very thick]  node {} (z)
			(z)     edge[->]  node {} (gl)
			(z)     edge[->]  node {} (gr)
			(r)     edge[->]  node {} (sa.north)
			(s)     edge[->]  node {} (sb.north)
			(t)     edge[->]  node {} (sg.north)
			(u)     edge[->]  node {} (sd.north)
			(v)     edge[->]  node {} (ss.north)
			(w)     edge[->]  node {} (st.north)
			(x)     edge[->]  node {} (sp.north)
			(y)     edge[->]  node {} (sl.north);		
\end{tikzpicture}
\caption{Operation $op(50)$ is suspended at node $Y$ during its traversal}
\end{figure}
\end{frame}

\begin{frame}[c]{Example}
\begin{figure}[htp]
\begin{tikzpicture}[scale=0.5, transform shape]   
	 \newcommand\NODEDX{1.25}
	 \newcommand\NODEDY{1.25}
	 \newcommand\SUBTREEDX{1.5}
	 \newcommand\SUBTREEDY{0.75}
	
   \node (r)	[treenode] 		                at (0, 0)       		                      	{$R$ \\ 100};
   \node (s)	[treenode, fill=black!20] 		at ([shift=({ -\NODEDX, -\NODEDY})]r)     	{$S$ \\ 10};
	 \node (t)	[treenode] 		                at ([shift=({  \NODEDX, -\NODEDY})]s)    		{$T$ \\ 90};
	 \node (u)	[treenode, fill=black!20] 	  at ([shift=({ -\NODEDX, -\NODEDY})]t)     	{$U$ \\ 20};
	 \node (v)	[treenode] 										at ([shift=({  \NODEDX, -\NODEDY})]u)     	{$V$ \\ 80};
	 \node (w)	[treenode] 										at ([shift=({ -\NODEDX, -\NODEDY})]v)     	{$W$ \\ 70};
	 \node (x)	[treenode, fill=black!20] 		at ([shift=({ -\NODEDX, -\NODEDY})]w)     	{$X$ \\ 30};
	 \node (y)	[treenode, fill=red]					at ([shift=({  \NODEDX, -\NODEDY})]x)     	{$Y$ \\ 60};
	 \node (z)	[treenode] 										at ([shift=({ -\NODEDX, -\NODEDY})]y)     	{$Z$ \\ 50};
	 \node (gl) [ground]                      at ([shift=({ -\NODEDX, -\NODEDY})]z)     	{ };
	 \node (gr) [ground]                      at ([shift=({  \NODEDX, -\NODEDY})]z)     	{ };

	 \node (sa) [subtree]                     at ([shift=({  \SUBTREEDX, -\SUBTREEDY})]r) {\Large $\alpha$};
	 \node (sb) [subtree]                     at ([shift=({ -\SUBTREEDX, -\SUBTREEDY})]s) {\Large $\beta$};
	 \node (sg) [subtree]                     at ([shift=({  \SUBTREEDX, -\SUBTREEDY})]t) {\Large $\gamma$};
	 \node (sd) [subtree]                     at ([shift=({ -\SUBTREEDX, -\SUBTREEDY})]u) {\Large $\delta$};
	 \node (ss) [subtree]                     at ([shift=({  \SUBTREEDX, -\SUBTREEDY})]v) {\Large $\sigma$};
	 \node (st) [subtree]                     at ([shift=({  \SUBTREEDX, -\SUBTREEDY})]w) {\Large $\tau$};
	 %% \node (sp) [subtree]                     at ([shift=({ -\SUBTREEDX, -\SUBTREEDY})]x) {\Large $\pi$};
	 \node (sp) [ground]                    	at ([shift=({ -\NODEDX, -\NODEDY})]x) { };
	 \node (sl) [subtree]                     at ([shift=({  \SUBTREEDX, -\SUBTREEDY})]y) {\Large $\lambda$};	

   \node (op) [label={right:{\large $op(50)^{z^{z^z}}$}}] at ([shift=({0.375, 0})]y) {};
	
	 \path[every node/.style={font=\sffamily\small}]
	    (0, 1)  edge[->,very thick]  node {} (r)
		  (r)     edge[->,very thick]  node {} (s)
			(s)     edge[->,very thick]  node {} (t)
			(t)     edge[->,very thick]  node {} (u)
			(u)     edge[->,very thick]  node {} (v)
			(v)     edge[->,very thick]  node {} (w)
			(w)     edge[->,very thick]  node {} (x)
			(x)     edge[->,very thick]  node {} (y)
			(y)     edge[->,very thick]  node {} (z)
			(z)     edge[->]  node {} (gl)
			(z)     edge[->]  node {} (gr)
			(r)     edge[->]  node {} (sa.north)
			(s)     edge[->]  node {} (sb.north)
			(t)     edge[->]  node {} (sg.north)
			(u)     edge[->]  node {} (sd.north)
			(v)     edge[->]  node {} (ss.north)
			(w)     edge[->]  node {} (st.north)
			(x)     edge[->]  node {} (sp)
			(y)     edge[->]  node {} (sl.north);	
\end{tikzpicture}
\caption{All keys in subtree $\pi$ are deleted one-by-one}
\end{figure}
\end{frame}

\begin{frame}[c]{Example}
\begin{figure}[htp]
\begin{tikzpicture}[scale=0.5, transform shape]
   
	 \newcommand\NODEDX{1.25}
	 \newcommand\NODEDY{1.25}
	 \newcommand\SUBTREEDX{1.5}
	 \newcommand\SUBTREEDY{0.75}
	
   \node (r)	[treenode] 		                at (0, 0)       		                      	{$R$ \\ 100};
   \node (s)	[treenode, fill=black!20] 		at ([shift=({ -\NODEDX, -\NODEDY})]r)     	{$S$ \\ 10};
	 \node (t)	[treenode] 		                at ([shift=({  \NODEDX, -\NODEDY})]s)    		{$T$ \\ 90};
	 \node (u)	[treenode, fill=black!20] 	  at ([shift=({ -\NODEDX, -\NODEDY})]t)     	{$U$ \\ 20};
	 \node (v)	[treenode] 										at ([shift=({  \NODEDX, -\NODEDY})]u)     	{$V$ \\ 80};
	 \node (w)	[treenode] 										at ([shift=({ -\NODEDX, -\NODEDY})]v)     	{$W$ \\ 70};
	 \node (x)	[treenode, fill=black!20, dotted] 		at ([shift=({ -\NODEDX, -\NODEDY})]w)     	{$X$ \\ 30};
	 \node (y)	[treenode] 										at ([shift=({  \NODEDX, -\NODEDY})]x)     	{$Y$ \\ 60};
	 \node (z)	[treenode] 										at ([shift=({ -\NODEDX, -\NODEDY})]y)     	{$Z$ \\ 50};
	 \node (gl) [ground]                      at ([shift=({ -\NODEDX, -\NODEDY})]z)     	{ };
	 \node (gr) [ground]                      at ([shift=({  \NODEDX, -\NODEDY})]z)     	{ };
		
	 \node (sa) [subtree]                     at ([shift=({  \SUBTREEDX, -\SUBTREEDY})]r) {\Large $\alpha$};
	 \node (sb) [subtree]                     at ([shift=({ -\SUBTREEDX, -\SUBTREEDY})]s) {\Large $\beta$};
	 \node (sg) [subtree]                     at ([shift=({  \SUBTREEDX, -\SUBTREEDY})]t) {\Large $\gamma$};
	 \node (sd) [subtree]                     at ([shift=({ -\SUBTREEDX, -\SUBTREEDY})]u) {\Large $\delta$};
	 \node (ss) [subtree]                     at ([shift=({  \SUBTREEDX, -\SUBTREEDY})]v) {\Large $\sigma$};
	 \node (st) [subtree]                     at ([shift=({  \SUBTREEDX, -\SUBTREEDY})]w) {\Large $\tau$};
	 %% \node (sp) [subtree]                     at ([shift=({ -\SUBTREEDX, -\SUBTREEDY})]x) {\Large $\pi$};
	 \node (sp) [ground]                    	at ([shift=({ -\NODEDX, -\NODEDY})]x) { };
	 \node (sl) [subtree]                     at ([shift=({  \SUBTREEDX, -\SUBTREEDY})]y) {\Large $\lambda$};	
	
	 \node (op) [label={right:{\large $op(50)^{z^{z^z}}$}}] at ([shift=({0.375, 0})]y) {};
	
	 \path[every node/.style={font=\sffamily\small}]
	    (0, 1)  edge[->,very thick]  node {} (r)
		  (r)     edge[->,very thick]  node {} (s)
			(s)     edge[->,very thick]  node {} (t)
			(t)     edge[->,very thick]  node {} (u)
			(u)     edge[->,very thick]  node {} (v)
			(v)     edge[->,very thick]  node {} (w)
			%% (w)     edge[->]  node {} (x)
			(w)     edge[->,very thick]  node {} (y)
			(x)     edge[->]  node {} (y)
			(y)     edge[->,very thick]  node {} (z)
			(z)     edge[->]  node {} (gl)
			(z)     edge[->]  node {} (gr)
			(r)     edge[->]  node {} (sa.north)
			(s)     edge[->]  node {} (sb.north)
			(t)     edge[->]  node {} (sg.north)
			(u)     edge[->]  node {} (sd.north)
			(v)     edge[->]  node {} (ss.north)
			(w)     edge[->]  node {} (st.north)
			(x)     edge[->]  node {} (sp)
			(y)     edge[->]  node {} (sl.north);	
\end{tikzpicture}
\caption{Key 30 is deleted (simple delete); node $X$ is removed}
\end{figure}
\end{frame}

\begin{frame}[c]{Example}
\begin{figure}[htp]
\begin{tikzpicture}[scale=0.5, transform shape]
   
	 \newcommand\NODEDX{1.25}
	 \newcommand\NODEDY{1.25}
	 \newcommand\SUBTREEDX{1.5}
	 \newcommand\SUBTREEDY{0.75}
	
   \node (r)	[treenode] 		                at (0, 0)       		                      	{$R$ \\ 100};
   \node (s)	[treenode, fill=black!20] 		at ([shift=({ -\NODEDX, -\NODEDY})]r)     	{$S$ \\ 10};
	 \node (t)	[treenode] 		                at ([shift=({  \NODEDX, -\NODEDY})]s)    		{$T$ \\ 90};
	 \node (u)	[treenode, fill=black!20] 	  at ([shift=({ -\NODEDX, -\NODEDY})]t)     	{$U$ \\ 50};
	 \node (v)	[treenode] 										at ([shift=({  \NODEDX, -\NODEDY})]u)     	{$V$ \\ 80};
	 \node (w)	[treenode] 										at ([shift=({ -\NODEDX, -\NODEDY})]v)     	{$W$ \\ 70};
	 \node (x)	[treenode, fill=black!20, dotted] 		at ([shift=({ -\NODEDX, -\NODEDY})]w)     	{$X$ \\ 30};
	 \node (y)	[treenode] 										at ([shift=({  \NODEDX, -\NODEDY})]x)     	{$Y$ \\ 60};
	 \node (z)	[treenode, dotted] 						at ([shift=({ -\NODEDX, -\NODEDY})]y)     	{$Z$ \\ 50};
	 \node (gl) [ground]                      at ([shift=({ -\NODEDX, -\NODEDY})]z)     	{ };
	 \node (gr) [ground]                      at ([shift=({  \NODEDX, -\NODEDY})]z)     	{ };
		
	 \node (sa) [subtree]                     at ([shift=({  \SUBTREEDX, -\SUBTREEDY})]r) {\Large $\alpha$};
	 \node (sb) [subtree]                     at ([shift=({ -\SUBTREEDX, -\SUBTREEDY})]s) {\Large $\beta$};
	 \node (sg) [subtree]                     at ([shift=({  \SUBTREEDX, -\SUBTREEDY})]t) {\Large $\gamma$};
	 \node (sd) [subtree]                     at ([shift=({ -\SUBTREEDX, -\SUBTREEDY})]u) {\Large $\delta$};
	 \node (ss) [subtree]                     at ([shift=({  \SUBTREEDX, -\SUBTREEDY})]v) {\Large $\sigma$};
	 \node (st) [subtree]                     at ([shift=({  \SUBTREEDX, -\SUBTREEDY})]w) {\Large $\tau$};
	 %% \node (sp) [subtree]                     at ([shift=({ -\SUBTREEDX, -\SUBTREEDY})]x) {\Large $\pi$};
	 \node (sp) [ground]                    	at ([shift=({ -\NODEDX, -\NODEDY})]x) { };
	 \node (sl) [subtree]                     at ([shift=({  \SUBTREEDX, -\SUBTREEDY})]y) {\Large $\lambda$};	
	
	 \node (op) [label={right:{\large $op(50)^{z^{z^z}}$}}] at ([shift=({0.375, 0})]y) {};
	
	 \path[every node/.style={font=\sffamily\small}]
	    (0, 1)  edge[->,very thick]  node {} (r)
		  (r)     edge[->,very thick]  node {} (s)
			(s)     edge[->,very thick]  node {} (t)
			(t)     edge[->,very thick]  node {} (u)
			(u)     edge[->,very thick]  node {} (v)
			(v)     edge[->,very thick]  node {} (w)
			%% (w)     edge[->]  node {} (x)
			(w)     edge[->,very thick]  node {} (y)
			(x)     edge[->]  node {} (y)
			%% (y)     edge[->]  node {} (z)
			(y)     edge[->, very thick]  node {} (gr)
			(z)     edge[->]  node {} (gl)
			(z)     edge[->]  node {} (gr)
			(r)     edge[->]  node {} (sa.north)
			(s)     edge[->]  node {} (sb.north)
			(t)     edge[->]  node {} (sg.north)
			(u)     edge[->]  node {} (sd.north)
			(v)     edge[->]  node {} (ss.north)
			(w)     edge[->]  node {} (st.north)
			(x)     edge[->]  node {} (sp)
			(y)     edge[->]  node {} (sl.north);		
\end{tikzpicture}
\caption{Key 20 is deleted (complex delete); key 20 is replaced with key 50 in node $U$ and node $Z$ is removed}
\end{figure}
\end{frame}

\begin{frame}[c]{Traversal Stack}
\begin{itemize}
\item a stack to keep track of anchor nodes of all nodes in the traversal path
\item reduces tree traversal cost during failures by restarting closer to an operation�s window
\end{itemize}
\end{frame}

\begin{frame}[c]{Traversal Stack}
\begin{figure}[htp]
\begin{tikzpicture}[scale=0.5, transform shape] 
	 \newcommand\NODEDX{1.25}
	 \newcommand\NODEDY{1.25}
	 \newcommand\SUBTREEDX{1.5}
	 \newcommand\SUBTREEDY{0.75}
	
   \node (r)	[treenode, fill=black!20] 		at (0, 0)       		                      	{$R$ \\  -$\infty$};
   \node (s)	[treenode] 										at ([shift=({ \NODEDX, -\NODEDY})]r)     	  {$S$ \\  $\infty$};
	 \node (t)	[treenode] 		                at ([shift=({  -\NODEDX, -\NODEDY})]s)    	{$T$ \\ 90};
	 \node (u)	[treenode, fill=black!20] 	  at ([shift=({ -\NODEDX, -\NODEDY})]t)     	{$U$ \\ 20};
	 \node (v)	[treenode] 										at ([shift=({  \NODEDX, -\NODEDY})]u)     	{$V$ \\ 80};
	 \node (w)	[treenode] 										at ([shift=({ -\NODEDX, -\NODEDY})]v)     	{$W$ \\ 70};
	 \node (x)	[treenode, fill=black!20] 		at ([shift=({ -\NODEDX, -\NODEDY})]w)     	{$X$ \\ 30};
	 \node (y)	[treenode] 										at ([shift=({  \NODEDX, -\NODEDY})]x)     	{$Y$ \\ 60};
	 \node (z)	[treenode] 										at ([shift=({ -\NODEDX, -\NODEDY})]y)     	{$Z$ \\ 50};
	 \node (gl) [ground]                      at ([shift=({ -\NODEDX, -\NODEDY})]z)     	{ };
	 \node (gr) [ground]                      at ([shift=({  \NODEDX, -\NODEDY})]z)     	{ };
		
	 \node (sa) [ground]                      at ([shift=({ -\SUBTREEDX, -\SUBTREEDY})]r) { };
	 \node (sb) [ground]                      at ([shift=({ \SUBTREEDX, -\SUBTREEDY})]s) 	{ };
	 \node (sg) [subtree]                     at ([shift=({  \SUBTREEDX, -\SUBTREEDY})]t) {\Large $\gamma$};
	 \node (sd) [subtree]                     at ([shift=({ -\SUBTREEDX, -\SUBTREEDY})]u) {\Large $\delta$};
	 \node (ss) [subtree]                     at ([shift=({  \SUBTREEDX, -\SUBTREEDY})]v) {\Large $\sigma$};
	 \node (st) [subtree]                     at ([shift=({  \SUBTREEDX, -\SUBTREEDY})]w) {\Large $\tau$};
	 \node (sp) [subtree]                     at ([shift=({ -\SUBTREEDX, -\SUBTREEDY})]x) {\Large $\pi$};
	 \node (sl) [subtree]                     at ([shift=({  \SUBTREEDX, -\SUBTREEDY})]y) {\Large $\lambda$};	
	
	 \node (op) [label={right:{\large $op(50) is here$}}] at ([shift=({0.5, 0})]s) {};
	
	 \path[every node/.style={font=\sffamily\small}]
	    %(0, 1)  edge[->,very thick]  node {} (r)
		  (r)     edge[->,very thick]  node {} (s)
			(s)     edge[->,very thick]  node {} (t)
			(t)     edge[->,very thick]  node {} (u)
			(u)     edge[->,very thick]  node {} (v)
			(v)     edge[->,very thick]  node {} (w)
			(w)     edge[->,very thick]  node {} (x)
			(x)     edge[->,very thick]  node {} (y)
			(y)     edge[->,very thick]  node {} (z)
			(z)     edge[->]  node {} (gl)
			(z)     edge[->]  node {} (gr)
			(r)     edge[->]  node {} (sa)
			(s)     edge[->]  node {} (sb)
			(t)     edge[->]  node {} (sg.north)
			(u)     edge[->]  node {} (sd.north)
			(v)     edge[->]  node {} (ss.north)
			(w)     edge[->]  node {} (st.north)
			(x)     edge[->]  node {} (sp.north)
			(y)     edge[->]  node {} (sl.north);		
\end{tikzpicture}
%\qquad
%\begin{tikzpicture}[scale=1.0, transform shape]
%\node[stack=9]  {
%0,\nodepart{one}Z,left,6
%\nodepart{two}Y,rigt,6
%\nodepart{three}X,left,3
%\nodepart{four}W,left,3
%\nodepart{five}V,right,3
%\nodepart{six}U,left,1
%\nodepart{seven}T,right,1
%\nodepart{eight}S,right,0
%\nodepart{nine}R,right,-1
%};
%\end{tikzpicture}
\qquad
\begin{tikzpicture}[scale=0.5, transform shape]
  \stacktop{} \cellptr{top of stack}
	\separator
	\cell{\texttt{S,R}}        \cellcomL{1} \coordinate () at (currentcell.east);
  \separator
	\cell{\texttt{R,null}}     \cellcomL{0} \coordinate () at (currentcell.east);
  \separator
\end{tikzpicture}
\caption{Operation $op(50)$ starting at R and ending at Z along with the stack}
\end{figure}
\end{frame}
\begin{frame}[c]{Traversal Stack}
\begin{figure}[htp]
\begin{tikzpicture}[scale=0.5, transform shape] 
	 \newcommand\NODEDX{1.25}
	 \newcommand\NODEDY{1.25}
	 \newcommand\SUBTREEDX{1.5}
	 \newcommand\SUBTREEDY{0.75}
	
   \node (r)	[treenode, fill=black!20] 		at (0, 0)       		                      	{$R$ \\  -$\infty$};
   \node (s)	[treenode] 										at ([shift=({ \NODEDX, -\NODEDY})]r)     	  {$S$ \\  $\infty$};
	 \node (t)	[treenode, fill=red]          at ([shift=({  -\NODEDX, -\NODEDY})]s)    	{$T$ \\ 90};
	 \node (u)	[treenode, fill=black!20] 	  at ([shift=({ -\NODEDX, -\NODEDY})]t)     	{$U$ \\ 20};
	 \node (v)	[treenode] 										at ([shift=({  \NODEDX, -\NODEDY})]u)     	{$V$ \\ 80};
	 \node (w)	[treenode] 										at ([shift=({ -\NODEDX, -\NODEDY})]v)     	{$W$ \\ 70};
	 \node (x)	[treenode, fill=black!20] 		at ([shift=({ -\NODEDX, -\NODEDY})]w)     	{$X$ \\ 30};
	 \node (y)	[treenode] 										at ([shift=({  \NODEDX, -\NODEDY})]x)     	{$Y$ \\ 60};
	 \node (z)	[treenode] 										at ([shift=({ -\NODEDX, -\NODEDY})]y)     	{$Z$ \\ 50};
	 \node (gl) [ground]                      at ([shift=({ -\NODEDX, -\NODEDY})]z)     	{ };
	 \node (gr) [ground]                      at ([shift=({  \NODEDX, -\NODEDY})]z)     	{ };
		
	 \node (sa) [ground]                      at ([shift=({ -\SUBTREEDX, -\SUBTREEDY})]r) { };
	 \node (sb) [ground]                      at ([shift=({ \SUBTREEDX, -\SUBTREEDY})]s) 	{ };
	 \node (sg) [subtree]                     at ([shift=({  \SUBTREEDX, -\SUBTREEDY})]t) {\Large $\gamma$};
	 \node (sd) [subtree]                     at ([shift=({ -\SUBTREEDX, -\SUBTREEDY})]u) {\Large $\delta$};
	 \node (ss) [subtree]                     at ([shift=({  \SUBTREEDX, -\SUBTREEDY})]v) {\Large $\sigma$};
	 \node (st) [subtree]                     at ([shift=({  \SUBTREEDX, -\SUBTREEDY})]w) {\Large $\tau$};
	 \node (sp) [subtree]                     at ([shift=({ -\SUBTREEDX, -\SUBTREEDY})]x) {\Large $\pi$};
	 \node (sl) [subtree]                     at ([shift=({  \SUBTREEDX, -\SUBTREEDY})]y) {\Large $\lambda$};	
	
	 \node (op) [label={left:{\large $op(50)$}}] at ([shift=({-0.5, 0})]t) {};
	
	 \path[every node/.style={font=\sffamily\small}]
	    %(0, 1)  edge[->,very thick]  node {} (r)
		  (r)     edge[->,very thick]  node {} (s)
			(s)     edge[->,very thick]  node {} (t)
			(t)     edge[->,very thick]  node {} (u)
			(u)     edge[->,very thick]  node {} (v)
			(v)     edge[->,very thick]  node {} (w)
			(w)     edge[->,very thick]  node {} (x)
			(x)     edge[->,very thick]  node {} (y)
			(y)     edge[->,very thick]  node {} (z)
			(z)     edge[->]  node {} (gl)
			(z)     edge[->]  node {} (gr)
			(r)     edge[->]  node {} (sa)
			(s)     edge[->]  node {} (sb)
			(t)     edge[->]  node {} (sg.north)
			(u)     edge[->]  node {} (sd.north)
			(v)     edge[->]  node {} (ss.north)
			(w)     edge[->]  node {} (st.north)
			(x)     edge[->]  node {} (sp.north)
			(y)     edge[->]  node {} (sl.north);		
\end{tikzpicture}
%\qquad
%\begin{tikzpicture}[scale=1.0, transform shape]
%\node[stack=9]  {
%0,\nodepart{one}Z,left,6
%\nodepart{two}Y,rigt,6
%\nodepart{three}X,left,3
%\nodepart{four}W,left,3
%\nodepart{five}V,right,3
%\nodepart{six}U,left,1
%\nodepart{seven}T,right,1
%\nodepart{eight}S,right,0
%\nodepart{nine}R,right,-1
%};
%\end{tikzpicture}
\qquad
\begin{tikzpicture}[scale=0.5, transform shape]
  \stacktop{} \cellptr{top of stack}
	\separator
	\cell{\Large \texttt{T,R}}        \cellcomL{2} \coordinate () at (currentcell.east);
  \separator
	\cell{\Large \texttt{S,R}}        \cellcomL{1} \coordinate () at (currentcell.east);
  \separator
	\cell{\Large \texttt{R,null}}     \cellcomL{0} \coordinate () at (currentcell.east);
  \separator
\end{tikzpicture}
%\caption{Operation $op(50)$ starting at R and suspended at Y along with the stack}
\end{figure}
\end{frame}


\begin{frame}[c]{Search}
search operations do not restart
\begin{figure}[htp]
\centering
{
	\begin{tikzpicture}[scale=0.5, transform shape]
	\node (p0) [] {search(K)};
	\node (p1) [process, below of=p0, text width=4cm] {Do a binary search for key K in the tree};
	\node (p2) [process, below of=p1, yshift=-1.5cm, text width=4.5cm] {Examine anchor node $A$ of top entry in the stack};
	\node (p3) [decision, below of=p2, yshift=-1.5cm, text width=2cm] {is anchor node marked?};
	\node (p4) [process, right of=p3, xshift=4cm, text width=4.5cm] {pop all entries upto anchor node $A$};
	\node (retT) [process, right of=p1, xshift=4cm, text width=1cm, minimum width=1cm] {return true};
	\node (retF) [process, left of=p2, xshift=-5cm, text width=1cm, minimum width=1cm] {return false};

	\draw [arrow] (p1) -- node[anchor=west] {K not found} (p2);
	\draw [arrow] (p1) -- node[anchor=south] {K found} (retT);
	\draw [arrow] (p2.east) -| node[anchor=north,pos=0.5] {K = A.key}    (retT.south);
	\draw [arrow] (p2.west) -- node (temp) [anchor=north,pos=0.5] {K $<$ A.key}  (retF.east);
	\draw [arrow] (p2) -- node[anchor=east] {K $>$ A.key} (p3);
	\draw [arrow] (p3.west) -| (retF.south) node[below, pos=0.5]{No}  (retF.south);
	\draw [arrow] (p3.east) -- node[anchor=north]{Yes} (p4.west);
	\draw [arrow] (p4.north) -- (p2.south);
	\node (temp1) [above of=temp,xshift=0.5cm,yshift=0.4cm] {(Aold.key $<$ K $<$ Anew.key)};
	\end{tikzpicture}
}
\caption{Sequence of steps in a search operation}
\end{figure}
\end{frame}

\begin{frame}[c]{Search}
\begin{figure}[htp]
\begin{tikzpicture}[scale=0.5, transform shape] 
	 \newcommand\NODEDX{1.25}
	 \newcommand\NODEDY{1.25}
	 \newcommand\SUBTREEDX{1.5}
	 \newcommand\SUBTREEDY{0.75}
	
   \node (r)	[treenode, fill=black!20] 		at (0, 0)       		                      	{$R$ \\  -$\infty$};
   \node (s)	[treenode] 										at ([shift=({ \NODEDX, -\NODEDY})]r)     	  {$S$ \\  $\infty$};
	 \node (t)	[treenode, fill=red]          at ([shift=({  -\NODEDX, -\NODEDY})]s)    	{$T$ \\ 90};
	 \node (u)	[treenode, fill=black!20] 	  at ([shift=({ -\NODEDX, -\NODEDY})]t)     	{$U$ \\ 20};
	 \node (v)	[treenode] 										at ([shift=({  \NODEDX, -\NODEDY})]u)     	{$V$ \\ 80};
	 \node (w)	[treenode] 										at ([shift=({ -\NODEDX, -\NODEDY})]v)     	{$W$ \\ 70};
	 \node (x)	[treenode, fill=black!20] 		at ([shift=({ -\NODEDX, -\NODEDY})]w)     	{$X$ \\ 30};
	 \node (y)	[treenode] 										at ([shift=({  \NODEDX, -\NODEDY})]x)     	{$Y$ \\ 60};
	 \node (z)	[treenode] 										at ([shift=({ -\NODEDX, -\NODEDY})]y)     	{$Z$ \\ 50};
	 \node (gl) [ground]                      at ([shift=({ -\NODEDX, -\NODEDY})]z)     	{ };
	 \node (gr) [ground]                      at ([shift=({  \NODEDX, -\NODEDY})]z)     	{ };
		
	 \node (sa) [ground]                      at ([shift=({ -\SUBTREEDX, -\SUBTREEDY})]r) { };
	 \node (sb) [ground]                      at ([shift=({ \SUBTREEDX, -\SUBTREEDY})]s) 	{ };
	 \node (sg) [subtree]                     at ([shift=({  \SUBTREEDX, -\SUBTREEDY})]t) {\Large $\gamma$};
	 \node (sd) [subtree]                     at ([shift=({ -\SUBTREEDX, -\SUBTREEDY})]u) {\Large $\delta$};
	 \node (ss) [subtree]                     at ([shift=({  \SUBTREEDX, -\SUBTREEDY})]v) {\Large $\sigma$};
	 \node (st) [subtree]                     at ([shift=({  \SUBTREEDX, -\SUBTREEDY})]w) {\Large $\tau$};
	 \node (sp) [subtree]                     at ([shift=({ -\SUBTREEDX, -\SUBTREEDY})]x) {\Large $\pi$};
	 \node (sl) [subtree]                     at ([shift=({  \SUBTREEDX, -\SUBTREEDY})]y) {\Large $\lambda$};	
	
	 \node (op) [label={left:{\large $op(50)$}}] at ([shift=({-0.5, 0})]t) {};
	
	 \path[every node/.style={font=\sffamily\small}]
	    %(0, 1)  edge[->,very thick]  node {} (r)
		  (r)     edge[->,very thick]  node {} (s)
			(s)     edge[->,very thick]  node {} (t)
			(t)     edge[->,very thick]  node {} (u)
			(u)     edge[->,very thick]  node {} (v)
			(v)     edge[->,very thick]  node {} (w)
			(w)     edge[->,very thick]  node {} (x)
			(x)     edge[->,very thick]  node {} (y)
			(y)     edge[->,very thick]  node {} (z)
			(z)     edge[->]  node {} (gl)
			(z)     edge[->]  node {} (gr)
			(r)     edge[->]  node {} (sa)
			(s)     edge[->]  node {} (sb)
			(t)     edge[->]  node {} (sg.north)
			(u)     edge[->]  node {} (sd.north)
			(v)     edge[->]  node {} (ss.north)
			(w)     edge[->]  node {} (st.north)
			(x)     edge[->]  node {} (sp.north)
			(y)     edge[->]  node {} (sl.north);		
\end{tikzpicture}
%\qquad
%\begin{tikzpicture}[scale=1.0, transform shape]
%\node[stack=9]  {
%0,\nodepart{one}Z,left,6
%\nodepart{two}Y,rigt,6
%\nodepart{three}X,left,3
%\nodepart{four}W,left,3
%\nodepart{five}V,right,3
%\nodepart{six}U,left,1
%\nodepart{seven}T,right,1
%\nodepart{eight}S,right,0
%\nodepart{nine}R,right,-1
%};
%\end{tikzpicture}
\qquad
\begin{tikzpicture}[scale=0.5, transform shape]
  \stacktop{} \cellptr{top of stack}
	\separator
	\cell{\Large \texttt{T,R}}        \cellcomL{2} \coordinate () at (currentcell.east);
  \separator
	\cell{\Large \texttt{S,R}}        \cellcomL{1} \coordinate () at (currentcell.east);
  \separator
	\cell{\Large \texttt{R,null}}     \cellcomL{0} \coordinate () at (currentcell.east);
  \separator
\end{tikzpicture}
%\caption{Operation $op(50)$ starting at R and suspended at Y along with the stack}
\end{figure}
\end{frame}
\begin{frame}[c]{Search}
\begin{figure}[htp]
\begin{tikzpicture}[scale=0.33, transform shape]
   
	 \newcommand\NODEDX{1.25}
	 \newcommand\NODEDY{1.25}
	 \newcommand\SUBTREEDX{1.5}
	 \newcommand\SUBTREEDY{0.75}
	
	 \node (r)	[treenode, fill=black!20] 		at (0, 0)       		                      	{$R$ \\  -$\infty$};
   \node (s)	[treenode] 										at ([shift=({ \NODEDX, -\NODEDY})]r)     	  {$S$ \\  $\infty$};
	 \node (t)	[treenode] 		                at ([shift=({  -\NODEDX, -\NODEDY})]s)    	{$T$ \\ 90};
	 \node (u)	[treenode, fill=black!20] 	  at ([shift=({ -\NODEDX, -\NODEDY})]t)     	{$U$ \\ 50};
	 \node (v)	[treenode] 										at ([shift=({  \NODEDX, -\NODEDY})]u)     	{$V$ \\ 80};
	 \node (w)	[treenode] 										at ([shift=({ -\NODEDX, -\NODEDY})]v)     	{$W$ \\ 70};
	 \node (x)	[treenode, fill=black!20, dotted] 		at ([shift=({ -\NODEDX, -\NODEDY})]w)     	{$X$ \\ 30};
	 \node (y)	[treenode] 										at ([shift=({  \NODEDX, -\NODEDY})]x)     	{$Y$ \\ 60};
	 \node (z)	[treenode, dotted] 						at ([shift=({ -\NODEDX, -\NODEDY})]y)     	{$Z$ \\ 50};
	 \node (gl) [ground]                      at ([shift=({ -\NODEDX, -\NODEDY})]z)     	{ };
	 \node (gr) [ground]                      at ([shift=({  \NODEDX, -\NODEDY})]z)     	{ };
		
	 \node (sa) [ground]                      at ([shift=({ -\SUBTREEDX, -\SUBTREEDY})]r) { };
	 \node (sb) [ground]                      at ([shift=({ \SUBTREEDX, -\SUBTREEDY})]s) 	{ };
	 \node (sg) [subtree]                     at ([shift=({  \SUBTREEDX, -\SUBTREEDY})]t) {\Large $\gamma$};
	 \node (sd) [subtree]                     at ([shift=({ -\SUBTREEDX, -\SUBTREEDY})]u) {\Large $\delta$};
	 \node (ss) [subtree]                     at ([shift=({  \SUBTREEDX, -\SUBTREEDY})]v) {\Large $\sigma$};
	 \node (st) [subtree]                     at ([shift=({  \SUBTREEDX, -\SUBTREEDY})]w) {\Large $\tau$};
	 %% \node (sp) [subtree]                     at ([shift=({ -\SUBTREEDX, -\SUBTREEDY})]x) {\Large $\pi$};
	 \node (sp) [ground]                    	at ([shift=({ -\NODEDX, -\NODEDY})]x) { };
	 \node (sl) [subtree]                     at ([shift=({  \SUBTREEDX, -\SUBTREEDY})]y) {\Large $\lambda$};	
	
	 \node (op) [label={right:{\large $search(50)$}}] at ([shift=({0.375, 0})]gr) {};
	
	 \path[every node/.style={font=\sffamily\small}]
	    %(0, 1)  edge[->,very thick]  node {} (r)
		  (r)     edge[->,very thick]  node {} (s)
			(s)     edge[->,very thick]  node {} (t)
			(t)     edge[->,very thick]  node {} (u)
			(u)     edge[->,very thick]  node {} (v)
			(v)     edge[->,very thick]  node {} (w)
			%% (w)     edge[->]  node {} (x)
			(w)     edge[->,very thick]  node {} (y)
			(x)     edge[->]  node {} (y)
			%% (y)     edge[->]  node {} (z)
			(y)     edge[->, very thick]  node {} (gr)
			(z)     edge[->]  node {} (gl)
			(z)     edge[->]  node {} (gr)
			(r)     edge[->]  node {} (sa)
			(s)     edge[->]  node {} (sb)
			(t)     edge[->]  node {} (sg.north)
			(u)     edge[->]  node {} (sd.north)
			(v)     edge[->]  node {} (ss.north)
			(w)     edge[->]  node {} (st.north)
			(x)     edge[->]  node {} (sp)
			(y)     edge[->]  node {} (sl.north);		
\end{tikzpicture}
\quad
\begin{tikzpicture}[scale=0.28, transform shape]
  \stacktop{} \cellptr{top}
	\separator
	\cell{\Large \texttt{Y,X}}        \cellcomL{7} \coordinate () at (currentcell.east);
  \separator
	\cell{\Large \texttt{X,U}}        \cellcomL{6} \coordinate () at (currentcell.east);
  \separator
	\cell{\Large \texttt{W,U}}        \cellcomL{5} \coordinate () at (currentcell.east);
  \separator
	\cell{\Large \texttt{V,U}}        \cellcomL{4} \coordinate () at (currentcell.east);
  \separator
	\cell{\Large \texttt{U,R}}        \cellcomL{3} \coordinate () at (currentcell.east);
  \separator
	\cell{\Large \texttt{T,R}}        \cellcomL{2} \coordinate () at (currentcell.east);
  \separator
	\cell{\Large \texttt{S,R}}        \cellcomL{1} \coordinate () at (currentcell.east);
  \separator
	\cell{\Large \texttt{R,null}}     \cellcomL{0} \coordinate () at (currentcell.east);
  \separator
\end{tikzpicture}
\quad
\begin{tikzpicture}[scale=0.4, transform shape]
	\node (p0) [] {search(K)};
	\node (p1) [process, below of=p0, text width=4cm] {Do a binary search for key K in the tree};
	\node (p2) [process, below of=p1, yshift=-1.5cm, text width=4.5cm,fill=black!20] {Examine anchor node $A$ of top entry in the stack};
	\node (p3) [decision, below of=p2, yshift=-1.5cm, text width=2cm,fill=black!20] {is anchor node marked?};
	\node (p4) [process, right of=p3, xshift=4cm, text width=4.5cm,fill=black!20] {pop all entries upto anchor node $A$};
	\node (retT) [process, right of=p1, xshift=4cm, text width=1cm, minimum width=1cm] {return true};
	\node (retF) [process, left of=p2, xshift=-5cm, text width=1cm, minimum width=1cm] {return false};

	\draw [arrow] (p1) -- node[anchor=west] {K not found} (p2);
	\draw [arrow] (p1) -- node[anchor=south] {K found} (retT);
	\draw [arrow] (p2.east) -| node[anchor=north,pos=0.5] {K = A.key}    (retT.south);
	\draw [arrow] (p2.west) -- node (temp) [anchor=north,pos=0.5] {K $<$ A.key}  (retF.east);
	\draw [arrow] (p2) -- node[anchor=east] {K $>$ A.key} (p3);
	\draw [arrow] (p3.west) -| (retF.south) node[below, pos=0.5]{No}  (retF.south);
	\draw [arrow] (p3.east) -- node[anchor=north]{Yes} (p4.west);
	\draw [arrow] (p4.north) -- (p2.south);
	\node (temp1) [above of=temp,xshift=0.5cm,yshift=0.4cm] {(A.oldKey $<$ K $<$ A.newKey)};
	\end{tikzpicture}
\caption{Key 30 is deleted;key 20 is deleted $\&$ replaced with key 50 in node $U$ and node $Z$ is removed}
\end{figure}
\end{frame}
\begin{frame}[c]{Search}
\begin{figure}[htp]
\begin{tikzpicture}[scale=0.5, transform shape]
   
	 \newcommand\NODEDX{1.25}
	 \newcommand\NODEDY{1.25}
	 \newcommand\SUBTREEDX{1.5}
	 \newcommand\SUBTREEDY{0.75}
	
	 \node (r)	[treenode, fill=black!20] 		at (0, 0)       		                      	{$R$ \\  -$\infty$};
   \node (s)	[treenode] 										at ([shift=({ \NODEDX, -\NODEDY})]r)     	  {$S$ \\  $\infty$};
	 \node (t)	[treenode] 		                at ([shift=({  -\NODEDX, -\NODEDY})]s)    	{$T$ \\ 90};
	 \node (u)	[treenode, fill=black!20] 	  at ([shift=({ -\NODEDX, -\NODEDY})]t)     	{$U$ \\ 50};
	 \node (v)	[treenode] 										at ([shift=({  \NODEDX, -\NODEDY})]u)     	{$V$ \\ 80};
	 \node (w)	[treenode] 										at ([shift=({ -\NODEDX, -\NODEDY})]v)     	{$W$ \\ 70};
	 \node (x)	[treenode, fill=black!20, dotted] 		at ([shift=({ -\NODEDX, -\NODEDY})]w)     	{$X$ \\ 30};
	 \node (y)	[treenode] 										at ([shift=({  \NODEDX, -\NODEDY})]x)     	{$Y$ \\ 60};
	 \node (z)	[treenode, dotted] 						at ([shift=({ -\NODEDX, -\NODEDY})]y)     	{$Z$ \\ 50};
	 \node (gl) [ground]                      at ([shift=({ -\NODEDX, -\NODEDY})]z)     	{ };
	 \node (gr) [ground]                      at ([shift=({  \NODEDX, -\NODEDY})]z)     	{ };
		
	 \node (sa) [ground]                      at ([shift=({ -\SUBTREEDX, -\SUBTREEDY})]r) { };
	 \node (sb) [ground]                      at ([shift=({ \SUBTREEDX, -\SUBTREEDY})]s) 	{ };
	 \node (sg) [subtree]                     at ([shift=({  \SUBTREEDX, -\SUBTREEDY})]t) {\Large $\gamma$};
	 \node (sd) [subtree]                     at ([shift=({ -\SUBTREEDX, -\SUBTREEDY})]u) {\Large $\delta$};
	 \node (ss) [subtree]                     at ([shift=({  \SUBTREEDX, -\SUBTREEDY})]v) {\Large $\sigma$};
	 \node (st) [subtree]                     at ([shift=({  \SUBTREEDX, -\SUBTREEDY})]w) {\Large $\tau$};
	 %% \node (sp) [subtree]                     at ([shift=({ -\SUBTREEDX, -\SUBTREEDY})]x) {\Large $\pi$};
	 \node (sp) [ground]                    	at ([shift=({ -\NODEDX, -\NODEDY})]x) { };
	 \node (sl) [subtree]                     at ([shift=({  \SUBTREEDX, -\SUBTREEDY})]y) {\Large $\lambda$};	
	
	 \node (op) [label={left:{\large $op(50) is here$}}] at ([shift=({-0.5, 0})]w) {};
	
	 \path[every node/.style={font=\sffamily\small}]
	    %(0, 1)  edge[->,very thick]  node {} (r)
		  (r)     edge[->,very thick]  node {} (s)
			(s)     edge[->,very thick]  node {} (t)
			(t)     edge[->,very thick]  node {} (u)
			(u)     edge[->,very thick]  node {} (v)
			(v)     edge[->,very thick]  node {} (w)
			%% (w)     edge[->]  node {} (x)
			(w)     edge[->,very thick]  node {} (y)
			(x)     edge[->]  node {} (y)
			%% (y)     edge[->]  node {} (z)
			(y)     edge[->, very thick]  node {} (gr)
			(z)     edge[->]  node {} (gl)
			(z)     edge[->]  node {} (gr)
			(r)     edge[->]  node {} (sa)
			(s)     edge[->]  node {} (sb)
			(t)     edge[->]  node {} (sg.north)
			(u)     edge[->]  node {} (sd.north)
			(v)     edge[->]  node {} (ss.north)
			(w)     edge[->]  node {} (st.north)
			(x)     edge[->]  node {} (sp)
			(y)     edge[->]  node {} (sl.north);		
\end{tikzpicture}
\qquad
\begin{tikzpicture}[scale=0.5, transform shape]
  \stacktop{} \cellptr{top of stack}
	\separator
	\cell{\texttt{W,U}}        \cellcomL{5} \coordinate () at (currentcell.east);
  \separator
	\cell{\texttt{V,U}}        \cellcomL{4} \coordinate () at (currentcell.east);
  \separator
	\cell{\texttt{U,R}}        \cellcomL{3} \coordinate () at (currentcell.east);
  \separator
	\cell{\texttt{T,R}}        \cellcomL{2} \coordinate () at (currentcell.east);
  \separator
	\cell{\texttt{S,R}}        \cellcomL{1} \coordinate () at (currentcell.east);
  \separator
	\cell{\texttt{R,null}}     \cellcomL{0} \coordinate () at (currentcell.east);
  \separator
\end{tikzpicture}
\caption{Pop upto marked anchor node $X$. Top of stack is now $W$. Examine anchor node $U$}
\end{figure}
\end{frame}

\begin{frame}[c]{Insert}
An insert operation needs to restart only if one of the anchor nodes in the path has become inconsistent
\begin{figure}[htp]
\centering
{
	\begin{tikzpicture}[scale=0.5, transform shape]
	\node (p0) [] {insert(K)};
	\node (p1) [process, below of=p0, text width=4cm] {Do a binary search for key K in the tree};
	\node (p2) [process, below of=p1, yshift=-1.5cm, text width=4.5cm] {Examine anchor node $A$ of top entry in the stack};
	\node (p3) [decision, below of=p2, yshift=-1.5cm, text width=2cm] {is anchor node marked?};
	\node (p4) [process, right of=p3, xshift=4cm, text width=4.5cm] {pop all entries upto anchor node $A$};
	\node (retF) [process, right of=p1, xshift=4cm, text width=1cm, minimum width=1cm] {return false};
	\node (retT) [process, left of=p2, xshift=-6cm, text width=4.5cm, minimum width=1cm] {discard suffix of the path after anchor node and find a restart point};
	\node (p5) [process, left of=p3, xshift=-6cm, text width=4.5cm, minimum width=1cm] {traversal terminates. Terminal node is returned as the injection point};

	\draw [arrow] (p1) -- node[anchor=west] {K not found} (p2);
	\draw [arrow] (p1) -- node[anchor=south] {K found} (retF);
	\draw [arrow] (p2.east) -| node[anchor=north,pos=0.5] {K = A.key}    (retF.south);
	\draw [arrow] (p2.west) -- node[anchor=north,pos=0.5] {K $<$ A.key}  (retT.east);
	\draw [arrow] (p2) -- node[anchor=east] {K $>$ A.key} (p3);
	\draw [arrow] (p3.west)  -- node[anchor=north]{No}  (p5.east);
	\draw [arrow] (p3.east) -- node[anchor=north]{Yes} (p4.west);
	\draw [arrow] (p4.north) -- (p2.south);

	\end{tikzpicture}
}
\caption{Sequence of steps in an insert operation}
\end{figure}
\end{frame}

\begin{frame}[c]{Delete}
A delete operation do not restart except when there is a failure in the execution phase
\begin{figure}[htp]
\centering
{
	\begin{tikzpicture}[scale=0.5, transform shape]
	\node (p0) [] {delete(K)};
	\node (p1) [process, below of=p0, text width=4cm] {Do a binary search for key K in the tree};
	\node (p2) [process, below of=p1, yshift=-1.5cm, text width=4.5cm] {Examine anchor node $A$ of top entry in the stack};
	\node (p3) [decision, below of=p2, yshift=-1.5cm, text width=2cm] {is anchor node marked?};
	\node (p4) [process, right of=p3, xshift=4cm, text width=4.5cm] {pop all entries upto anchor node $A$};
	\node (ex) [process, right of=p1, xshift=6cm, text width=4.5cm, minimum width=1cm] {go to execution phase};
	\node (retF) [process, left of=p2, xshift=-4cm, text width=1cm, minimum width=1cm] {return false};

	\draw [arrow] (p1) -- node[anchor=west] {K not found} (p2);
	\draw [arrow] (p1) -- node[anchor=south] {K found} (ex);
	\draw [arrow] (p2.east) -| node[anchor=north,pos=0.5] {K = A.key}    (ex.south);
	\draw [arrow] (p2.west) -- node[anchor=north,pos=0.5] {K $<$ A.key}  (retF.east);
	\draw [arrow] (p2) -- node[anchor=east] {K $>$ A.key} (p3);
	\draw [arrow] (p3.west)  -| node[anchor=north]{No}  (retF.south);
	\draw [arrow] (p3.east) -- node[anchor=north]{Yes} (p4.west);
	\draw [arrow] (p4.north) -- (p2.south);

	\end{tikzpicture}
}
\caption{Sequence of steps in a delete operation}
\end{figure}
\end{frame}

\chapter{Wait free search}
\label{chapter:waitFreeSearch}
\begin{limitscope}
%%%%% localRecovery/wait free search macros - begin
\newcommand{\remove}[1]{}
\NewDocumentCommand\accesspath{ g g }{\IfNoValueTF{#1}{access-path\xspace}{\IfNoValueTF{#2}{A(#1)\xspace}{A(#1,#2)\xspace}}}
\newcommand{\myterminal}{terminal}
\newcommand{\myanchor}{anchor}
\newcommand{\Myanchor}{Anchor}
\newcommand{\mytarget}{target}
\newcommand{\myadmissible}{admissible}
\newcommand{\mycritical}{critical}
\newcommand{\mysadmissible}{strongly admissible}
\newcommand{\mysafe}{safe}
\newcommand{\myconsistent}{consistent}
\newcommand{\myinconsistent}{inconsistent}
\newcommand{\storedpath}{\Pi}
\newcommand{\prefixpath}[1]{\Pi(#1)}
\newcommand{\injection}{injection}
\newcommand{\myleft}{le\!f\!t}
\newcommand{\myright}{right}
\newcommand{\myparent}{parent}
\newcommand{\InitializeTraversalRecord}{\textsc{InitializeTraversalState}}
\newcommand{\TestForTermination}{\textsc{CanTerminate}}
\newcommand{\FindStartPoint}{\textsc{FindASafeNode}}
\newcommand{\FindAdmissible}{\textsc{ValidatePath}}
\newcommand{\RemoveFromTop}{\textsc{RemoveFromTop}}
\newcommand{\AddToTop}{\textsc{AddToTop}}
\newcommand{\GetTop}{\textsc{GetTop}}
\newcommand{\GetSecondToTop}{\textsc{GetSecondToTop}}
\newcommand{\RemoveUntilCritical}{\textsc{RemoveUntil}}
\newcommand{\RememberCritical}{\textsc{RememberCritical}}
\newcommand{\GetFullEntry}{\textsc{GetFullEntry}}
\newcommand{\IsMarked}{\textsc{IsMarked}}
\newcommand{\IsClean}{\textsc{IsClean}}
\newcommand{\NeedCleanParentNode}{\textsc{NeedCleanParentNode}}
\newcommand{\AddToBottom}{\textsc{AddToBottom}}
\newcommand{\MoveFromTargetToSuccessor}{\textsc{MoveFromTargetToSuccessor}}
\newcommand{\IsEmpty}{\textsc{IsEmpty}}
\newcommand{\Size}{\textsc{Size}}
\newcommand{\GetKey}{\textsc{GetKey}}
\newcommand{\SeekForSuccessor}{\textsc{SeekForSuccessor}}
\newcommand{\NeedSuccessorKey}{\textsc{NeedSuccessorKey}}
\newcommand{\GetChild}{\textsc{GetChild}}
\newcommand{\Move}{\textsc{Move}}
\newcommand{\GetAddress}{\textsc{GetAddress}}
\newcommand{\IsNull}{\textsc{IsNull}}
\newcommand{\PopulateSeekRecord}{\textsc{PopulateSeekRecord}}
\newcommand{\SeekForModify}{\textsc{SeekForModify}}
\newcommand{\SeekForSearch}{\textsc{SeekForSearch}}
\newcommand{\TraverseTree}{\textsc{Traverse}}
\newcommand{\ExamineStack}{\textsc{ExamineStack}}
\newcommand{\numberOfProcesses}{p}
\newcommand{\STELLAR}{\textsc{STELLAR}}
\newcommand{\sNodeOne}{\mathbb{R}}
\newcommand{\sNodeTwo}{\mathbb{S}}
\newcommand{\sKeyOne}{-\infty}
\newcommand{\sKeyTwo}{\infty}
\newcommand{\traversalRecord}{state}
\newcommand{\TraversalRecord}{\textsf{State}}
\newcommand{\opRecord}{opRecord}
\newcommand{\OpRecord}{\textsf{OpRecord}}
\newcommand{\seekRecord}{seekRecord}
\newcommand{\SeekRecord}{\textsf{SeekRecord}}
\newcommand{\maximumgap}{49\%}
\newcommand{\maximumdrop}{10\%}
\newcommand{\dCounters}{DC}
\newcommand{\iCounters}{IC}
\newcommand{\labels}{labels}
\newcommand{\dcounters}{DC}
\newcommand{\icounters}{IC}
\newcommand{\myfigurescaletwo}{0.5}
%%%%% localRecovery/wait free search macros - end
In this chapter we present two light-weight techniques to make search operations for concurrent binary search trees based on internal representation, \emph{wait-free} with low additional overhead. Both of our techniques have the desirable feature that a search operation does not need to perform any write instructions on the share memory thereby minimizing the cache coherence traffic.

\begin{limitscope}
%%%%% localRecovery/wait free search macros - begin
\newcommand{\remove}[1]{}
\NewDocumentCommand\accesspath{ g g }{\IfNoValueTF{#1}{access-path\xspace}{\IfNoValueTF{#2}{A(#1)\xspace}{A(#1,#2)\xspace}}}
\newcommand{\myterminal}{terminal}
\newcommand{\myanchor}{anchor}
\newcommand{\Myanchor}{Anchor}
\newcommand{\mytarget}{target}
\newcommand{\myadmissible}{admissible}
\newcommand{\mycritical}{critical}
\newcommand{\mysadmissible}{strongly admissible}
\newcommand{\mysafe}{safe}
\newcommand{\myconsistent}{consistent}
\newcommand{\myinconsistent}{inconsistent}
\newcommand{\storedpath}{\Pi}
\newcommand{\prefixpath}[1]{\Pi(#1)}
\newcommand{\injection}{injection}
\newcommand{\myleft}{le\!f\!t}
\newcommand{\myright}{right}
\newcommand{\myparent}{parent}
\newcommand{\InitializeTraversalRecord}{\textsc{InitializeTraversalState}}
\newcommand{\TestForTermination}{\textsc{CanTerminate}}
\newcommand{\FindStartPoint}{\textsc{FindASafeNode}}
\newcommand{\FindAdmissible}{\textsc{ValidatePath}}
\newcommand{\RemoveFromTop}{\textsc{RemoveFromTop}}
\newcommand{\AddToTop}{\textsc{AddToTop}}
\newcommand{\GetTop}{\textsc{GetTop}}
\newcommand{\GetSecondToTop}{\textsc{GetSecondToTop}}
\newcommand{\RemoveUntilCritical}{\textsc{RemoveUntil}}
\newcommand{\RememberCritical}{\textsc{RememberCritical}}
\newcommand{\GetFullEntry}{\textsc{GetFullEntry}}
\newcommand{\IsMarked}{\textsc{IsMarked}}
\newcommand{\IsClean}{\textsc{IsClean}}
\newcommand{\NeedCleanParentNode}{\textsc{NeedCleanParentNode}}
\newcommand{\AddToBottom}{\textsc{AddToBottom}}
\newcommand{\MoveFromTargetToSuccessor}{\textsc{MoveFromTargetToSuccessor}}
\newcommand{\IsEmpty}{\textsc{IsEmpty}}
\newcommand{\Size}{\textsc{Size}}
\newcommand{\GetKey}{\textsc{GetKey}}
\newcommand{\SeekForSuccessor}{\textsc{SeekForSuccessor}}
\newcommand{\NeedSuccessorKey}{\textsc{NeedSuccessorKey}}
\newcommand{\GetChild}{\textsc{GetChild}}
\newcommand{\Move}{\textsc{Move}}
\newcommand{\GetAddress}{\textsc{GetAddress}}
\newcommand{\IsNull}{\textsc{IsNull}}
\newcommand{\PopulateSeekRecord}{\textsc{PopulateSeekRecord}}
\newcommand{\SeekForModify}{\textsc{SeekForModify}}
\newcommand{\SeekForSearch}{\textsc{SeekForSearch}}
\newcommand{\TraverseTree}{\textsc{Traverse}}
\newcommand{\ExamineStack}{\textsc{ExamineStack}}
\newcommand{\numberOfProcesses}{p}
\newcommand{\STELLAR}{\textsc{STELLAR}}
\newcommand{\sNodeOne}{\mathbb{R}}
\newcommand{\sNodeTwo}{\mathbb{S}}
\newcommand{\sKeyOne}{-\infty}
\newcommand{\sKeyTwo}{\infty}
\newcommand{\traversalRecord}{state}
\newcommand{\TraversalRecord}{\textsf{State}}
\newcommand{\opRecord}{opRecord}
\newcommand{\OpRecord}{\textsf{OpRecord}}
\newcommand{\seekRecord}{seekRecord}
\newcommand{\SeekRecord}{\textsf{SeekRecord}}
\newcommand{\maximumgap}{49\%}
\newcommand{\maximumdrop}{10\%}
\newcommand{\dCounters}{DC}
\newcommand{\iCounters}{IC}
\newcommand{\labels}{labels}
\newcommand{\dcounters}{DC}
\newcommand{\icounters}{IC}
\newcommand{\myfigurescaletwo}{0.5}
%%%%% localRecovery/wait free search macros - end
In this chapter we present two light-weight techniques to make search operations for concurrent binary search trees based on internal representation, \emph{wait-free} with low additional overhead. Both of our techniques have the desirable feature that a search operation does not need to perform any write instructions on the share memory thereby minimizing the cache coherence traffic.

\begin{limitscope}
%%%%% localRecovery/wait free search macros - begin
\newcommand{\remove}[1]{}
\NewDocumentCommand\accesspath{ g g }{\IfNoValueTF{#1}{access-path\xspace}{\IfNoValueTF{#2}{A(#1)\xspace}{A(#1,#2)\xspace}}}
\newcommand{\myterminal}{terminal}
\newcommand{\myanchor}{anchor}
\newcommand{\Myanchor}{Anchor}
\newcommand{\mytarget}{target}
\newcommand{\myadmissible}{admissible}
\newcommand{\mycritical}{critical}
\newcommand{\mysadmissible}{strongly admissible}
\newcommand{\mysafe}{safe}
\newcommand{\myconsistent}{consistent}
\newcommand{\myinconsistent}{inconsistent}
\newcommand{\storedpath}{\Pi}
\newcommand{\prefixpath}[1]{\Pi(#1)}
\newcommand{\injection}{injection}
\newcommand{\myleft}{le\!f\!t}
\newcommand{\myright}{right}
\newcommand{\myparent}{parent}
\newcommand{\InitializeTraversalRecord}{\textsc{InitializeTraversalState}}
\newcommand{\TestForTermination}{\textsc{CanTerminate}}
\newcommand{\FindStartPoint}{\textsc{FindASafeNode}}
\newcommand{\FindAdmissible}{\textsc{ValidatePath}}
\newcommand{\RemoveFromTop}{\textsc{RemoveFromTop}}
\newcommand{\AddToTop}{\textsc{AddToTop}}
\newcommand{\GetTop}{\textsc{GetTop}}
\newcommand{\GetSecondToTop}{\textsc{GetSecondToTop}}
\newcommand{\RemoveUntilCritical}{\textsc{RemoveUntil}}
\newcommand{\RememberCritical}{\textsc{RememberCritical}}
\newcommand{\GetFullEntry}{\textsc{GetFullEntry}}
\newcommand{\IsMarked}{\textsc{IsMarked}}
\newcommand{\IsClean}{\textsc{IsClean}}
\newcommand{\NeedCleanParentNode}{\textsc{NeedCleanParentNode}}
\newcommand{\AddToBottom}{\textsc{AddToBottom}}
\newcommand{\MoveFromTargetToSuccessor}{\textsc{MoveFromTargetToSuccessor}}
\newcommand{\IsEmpty}{\textsc{IsEmpty}}
\newcommand{\Size}{\textsc{Size}}
\newcommand{\GetKey}{\textsc{GetKey}}
\newcommand{\SeekForSuccessor}{\textsc{SeekForSuccessor}}
\newcommand{\NeedSuccessorKey}{\textsc{NeedSuccessorKey}}
\newcommand{\GetChild}{\textsc{GetChild}}
\newcommand{\Move}{\textsc{Move}}
\newcommand{\GetAddress}{\textsc{GetAddress}}
\newcommand{\IsNull}{\textsc{IsNull}}
\newcommand{\PopulateSeekRecord}{\textsc{PopulateSeekRecord}}
\newcommand{\SeekForModify}{\textsc{SeekForModify}}
\newcommand{\SeekForSearch}{\textsc{SeekForSearch}}
\newcommand{\TraverseTree}{\textsc{Traverse}}
\newcommand{\ExamineStack}{\textsc{ExamineStack}}
\newcommand{\numberOfProcesses}{p}
\newcommand{\STELLAR}{\textsc{STELLAR}}
\newcommand{\sNodeOne}{\mathbb{R}}
\newcommand{\sNodeTwo}{\mathbb{S}}
\newcommand{\sKeyOne}{-\infty}
\newcommand{\sKeyTwo}{\infty}
\newcommand{\traversalRecord}{state}
\newcommand{\TraversalRecord}{\textsf{State}}
\newcommand{\opRecord}{opRecord}
\newcommand{\OpRecord}{\textsf{OpRecord}}
\newcommand{\seekRecord}{seekRecord}
\newcommand{\SeekRecord}{\textsf{SeekRecord}}
\newcommand{\maximumgap}{49\%}
\newcommand{\maximumdrop}{10\%}
\newcommand{\dCounters}{DC}
\newcommand{\iCounters}{IC}
\newcommand{\labels}{labels}
\newcommand{\dcounters}{DC}
\newcommand{\icounters}{IC}
\newcommand{\myfigurescaletwo}{0.5}
%%%%% localRecovery/wait free search macros - end
In this chapter we present two light-weight techniques to make search operations for concurrent binary search trees based on internal representation, \emph{wait-free} with low additional overhead. Both of our techniques have the desirable feature that a search operation does not need to perform any write instructions on the share memory thereby minimizing the cache coherence traffic.

\input{Figures/waitFreeSearch}
The search operations in~\cite{HowJon:2012:SPAA,DraVec+:2014:PPoPP,ArbAtt:2014:PODC,RamMit:2015:ICDCN,RamMit:2015:PPoPP} are \emph{not wait-free} even for a bounded key space. For example, in \figref{waitFreeSearch} thread $A$ executes \textsf{contains(15)} and thread $B$ executes a series of operations preventing the \textsf{contains} operation of thread $A$ to terminate. Note that the tree in (a) and (e) are same and this sequence of operations can occur ad infinitum. Hence the search operation is not wait-free.

In our first approach, we keep track of the count of modify operations. Here we do not make any new modifications to the tree node. In our second approach, we append a timestamp to the tree node but it has better complexity than the previous one.

\section{No Modification to Tree Node}
Due to the limited manner in which the tree can evolve in the concurrent algorithms described in~\cite{HowJon:2012:SPAA,DraVec+:2014:PPoPP,ArbAtt:2014:PODC,RamMit:2015:ICDCN,RamMit:2015:PPoPP}, it is possible to design a light-weight wait-free algorithm for a search operation for all of the algorithms. The main property we use is that as long as a key is \emph{continuously} present in the tree, its distance from the root of the tree is \emph{monotonically non-increasing}. As a result, if a key is not found after visiting a ``certain'' number of nodes in the tree, then the traversal can stop and it is sufficient to examine the path traversed to check whether or not the key has moved up. In case the key is not continuously present in the tree, while a search operation is in progress, it is acceptable to return either of the 
outcomes---present or not present---to the application. In the first case, the search operation can be linearized after the insert operation that added the key to the tree. In the second case, it can be linearized after the delete operation that removed the key from the tree. 

The main question is: ``How do we \emph{efficiently} determine the number of nodes to visit in the tree before stopping the downward traversal \emph{without missing} the key that is continuously present in the tree?'' To that end, we maintain two arrays with one entry for each process in the array, denoted by $\iCounters$ and $\dCounters$. Roughly speaking, entries $\iCounters[i]$ and $\dCounters[i]$ denote the number of insert and delete operations, respectively, process $P_i$ has performed so far. A process increments its insert counter before adding a key to the tree and its delete counter after removing a key from the tree. As a result, the insert (delete) counter at a process is an upper (lower) bound on the number of keys that the process has added to (removed from) the tree. Before starting downward traversal for a search operation, a process first reads the delete counter values of all processes and then reads the insert counter values of all processes. It then computes an estimate for the number of nodes to visit as the sum of all the insert counter values minus the sum of all the delete counter values. We show that the estimate computed by a process is \emph{safe} in the sense that a search operation will not miss a key continuously present in the tree while the operation is in progress.

To show that our algorithm works, we introduce some notation. Consider a time $t$ \emph{after} an operation has read all the delete counter values but \emph{before} its starts reading any of the insert counter values. Let $I_{t,i}$ ($D_{t,i}$) denote the \emph{actual} number of keys added to (removed from) the tree by process $P_i$ at or before time $t$. Also, let $\icounters[i]$ ($\dcounters[i]$) denote the value read for $\iCounters[i]$ ($\dCounters[i]$) by the operation. Note that, the way counters are maintained, $\icounters[i] \geq I_{t,i}$ and $\dcounters[i] \leq D_{t,i}$. Also, let $S_t$ and $\Delta_t$ denote the \emph{actual} size of the tree and the \emph{actual} distance of the target key from the root of the tree, respectively, at time $t$. Clearly, we have:
%%
\[
S_t = \sum_{0 \leq i < \numberOfProcesses} I_{t,i} - \sum_{0 \leq i < \numberOfProcesses} D_{t,i} \text{\quad and \quad} \Delta_t \leq S_t
\]
%%
Thus we have:
\[
\sum_{0 \leq i < \numberOfProcesses} \icounters[i] - \sum_{0 \leq i < \numberOfProcesses} \dcounters[i]  \; \geq \; \sum_{0 \leq i < \numberOfProcesses} I_{t,i} - \sum_{0 \leq i < \numberOfProcesses} D_{t,i} \; = \; S_t 
\]
%%
In other words, the estimate computed by our algorithm is an upper bound on the actual distance of the key from the root of tree when the operation starts traversing the tree (which is monotonically non-increasing).

A pseudo-code of the algorithm is given in \pseudoref{wf:search:size}. In the pseudo-code, $\numberOfProcesses$ denotes the number of processes. To amortize the overhead of reading $O(\numberOfProcesses)$ counters, an operation first visits $\numberOfProcesses$ nodes in the tree. If it does not find the key, then it reads the counter values and proceeds as described above. Thus $O(\numberOfProcesses)$ overhead is incurred only for ``large'' trees.

Some advantages of our approach are as follows. First, it works even if a key space is unbounded. Second, it does not require a search operation to perform any write instruction on shared memory. Third, it does not require a modify operation to perform any additional atomic instruction or helping (besides that performed by the original algorithm). 

\input{waitFreeSearch/pseudocode-waitfree}

\section{With Modification to Tree Node}
A disadvantage of the previous approach is that the time complexity of a search operation depends on the tree size. We now describe another approach to achieve wait-freedom for which the time complexity of a search operation depends on the tree height. This approach, however, requires modifying tree node to store a time-stamp of when the node was created. It consists of the identifier of the process that created the node and the process-specific sequence number (which is incremented before the node is added to the tree). This time-stamp is copied if a node is \emph{replaced} with a new node (in a complex delete operation) as in~\cite{RamMit:2015:ICDCN}. Before a search operation starts traversing the tree, it reads the current sequence number values of all processes. Let $\labels[i]$ denote the value read for process $P_i$. The operation then stops the downward traversal of the tree once it encounters a node with time-stamp $\ang{i,v}$ such that $v > \labels[i]$. Clearly, this node and its descendants were added to the tree after the operation read the sequence number value of process $P_i$.
%%
A pseudo-code of the algorithm is given in \pseudoref{wf:search:height}.
\end{limitscope}
The search operations in~\cite{HowJon:2012:SPAA,DraVec+:2014:PPoPP,ArbAtt:2014:PODC,RamMit:2015:ICDCN,RamMit:2015:PPoPP} are \emph{not wait-free} even for a bounded key space. For example, in \figref{waitFreeSearch} thread $A$ executes \textsf{contains(15)} and thread $B$ executes a series of operations preventing the \textsf{contains} operation of thread $A$ to terminate. Note that the tree in (a) and (e) are same and this sequence of operations can occur ad infinitum. Hence the search operation is not wait-free.

In our first approach, we keep track of the count of modify operations. Here we do not make any new modifications to the tree node. In our second approach, we append a timestamp to the tree node but it has better complexity than the previous one.

\section{No Modification to Tree Node}
Due to the limited manner in which the tree can evolve in the concurrent algorithms described in~\cite{HowJon:2012:SPAA,DraVec+:2014:PPoPP,ArbAtt:2014:PODC,RamMit:2015:ICDCN,RamMit:2015:PPoPP}, it is possible to design a light-weight wait-free algorithm for a search operation for all of the algorithms. The main property we use is that as long as a key is \emph{continuously} present in the tree, its distance from the root of the tree is \emph{monotonically non-increasing}. As a result, if a key is not found after visiting a ``certain'' number of nodes in the tree, then the traversal can stop and it is sufficient to examine the path traversed to check whether or not the key has moved up. In case the key is not continuously present in the tree, while a search operation is in progress, it is acceptable to return either of the 
outcomes---present or not present---to the application. In the first case, the search operation can be linearized after the insert operation that added the key to the tree. In the second case, it can be linearized after the delete operation that removed the key from the tree. 

The main question is: ``How do we \emph{efficiently} determine the number of nodes to visit in the tree before stopping the downward traversal \emph{without missing} the key that is continuously present in the tree?'' To that end, we maintain two arrays with one entry for each process in the array, denoted by $\iCounters$ and $\dCounters$. Roughly speaking, entries $\iCounters[i]$ and $\dCounters[i]$ denote the number of insert and delete operations, respectively, process $P_i$ has performed so far. A process increments its insert counter before adding a key to the tree and its delete counter after removing a key from the tree. As a result, the insert (delete) counter at a process is an upper (lower) bound on the number of keys that the process has added to (removed from) the tree. Before starting downward traversal for a search operation, a process first reads the delete counter values of all processes and then reads the insert counter values of all processes. It then computes an estimate for the number of nodes to visit as the sum of all the insert counter values minus the sum of all the delete counter values. We show that the estimate computed by a process is \emph{safe} in the sense that a search operation will not miss a key continuously present in the tree while the operation is in progress.

To show that our algorithm works, we introduce some notation. Consider a time $t$ \emph{after} an operation has read all the delete counter values but \emph{before} its starts reading any of the insert counter values. Let $I_{t,i}$ ($D_{t,i}$) denote the \emph{actual} number of keys added to (removed from) the tree by process $P_i$ at or before time $t$. Also, let $\icounters[i]$ ($\dcounters[i]$) denote the value read for $\iCounters[i]$ ($\dCounters[i]$) by the operation. Note that, the way counters are maintained, $\icounters[i] \geq I_{t,i}$ and $\dcounters[i] \leq D_{t,i}$. Also, let $S_t$ and $\Delta_t$ denote the \emph{actual} size of the tree and the \emph{actual} distance of the target key from the root of the tree, respectively, at time $t$. Clearly, we have:
%%
\[
S_t = \sum_{0 \leq i < \numberOfProcesses} I_{t,i} - \sum_{0 \leq i < \numberOfProcesses} D_{t,i} \text{\quad and \quad} \Delta_t \leq S_t
\]
%%
Thus we have:
\[
\sum_{0 \leq i < \numberOfProcesses} \icounters[i] - \sum_{0 \leq i < \numberOfProcesses} \dcounters[i]  \; \geq \; \sum_{0 \leq i < \numberOfProcesses} I_{t,i} - \sum_{0 \leq i < \numberOfProcesses} D_{t,i} \; = \; S_t 
\]
%%
In other words, the estimate computed by our algorithm is an upper bound on the actual distance of the key from the root of tree when the operation starts traversing the tree (which is monotonically non-increasing).

A pseudo-code of the algorithm is given in \pseudoref{wf:search:size}. In the pseudo-code, $\numberOfProcesses$ denotes the number of processes. To amortize the overhead of reading $O(\numberOfProcesses)$ counters, an operation first visits $\numberOfProcesses$ nodes in the tree. If it does not find the key, then it reads the counter values and proceeds as described above. Thus $O(\numberOfProcesses)$ overhead is incurred only for ``large'' trees.

Some advantages of our approach are as follows. First, it works even if a key space is unbounded. Second, it does not require a search operation to perform any write instruction on shared memory. Third, it does not require a modify operation to perform any additional atomic instruction or helping (besides that performed by the original algorithm). 

\begin{limitscope}

%% To limit the scope of the macros defined below

%% macros for pseudocode

\newcommand{\child}{child}
\newcommand{\node}{node}
\newcommand{\parent}{parent}

\newcommand{\mainSeekRecord}{seekTargetKey}
\newcommand{\successorSeekRecord}{seekSuccessorKey}


\newcommand{\targetStack}{targetStack}
\newcommand{\successorStack}{successorStack}


\newcommand{\successorStackInUse}{successorStackInUse}
\newcommand{\targetNode}{targetNode}




\newcommand{\key}{key}

\newcommand{\done}{done}
\newcommand{\result}{result}
\newcommand{\status}{status}
\newcommand{\restart}{restart}





\newcommand{\cKey}{key}
\newcommand{\nKey}{key}
\newcommand{\cNode}{current}
\newcommand{\pNode}{parent}
\newcommand{\nMarked}{marked}



\newcommand{\which}{which}
\newcommand{\address}{address}

\newcommand{\anchor}{anchor}

\newcommand{\stack}{stack}
\newcommand{\sTop}{top}
\newcommand{\sBottom}{bottom}
\newcommand{\current}{current}

\remove{
\newcommand{\traversalRecord}{state}
\newcommand{\TraversalRecord}{State}
\newcommand{\opRecord}{opRecord}
\newcommand{\OpRecord}{OpRecord}
\newcommand{\seekRecord}{seekRecord}
\newcommand{\SeekRecord}{SeekRecord}
}

\newcommand{\admissible}{admissible}
\newcommand{\critical}{critical}
\newcommand{\reference}{re\!f\!erence}

\newcommand{\OptReturn}[1][]{\Return #1\;}

\newcommand{\injectionPoint}{injectionPoint}



\newcommand{\Search}{\textsc{Search}}
\newcommand{\Insert}{\textsc{Insert}}
\newcommand{\Delete}{\textsc{Delete}}
\newcommand{\Seek}{\textsc{Seek}}

\newcommand{\Inject}{\textsc{Inject}}


%%
%% \newcommand{\WFSeekForSearchBOSize}{\textsc{WFSeekForSearchBasedOnSize}}
\newcommand{\WFSeekForSearchBOSize}{\textsc{WFSeekForSearchBOSize}}
%% \newcommand{\WFSeekForSearchBOHeight}{\textsc{WFSeekForSearchBasedOnHeight}}
\newcommand{\WFSeekForSearchBOHeight}{\textsc{WFSeekForSearchBOHeight}}
%%
%% \newcommand{\WFTraverseTreeBOCount}{\textsc{TraverseBasedOnCount}}
\newcommand{\WFTraverseTreeBOCount}{\textsc{LimitedSeek}}
%% \newcommand{\WFTraverseTreeBOTimeStamp}{\textsc{TraverseBasedOnTimeStamp}}
\newcommand{\WFTraverseTreeBOTimeStamp}{\textsc{TimeStampSeek}}
%%
\remove{
\newcommand{\SeekForSuccessor}{\textsc{SeekForSuccessor}}
\newcommand{\NeedSuccessorKey}{\textsc{NeedSuccessorKey}}
\newcommand{\GetChild}{\textsc{GetChild}}
\newcommand{\Move}{\textsc{Move}}
\newcommand{\GetAddress}{\textsc{GetAddress}}
\newcommand{\IsNull}{\textsc{IsNull}}
\newcommand{\PopulateSeekRecord}{\textsc{PopulateSeekRecord}}
}

%%


\newcommand{\dSum}{\mathcal{D}}
\newcommand{\iSum}{\mathcal{I}}

\newcommand{\copyOfLabels}{copyO\!f\!Labels}
\newcommand{\timeStamp}{timeStamp}
\newcommand{\myLabel}{label}
\newcommand{\myPID}{pid}

\newcommand{\mline}[1]{\DontPrintSemicolon #1 \PrintSemicolon}


\newcommand{\LEFT}{\textsf{LEFT}}
\newcommand{\RIGHT}{\textsf{RIGHT}}


\newcommand{\rarrow}{\!\rightarrow\!}
\newcommand{\type}{type}
\newcommand{\limit}{limit}


\newcommand{\SEARCH}{\textsf{SEARCH}}
\newcommand{\INSERT}{\textsf{INSERT}}
\newcommand{\DELETE}{\textsf{DELETE}}

\newcommand{\STOPFOUND}{\textsf{FOUND}}
\newcommand{\STOPNOTFOUND}{\textsf{NOT\_FOUND}}
\newcommand{\ADMISSIBLE}{\textsf{SAFE}}
\newcommand{\INADMISSIBLE}{\textsf{NOT\_SAFE}}

\newcommand{\TARGETSTACK}{\textsf{TARGET\_STACK}}
\newcommand{\SUCCESSORSTACK}{\textsf{SUCCESSOR\_STACK}}

%%%%%%%%%%%%%%%%%%%%%%%%%%%%%%%%%%%%%%%%%%%%%%%%%%%%%%%%%%%%%%%%%%%%%%%%%%%%%%%%%%%%

\newcommand{\DefineKeyWords}{
%%
\SetKw{Boolean}{boolean}
\SetKw{Integer}{integer}
\SetKw{LAnd}{~and~}
\SetKw{LOr}{~or~}
\SetKw{LNot}{not}
\SetKw{Struct}{struct}
\SetKw{Null}{null}
\SetKw{True}{true}
\SetKw{False}{false}
\SetKw{Break}{break}
\SetKw{Continue}{continue}
\SetKw{Enum}{enum}
\SetKw{Word}{word}
%%
}

%%%%%%%%%%%%%%%%%%%%%%%%%%%%%%%%%%%%%%%%%%%%%%%%%%%%%%%%%%%%%%%%%%%%%%%%%%%%%%%%%%%%%

%% Functions for wait-free search

%%%%%%%%%%%%%%%%%%%%%%%%%%%%%%%%%%%%%%%%%%%%%%%%%%%%%%%%%%%%%%%%%%%%%%%%%%%%%%%%%%%%%




\begin{algorithm}[htp]
\caption{Seek Function for Target Key based on Estimating Tree Size} 
\label{algo:wf:search:size}
%%
\DefineKeyWords
%%
\Integer $\iCounters[\numberOfProcesses]$\;
\Integer $\dCounters[\numberOfProcesses]$\;
%%
\BlankLine
%%
\tcp{Traverses the tree starting from the root node but visits a limited number of nodes}
\DontPrintSemicolon
\Boolean \WFTraverseTreeBOCount( $\opRecord$, $\seekRecord$, $\limit$ )\;
%% \Boolean \WFTraverseTreeBOCount( $\opRecord$,  $\limit$ )\;
\PrintSemicolon
\Begin
{
%%
\tcp{similar to \TraverseTree{} except that the while loop from \linesref{local-seek:while:traversal:begin}{local-seek:while:traversal:end} is executed at most $\limit$ times}
%%
}
%%
\BlankLine
%%
\tcp{A wait-free seek function for a search operation based on computing an upper-bound on tree size}
\DontPrintSemicolon
\Boolean \WFSeekForSearchBOSize( $\opRecord$, $\seekRecord$ )\;
\PrintSemicolon
\Begin
{
%%
   %% $\limit$ := $\numberOfProcesses$\;
	 $\result$ := \WFTraverseTreeBOCount(  $\opRecord$, $\seekRecord$, $\numberOfProcesses$ )\;
	 \If{\LNot($\result$)} 
	 {
	    %% $\dSum$ := $\sum\limits_{i = 0}^{\numberOfProcesses-1}  \dCounters[i]$\;
			%% $\iSum$ := $\sum\limits_{i = 0}^{\numberOfProcesses-1}  \iCounters[i]$\;
			$\dSum$ := $\dCounters[0] + \dCounters[1] + \cdots + \dCounters[\numberOfProcesses-1]$\;
		  $\iSum$ := $\iCounters[0] + \iCounters[1] + \cdots + \iCounters[\numberOfProcesses-1]$\;
			$S$ := $\iSum - \dSum$\;
			$\result$ :=  \WFTraverseTreeBOCount(  $\opRecord$, $\seekRecord$, $S$ )\;
			
	 }
	 
	 \If{\LNot($\result$)}
	 {
	    \tcp{examine the stack}
			$\result$ := \ExamineStack( $\opRecord$, $\seekRecord$ )\;
	 }
	 
	 \tcp{return the outcome}
	 %% \tcp{return the outcome}
	 \PopulateSeekRecord( $\seekRecord$, $\traversalRecord$ )\;
	 \Return $\result$\;
	
%%
}
\end{algorithm}




\begin{algorithm}[htp]
\caption{Seek Function for Target Key based on Time-Stamps} 
\label{algo:wf:search:height}
%%
\DefineKeyWords
%%
\Integer $\labels[\numberOfProcesses]$\;
%%
\BlankLine
%%
\tcp{Traverses the tree starting from the root node but stops if recently added key is found}
%%
\DontPrintSemicolon
\Boolean \WFTraverseTreeBOTimeStamp( $\opRecord$, $\seekRecord$, $\labels$ )\;
\PrintSemicolon
\Begin
{
%%
\tcp{similar to \TraverseTree{} except that the while loop from \linesref{local-seek:while:traversal:begin}{local-seek:while:traversal:end} is terminated as soon as a node with a ``recent'' time-stamp is encountered}
\tcp{specifically, the following lines are inserted between \linesref[ \& ]{local-seek:while:traversal:begin}{local-seek:while:traversal:first}}
%%
$\ang{ \myPID, \myLabel }$ := $\node \rarrow \timeStamp$\;
\lIf{$\myLabel$ $>$ $\labels[\myPID]$}{ \Break }
}
%%
\BlankLine
%%
\tcp{A wait-free  seek function for a search operation based on estimating tree height}
%%
\DontPrintSemicolon
\Boolean \WFSeekForSearchBOHeight( $\opRecord$, $\seekRecord$ )\;
\PrintSemicolon
\Begin
{
%%
   %% $\limit$ := $\numberOfProcesses$\;
	 $\result$ := \WFTraverseTreeBOCount(  $\opRecord$, $\seekRecord$, $\numberOfProcesses$ )\;
	 \If{\LNot($\result$)} 
	 {
	    $\copyOfLabels$ := $\labels$\;
			%\mline{$\result$ := \WFTraverseTreeBOTimeStamp( \parbox[t]{1.125in}{$\opRecord$, $\seekRecord$, \\ $\copyOfLabels$ );} \;}
			$\result$ := \WFTraverseTreeBOTimeStamp($\opRecord$, $\seekRecord$, $\copyOfLabels$)\;
			
	 }
	 
	 \If{\LNot($\result$)}
	 {
	    \tcp{examine the stack }
			$\result$ := \ExamineStack( $\opRecord$, $\seekRecord$ )\;
	 }
	 
	 \tcp{return the outcome}
	 %% \tcp{return the outcome}
	 \PopulateSeekRecord( $\seekRecord$, $\traversalRecord$ )\;
	 \Return $\result$\;
	
%%
}
\end{algorithm}



\end{limitscope}

\section{With Modification to Tree Node}
A disadvantage of the previous approach is that the time complexity of a search operation depends on the tree size. We now describe another approach to achieve wait-freedom for which the time complexity of a search operation depends on the tree height. This approach, however, requires modifying tree node to store a time-stamp of when the node was created. It consists of the identifier of the process that created the node and the process-specific sequence number (which is incremented before the node is added to the tree). This time-stamp is copied if a node is \emph{replaced} with a new node (in a complex delete operation) as in~\cite{RamMit:2015:ICDCN}. Before a search operation starts traversing the tree, it reads the current sequence number values of all processes. Let $\labels[i]$ denote the value read for process $P_i$. The operation then stops the downward traversal of the tree once it encounters a node with time-stamp $\ang{i,v}$ such that $v > \labels[i]$. Clearly, this node and its descendants were added to the tree after the operation read the sequence number value of process $P_i$.
%%
A pseudo-code of the algorithm is given in \pseudoref{wf:search:height}.
\end{limitscope}
The search operations in~\cite{HowJon:2012:SPAA,DraVec+:2014:PPoPP,ArbAtt:2014:PODC,RamMit:2015:ICDCN,RamMit:2015:PPoPP} are \emph{not wait-free} even for a bounded key space. For example, in \figref{waitFreeSearch} thread $A$ executes \textsf{contains(15)} and thread $B$ executes a series of operations preventing the \textsf{contains} operation of thread $A$ to terminate. Note that the tree in (a) and (e) are same and this sequence of operations can occur ad infinitum. Hence the search operation is not wait-free.

In our first approach, we keep track of the count of modify operations. Here we do not make any new modifications to the tree node. In our second approach, we append a timestamp to the tree node but it has better complexity than the previous one.

\section{No Modification to Tree Node}
Due to the limited manner in which the tree can evolve in the concurrent algorithms described in~\cite{HowJon:2012:SPAA,DraVec+:2014:PPoPP,ArbAtt:2014:PODC,RamMit:2015:ICDCN,RamMit:2015:PPoPP}, it is possible to design a light-weight wait-free algorithm for a search operation for all of the algorithms. The main property we use is that as long as a key is \emph{continuously} present in the tree, its distance from the root of the tree is \emph{monotonically non-increasing}. As a result, if a key is not found after visiting a ``certain'' number of nodes in the tree, then the traversal can stop and it is sufficient to examine the path traversed to check whether or not the key has moved up. In case the key is not continuously present in the tree, while a search operation is in progress, it is acceptable to return either of the 
outcomes---present or not present---to the application. In the first case, the search operation can be linearized after the insert operation that added the key to the tree. In the second case, it can be linearized after the delete operation that removed the key from the tree. 

The main question is: ``How do we \emph{efficiently} determine the number of nodes to visit in the tree before stopping the downward traversal \emph{without missing} the key that is continuously present in the tree?'' To that end, we maintain two arrays with one entry for each process in the array, denoted by $\iCounters$ and $\dCounters$. Roughly speaking, entries $\iCounters[i]$ and $\dCounters[i]$ denote the number of insert and delete operations, respectively, process $P_i$ has performed so far. A process increments its insert counter before adding a key to the tree and its delete counter after removing a key from the tree. As a result, the insert (delete) counter at a process is an upper (lower) bound on the number of keys that the process has added to (removed from) the tree. Before starting downward traversal for a search operation, a process first reads the delete counter values of all processes and then reads the insert counter values of all processes. It then computes an estimate for the number of nodes to visit as the sum of all the insert counter values minus the sum of all the delete counter values. We show that the estimate computed by a process is \emph{safe} in the sense that a search operation will not miss a key continuously present in the tree while the operation is in progress.

To show that our algorithm works, we introduce some notation. Consider a time $t$ \emph{after} an operation has read all the delete counter values but \emph{before} its starts reading any of the insert counter values. Let $I_{t,i}$ ($D_{t,i}$) denote the \emph{actual} number of keys added to (removed from) the tree by process $P_i$ at or before time $t$. Also, let $\icounters[i]$ ($\dcounters[i]$) denote the value read for $\iCounters[i]$ ($\dCounters[i]$) by the operation. Note that, the way counters are maintained, $\icounters[i] \geq I_{t,i}$ and $\dcounters[i] \leq D_{t,i}$. Also, let $S_t$ and $\Delta_t$ denote the \emph{actual} size of the tree and the \emph{actual} distance of the target key from the root of the tree, respectively, at time $t$. Clearly, we have:
%%
\[
S_t = \sum_{0 \leq i < \numberOfProcesses} I_{t,i} - \sum_{0 \leq i < \numberOfProcesses} D_{t,i} \text{\quad and \quad} \Delta_t \leq S_t
\]
%%
Thus we have:
\[
\sum_{0 \leq i < \numberOfProcesses} \icounters[i] - \sum_{0 \leq i < \numberOfProcesses} \dcounters[i]  \; \geq \; \sum_{0 \leq i < \numberOfProcesses} I_{t,i} - \sum_{0 \leq i < \numberOfProcesses} D_{t,i} \; = \; S_t 
\]
%%
In other words, the estimate computed by our algorithm is an upper bound on the actual distance of the key from the root of tree when the operation starts traversing the tree (which is monotonically non-increasing).

A pseudo-code of the algorithm is given in \pseudoref{wf:search:size}. In the pseudo-code, $\numberOfProcesses$ denotes the number of processes. To amortize the overhead of reading $O(\numberOfProcesses)$ counters, an operation first visits $\numberOfProcesses$ nodes in the tree. If it does not find the key, then it reads the counter values and proceeds as described above. Thus $O(\numberOfProcesses)$ overhead is incurred only for ``large'' trees.

Some advantages of our approach are as follows. First, it works even if a key space is unbounded. Second, it does not require a search operation to perform any write instruction on shared memory. Third, it does not require a modify operation to perform any additional atomic instruction or helping (besides that performed by the original algorithm). 

\begin{limitscope}

%% To limit the scope of the macros defined below

%% macros for pseudocode

\newcommand{\child}{child}
\newcommand{\node}{node}
\newcommand{\parent}{parent}

\newcommand{\mainSeekRecord}{seekTargetKey}
\newcommand{\successorSeekRecord}{seekSuccessorKey}


\newcommand{\targetStack}{targetStack}
\newcommand{\successorStack}{successorStack}


\newcommand{\successorStackInUse}{successorStackInUse}
\newcommand{\targetNode}{targetNode}




\newcommand{\key}{key}

\newcommand{\done}{done}
\newcommand{\result}{result}
\newcommand{\status}{status}
\newcommand{\restart}{restart}





\newcommand{\cKey}{key}
\newcommand{\nKey}{key}
\newcommand{\cNode}{current}
\newcommand{\pNode}{parent}
\newcommand{\nMarked}{marked}



\newcommand{\which}{which}
\newcommand{\address}{address}

\newcommand{\anchor}{anchor}

\newcommand{\stack}{stack}
\newcommand{\sTop}{top}
\newcommand{\sBottom}{bottom}
\newcommand{\current}{current}

\remove{
\newcommand{\traversalRecord}{state}
\newcommand{\TraversalRecord}{State}
\newcommand{\opRecord}{opRecord}
\newcommand{\OpRecord}{OpRecord}
\newcommand{\seekRecord}{seekRecord}
\newcommand{\SeekRecord}{SeekRecord}
}

\newcommand{\admissible}{admissible}
\newcommand{\critical}{critical}
\newcommand{\reference}{re\!f\!erence}

\newcommand{\OptReturn}[1][]{\Return #1\;}

\newcommand{\injectionPoint}{injectionPoint}



\newcommand{\Search}{\textsc{Search}}
\newcommand{\Insert}{\textsc{Insert}}
\newcommand{\Delete}{\textsc{Delete}}
\newcommand{\Seek}{\textsc{Seek}}

\newcommand{\Inject}{\textsc{Inject}}


%%
%% \newcommand{\WFSeekForSearchBOSize}{\textsc{WFSeekForSearchBasedOnSize}}
\newcommand{\WFSeekForSearchBOSize}{\textsc{WFSeekForSearchBOSize}}
%% \newcommand{\WFSeekForSearchBOHeight}{\textsc{WFSeekForSearchBasedOnHeight}}
\newcommand{\WFSeekForSearchBOHeight}{\textsc{WFSeekForSearchBOHeight}}
%%
%% \newcommand{\WFTraverseTreeBOCount}{\textsc{TraverseBasedOnCount}}
\newcommand{\WFTraverseTreeBOCount}{\textsc{LimitedSeek}}
%% \newcommand{\WFTraverseTreeBOTimeStamp}{\textsc{TraverseBasedOnTimeStamp}}
\newcommand{\WFTraverseTreeBOTimeStamp}{\textsc{TimeStampSeek}}
%%
\remove{
\newcommand{\SeekForSuccessor}{\textsc{SeekForSuccessor}}
\newcommand{\NeedSuccessorKey}{\textsc{NeedSuccessorKey}}
\newcommand{\GetChild}{\textsc{GetChild}}
\newcommand{\Move}{\textsc{Move}}
\newcommand{\GetAddress}{\textsc{GetAddress}}
\newcommand{\IsNull}{\textsc{IsNull}}
\newcommand{\PopulateSeekRecord}{\textsc{PopulateSeekRecord}}
}

%%


\newcommand{\dSum}{\mathcal{D}}
\newcommand{\iSum}{\mathcal{I}}

\newcommand{\copyOfLabels}{copyO\!f\!Labels}
\newcommand{\timeStamp}{timeStamp}
\newcommand{\myLabel}{label}
\newcommand{\myPID}{pid}

\newcommand{\mline}[1]{\DontPrintSemicolon #1 \PrintSemicolon}


\newcommand{\LEFT}{\textsf{LEFT}}
\newcommand{\RIGHT}{\textsf{RIGHT}}


\newcommand{\rarrow}{\!\rightarrow\!}
\newcommand{\type}{type}
\newcommand{\limit}{limit}


\newcommand{\SEARCH}{\textsf{SEARCH}}
\newcommand{\INSERT}{\textsf{INSERT}}
\newcommand{\DELETE}{\textsf{DELETE}}

\newcommand{\STOPFOUND}{\textsf{FOUND}}
\newcommand{\STOPNOTFOUND}{\textsf{NOT\_FOUND}}
\newcommand{\ADMISSIBLE}{\textsf{SAFE}}
\newcommand{\INADMISSIBLE}{\textsf{NOT\_SAFE}}

\newcommand{\TARGETSTACK}{\textsf{TARGET\_STACK}}
\newcommand{\SUCCESSORSTACK}{\textsf{SUCCESSOR\_STACK}}

%%%%%%%%%%%%%%%%%%%%%%%%%%%%%%%%%%%%%%%%%%%%%%%%%%%%%%%%%%%%%%%%%%%%%%%%%%%%%%%%%%%%

\newcommand{\DefineKeyWords}{
%%
\SetKw{Boolean}{boolean}
\SetKw{Integer}{integer}
\SetKw{LAnd}{~and~}
\SetKw{LOr}{~or~}
\SetKw{LNot}{not}
\SetKw{Struct}{struct}
\SetKw{Null}{null}
\SetKw{True}{true}
\SetKw{False}{false}
\SetKw{Break}{break}
\SetKw{Continue}{continue}
\SetKw{Enum}{enum}
\SetKw{Word}{word}
%%
}

%%%%%%%%%%%%%%%%%%%%%%%%%%%%%%%%%%%%%%%%%%%%%%%%%%%%%%%%%%%%%%%%%%%%%%%%%%%%%%%%%%%%%

%% Functions for wait-free search

%%%%%%%%%%%%%%%%%%%%%%%%%%%%%%%%%%%%%%%%%%%%%%%%%%%%%%%%%%%%%%%%%%%%%%%%%%%%%%%%%%%%%




\begin{algorithm}[htp]
\caption{Seek Function for Target Key based on Estimating Tree Size} 
\label{algo:wf:search:size}
%%
\DefineKeyWords
%%
\Integer $\iCounters[\numberOfProcesses]$\;
\Integer $\dCounters[\numberOfProcesses]$\;
%%
\BlankLine
%%
\tcp{Traverses the tree starting from the root node but visits a limited number of nodes}
\DontPrintSemicolon
\Boolean \WFTraverseTreeBOCount( $\opRecord$, $\seekRecord$, $\limit$ )\;
%% \Boolean \WFTraverseTreeBOCount( $\opRecord$,  $\limit$ )\;
\PrintSemicolon
\Begin
{
%%
\tcp{similar to \TraverseTree{} except that the while loop from \linesref{local-seek:while:traversal:begin}{local-seek:while:traversal:end} is executed at most $\limit$ times}
%%
}
%%
\BlankLine
%%
\tcp{A wait-free seek function for a search operation based on computing an upper-bound on tree size}
\DontPrintSemicolon
\Boolean \WFSeekForSearchBOSize( $\opRecord$, $\seekRecord$ )\;
\PrintSemicolon
\Begin
{
%%
   %% $\limit$ := $\numberOfProcesses$\;
	 $\result$ := \WFTraverseTreeBOCount(  $\opRecord$, $\seekRecord$, $\numberOfProcesses$ )\;
	 \If{\LNot($\result$)} 
	 {
	    %% $\dSum$ := $\sum\limits_{i = 0}^{\numberOfProcesses-1}  \dCounters[i]$\;
			%% $\iSum$ := $\sum\limits_{i = 0}^{\numberOfProcesses-1}  \iCounters[i]$\;
			$\dSum$ := $\dCounters[0] + \dCounters[1] + \cdots + \dCounters[\numberOfProcesses-1]$\;
		  $\iSum$ := $\iCounters[0] + \iCounters[1] + \cdots + \iCounters[\numberOfProcesses-1]$\;
			$S$ := $\iSum - \dSum$\;
			$\result$ :=  \WFTraverseTreeBOCount(  $\opRecord$, $\seekRecord$, $S$ )\;
			
	 }
	 
	 \If{\LNot($\result$)}
	 {
	    \tcp{examine the stack}
			$\result$ := \ExamineStack( $\opRecord$, $\seekRecord$ )\;
	 }
	 
	 \tcp{return the outcome}
	 %% \tcp{return the outcome}
	 \PopulateSeekRecord( $\seekRecord$, $\traversalRecord$ )\;
	 \Return $\result$\;
	
%%
}
\end{algorithm}




\begin{algorithm}[htp]
\caption{Seek Function for Target Key based on Time-Stamps} 
\label{algo:wf:search:height}
%%
\DefineKeyWords
%%
\Integer $\labels[\numberOfProcesses]$\;
%%
\BlankLine
%%
\tcp{Traverses the tree starting from the root node but stops if recently added key is found}
%%
\DontPrintSemicolon
\Boolean \WFTraverseTreeBOTimeStamp( $\opRecord$, $\seekRecord$, $\labels$ )\;
\PrintSemicolon
\Begin
{
%%
\tcp{similar to \TraverseTree{} except that the while loop from \linesref{local-seek:while:traversal:begin}{local-seek:while:traversal:end} is terminated as soon as a node with a ``recent'' time-stamp is encountered}
\tcp{specifically, the following lines are inserted between \linesref[ \& ]{local-seek:while:traversal:begin}{local-seek:while:traversal:first}}
%%
$\ang{ \myPID, \myLabel }$ := $\node \rarrow \timeStamp$\;
\lIf{$\myLabel$ $>$ $\labels[\myPID]$}{ \Break }
}
%%
\BlankLine
%%
\tcp{A wait-free  seek function for a search operation based on estimating tree height}
%%
\DontPrintSemicolon
\Boolean \WFSeekForSearchBOHeight( $\opRecord$, $\seekRecord$ )\;
\PrintSemicolon
\Begin
{
%%
   %% $\limit$ := $\numberOfProcesses$\;
	 $\result$ := \WFTraverseTreeBOCount(  $\opRecord$, $\seekRecord$, $\numberOfProcesses$ )\;
	 \If{\LNot($\result$)} 
	 {
	    $\copyOfLabels$ := $\labels$\;
			%\mline{$\result$ := \WFTraverseTreeBOTimeStamp( \parbox[t]{1.125in}{$\opRecord$, $\seekRecord$, \\ $\copyOfLabels$ );} \;}
			$\result$ := \WFTraverseTreeBOTimeStamp($\opRecord$, $\seekRecord$, $\copyOfLabels$)\;
			
	 }
	 
	 \If{\LNot($\result$)}
	 {
	    \tcp{examine the stack }
			$\result$ := \ExamineStack( $\opRecord$, $\seekRecord$ )\;
	 }
	 
	 \tcp{return the outcome}
	 %% \tcp{return the outcome}
	 \PopulateSeekRecord( $\seekRecord$, $\traversalRecord$ )\;
	 \Return $\result$\;
	
%%
}
\end{algorithm}



\end{limitscope}

\section{With Modification to Tree Node}
A disadvantage of the previous approach is that the time complexity of a search operation depends on the tree size. We now describe another approach to achieve wait-freedom for which the time complexity of a search operation depends on the tree height. This approach, however, requires modifying tree node to store a time-stamp of when the node was created. It consists of the identifier of the process that created the node and the process-specific sequence number (which is incremented before the node is added to the tree). This time-stamp is copied if a node is \emph{replaced} with a new node (in a complex delete operation) as in~\cite{RamMit:2015:ICDCN}. Before a search operation starts traversing the tree, it reads the current sequence number values of all processes. Let $\labels[i]$ denote the value read for process $P_i$. The operation then stops the downward traversal of the tree once it encounters a node with time-stamp $\ang{i,v}$ such that $v > \labels[i]$. Clearly, this node and its descendants were added to the tree after the operation read the sequence number value of process $P_i$.
%%
A pseudo-code of the algorithm is given in \pseudoref{wf:search:height}.
\end{limitscope}

\chapter{Experimental evaluation}
\label{chapter:experiments}
We now describe the results of the comparative evaluation of different implementations of a concurrent BST using simulated workloads. This chapter is organized as follows. Performance evaluation of our lock-based algorithm is described in \secref{experiments:castle} followed by our lock-free algorithm described in \secref{experiments:icdcn}. Performance evaluation of our local recovery technique is described in \secref{experiments:localRecovery}.

\section{Experimental Setup} 
We conducted our experiments on a single large-memory node in stampede\footnote{https://www.tacc.utexas.edu/systems/stampede} cluster at TACC (Texas Advanced Computing Center). This node is a Dell PowerEdge R820 server with 4 Intel E5-4650 8-core processors (32 cores in total) and 1TB of DDR3 memory. Hyper-threading has been disabled on the node. It runs CentOS 6.3 operating system.  

To better understand the scalability of our algorithms we also conducted experiments on a single Intel Xeon Phi SE10P Coprocessor\footnote{http://www.intel.com/content/www/us/en/processors/xeon/xeon-phi-detail.html} having 61 1.1 GHz cores with 4 hardware threads per core and 8GB of GDDR5 memory.

We used Intel C/C++ compiler (version 2013.2.146) with optimization flag set to O3. We used GNU Scientific Library to generate random numbers. We used Intel's \emph{TBB Malloc}~\cite{Rei:2007:Book} as the dynamic memory allocator since it provided superior performance to C/C+ default allocator in a multi-threaded environment.

To compare the performance of different implementations, we considered the following parameters:
\begin{enumerate}[leftmargin=*, noitemsep]
\item \textbf{Relative Distribution of Operations:} We considered three different workload  distributions: 
						\begin{enumerate*}[label=(\alph*)]
						\item \emph{read-dominated:} 90\% search, 5\% insert and 5\% delete, 
						\item \emph{mixed:} 70\% search, 15\% insert and 15\% delete, and
						\item \emph{write-dominated:} 0\% search, 50\% insert and 50\% delete.
						\end{enumerate*}
\item \textbf{Maximum Degree of Contention:} This depends on number of threads that can concurrently operate on the tree. On 32 core machine, we varied the number of threads 
from 1 to 32 in powers of two. On 61 core machine we varied the number of threads from 1 to 244 in multiples of 61.
\item \textbf{Maximum Tree Size:} This depends on the size of the key space. To get the peak throughput, we set the number of threads to be the value where the peak performance is achieved and we varied key space size from 2\textsuperscript{13} (8Ki) to 2\textsuperscript{24} (16Mi).  To understand the scalability of the algorithms, we varied the number of threads and considered four different key ranges: 2,000 (2k), 20,000 (20K), 200,000 (200K) and 2 million (2M) keys.
\end{enumerate}

We compared the performance of different algorithms with respect to \emph{system throughput}, given by the number of operations executed per unit time. In each run of the experiment, we ran each algorithm for 10 seconds, and calculated the total number of operations completed by the end of the run to determine the system throughput. The results were averaged over 10 runs. To capture only the steady state behaviour, we \textit{pre-populated} the tree to 50\% of its maximum size, prior to starting a simulation run. The beginning of each run consisted of a 1 second ``warm-up'' phase whose numbers were excluded in the computed statistics to avoid initial caching effects. 

\section{Lock based tree}
\label{sec:experiments:castle}
\newenvironment{limitscope}{}{}
\begin{limitscope}
%%%%% castle macros - begin
\newcommand{\accesspath}{access-path}
\newcommand{\terminalnode}{terminal node}

\newcommand{\true}{\textsf{true}}
\newcommand{\false}{\textsf{false}}

\newcommand{\CAS}{\textsf{CAS}}

\newcommand{\sNodeOne}{\mathbb{R}}
\newcommand{\sNodeTwo}{\mathbb{S}}
\newcommand{\sKeyOne}{\infty_1}
\newcommand{\sKeyTwo}{\infty_2}

\newcommand{\targetnode}{target node}
\newcommand{\anchornode}{anchor node}

\newcommand{\myparent}{parent}
\newcommand{\myleft}{le\!f\!t}
\newcommand{\myright}{right}

\newcommand{\CASTLE}{\textsc{CASTLE}}
\newcommand{\CITRUS}{\textsc{CITRUS}}
\newcommand{\HJBST}{\textsc{LF-IBST}}
\newcommand{\NMBST}{\textsc{LF-EBST}}

\newcommand{\RemoveChild}{\textsc{RemoveChild}}
\newcommand{\LockAll}{\textsc{LockAll}}
\newcommand{\UnlockAll}{\textsc{UnlockAll}}
\newcommand{\ClearFlags}{\textsc{ClearFlags}}
\newcommand{\FindSmallest}{\textsc{FindSmallest}}

\newcommand{\lFlag}{lFlag}
\newcommand{\mFlag}{mFlag}
\newcommand{\nFlag}{nFlag}

%%%%% castle macros - end

\section{The Lock-Based Algorithm}
\label{sec:castle-algorithm}
We first provide an overview of our algorithm. We then describe the algorithm in more detail and also give its pseudo-code. For ease of exposition, we describe our algorithm assuming no memory reclamation, which can be performed using the well-known technique of hazard pointers~\cite{Mic:2004:TPDS}.

\section{Overview of the Algorithm}
Every operation in our algorithm uses \emph{seek} function as a subroutine. The seek function traverses the  tree from the root node until it either finds the target key or reaches a non-binary node whose next edge to be followed points to a null node. We refer to the path traversed by the operation during the seek  as the \emph{\accesspath}, and the last node in the \accesspath{} as the \emph{\terminalnode}. The operation then compares the target key with the stored key (the key present in the \terminalnode). Depending on the result of the comparison and the type of the operation, the operation either terminates or moves to the execution phase. In certain cases in which a key may have moved upward along the \accesspath, the seek function may have to restart and traverse the tree again; details about restarting are provided later. We now describe the next steps for each of the type of operation one-by-one. 

\paragraph{Search:} 
A search operation starts by invoking seek operation. It returns \true{} if the stored key matches the target key and \false{} otherwise. 

\paragraph{Insert:}
An insert operation starts by invoking seek operation. It returns \false{} if the target key matches the stored key; otherwise, it moves to the execution phase. In the execution phase, it attempts to insert the key into the tree as a child node of the last node in the \accesspath{} using a \CAS{} instruction. If the instruction succeeds, then the operation returns \true{}; otherwise, it restarts by invoking the seek function again.

\paragraph{Delete:} 
A delete operation starts by invoking seek function. It returns \false{} if the stored key does not match the target key; otherwise, it moves to the execution phase. In the execution phase, it attempts to remove the key stored in the \terminalnode{} of the \accesspath. There are two cases depending on whether the \terminalnode{} is a binary node (has two children) or not (has at most one child). In the first case, the operation is referred to as \emph{complex delete operation}. In the second case, it is referred to as \emph{simple delete operation}. In the case of simple delete, the \terminalnode{} is removed by changing the pointer at the parent node of the \terminalnode. In the 
case of complex delete, the key to be deleted is replaced with the \emph{next largest} key in the tree, which will be stored in the \emph{leftmost node} of the \emph{right subtree} of the \terminalnode.

\section{Details of the Algorithm}
\label{sec:description}

\input{localRecovery/pseudocode-localrecovery}

A pseudo-code of the local recovery algorithm is given in \pseudosref{local-data|structures}{local-seek:modify}. The pseudo-code only shows the seek phase of an algorithm and not its execution phase since the execution phase is algorithm-specific. We have also moved the pseudo-code for local recovery when looking for a successor key to the appendix due to lack of space.


The local recovery algorithm assumes that the original algorithm supports the following functions:
\begin{enumerate*}[label=(\alph*)]
%%
\item \GetKey(~), \IsMarked(~) and \GetChild(~) returns the various attributes of a tree node,
\item \IsNull(~) returns true if a reference is null and false otherwise,
\item \GetAddress(~) returns the node address stored in a reference, if non-null,
\item \Move(~) enables the original algorithm to move along an edge, which may invoke helping and restarting of the traversal as in~\cite{HowJon:2012:SPAA},
\item \NeedCleanParentNode(~) returns true if the operation needs the parent node to be clean and have no operation in progress (needed for a delete operation since it needs to modify a child pointer at the parent node), and
\item \PopulateSeekRecord(~) copies the relevant information from the traversal state required by the algorithm into a seek record.
\end{enumerate*}

\begin{comment}
\NeedSuccessorKey(~) evaluates if the successor key is still needed for the target key and returns a reference which is null if no successor key is needed and an address of the terminal node's right child otherwise
\end{comment}

\Pseudoref{local-data|structures} shows the data structures used by the local recovery algorithm. Note that all the data structures shown in \Pseudoref{local-data|structures} are \emph{local} to a process not shared among processes. A process uses three main data structures, namely \TraversalRecord{}, \OpRecord{} and \SeekRecord{}. A \TraversalRecord{} (\linesref{local-traversal|record:begin}{local-traversal|record:end}) is essentially a stack used to store the nodes visited during tree traversal when looking for a key (target or successor). Note that the traversal stack satisfies the last-in-first-out (LIFO) semantics but our algorithm sometimes uses it in a non-traditional way by accessing entries in the middle of the stack. One way to implement such an ``augmented'' stack is to use an auto-resizing vector provided as part of C++ STL library or Java package. Each entry in a traversal stack (\linesref{local-stack|entry:begin}{local-stack|entry:end}) stores the address of the node, the location of its closest \myanchor{} node (within the stack's vector) and whether the node is a left or right child of its parent. An \OpRecord{} (\linesref{local-op|record:begin}{local-op|record:end}) stores information about the operation such as type and key as well two stacks: one used when looking for the target key (all operations) and one used when looking for the successor key (only complex delete operations). Finally, a \SeekRecord{} (\linesref{local-seek|record:begin}{local-seek|record:end}) is used to return the outcome of a tree traversal to the original algorithm. Its fields are algorithm-specific. For example, for \CASTLE{}, \SeekRecord{} contains three fields: 
\begin{enumerate*}[label=(\alph*)]
\item two addresses, namely those of the target node and its parent, and
\item the contents of the injection point where an insert operation needs to attach the new node. 
\end{enumerate*}

\Pseudoref{local-stack|functions} shows the functions used to manipulate a traversal stack. The function \Size{} (\linesref{local-size:begin}{local-size:end}) returns the number of entries in the stack. The functions \GetTop{} (\linesref{local-get|top:begin}{local-get|top:end}) and \GetSecondToTop{} (\linesref{local-get|second|to|top:begin}{local-get|second|to|top:end}) return the address of the node stored in the topmost entry and the entry below it, respectively. The function \AddToTop{} (\linesref{local-add|to|top:begin}{local-add|to|top:end}) adds an entry to the top of the stack while \RemoveFromTop{} (\linesref{local-remove|from|top:begin}{local-remove|from|top:end}) removes an entry from the top of the stack. The function \RemoveUntilCritical{} (\linesref{local-remove|until|critical:begin}{local-remove|until|critical:end}) removes the entries from the top of the stack until a given point. The function \RememberCritical{} (\linesref{local-remember|critical:begin}{local-remember|critical:end}) updates the \myanchor{} field of the \myanchor{} node of the topmost entry in the stack. The function \GetFullEntry{} (\linesref{local-get|full|entry:begin}{local-get|full|entry:end} returns all the three fields of a given entry in the stack (may not be the topmost entry). The function \InitializeTraversalRecord{} (\linesref{local-initialize|traversal|record:begin}{local-initialize|traversal|record:end}) initializes a traversal stack. The stack for target key %is initialized using sentinel nodes while the stack for successor key is initialized as empty.

\Pseudosref[ \& ] {local-seek:search}{local-seek:search:2} shows the functions used to find the target key by a search operation. The function \SeekForSearch{} (\linesref{local-seek|search:begin}{local-seek|search:end}) first traverses the tree starting from the root node (\lineref{local-seek|search:traverse|tree}). If the traversal fails to locate the key, then the key may have moved up the tree. To address this possibility, the function examines the traversal stack to determine whether or not that is the case (\lineref{local-seek|search:examine|stack}). The function \TraverseTree{} (\linesref{local-traverse|tree:begin}{local-traverse|tree:end}) first initializes the traversal stack (\lineref{local-traverse|tree:initialize}) and then, starting from the topmost node in the stack (\lineref{local-traverse|tree:start}), follows either the left or the right child pointer (\lineref{local-traverse|tree:select}) until it either finds the key (\lineref{local-traverse|tree:match}) or encounters a null pointer (\lineref{local-traverse|tree:null}). It also populates the traversal stack as it moves (\lineref{local-traverse|tree:stack}). The function \ExamineStack (\linesref{local-examine|stack:begin}{local-examine|stack:end}) examines the \myanchor{} nodes stored in the stack in the reverse order in which they were visited, starting from the \myanchor{} node closest to the topmost node in the traversal stack (\lineref{local-examine|stack:start}). If the \myanchor{} node's key matches the target key, then the function returns true (\linesref{local-examine|stack:while:found:begin}{local-examine|stack:while:found:end}). If the \myanchor{} node is no longer \myconsistent{} or is unmarked, then the function returns false (\linesref{local-examine|stack:while:not|found:begin}{local-examine|stack:while:not|found:end}). Otherwise, the function backtracks and examines the preceding \myanchor{} node in the stack (\linesref{local-examine|stack:while:continue:begin}{local-examine|stack:while:continue:end}).

\Pseudosref{local-local:recovery}{local-local:recovery:2} $\&$ ~\ref{algo:local-seek:modify} show the functions used to find the target key by a modify (insert or delete) operation. The function \SeekForModify{} (\linesref{local-seek|modify:begin}{local-seek|modify:end}) first backtracks to a \mysafe{} node in the stack (\lineref{local-seek|modify:while:find|start|point}). Initially, the starting point is typically a sentinel node which is a \mysafe{} node. The function then traverses the tree from top to down by following either the left or the right child pointer (\lineref{local-seek|modify:while:traversal:select}) until it either finds the key or encounters a null pointer (\linesref{local-seek|modify:while:traversal:stop:begin}{local-seek|modify:while:traversal:stop:end}). In case the terminal node's key is greater than the target key, the function checks whether the path stored in the traversal stack is still valid (\lineref{local-seek|modify:while:traversal:find|admissible}). If not, the traversal is restarted. As the traversal moves down the tree, the function also populates the traversal stack (\linesref{local-seek|modify:while:traversal:move:begin}{local-seek|modify:while:traversal:move:end}). The function \FindAdmissible{} (\linesref{local-find|admissible:begin}{local-find|admissible:end}) checks whether or not the path stored in the stack is still valid. To that end, it examines  the \myanchor{} nodes in the stack in the reverse order in which they were visited, starting from the \myanchor{} node closest to the topmost node in the traversal stack. There are three possible cases. First, the \myanchor{} node is still consistent (\linesref{local-find|admissible:while:consistent:begin}{local-find|admissible:while:consistent:end}). In this case, the path is deemed to be valid if the \myanchor{} node is unmarked; otherwise, the function moves to the preceding \myanchor{} node. Second, the \myanchor{} node is no longer consistent (\linesref{local-find|admissible:while:not|consistent:begin}{local-find|admissible:while:not|consistent:end}). In this case, the path is deemed to be invalid. However, if the operation is a delete operation, then it can be deduced that the key did not exist in the tree  continuously and the function returns indicating that the key was not found (thereby causing the operation to terminate). Finally, the \myanchor{} node's key matches the target key (\linesref{local-find|admissible:while:match:begin}{local-find|admissible:while:match:end}). In this case, if the \myanchor{} node is marked and the operation is a delete operation, then the path is deemed to be invalid (and further backtracking is required). This is because the key may be in the process of moving up the tree. Otherwise, the function returns indicating that the key was found. The function \FindStartPoint{} (\linesref{local-find|start|point:begin}{local-find|start|point:end}) finds a \mysafe{} node on the path stored in the stack from which the operation can restart its traversal. To that end, it backtracks to an unmarked node with a clean parent if required (\linesref{local-find|start|point:while:backtrack:begin}{local-find|start|point:while:backtrack:end}). It then checks whether or not the remaining path in the stack is still valid (\lineref{local-find|start|point:while:find|admissible}). If not, it repeats the above-mentioned steps.

\input{localRecovery/pseudocode-seekForSuccessor}

\Pseudoref{seek:successor} shows the function \SeekForSuccessor{} used to locate the successor key by a complex delete operation (\linesref{local-seek|successor:begin}{local-seek|successor:end}). The function first backtracks to an unmarked node with a clean parent if required (\linesref{local-seek|successor:while:backtrack:begin}{local-seek|successor:while:clean:end}). It then checks whether or not the successor key is still needed by invoking \NeedSuccessorKey{} function (\lineref{local-seek|successor:while:need|successor}). The function \NeedSuccessorKey{} returns a reference, which is null if the successor key is no longer needed and contains the address of the target node's right child otherwise. This address is used as a traversal point if the stack only contains a single entry (the node whose key needs to be replaced). If the successor key is still needed, then the function repeatedly follows the left child pointer until it encounters a null pointer (\linesref{local-seek|successor:while:traversal:begin}{local-seek|successor:while:traversal:end}). While moving down the tree, the function also populates the traversal stack (\lineref{local-seek|successor:while:traversal:stack}).

\subsection{Formal Description}

We refer to our algorithm as \CASTLE{} (\underline{C}oncurrent \underline{A}lgorithm for Binary \underline{S}earch \underline{T}ree by \underline{L}ocking \underline{E}dges). 

\begin{limitscope}

%% To limit the scope of the macros defined below

%% macros for pseudocode
\newcommand{\leftChild}{le\!f\!t}
\newcommand{\rightChild}{right}
\newcommand{\child}{child}
\newcommand{\canReplace}{readyToReplace}
\newcommand{\markAndKey}{mKey}

\newcommand{\node}{node}
\newcommand{\parent}{parent}

\newcommand{\terminalEdge}{lastEdge}
\newcommand{\targetEdge}{targetEdge}
\newcommand{\parentTargetEdge}{pTargetEdge}
\newcommand{\successorEdge}{successorEdge}
\newcommand{\parentSuccessorEdge}{pSuccessorEdge}
\newcommand{\injectionEdge}{injectionEdge}
\newcommand{\penultimateEdge}{pLastEdge}

\newcommand{\targetKey}{targetKey}
\newcommand{\currentKey}{currentKey}

\newcommand{\newNode}{newNode}
\newcommand{\reference}{re\!f\!erence}
\newcommand{\state}{state}

\newcommand{\StateRecord}{StateRecord}
\newcommand{\AnchorRecord}{AnchorRecord}

\newcommand{\mline}[1]{\DontPrintSemicolon #1 \PrintSemicolon}

\newcommand{\prev}{prev}
\newcommand{\curr}{curr}

\newcommand{\prevSeekRecord}{pSeekRecord}
\newcommand{\prevAnchorRecord}{pAnchorRecord}
%% \newcommand{\currSeekRecord}{cSeekRecord}
\newcommand{\anchorRecord}{anchorRecord}

\newcommand{\oldContents}{oldValue}
\newcommand{\newContents}{newValue}

\newcommand{\INJECTION}{\textsf{INJECTION}}
\newcommand{\DISCOVERY}{\textsf{DISCOVERY}}
\newcommand{\CLEANUP}{\textsf{CLEANUP}}
\newcommand{\FINISHED}{\textsf{FINISHED}}

\newcommand{\DELETEFLAG}{\textsf{DELETE\_FLAG}}
\newcommand{\PROMOTEFLAG}{\textsf{PROMOTE\_FLAG}}
\newcommand{\INTENTFLAG}{\textsf{INTENT\_FLAG}}
\newcommand{\flag}{f\!lag}

\newcommand{\COMPLEX}{\textsf{COMPLEX}}
\newcommand{\SIMPLE}{\textsf{SIMPLE}}

\newcommand{\LEFT}{\textsf{LEFT}}
\newcommand{\RIGHT}{\textsf{RIGHT}}

\newcommand{\targetSeekRecord}{targetRecord}
\newcommand{\successorSeekRecord}{successorRecord}

\newcommand{\dFlag}{d}
\newcommand{\iFlag}{i}
\newcommand{\pFlag}{p}
\newcommand{\nFlag}{n}
\newcommand{\mFlag}{m}
\newcommand{\lNFlag}{lN}
\newcommand{\rNFlag}{rN}

\newcommand{\rarrow}{\!\rightarrow\!}


%%%%%%%%%%%%%%%%%%%%%%%%%%%%%%%%%%%%%%%%%%%%%%%%%%%%%%%%%%%%%%%%%%%%%%%%%%%%%%%%%%%%

\newcommand{\DefineKeyWords}{
%%
\SetKw{Boolean}{Boolean}
\SetKw{LAnd}{~and~}
\SetKw{LOr}{~or~}
\SetKw{LNot}{not}
\SetKw{Struct}{struct}
\SetKw{Null}{null}
\SetKw{True}{true}
\SetKw{False}{false}
\SetKw{Break}{break}
\SetKw{Continue}{continue}
\SetKw{Enum}{enum}
%%
}

%%%%%%%%%%%%%%%%%%%%%%%%%%%%%%%%%%%%%%%%%%%%%%%%%%%%%%%%%%%%%%%%%%%%%%%%%%%%%%%%%%%%%

%% DATA STRUCTURES


\begin{algorithm}[htp]
%%
\DefineKeyWords
%%

%% define data structures used in the algorithm

\DontPrintSemicolon
\Struct Node \{\;
\label{ln:icdcn-node|begin}
\PrintSemicolon
\Indp 
   $\{ \Boolean, \text{Key} \}$ $\markAndKey$\;
   $\{ \Boolean, \Boolean, \Boolean, \Boolean, \text{NodePtr} \}$ $\child[2]$\;
   \Boolean $\canReplace$\;
\Indm
\}\;
\label{ln:icdcn-node|end}
%%
\BlankLine

\DontPrintSemicolon
\Struct Edge \{\;
\label{ln:icdcn-edge|begin}
\PrintSemicolon
\Indp 
   %% NodePtr $\parent$\;
   %% NodePtr $\child$\;
	 NodePtr $\parent$, $\child$\;
   \Enum $which$ \{ \LEFT{}, \RIGHT{} \}\;
\Indm
\}\;
\label{ln:icdcn-edge|end}
%%
\BlankLine

\DontPrintSemicolon
\Struct SeekRecord \{\;
\PrintSemicolon
\Indp 
%%
   %% Edge $\terminalEdge$\;
%%
   %% Edge $\penultimateEdge$\;
%%
   %% Edge $\injectionEdge$\;
   Edge $\terminalEdge$, $\penultimateEdge$, $\injectionEdge$\;
\Indm
\}\;
%%
\BlankLine


\BlankLine
\DontPrintSemicolon
\Struct \AnchorRecord{} \{\;
\PrintSemicolon
\Indp 
   NodePtr $\node$\;
   Key $key$\;
\Indm
\}\;
%%

\BlankLine
\DontPrintSemicolon
\Struct \StateRecord{} \{\;
\PrintSemicolon
\Indp 
%%
   %% int $depth$\;
   %% Edge $\targetEdge$\;
	 %% Edge $\parentTargetEdge$\;
	 Edge $\targetEdge$, $\parentTargetEdge$\;
%%
   %% Key $\targetKey$\;
	 %% Key $\currentKey$\;
	 Key $\targetKey$, $\currentKey$\;
   \Enum $mode$ \{ \INJECTION{}, \DISCOVERY{}, \CLEANUP{} \}\;
   \Enum $type$ \{ \SIMPLE{}, \COMPLEX{} \} \;
%%
   \tcp{the next field stores pointer to a seek record; it is used for finding the successor if the delete operation is complex}
   SeekRecordPtr $\successorSeekRecord$\; 
\Indm
\}\;
%%
\BlankLine
\tcp{object to store information about the tree traversal when looking for a given key (used by the seek function)}
SeekRecordPtr $\targetSeekRecord$ := new seek record\;
\tcp{object to store information about process' own delete operation}
\StateRecord{Ptr} $myState$ := new state\;


\caption{Data Structures Used}
\label{algo:icdcn-data|structures}
\end{algorithm}



\begin{algorithm}[htp]
%%
\DefineKeyWords


%% SEEK


%%
%% traverses the tree from the root node to a leaf node looking for a given key
%%
\DontPrintSemicolon
\Seek( $key$, $seekRecord$ )\;
\PrintSemicolon
\Begin
{
   $\prevAnchorRecord$ := $\curly{ \snodetwo{}, \skey{1} }$\;
   \While{\True}
   {
	    \tcp{initialize all variables used in traversal}
		  $\penultimateEdge$ := $\curly{ \snodeone, \snodetwo, \RIGHT }$; \qquad
			$\terminalEdge$ := $\curly{ \snodetwo, \snodethree, \RIGHT }$\;
			$\curr$ := $\snodethree$; \qquad
			$\anchorRecord$ := $\curly{ \snodetwo{}, \skey{1} }$\;
			\BlankLine
			\While{\True}
			{
			    \tcp{read the key stored in the current node}
			    $\ang{ \ast, cKey }$ := $\curr \rarrow \markAndKey$\;
				  \tcp{find the next edge to follow}
					$which$ := $key < cKey$ ? \LEFT : \RIGHT\;
				  $\ang{ \nFlag, \ast, \dFlag, \pFlag, next }$ := $\curr \rarrow \child[which]$\;
					\tcp{check for the completion of the traversal}
				  \If{$key = cKey$ \LOr $\nFlag$}
				  {
				     \tcp{either key found or no next edge to follow; stop the traversal}
						 $seekRecord \rarrow \penultimateEdge$ := $\penultimateEdge$\;
						 $seekRecord \rarrow \terminalEdge$ := $\terminalEdge$\;
						 $seekRecord \rarrow \injectionEdge$ := $\curly{ \curr, next, which }$\;
						 \BlankLine
						 \uIf(\tcp*[h]{keys match}){$key = cKey$}
						 { 
						    \Return\;
						 }
						 \lElse { \Break }
				  }
				  \BlankLine			   
				  \If{$which$ = \RIGHT}
				  {
				     \tcp{next edge to be traversed is a right edge; keep track of the current node and its key}
						 $\anchorRecord$ := $\ang{ \curr, cKey }$\;
				  }	
				  \BlankLine
				  \tcp{traverse the next edge}
					$\penultimateEdge$ := $\terminalEdge$; \qquad
					$\terminalEdge$ := $\curly{ \curr, next, which }$; \qquad
				  $\curr$ := $next$\; 
		  }
		  \tcp{key was not found; check if can stop}
		  $\ang{ \ast, \ast, \dFlag, \pFlag, \ast }$ := $\anchorRecord.\node \rarrow \child[\RIGHT]$\;			
			\uIf{\LNot($\dFlag$) \LAnd \LNot($\pFlag$)}
			{
			   \tcp{anchor node is still part of the tree; check if anchor node's key has changed}
				 $\ang{ \ast, aKey }$ := $\anchorRecord.\node \rarrow \markAndKey$\;
				 \lIf{$\anchorRecord.key$ = $aKey$}
			   {  
				    \Return
				 } 
			}	
			\Else
			{ 
			   \tcp{check if the anchor record (the node and its key) matches that of the previous traversal}
			   \If{$\prevAnchorRecord = \anchorRecord$}
			   {
				    \tcp{return the results of the previous traversal}
					  $seekRecord$ := $\prevSeekRecord$\;
				    \Return\;
		     }
			}
			\tcp{store the results of the traversal and restart}
			$\prevSeekRecord$ := $seekRecord$; \qquad
			$\prevAnchorRecord$ := $\anchorRecord$;					
   }
} 
%% End of seek function
\caption{Seek Function}
\label{algo:icdcn-seek}
\end{algorithm}


%% SEARCH
\begin{algorithm}[htp]
%%
\DefineKeyWords
\DontPrintSemicolon
\Boolean \Search( $key$ )\;
\PrintSemicolon
\Begin
{
   \Seek( $key$, $mySeekRecord$ )\;
	 \BlankLine
	 %% $\node$ := $mySeekRecord \rarrow \node$\;
   $\node$ := $mySeekRecord \rarrow \terminalEdge.\child$\;
   $\ang{ \ast , nKey }$ := $\node \rarrow \markAndKey$\;
	 \BlankLine
   \lIf{nKey = key}{\Return \True}
   \lElse{\Return \False}
}
\caption{Search Operation}
\label{algo:icdcn-search}
\end{algorithm}

%% INSERT
\begin{algorithm}[htp]
%%
\DefineKeyWords
\DontPrintSemicolon
\Boolean \Insert( $key$ )\;
\PrintSemicolon
\Begin
{
   \While{\True}
	 {
      \Seek( $key$, $\targetSeekRecord$ )\;
			\BlankLine
			$\targetEdge$ := $\targetSeekRecord \rarrow \terminalEdge$\;
			$\node$ := $\targetEdge.\child$\;
			$\ang{ \ast , nKey }$ := $\node \rarrow \markAndKey$\; 
			\lIf{$key = nKey$}{\Return \False}
			\BlankLine
			\tcp{create a new node and initialize its fields}
			$\newNode$ := create a new node\;
			$\newNode \rarrow \markAndKey$ := $\ang{ 0_m, key }$\;
			$\newNode \rarrow \child[\LEFT]$ := $\ang{ 1_n, 0_i, 0_d, 0_p, \Null }$\;
			$\newNode \rarrow \child[\RIGHT]$ := $\ang{ 1_n, 0_i, 0_d, 0_p, \Null }$\;
			$\newNode \rarrow \canReplace$ := \False\;
			\BlankLine
			$which$ := $\targetSeekRecord \rarrow \injectionEdge.which$\;
			$address$ := $\targetSeekRecord \rarrow \injectionEdge.\child$\;
			$result$ := \CAS($\node \rarrow \child[which]$, $\ang{ 1_n, 0_i, 0_d, 0_p, address }$, $\ang{ 0_n, 0_i, 0_d, 0_p, \newNode }$)\;
			\lIf{$result$}{\Return \True}
			\BlankLine	
			\tcp{help if needed}
		  $\ang{ \ast, \ast, \dFlag, \pFlag, \ast }$ := $\node \rarrow \child[which]$\;
			\lIf{$\dFlag$}
			{
			   \HelpTargetNode( $\targetEdge$ )
			} 
			\lElseIf{$p$}
			{  
			   \HelpSuccessorNode( $\targetEdge$ )
			}
	}
}
\caption{Insert Operation}
\label{algo:icdcn-insert}
\end{algorithm}

%% DELETE
\begin{algorithm}[htp]
\DefineKeyWords
\DontPrintSemicolon
\Boolean \Delete( $key$ )\;
\PrintSemicolon
\Begin
{
   \tcp{initialize the state record}
 	 $myState \rarrow \targetKey$ := $key$; $\qquad$
	 $myState \rarrow \currentKey$ := $key$\;
	 $myState \rarrow mode$ := \INJECTION\;
	 \BlankLine
   \While{\True}
	 {
      \Seek( $myState \rarrow \currentKey$, $\targetSeekRecord$ )\;
			$\targetEdge$ := $\targetSeekRecord \rarrow \terminalEdge$; $\qquad$
			$\parentTargetEdge$ := $\targetSeekRecord \rarrow \penultimateEdge$\;
			$\ang{ \ast , nKey }$ := $\targetEdge.\child \rarrow \markAndKey$\; 
			\BlankLine
			\If{$myState \rarrow \currentKey \neq nKey$}
			{
			   \tcp{the key does not exist in the tree}
			   \lIf{$myState \rarrow mode$ = \INJECTION}{\Return \False}
				 \lElse{\Return \True}
			}
 	    \BlankLine		   	
			\tcp{perform appropriate action depending on the mode}
	    \If{$myState \rarrow mode$ = \INJECTION}
		  {
				 \tcp{store a reference to the target edge}
   	     $myState \rarrow \targetEdge$ := $\targetEdge$\;
   	     $myState \rarrow \parentTargetEdge$ := $\parentTargetEdge$\;
				 \tcp{attempt to inject the operation at the node}
				 %% $result$ := \Inject( $myState$ )\;
				 \Inject( $myState$ )\;					 								 
			}
			\BlankLine
			\tcp{mode would have changed if injection was successful}
				 
			\If{$myState \rarrow mode \neq$ \INJECTION}
			{
				 \tcp{check if the target node found by the seek function matches the one stored in the state record}			
			   %%\If{$\left(\text{\parbox[c]{1.75in}{$myState \rarrow \targetEdge.\child$ $\neq$  \\ \mbox{}\hfill$\targetEdge.\child$}}\right)$}
				 \lIf{$myState \rarrow \targetEdge.\child$ $\neq$ $\targetEdge.\child$}
				 {
				    \Return \True
				 }						
				 \tcp{update the target edge information using the most recent seek}
				 $myState \rarrow \targetEdge$ := $\targetEdge$\; 			 				
		  }				
			\BlankLine							
			\If{$myState \rarrow mode$ = \DISCOVERY}
			{
				 \tcp{complex delete operation; locate the successor node and mark its child edges with promote flag} 
			   \FindAndMarkSuccessor( $myState$ )\;			 
			}			
			\If{$myState \rarrow mode$ = \DISCOVERY}
			{
				 \tcp{complex delete operation; promote the successor node's key and remove the successor node}
		     \RemoveSuccessor( $myState$ )\;						
			}			
			\BlankLine				
			\If{$myState \rarrow mode$ = \CLEANUP}
			{
			   \tcp{either remove the target node (simple delete) or replace it with a new node with all fields unmarked  (complex delete)}
			   $result$ := \Cleanup( $myState$ )\;
				 \lIf{$result$}{\Return \True}
				 \Else{
				    $\ang{ \ast, nKey }$ := $\targetEdge.\child \rarrow \markAndKey$\;
						$myState \rarrow \currentKey$ := $nKey$\;
				 }					
		  }
	 }	
   %% \Return\;
}
\caption{Delete Operation}
\label{algo:icdcn-delete}
\end{algorithm}

%% INJECT
\begin{algorithm}[htp]
%%
\DefineKeyWords
\DontPrintSemicolon
\Inject( $\state$ )\;
\PrintSemicolon
\Begin
{
   $\targetEdge$ := $\state \rarrow \targetEdge$\;
	 \tcp{try to set the intent flag on the target edge}
	 \tcp{retrieve attributes of the target edge}
	 $\parent$ := $\targetEdge.\parent$\;
	 $\node$ := $\targetEdge.\child$\;
	 $which$ := $\targetEdge.which$\;
	 \BlankLine
	 \mline{$result$ := \CAS( \parbox[t]{2.075in}{$\parent \rarrow \child[which]$, \\ $\ang{ 0_n, 0_i, 0_d, 0_p, \node }$,  $\ang{ 0_n, 1_i, 0_d, 0_p, \node }$ );}\;}
	 \If{\LNot($result$)}
	 {
	    \tcp{unable to set the intent flag; help if needed}
			$\ang{ \ast, \iFlag, \dFlag, \pFlag, address }$ := $\parent \rarrow \child[which]$\;
			\lIf{$\iFlag$}
			{
			   \HelpTargetNode( $\targetEdge$ )
			} 
			\uElseIf{$\dFlag$}
			{
			   \HelpTargetNode( $\state \rarrow \parentTargetEdge$ )\;
			} 
			\ElseIf{$\pFlag$}
			{
			   \HelpSuccessorNode( $\state \rarrow \parentTargetEdge$ )\;
			}

      \Return;					
	 }

   \BlankLine
	 \tcp{mark the left edge for deletion}

	 $result$ := \MarkChildEdge( $\state$, \LEFT{} )\;
	 
	 \lIf{\LNot($result$)}
	 {
	    \Return
	 } 
	 \tcp{mark the right edge for deletion; cannot fail}
	 \MarkChildEdge( $\state$, \RIGHT{} )\;
	   
	 \BlankLine
	 \tcp{initialize the type and mode of the operation}
	 \InitializeTypeAndUpdateMode( $\state$ );	
}

\caption{Injecting a Deletion Operation}
\label{algo:icdcn-inject}
\end{algorithm}









%% FINDANDMARKSUCCESSOR


\begin{algorithm}[htp]
%%
\DefineKeyWords

\DontPrintSemicolon
\FindAndMarkSuccessor( $\state$ )\;
\PrintSemicolon
\Begin
{
   \tcp{retrieve the addresses from the state record}
   $\node := \state \rarrow \targetEdge.\child$\;
	 $seekRecord$ := $\state \rarrow \successorSeekRecord$\; 
   
	 \BlankLine
   \While{\True}
	 {
	 
	    \tcp{read the mark flag of the key in the target node}  
	    $\ang{ \mFlag, \ast}$ := $\node \rarrow \markAndKey$\; 
	    

	  	\tcp{find the node with the smallest key in the right subtree}
	    $result$ := \FindSmallest( $\state$ )\;
			
						
			\BlankLine
			\If{$\mFlag$ \LOr \LNot($result$)} 
			{
			   \tcp{successor node had already been selected \emph{before} the traversal or the right subtree is empty}
				 \Break\;
			}
			
				
			\tcp{retrieve the information from the seek record}
			$\successorEdge$ := $seekRecord \rarrow \terminalEdge$\;
			%% $\ang{ \nFlag, \ast, \ast, \ast, \leftChild}$ :=  $\successorEdge.\child \rarrow \child[\LEFT]$\;
			%% \lIf{\LNot($\nFlag$)}{ \Continue }
			$\leftChild$ := $seekRecord \rarrow \injectionEdge.\child$\;
			
			\BlankLine
			\tcp{read the mark flag of the key under deletion}
      $\ang{ \mFlag, \ast}$ := $\node \rarrow \markAndKey$\;
			
			\If(\tcp*[h]{successor node has already been selected}){$\mFlag$}
			{
			   %% \tcp{successor node has already been selected}
			   %% \lIf{$p$}{ \Break }
				 %% \lElse{ \Continue }
				 \Continue\;
				 
			}
			
			
		


          
			\tcp{try to set the promote flag on the left edge}
			\mline{$result$ := \CAS( \parbox[t]{1.875in}{$\successorEdge.\child \rarrow \child[\LEFT]$, \\ 
			                                             $\ang{ 1_n, 0_i, 0_d, 0_p, \leftChild }$, \\ $\ang{ 1_n, 0_i, 0_d, 1_p, \node }$ );}\;}
			
			\lIf{$result$}{\Break}
			
			\BlankLine
			\tcp{attempt to mark the edge failed; recover from the failure and retry if needed}
			%% $\ang{ n, \ast, d, p, \ast }$ := $\successorEdge.\child \rarrow \child[\LEFT]$\;
			$\ang{ \nFlag, \ast, \dFlag, \ast, \ast }$ := $\successorEdge.\child \rarrow \child[\LEFT]$\;
			
			
      %% \lIf{$p$}
      %% {
      %%    \Break
      %% }   

      %% \If{\LNot($n$)}
			%% { 
			%%   \tcp{the node found has since gained a left child}
			%%   \Continue\;
			%% }

			\If{$\nFlag$ \LAnd $\dFlag$}
			{
			    \tcp{the node found is undergoing deletion; need to help}
					
								
					%% \mline{\HelpTargetNode( \parbox[t]{1.5in}{$\successorEdge$, \\ $\state \rarrow depth + 1$ );}\;}
		      \HelpTargetNode( $\successorEdge$ )\;
       } 
	 }	
   \BlankLine
   \tcp{update the operation mode}
	 \UpdateMode( $\state$ );
}

\caption{Locating the Successor Node}
\label{algo:icdcn-findandmark}
\end{algorithm}




%% REMOVESUCCESSOR
\begin{algorithm}[htp]
%%
\DefineKeyWords
\DontPrintSemicolon
\RemoveSuccessor( $\state$ )\;
\PrintSemicolon
\Begin
{
   \tcp{retrieve addresses from the state record}
   $\node$ := $\state \rarrow \targetEdge.\child$\;
   $seekRecord$ := $\state \rarrow \successorSeekRecord$\;
   \tcp{extract information about the successor node}
	 %% \tcp{assumes that the state's seek record contains valid information}
   $\successorEdge$ := $seekRecord \rarrow \terminalEdge$\;
	 \BlankLine
	 \tcp{ascertain that seek record for successor node contains valid information}
	 $\ang{ \ast, \ast, \ast, \pFlag, address }$ := $\successorEdge.\child \rarrow \child[\LEFT]$\;
	 \If{\LNot($\pFlag$) \LOr ($address$ $\neq$ $\node$)}
	 {
	    $\node \rarrow \canReplace$ := \True\;
			\UpdateMode( $\state$ )\;
	    \Return\;
	 }
   \BlankLine
   \tcp{mark the right edge for promotion if unmarked}
   \MarkChildEdge( $\state$, \RIGHT{} )\; 
   \BlankLine
   \tcp{promote the key}
   $\node \rarrow \markAndKey$ := $\ang{ 1_m, \successorEdge.\child \rarrow \markAndKey }$\;
   \While{\True}
   {
      \tcp{check if the successor is the right child of the target node itself}
	    \uIf{$\successorEdge.\parent$ = $\node$}
	    {
	       \tcp{need to modify the right edge of target node whose delete flag is set}
				 $dFlag$ := 1; \qquad
			   $which$ := \RIGHT\;
	    }
	    \Else
	    {
			   $dFlag$ := 0; \qquad
			   $which$ := \LEFT\;
	    }
      $\ang{ \ast, \iFlag, \ast, \ast, \ast }$ := $\successorEdge.\parent \rarrow \child[which]$\;			
      \BlankLine			
	    $\ang{ \nFlag, \ast, \ast, \ast, \rightChild }$ := $\successorEdge.\child \rarrow \child[\RIGHT]$\;	
	    $\oldContents$ := $\ang{ 0_n, \iFlag, dFlag, 0_p, \successorEdge.\child }$\;	    
			\uIf(\tcp*[f]{only set the null flag; do not change the address}){$\nFlag$}
	    {				
				 %\mline{\parbox[t]{1.75in}{$\newContents$ := \\ \mbox{} \qquad $\ang{ 1_n, 0_i, dFlag, 0_p,  \successorEdge.\child }$;}}
				 $\newContents$ := $\ang{ 1_n, 0_i, dFlag, 0_p,  \successorEdge.\child }$\;
	    }
	    \Else(\tcp*[f]{switch the pointer to point to the grand child})
	    {	 				 
				 $\newContents$ := $\ang{ 0_n, 0_i, dFlag, 0_p, \rightChild }$ \;		 
	    }	 
      \remove{ \lIf{$result$}{\Break} }			
			%\mline{$result$ := \CAS( \parbox[t]{1.77in}{$\successorEdge.\parent \rarrow \child[which]$, \\ $\oldContents$, $\newContents$ );}\;}
			$result$ := \CAS($\successorEdge.\parent \rarrow \child[which]$,$\oldContents$, $\newContents$)\;
			\lIf{$result$ \LOr $dFlag$}{ \Break }
	    %\BlankLine			
			$\ang{ \ast, \ast, \dFlag, \ast, \ast }$ := $\successorEdge.\parent \rarrow \child[which]$\;
			$\penultimateEdge$ := $seekRecord \rarrow \penultimateEdge$\;
			\If{$\dFlag$ \LAnd ($\penultimateEdge.\parent$ $\neq$ \Null)}
			{
			   %% \mline{\HelpTargetNode( \parbox[t]{1.25in}{$\penultimateEdge$, \\ $\state \rarrow depth + 1$ );}\;}
				 \HelpTargetNode( $\penultimateEdge$ )\;
			}			
      \BlankLine			
 	    $result$ := \FindSmallest( $\state$ )\;
			$\terminalEdge$ := $seekRecord \rarrow \terminalEdge$\;
	    %\If{$\left(\text{\parbox[c]{1.875in}{\LNot($result$) \LOr \\ $\terminalEdge.\child$ $\neq$ $\successorEdge.\child$}}\right)$}
			\If{\LNot($result$) \LOr $\terminalEdge.\child$ $\neq$ $\successorEdge.\child$}
			{
			   \Break;
				 \tcp*[f]{the successor node has already been removed}
			} 
			\lElse
			{
			   $\successorEdge$ := $seekRecord \rarrow \terminalEdge$
			}
   }
   \BlankLine
	 $\node \rarrow \canReplace$ := \True\;
   \UpdateMode( $\state$ )\;	
}
\caption{Removing the Successor Node}
\label{algo:icdcn-remove}
\end{algorithm}



%% CLEANUP

\begin{algorithm}[htp]
%%
\DefineKeyWords



\DontPrintSemicolon
\Boolean \Cleanup( $\state$ )\;
\PrintSemicolon
\Begin
{
   %% \tcp{retrieve the addresses from the state record}
   %% $\node$ := $\state \rarrow \node$\;
	 %% $\parent$ := $\state \rarrow \parent$\;
	
	 %% \BlankLine
	
	 %% \tcp{determine which edge of the parent needs to be switched} 
	 %% $\ang{ \ast, pKey }$ := $\parent \rarrow \markAndKey$\;
	 %% $\ang{ \ast, nKey }$ := $\node \rarrow \markAndKey$\;
	 %% $pWhich$ := $nKey < pKey$ ? \LEFT : \RIGHT\;
	 $\ang{\parent, \node, pWhich}$ := $\state \rarrow \targetEdge$\;
	 
	
	 \BlankLine
	 
	 \uIf{$\state \rarrow type$ = \COMPLEX}
	 {
	  	   
	    \tcp{replace the node with a new copy in which all fields are unmarked} 
			$\ang{ \ast, nKey }$ := $\node \rarrow \markAndKey$\;
			$newNode \rarrow \markAndKey$ := $\ang{ 0_m, nKey }$\;		
			\BlankLine
			\tcp{initialize left and right child pointers}
  		$\ang{ \ast, \ast, \ast, \ast, \leftChild }$ := $\node \rarrow \child[\LEFT]$\;
			$\newNode \rarrow \child[\LEFT]$  := $\ang{ 0_n, 0_i, 0_d, 0_p, \leftChild }$\;
			$\ang{ \nFlag, \ast, \ast, \ast, \rightChild }$ := $\node \rarrow \child[\RIGHT]$\;
			\uIf{$\nFlag$}
			{
			  $\newNode \rarrow \child[\RIGHT]$  := $\ang{ 1_n, 0_i, 0_d, 0_p, \Null }$\;
			}
			\lElse
			{
			  $\newNode \rarrow \child[\RIGHT]$  := $\ang{ 0_n, 0_i, 0_d, 0_p, \rightChild }$
			}
			\BlankLine
			\tcp{initialize the arguments of \CAS{} instruction}
			$\oldContents$ := $\ang{ 0_n, 1_i, 0_d, 0_p, \node }$\;
			$\newContents$ := $\ang{ 0_n, 0_i, 0_d, 0_p, \newNode }$\;
			
			%% \tcp{switch the edge at the parent}
			%% \mline{$result$ := \CAS( \parbox[t]{1.875in}{$\parent \rarrow \child[pWhich]$, \\ $\ang{ 0_n, 1_i, 0_d, 0_p, \node }$, $\ang{ 0_n, 0_i, 0_d, 0_p, \newNode }$ );}\;}
			 
	
	 }
	 \Else(\tcp*[h]{remove the node})
	 {
	   			
	    %% \tcp{remove the node}
			
			\tcp{determine to which grand child will the edge at the parent be switched}
			\uIf{$\node \rarrow \child[\LEFT]$ = $\ang{ 1_n, \ast, \ast, \ast, \ast }$}
			{
		     $nWhich$ := \RIGHT\;
			}
			\lElse{$nWhich$ := \LEFT}
			
			\BlankLine
			\tcp{initialize the arguments of the \CAS{} instruction}
			$\oldContents$ := $\ang{ 0_n, 1_i, 0_d, 0_p, \node }$\;
			$\ang{ \nFlag, \ast, \ast, \ast, address }$ := $\node \rarrow \child[nWhich]$\; 
  		\uIf(\tcp*[h]{set the null flag only}){$\nFlag$}
			{
			   $\newContents$ := $\ang{ 1_n, 0_i, 0_d, 0_p, \node }$\;
			   %% \tcp{set the null flag only; do not change the address}
			   %% \mline{$result$ := \CAS( \parbox[t]{1.25in}{$\parent \rarrow \child[pWhich]$, \\ $\ang{ 0_n, 1_i, 0_d, 0_p, \node }$, \\ $\ang{ 1_n, 0_i, 0_d, 0_p, \node }$ );}\;}
			}
			\Else(\tcp*[h]{change the pointer to the grand child})
			{
			   $\newContents$ := $\ang{ 0_n, 0_i, 0_d, 0_p, address }$ \;
				 %% \mline{$result$ := \CAS( \parbox[t]{1.25in}{$\parent \rarrow \child[pWhich]$, \\ $\ang{ 0_n, 1_i, 0_d, 0_p, \node }$, \\ $\ang{ 0_n, 0_i, 0_d, 0_p, address }$ );}\;}
			}
			
			
			
	 }
	
	  \BlankLine
		\mline{$result$ := \CAS( \parbox[t]{1.75in}{$\parent \rarrow \child[pWhich]$, \\ $\oldContents$, $\newContents$ );}\;}
		\Return $result$\;
		



}

\caption{Cleaning Up the Tree}
\label{algo:icdcn-cleanup}
\end{algorithm}





\begin{algorithm}[htp]
%%
\DefineKeyWords
\DontPrintSemicolon
\Boolean \MarkChildEdge( $\state$, $which$ )\;
\PrintSemicolon
\Begin
{

   \uIf{$\state \rarrow mode$ = \INJECTION}
	 {
	    $edge$ := $\state \rarrow \targetEdge$\; 
	    $\flag$ := \DELETEFLAG\;
	 }
	 \Else
	 {
	    $edge$ := $( \state \rarrow \successorSeekRecord ) \rarrow \terminalEdge$\; 
	    $\flag$ := \PROMOTEFLAG\;
	 }
	 
	 
   $\node$ := $edge.\child$\;
	
   \BlankLine
  
	 \While{\True}
	 {
	    $\ang{\nFlag, \iFlag, \dFlag, \pFlag, address}$ := $\node \rarrow \child[which]$\;
			
			\uIf{$\iFlag$}
			{
			   $helpeeEdge$ := $\curly{ \node, address, which }$\;
				 %% \HelpTargetNode( $helpeeEdge$, $\state \rarrow depth + 1$ )\;
				 \HelpTargetNode( $helpeeEdge$ )\;
				 \Continue\;
			}
			\uElseIf{$\dFlag$}
			{
			   \uIf{$\flag$ = \PROMOTEFLAG}
				 {
				    %% \HelpTargetNode( $edge$, $\state \rarrow depth + 1$  )\;
						\HelpTargetNode( $edge$ )\;
						\Return \False\;
				 } 
				 \lElse
				 {
				    \Return \True
				 }
			}
			\ElseIf{$\pFlag$}
			{
			   \uIf{$\flag$ = \DELETEFLAG}
				 {
				    %% \HelpSuccessorNode( $edge$, $\state \rarrow depth + 1$  )\;
						\HelpSuccessorNode( $edge$ )\;
						\Return \False\;
				 } 
				 \lElse
				 {
				    \Return \True
				 }
			}
			
			$\oldContents$ := $\ang{ \nFlag, 0_i, 0_d, 0_p, address }$\;
			$\newContents$ := $\oldContents \: | \: \flag$\;
			\mline{$result$ := \CAS( \parbox[t]{1.5in}{$\node \rarrow \child[which]$, $\oldContents$, \\ $\newContents$ );}\;}
			
			\lIf{$result$}{ \Break }
			
			
	 }

   \Return \True\;
}
\caption{Mark Child Edge}
\label{algo:icdcn-markChildEdge}


%\remove{

\end{algorithm}

%% FINDSMALLEST

\begin{algorithm}[htp]
%%
\DefineKeyWords
%}

\BlankLine

\DontPrintSemicolon
\Boolean \FindSmallest( $\state$ )\;
\PrintSemicolon
\Begin
{
   \tcp{find the node with the smallest key in the subtree rooted at the right child of the target node}
	 $\node$ := $\state \rarrow \targetEdge.\child$\;
	 $seekRecord$ := $\state \rarrow seekRecord$\;
	 $\ang{ \nFlag, \ast, \ast, \ast, \rightChild }$ := $\node \rarrow \child[\RIGHT]$\;
	 \If(\tcp*[h]{the right subtree is empty}){$\nFlag$}
	 {
	    %% \tcp{the right subtree is empty}
			\Return \False\;
	 }
	
	 \BlankLine	
		
	 \tcp{initialize the variables used in the traversal}
	 
	
	 %% $\ang{ \ast, \ast, \ast, \ast, \rightChild }$ := $\node \rarrow \child[RIGHT]$\;
	 $\terminalEdge$ := $\ang{ \node, \rightChild, \RIGHT }$\;
	 $\penultimateEdge$ := $\ang{ \node, \rightChild, \RIGHT }$\;
		 
	 %% \BlankLine
	 	
	 \While{\True}
	 {
	    $\curr$ := $\terminalEdge.\child$\;
      $\ang{ \nFlag, \ast, \ast, \ast, \leftChild }$ := $\curr \rarrow \child[\LEFT]$\;			
			\If(\tcp*[h]{reached the node with the smallest key}){$\nFlag$}	
			{
			   $\injectionEdge$ := $\ang{\curr, \leftChild, \LEFT}$\;
			   \Break\;
			}				
			\BlankLine			
			\tcp{traverse the next edge}			
			$\penultimateEdge$ := $\terminalEdge$\;
	    $\terminalEdge$ := $\ang{ \curr, \leftChild, \LEFT }$\;			
	 }	
	 \BlankLine
	 \tcp{initialize seek record and return}
	 $seekRecord \rarrow \terminalEdge$ := $\terminalEdge$\;
	 $seekRecord \rarrow \penultimateEdge$ := $\penultimateEdge$\;
   $seekRecord \rarrow \injectionEdge$ := $\injectionEdge$\;
	 \Return \True\;	
}
\caption{Find Smallest}
\label{algo:icdcn-findSmallest}
\end{algorithm}


\begin{algorithm}[htp]
%%
\DefineKeyWords


\DontPrintSemicolon
\InitializeTypeAndUpdateMode( $\state$ )\;
\PrintSemicolon
\Begin
{

   \tcp{retrieve the target node's address from the state record}
   $\node$ := $\state \rarrow \targetEdge.\child$\;
	 
	
	 \BlankLine
	 %% $\canReplace$ := $\node \rarrow \canReplace$\;
	 $\ang{ \lNFlag, \ast, \ast, \ast, \ast }$ := $\node \rarrow \child[\LEFT]$\;
	 $\ang{ \rNFlag, \ast, \ast, \ast, \ast }$ := $\node \rarrow \child[\RIGHT]$\;
	
	 \uIf{$\lNFlag$ \LOr $\rNFlag$}
	 {
	    \tcp{one of the child pointers is null}
	    $\ang{\mFlag, \ast }$ := $\node \rarrow \markAndKey$\;
	    \lIf{$\mFlag$}
	    {
	      $\state \rarrow type$ := \COMPLEX
	      %% $\node \rarrow \canReplace$ := \True\;
	    }
	    \lElse
	    {
	      $\state \rarrow type$ := \SIMPLE
	     }
	 }
	 \Else(\tcp*[h]{both child pointers are non-null})
	 {
	    %% \tcp{both child pointers are non-null}
	    $\state \rarrow type$ := \COMPLEX\;
	 }
	
	 \UpdateMode( $\state$ )\;
	
	 %% \Return\;

}

\remove{

\end{algorithm}


%% UPDATEMODE

\begin{algorithm}[htp]
%%
\DefineKeyWords

}

\BlankLine

\DontPrintSemicolon
\UpdateMode( $\state$ )\;
\PrintSemicolon
\Begin
{
	
	 \tcp{update the operation mode}

	 \BlankLine
	 \uIf(\tcp*[h]{simple delete}){$\state \rarrow type$ = \SIMPLE}
	 {
	    %% \tcp{simple delete}	
			$\state \rarrow mode$ := \CLEANUP\;
	 }
	 \Else(\tcp*[h]{complex delete})
	 {
	  	%% \tcp{complex delete}	

      $\node$ := $\state \rarrow \targetEdge.\child$\;
			\uIf{$\node \rarrow \canReplace$}
			{
			   $\state \rarrow mode$ := \CLEANUP\;
			}
			\lElse{$\state \rarrow mode$ := \DISCOVERY}
	 }
	
	 %% \Return\;
}

\caption{Helper Routines}
\label{algo:icdcn-helper|2}
\end{algorithm}

%% HELP

\begin{algorithm}[htp]
%%
\DefineKeyWords




\DontPrintSemicolon
%% \HelpTargetNode( $helpeeEdge$, $depth$ )\;
\HelpTargetNode( $helpeeEdge$ )\;
\PrintSemicolon
\Begin
{
   %% \lIf{$depth$ = number of processes}{ \Return }
	 %% \BlankLine		
	 \tcp{intent flag must be set on the edge}
	 \tcp{obtain new state record and initialize it}
	 $\state \rarrow \targetEdge$ := $helpeeEdge$\;
	 %% $\state \rarrow depth$ := $depth$\;
	 $\state \rarrow mode$ := \INJECTION\;
	 \BlankLine	
	 \tcp{mark the left and right edges if unmarked}
	 $result$ := \MarkChildEdge( $\state$, \LEFT{} )\;
	 \lIf{\LNot($result$)}{ 
	    %% \tcp{promote flag must have been set on the left edge}
			%% \HelpSuccessorNode( $helpeeEdge$, $depth + 1$ )\;
	    \Return
	 }
	 \MarkChildEdge( $\state$, \RIGHT{} )\;
	 
	 \InitializeTypeAndUpdateMode( $\state$ )\;
	
			
	 
	 \BlankLine
	
	 \tcp{perform the remaining steps of a delete operation}
   \If{$\state \rarrow mode$ = \DISCOVERY}
	 {
			%% \tcp{complex delete operation; locate the successor node and mark its child edges with promote flag}		
	    \FindAndMarkSuccessor( $\state$ )\;
	 						
	 }
			
	 \BlankLine
			
	 \If{$\state \rarrow mode$ = \DISCOVERY}
	 {
						
			%% \tcp{complex delete operation; promote the successor node's key and remove the successor node}
	    \RemoveSuccessor( $\state$ )\;
		   						
	 }
				
	 \BlankLine	
				
	 \lIf{$\state \rarrow mode$ = \CLEANUP}
	 {
	    %% \tcp{either remove the target node (simple delete) or replace it with a new node with unmarked edges (complex delete)}
	    \Cleanup( $\state$ )
	 }
	
	 %% \Return\;
}

\remove{

\end{algorithm}	
	


\begin{algorithm}[htp]
%%
\DefineKeyWords

}

\BlankLine

\DontPrintSemicolon
%% \HelpSuccessorNode( $helpeeEdge$, $depth$ )\;
\HelpSuccessorNode( $helpeeEdge$ )\;
\PrintSemicolon
\Begin
{
   %% \lIf{$depth$ = number of processes}{ \Return }
	 %% \BlankLine
   \tcp{retrieve the address of the successor node}
   $\parent$ := $helpeeEdge.\parent$\;
	 $\node$ := $helpeeEdge.\child$\;
	 
	 \tcp{promote flat must be set on the successor node's left edge}
	 \tcp{retrieve the address of the target node}
	 $\ang{ \ast, \ast, \ast, \ast, \leftChild }$ := $\node \rarrow \child[\LEFT]$\;
	 \BlankLine	
	 \tcp{obtain new state record and initialize it}
	 $\state \rarrow \targetEdge$ := $\curly{ \Null, \leftChild, \_ }$\;
	 %% $\state \rarrow depth$ := $depth$\;
	 $\state \rarrow mode$ := \DISCOVERY\;
	 $seekRecord$ := $\state \rarrow \successorSeekRecord$\;
	 \tcp{initialize the seek record in the state record}
	 $seekRecord \rarrow \terminalEdge$ := $helpeeEdge$\;
	 $seekRecord \rarrow \penultimateEdge$ := $\curly{ \Null, \parent, \_ }$\;
   \tcp{promote the successor node's key and remove the successor node}
	 \RemoveSuccessor( $\state$ )\;
	 \tcp{no need to perform the cleanup}
	
	
	 %% \Return\;

}


\caption{Helping Conflicting Delete Operations}
\label{algo:icdcn-helping}
\end{algorithm}
\end{limitscope}

A pseudo-code of our algorithm is given in \algosref{castle-data|structures}{helper}.
Different data structures used in our algorithm are shown in \algoref{castle-data|structures}. Besides tree node, we use three additional records:
\begin{enumerate*}[label=(\alph*)]
\item  \emph{seek record:} to store the outcome of a tree traversal both when looking for the target key and the successor key, 
\item \emph{anchor record:} to store information about the \anchornode{} during the seek phase, and
\item  \emph{lock record:} to store information about a tree edge that needs to be locked. 
\end{enumerate*}

The pseudo-code for the seek function is shown in \algoref{castle-seek}. The 
pseudo-codes for search, insert and delete operations are shown  in 
\algoref{castle-search}, \algoref{castle-insert} and \algoref{castle-delete}, respectively. 
\Algoref{lock:unlock} contains the pseudo-code for locking and unlocking a 
set of tree edges, as specified in an array. Finally, \algoref{helper} contains the pseudo-codes for 
three helper functions used by a delete operation, namely:
\begin{enumerate*}[label=(\alph*)]
\item \ClearFlags{}: to clear lock and mark flags from a child field, 
\item \FindSmallest{}: to locate the smallest key in a subtree, and
\item \RemoveChild{}: to remove a given child of a node.
\end{enumerate*}

In the pseudo-code, to improve clarity, we sometimes use subscripts $l$, $m$ and $n$ to denote lock, mark and null flags, respectively.  
\section{Correctness Proof}
It can be shown that our algorithm satisfies linearizability and lock-freedom properties~\cite{HerSha:2012:Book}. Broadly speaking, linearizability requires that an operation should appear to take effect instantaneously at some point during its execution.  Lock-freedom requires that some process should be able to complete its operation in a finite number of its own steps.
It is convenient to treat insert and delete operations that do not change the tree as search operations. 
%%
We call a tree node \emph{active} if it is reachable from the root of the tree. We call a tree node  \emph{passive} if it was active earlier but is not active any more.
It can be verified that, if an insert operation completes successfully, then  its \targetnode{} was active when it performed the successful \CAS{} instruction on the node's child edge.
Likewise, if a delete operation completes successfully, then its \targetnode{} was active when it marked the node's left edge for deletion. Further, for a complex delete, 
the successor node was active when it marked the node's left edge for promotion.



%% We are now ready to prove the correctness of our algorithm.

\subsubsection{All Executions are Linearizable}

We show that an arbitrary execution of our algorithm is linearizable by specifying the \emph{linearization point} of each operation. Note that the linearization point of an operation is the point during its execution at which the operation appeared to have taken effect. Our algorithm supports three types of operations: search, insert and delete. 
%%   
We now specify the linearization point of each operation.
%%
\begin{enumerate}[leftmargin=*, noitemsep]

\item \emph{Insert operation:} The operation is linearized at the point at which it performed the successful \CAS{} instruction that resulted in its target key becoming part of the tree.
		
					
\item \emph{Delete operation:} There are two cases depending on whether the delete operation is simple or complex. If the operation is simple delete, then the operation is linearized at the point at which a successful \CAS{} instruction was performed at the parent of the \targetnode{} that resulted in the \targetnode{} becoming passive. Otherwise, it is linearized at the point at which the original key of the \targetnode{} was replaced with its successor key.
   
 
   
\item \emph{Search operation:} There are two cases depending on whether the \targetnode{} was active when the operation read the key stored in the node. 
If the \targetnode{} was not active, then the operation is linearized at the point at which the \targetnode{} became passive. Otherwise, it is linearized at the 
point at which the read action was performed.
                       

\end{enumerate}


It can be easily verified that, for any execution of the algorithm, the sequence of operations
obtained by ordering operations based on their linearization points is legal, \emph{i.e.}, all operations in the sequence satisfy their 
specification. 
%%
This establishes that \emph{our algorithm generates only linearizable executions}.

\begin{comment}
 
Thus we have:

\begin{theorem}
Every execution of our algorithm is linearizable.
\end{theorem}
	
\end{comment}
							 
\subsubsection{All Executions are Lock-Free}
	
	
We say that the system is in a \emph{quiescent state} if no modify operation 
completes hereafter. We say that the system is in a \emph{potent state} if it 
has one or more pending modify operations. Note that a quiescence is a \emph{
stable} property; once the system is in a quiescent state, it stays in a 
quiescent state. We show that our algorithm is lock-free by proving that 
a potent state is necessarily non-quiescent provided assuming that some 
process with a pending modify operation continues to take steps. 

Assume, by the way of contradiction, that there is an execution of the system 
in which the system eventually reaches a state that is potent as well as 
quiescent. Note that, once the system has reached a quiescent state, it will 
eventually reach a state after which the tree will not undergo any structural 
changes. This is because a modify operation makes at most two structural changes to the 
tree. So, if the tree is undergoing continuous structural changes, then it 
clearly implies that modify operations are continuously completing their 
responses, which contradicts the assumption that the system is in a quiescent 
state. Further, on reaching such a state, the system will reach a state after 
which no new edges in the tree are marked. Again, this is because a modify 
operation marks at most four edges and the set of edges in the tree does not 
change any more. We call such a system state after  which neither
the set of edges nor the set of \emph{marked} edges in the tree change any 
more as a \emph{strongly quiescent state}. Note that, like quiescence, strong 
quiescence is also a stable property. 

From the above discussion, it follows that the system in a quiescent state will eventually reach a state that is strongly quiescent. Consider the search tree in such a strongly quiescent state. It can be verified that no more modify operations can now be injected into the tree, and, moreover,  
all modify operations already injected into the tree are delete operations currently ``stuck'' in either \discovery{} or \cleanup{} mode. 
%%
Now, consider a process, say $p$, that continues to take steps to execute either its own operation or another operation blocking its progress (directly or indirectly) as part of helping. 
%%
Consider the recursive chain of the \emph{helpee} operations that $p$ proceeds to help in order to complete its own operation. Let $\alpha_i$ denote the $i^{th}$ helpee operation in the chain.
It can be shown that:

\begin{lemma}
Let $\mathcal{C}_D$ denote the set of all complex delete operations already injected into the tree that are ''stuck'' in the \discovery{} mode. 
Then,
%%
\begin{enumerate}[leftmargin=*, noitemsep]
%%
\item $\alpha_1 \in \mathcal{C}_D$, and
%%
\item Suppose $p$ is currently helping $\alpha_i$  for some $i \geq 1$ and assume that $\alpha_i \in \mathcal{C}_D$. Let $\alpha_{i+1}$ denote the next operation that $p$ selects to help. Then, 
%%
\begin{enumerate*}[label=(\alph*)]
\item $\alpha_{i+1}$ exists, 
\item $\alpha_{i+1} \in \mathcal{C}_D$, and
\item the target node of $\alpha_{i+1}$ is at strictly larger depth than the target node of $\alpha_i$.
\end{enumerate*}
%%
\end{enumerate}
%%
\end{lemma}


Using the above lemma, we can easily construct a chain of distinct helpee operations whose length exceeds the number of processes---a contradiction. 
%%
This establishes that \emph{our algorithm only generates lock-free executions}.
%%

\begin{comment}
Thus, we have:


\begin{theorem}
Every execution of our algorithm is lock-free.
\end{theorem}

\end{comment}

\end{limitscope}
\section{Lock free tree}
\label{sec:experiments:icdcn}
In this section we evaluate \ICDCN{} against three other implementations of a concurrent BST, namely those based on:
%%
\begin{enumerate}[label=(\roman*)]
\item the lock-free internal BST by Howley and Jones~\cite{HowJon:2012:SPAA}, denoted by \HJBST{},
\item the lock-free external BST by Natarajan and Mittal~\cite{NatMit:2014:PPoPP}, denoted by \NMBST{}, and 
\item the RCU-based internal BST by Arbel and Attiya~\cite{ArbAtt:2014:PODC}, denoted by \CITRUS{}.
\end{enumerate}

% Style to select only points from #1 to #2 (inclusive)
\pgfplotsset
{
	select coords between index/.style 2 args=
	{
    x filter/.code=
		{
        \ifnum\coordindex<#1\def\pgfmathresult{}\fi
        \ifnum\coordindex>#2\def\pgfmathresult{}\fi
    }
	}
}
\begin{figure}[htp]
\centering
\begin{tikzpicture}[scale=0.9, transform shape]
	\begin{groupplot}[group style={group size= 3 by 3},height=3cm,width=4cm,xmode=log,log basis x={2},max space between ticks=20,minor tick num=1,tick label style={font=\tiny},xlabel style={font=\tiny},ylabel style={font=\tiny}, title style={font=\tiny}]
		\nextgroupplot[title=Small,ylabel={Read-Dominated},xtick=data]
				%\addplot[brown, semithick,mark=square,mark size=1] 	[select coords between index={0}{3}] table[x=keyspace,y=Mops-citrus,col sep=space]{Data/snb32/keySweep.csv};  \label{plots:CITRUS:ika}
				\addplot[red,semithick,mark=triangle,mark size=1] 	[select coords between index={0}{3}] table[x=keyspace,y=Mops-howley,col sep=space]{Data/snb32/keySweep.csv};  \label{plots:HJ-BST:ika}
				\addplot[blue,semithick,mark=asterisk,mark size=1] 	[select coords between index={0}{3}] table[x=keyspace,y=Mops-ppop,col sep=space]{Data/snb32/keySweep.csv};  	\label{plots:NM-BST:ika}
				\addplot[green, semithick,mark=o,mark size=1] 			[select coords between index={0}{3}] table[x=keyspace,y=Mops-icdcn,col sep=space]{Data/snb32/keySweep.csv};  \label{plots:icdcn:ika}
				\coordinate (top) at (rel axis cs:0,1);% coordinate at top of the first plot
		\nextgroupplot[title=Medium,xtick=data]
				%\addplot[brown, semithick,mark=square,mark size=1] 	[select coords between index={4}{7}] table[x=keyspace,y=Mops-citrus,col sep=space]{Data/snb32/keySweep.csv}; 
				\addplot[red,semithick,mark=triangle,mark size=1] 	[select coords between index={4}{7}] table[x=keyspace,y=Mops-howley,col sep=space]{Data/snb32/keySweep.csv}; 
				\addplot[blue,semithick,mark=asterisk,mark size=1] 	[select coords between index={4}{7}] table[x=keyspace,y=Mops-ppop,col sep=space]{Data/snb32/keySweep.csv};   
				\addplot[green, semithick,mark=o,mark size=1] 			[select coords between index={4}{7}] table[x=keyspace,y=Mops-icdcn,col sep=space]{Data/snb32/keySweep.csv};
		\nextgroupplot[title=Large,xtick=data]
				%\addplot[brown, semithick,mark=square,mark size=1] 	[select coords between index={8}{11}] table[x=keyspace,y=Mops-citrus,col sep=space]{Data/snb32/keySweep.csv};
				\addplot[red,semithick,mark=triangle,mark size=1] 	[select coords between index={8}{11}] table[x=keyspace,y=Mops-howley,col sep=space]{Data/snb32/keySweep.csv};
				\addplot[blue,semithick,mark=asterisk,mark size=1] 	[select coords between index={8}{11}] table[x=keyspace,y=Mops-ppop,col sep=space]{Data/snb32/keySweep.csv};  
				\addplot[green, semithick,mark=o,mark size=1] 			[select coords between index={8}{11}] table[x=keyspace,y=Mops-icdcn,col sep=space]{Data/snb32/keySweep.csv};
		\nextgroupplot[ylabel={Mixed},xtick=data]
				%\addplot[brown, semithick,mark=square,mark size=1] 	[select coords between index={12}{15}] table[x=keyspace,y=Mops-citrus,col sep=space]{Data/snb32/keySweep.csv};
				\addplot[red,semithick,mark=triangle,mark size=1] 	[select coords between index={12}{15}] table[x=keyspace,y=Mops-howley,col sep=space]{Data/snb32/keySweep.csv};
				\addplot[blue,semithick,mark=asterisk,mark size=1] 	[select coords between index={12}{15}] table[x=keyspace,y=Mops-ppop,col sep=space]{Data/snb32/keySweep.csv};  
				\addplot[green, semithick,mark=o,mark size=1] 			[select coords between index={12}{15}] table[x=keyspace,y=Mops-icdcn,col sep=space]{Data/snb32/keySweep.csv};
		\nextgroupplot[xtick=data]
				%\addplot[brown, semithick,mark=square,mark size=1] 	[select coords between index={16}{19}] table[x=keyspace,y=Mops-citrus,col sep=space]{Data/snb32/keySweep.csv};
				\addplot[red,semithick,mark=triangle,mark size=1] 	[select coords between index={16}{19}] table[x=keyspace,y=Mops-howley,col sep=space]{Data/snb32/keySweep.csv};
				\addplot[blue,semithick,mark=asterisk,mark size=1] 	[select coords between index={16}{19}] table[x=keyspace,y=Mops-ppop,col sep=space]{Data/snb32/keySweep.csv};  
				\addplot[green, semithick,mark=o,mark size=1] 			[select coords between index={16}{19}] table[x=keyspace,y=Mops-icdcn,col sep=space]{Data/snb32/keySweep.csv};
		\nextgroupplot[xtick=data]
				%\addplot[brown, semithick,mark=square,mark size=1] 	[select coords between index={20}{23}] table[x=keyspace,y=Mops-citrus,col sep=space]{Data/snb32/keySweep.csv};
				\addplot[red,semithick,mark=triangle,mark size=1] 	[select coords between index={20}{23}] table[x=keyspace,y=Mops-howley,col sep=space]{Data/snb32/keySweep.csv};
				\addplot[blue,semithick,mark=asterisk,mark size=1] 	[select coords between index={20}{23}] table[x=keyspace,y=Mops-ppop,col sep=space]{Data/snb32/keySweep.csv};  
				\addplot[green, semithick,mark=o,mark size=1] 			[select coords between index={20}{23}] table[x=keyspace,y=Mops-icdcn,col sep=space]{Data/snb32/keySweep.csv};
		\nextgroupplot[xlabel={Key Space Size},ylabel={Write-Dominated},xtick=data]
				%\addplot[brown, semithick,mark=square,mark size=1] 	[select coords between index={24}{27}] table[x=keyspace,y=Mops-citrus,col sep=space]{Data/snb32/keySweep.csv}; 
				\addplot[red,semithick,mark=triangle,mark size=1] 	[select coords between index={24}{27}] table[x=keyspace,y=Mops-howley,col sep=space]{Data/snb32/keySweep.csv};
				\addplot[blue,semithick,mark=asterisk,mark size=1] 	[select coords between index={24}{27}] table[x=keyspace,y=Mops-ppop,col sep=space]{Data/snb32/keySweep.csv};  
				\addplot[green, semithick,mark=o,mark size=1] 			[select coords between index={24}{27}] table[x=keyspace,y=Mops-icdcn,col sep=space]{Data/snb32/keySweep.csv};	
		\nextgroupplot[xlabel={Key Space Size},xtick=data]
				%\addplot[brown, semithick,mark=square,mark size=1] 	[select coords between index={28}{31}] table[x=keyspace,y=Mops-citrus,col sep=space]{Data/snb32/keySweep.csv};
				\addplot[red,semithick,mark=triangle,mark size=1] 	[select coords between index={28}{31}] table[x=keyspace,y=Mops-howley,col sep=space]{Data/snb32/keySweep.csv};
				\addplot[blue,semithick,mark=asterisk,mark size=1] 	[select coords between index={28}{31}] table[x=keyspace,y=Mops-ppop,col sep=space]{Data/snb32/keySweep.csv};  
				\addplot[green, semithick,mark=o,mark size=1] 			[select coords between index={28}{31}] table[x=keyspace,y=Mops-icdcn,col sep=space]{Data/snb32/keySweep.csv};
		\nextgroupplot[xlabel={Key Space Size},xtick=data]
				%\addplot[brown, semithick,mark=square,mark size=1] 	[select coords between index={32}{35}] table[x=keyspace,y=Mops-citrus,col sep=space]{Data/snb32/keySweep.csv}; 
				\addplot[red,semithick,mark=triangle,mark size=1] 	[select coords between index={32}{35}] table[x=keyspace,y=Mops-howley,col sep=space]{Data/snb32/keySweep.csv};
				\addplot[blue,semithick,mark=asterisk,mark size=1] 	[select coords between index={32}{35}] table[x=keyspace,y=Mops-ppop,col sep=space]{Data/snb32/keySweep.csv};  
				\addplot[green, semithick,mark=o,mark size=1] 			[select coords between index={32}{35}] table[x=keyspace,y=Mops-icdcn,col sep=space]{Data/snb32/keySweep.csv};						
				\coordinate (bot) at (rel axis cs:1,0);% coordinate at bottom of the last plot
	\end{groupplot}
	\path (top-|current bounding box.west)-- node[anchor=south,rotate=90] {\tiny System throughput (million operations/second)} (bot-|current bounding box.west);
	\path (top|-current bounding box.north)-- coordinate(legendpos) (bot|-current bounding box.north);
	\matrix[matrix of nodes, anchor=south, draw, inner sep=0.2em, draw] at ([yshift=1ex]legendpos)
  {
    \ref{plots:HJ-BST:ika}& \tiny \HJBST{} & [5pt]
    \ref{plots:NM-BST:ika}& \tiny \NMBST{} & [5pt]
    %\ref{plots:CITRUS:ika}& \tiny \CITRUS{} & [5pt]
		\ref{plots:icdcn:ika}& \tiny \ICDCN{} \\
	};
\end{tikzpicture}
%\caption[\ICDCN{} - Comparison of throughput of different algorithms - key sweep]{Comparison of system throughput of different algorithms at 32 threads. Each row represents a workload type. Each column represents a range of key space range. Higher the ratio, better the performance of the algorithm.}
\label{fig:icdcn-keySweep-absolute}
\end{figure}
% Style to select only points from #1 to #2 (inclusive)
\pgfplotsset
{
	select coords between index/.style 2 args=
	{
    x filter/.code=
		{
        \ifnum\coordindex<#1\def\pgfmathresult{}\fi
        \ifnum\coordindex>#2\def\pgfmathresult{}\fi
    }
	}
}
\begin{figure}
\centering
\nextwithlateexternal% < added
\begin{tikzpicture}
	\begin{groupplot}[group style={group size= 3 by 3},height=5.5cm,width=5.5cm,max space between ticks=20,minor tick num=1,tick label style={font=\footnotesize}]
		\nextgroupplot[title=20K keys,ylabel={Read-Dominated},xtick=data]
				%\addplot[brown, semithick,mark=square] 	[select coords between index={6}{11}] table[x=threads,y=Mops-citrus,col sep=space]{Data/snb32/threadSweep.csv};  \label{plots:CITRUS:it}
				\addplot[red,semithick,mark=triangle] 	[select coords between index={6}{11}] table[x=threads,y=Mops-howley,col sep=space]{Data/snb32/threadSweep.csv};  \label{plots:HJ-BST:it}
				\addplot[blue,semithick,mark=asterisk] 	[select coords between index={6}{11}] table[x=threads,y=Mops-ppop,col sep=space]{Data/snb32/threadSweep.csv};  \label{plots:NM-BST:it}
				\addplot[green, semithick,mark=o] 			[select coords between index={6}{11}] table[x=threads,y=Mops-icdcn,col sep=space]{Data/snb32/threadSweep.csv};  \label{plots:icdcn:it}
				\coordinate (top) at (rel axis cs:0,1);% coordinate at top of the first plot
		\nextgroupplot[title=200K keys,xtick=data]
				%\addplot[brown, semithick,mark=square] 	[select coords between index={18}{23}] table[x=threads,y=Mops-citrus,col sep=space]{Data/snb32/threadSweep.csv};  
				\addplot[red,semithick,mark=triangle] 	[select coords between index={18}{23}] table[x=threads,y=Mops-howley,col sep=space]{Data/snb32/threadSweep.csv};  
				\addplot[blue,semithick,mark=asterisk] 	[select coords between index={18}{23}] table[x=threads,y=Mops-ppop,col sep=space]{Data/snb32/threadSweep.csv};  
				\addplot[green, semithick,mark=o] 			[select coords between index={18}{23}] table[x=threads,y=Mops-icdcn,col sep=space]{Data/snb32/threadSweep.csv}; 
		\nextgroupplot[title=2M keys,xtick=data]
				%\addplot[brown, semithick,mark=square] 	[select coords between index={30}{35}] table[x=threads,y=Mops-citrus,col sep=space]{Data/snb32/threadSweep.csv};
				\addplot[red,semithick,mark=triangle] 	[select coords between index={30}{35}] table[x=threads,y=Mops-howley,col sep=space]{Data/snb32/threadSweep.csv};
				\addplot[blue,semithick,mark=asterisk] 	[select coords between index={30}{35}] table[x=threads,y=Mops-ppop,col sep=space]{Data/snb32/threadSweep.csv};
				\addplot[green, semithick,mark=o] 			[select coords between index={30}{35}] table[x=threads,y=Mops-icdcn,col sep=space]{Data/snb32/threadSweep.csv};
		\nextgroupplot[ylabel={Mixed},xtick=data]
				%\addplot[brown, semithick,mark=square] 	[select coords between index={42}{47}] table[x=threads,y=Mops-citrus,col sep=space]{Data/snb32/threadSweep.csv}; 
				\addplot[red,semithick,mark=triangle] 	[select coords between index={42}{47}] table[x=threads,y=Mops-howley,col sep=space]{Data/snb32/threadSweep.csv};
				\addplot[blue,semithick,mark=asterisk] 	[select coords between index={42}{47}] table[x=threads,y=Mops-ppop,col sep=space]{Data/snb32/threadSweep.csv};
				\addplot[green, semithick,mark=o] 			[select coords between index={42}{47}] table[x=threads,y=Mops-icdcn,col sep=space]{Data/snb32/threadSweep.csv};
		\nextgroupplot[xtick=data]
				%\addplot[brown, semithick,mark=square] 	[select coords between index={54}{59}] table[x=threads,y=Mops-citrus,col sep=space]{Data/snb32/threadSweep.csv}; 
				\addplot[red,semithick,mark=triangle] 	[select coords between index={54}{59}] table[x=threads,y=Mops-howley,col sep=space]{Data/snb32/threadSweep.csv};
				\addplot[blue,semithick,mark=asterisk] 	[select coords between index={54}{59}] table[x=threads,y=Mops-ppop,col sep=space]{Data/snb32/threadSweep.csv};
				\addplot[green, semithick,mark=o] 			[select coords between index={54}{59}] table[x=threads,y=Mops-icdcn,col sep=space]{Data/snb32/threadSweep.csv};
		\nextgroupplot[xtick=data]
				%\addplot[brown, semithick,mark=square] 	[select coords between index={66}{71}] table[x=threads,y=Mops-citrus,col sep=space]{Data/snb32/threadSweep.csv}; 
				\addplot[red,semithick,mark=triangle] 	[select coords between index={66}{71}] table[x=threads,y=Mops-howley,col sep=space]{Data/snb32/threadSweep.csv};
				\addplot[blue,semithick,mark=asterisk] 	[select coords between index={66}{71}] table[x=threads,y=Mops-ppop,col sep=space]{Data/snb32/threadSweep.csv};
				\addplot[green, semithick,mark=o] 			[select coords between index={66}{71}] table[x=threads,y=Mops-icdcn,col sep=space]{Data/snb32/threadSweep.csv};
		\nextgroupplot[xlabel={Number of Threads},ylabel={Write-Dominated},xtick=data]
				%\addplot[brown, semithick,mark=square] 	[select coords between index={78}{83}] table[x=threads,y=Mops-citrus,col sep=space]{Data/snb32/threadSweep.csv}; 
				\addplot[red,semithick,mark=triangle] 	[select coords between index={78}{83}] table[x=threads,y=Mops-howley,col sep=space]{Data/snb32/threadSweep.csv};
				\addplot[blue,semithick,mark=asterisk] 	[select coords between index={78}{83}] table[x=threads,y=Mops-ppop,col sep=space]{Data/snb32/threadSweep.csv};
				\addplot[green, semithick,mark=o] 			[select coords between index={78}{83}] table[x=threads,y=Mops-icdcn,col sep=space]{Data/snb32/threadSweep.csv};	
		\nextgroupplot[xlabel={Number of Threads},xtick=data]
				%\addplot[brown, semithick,mark=square] 	[select coords between index={90}{95}] table[x=threads,y=Mops-citrus,col sep=space]{Data/snb32/threadSweep.csv}; 
				\addplot[red,semithick,mark=triangle] 	[select coords between index={90}{95}] table[x=threads,y=Mops-howley,col sep=space]{Data/snb32/threadSweep.csv};
				\addplot[blue,semithick,mark=asterisk] 	[select coords between index={90}{95}] table[x=threads,y=Mops-ppop,col sep=space]{Data/snb32/threadSweep.csv};
				\addplot[green, semithick,mark=o] 			[select coords between index={90}{95}] table[x=threads,y=Mops-icdcn,col sep=space]{Data/snb32/threadSweep.csv};	
		\nextgroupplot[xlabel={Number of Threads},xtick=data]
				%\addplot[brown, semithick,mark=square] 	[select coords between index={102}{107}] table[x=threads,y=Mops-citrus,col sep=space]{Data/snb32/threadSweep.csv}; 
				\addplot[red,semithick,mark=triangle] 	[select coords between index={102}{107}] table[x=threads,y=Mops-howley,col sep=space]{Data/snb32/threadSweep.csv};
				\addplot[blue,semithick,mark=asterisk] 	[select coords between index={102}{107}] table[x=threads,y=Mops-ppop,col sep=space]{Data/snb32/threadSweep.csv};
				\addplot[green, semithick,mark=o] 			[select coords between index={102}{107}] table[x=threads,y=Mops-icdcn,col sep=space]{Data/snb32/threadSweep.csv};						
				\coordinate (bot) at (rel axis cs:1,0);% coordinate at bottom of the last plot
\end{groupplot}
	\path (top-|current bounding box.west)-- node[anchor=south,rotate=90] {\small System throughput (million operations/second)} (bot-|current bounding box.west);
	%\node[right,rotate=90] at (-1.4,-7.2){\small ~Throughput (million ops/sec)};
	\path (top|-current bounding box.north)-- coordinate(legendpos) (bot|-current bounding box.north);
	\matrix[matrix of nodes, anchor=south, draw, inner sep=0.2em, draw] at ([xshift=-5ex,yshift=1ex]legendpos)
  {
    \ref{plots:HJ-BST:it}& \HJBST{} & [5pt]
    \ref{plots:NM-BST:it}& \NMBST{} & [5pt]
    %\ref{plots:CITRUS:it}& \CITRUS{} & [5pt]
		\ref{plots:icdcn:it}& \ICDCN{} \\
	};
\end{tikzpicture}
\caption[\ICDCN{} - Comparison of throughput of different algorithms - thread sweep]{Comparison of system throughput of different algorithms. Each row represents a workload type and each column represents a key space size. Higher the throughput, better the performance of the algorithm.}
\label{fig:icdcn-threadSweep}
\end{figure}

The results of our experiments are shown in \figref{icdcn-keySweep-absolute} and \figref{icdcn-threadSweep}. In \figref{icdcn-keySweep-absolute}, each row represents a specific workload (read-dominated, mixed or write-dominated) and each column represents a specific key space size; \textit{small} (8Ki to 64Ki), \textit{medium} (128Ki to 1Mi) and \textit{large} (2Mi to 16Mi). \figref{icdcn-threadSweep} shows the scaling with respect to the number of threads for key space size of 2\textsuperscript{19} (512Ki). We do not show the numbers for \CITRUS{} in the graphs as it had the worst performance among all implementations (slower by a factor of four in some cases). This is not surprising as \CITRUS{} is optimized for read operations (\emph{e.g.}, 98\% reads \& 2\% updates)~\cite{ArbAtt:2014:PODC}.


As the graphs show, \ICDCN{} achieved nearly same or higher throughput than the other two implementations for medium and large key space sizes (except for medium key space size with write-dominated workload). Specifically, at 32 threads and for a read-dominated workload, \ICDCN{} had \icdcnMaximumgap{} and 15\% higher throughput than the next best performer for key space sizes of 512Ki and 1Mi, respectively. Also, at 32 threads and for a mixed workload, \ICDCN{} had 22\% and 15\% higher throughput than the next best performer for key space sizes of 1Mi and 2Mi, respectively. Overall, \ICDCN{} outperformed the next best implementation by as much as \icdcnMaximumgap{}; it outperformed \HJBST{} by as much as 38\% (mixed) and \NMBST{} by as much as 30\%(read-dominated). For large key space sizes, the overhead of traversing the tree appears to dominate the overhead of actually modifying the operation's window, and the gap between various implementations becomes smaller.

\begin{table}[htp]
\centering
\caption{Comparison of different concurrent algorithms in the absence of contention.}
\label{tab:castle-comparison}
\renewcommand{\arraystretch}{1.25}
\small
\begin{tabular}{|c|c|c|c|c|}
\hline
\multirow{2}{*}{\textbf{Algorithm}} & 
\multicolumn{2}{c|}{\textbf{\begin{tabular}[c]{@{}c@{}}Number of  Objects \\  Allocated \end{tabular}}} & 
\multicolumn{2}{c|}{\textbf{\begin{tabular}[c]{@{}c@{}}Number of Synchronization \\ Primitives Executed \end{tabular}}} \\ \cline{2-5} 
%%
 & \textbf{Insert} & \textbf{Delete} & \textbf{Insert} & \textbf{Delete} \\ \hline
%%
\multirow{2}{*}{\HJBST} & \multirow{2}{*}{2} & \multirow{2}{*}{1} & \multirow{2}{*}{3} & simple: 4 \\ 
 &  &  &  & complex: 9 \\ \hline
%%
\NMBST{} & 2 & 0 & 1 & 3 \\ \hline
%%
\multirow{2}{*}{\CASTLE} & \multirow{2}{*}{1} & \multirow{2}{*}{0} & \multirow{2}{*}{1} & simple: 3 \\ 
 &  &  &  & complex: 4 \\ \hline
\end{tabular}
\end{table}
There are several reasons why \ICDCN{} outperformed the other two implementations in many cases. First, as \tabref{icdcn-comparison} shows, our algorithm allocates fewer objects than the two other algorithms on average considering the fact that the fraction of insert operations will generally be larger than the fraction of delete operations in any realistic workload. Further, we observed in our experiments that the number of simple delete operations outnumbered the number of complex delete operations by two to one, and our algorithm does not allocate any object for a simple delete operation. Second, again as \tabref{icdcn-comparison} shows, our algorithm executes the same number of atomic instructions as in~\cite{NatMit:2014:PPoPP} for insert operations;  and, in all the cases,  executes same or fewer atomic instructions than in~\cite{HowJon:2012:SPAA}. This is important since an atomic instruction is more expensive to execute than a simple read or write instruction. Third, we observed in our experiments that \ICDCN{} had a smaller memory footprint than the other two implementations (by almost a factor of two) since it uses internal representation and allocates fewer objects. As a result, it was likely able to benefit from caching to a larger degree than \HJBST{} and \NMBST{}.
\section{Impact of local recovery}
\label{sec:experiments:localRecovery}
\begin{frame}[c]{Local recovery[PPoPP'16 Poster]}
Overview
\begin{itemize}
\item a general technique for local recovery for concurrent BSTs
\item reduces tree traversal cost during failures by restarting closer to an operation�s window
\end{itemize}
Motivation
\begin{itemize}
\item in most concurrent BSTs, execution phase of an operation have constant time complexity
\item seek phase is where an operation may end up spending most of its time (esp for large trees)
\item this technique reduces the seek time
\end{itemize}
\end{frame}

\begin{frame}[c]{Example}
\begin{figure}[htp]
\begin{tikzpicture}[scale=0.5, transform shape] 
	 \newcommand\NODEDX{1.25}
	 \newcommand\NODEDY{1.25}
	 \newcommand\SUBTREEDX{1.5}
	 \newcommand\SUBTREEDY{0.75}
	
   \node (r)	[treenode] 		                at (0, 0)       		                      	{$R$ \\ 100};
   \node (s)	[treenode, fill=black!20] 		at ([shift=({ -\NODEDX, -\NODEDY})]r)     	{$S$ \\ 10};
	 \node (t)	[treenode] 		                at ([shift=({  \NODEDX, -\NODEDY})]s)    		{$T$ \\ 90};
	 \node (u)	[treenode, fill=black!20] 	  at ([shift=({ -\NODEDX, -\NODEDY})]t)     	{$U$ \\ 20};
	 \node (v)	[treenode] 										at ([shift=({  \NODEDX, -\NODEDY})]u)     	{$V$ \\ 80};
	 \node (w)	[treenode] 										at ([shift=({ -\NODEDX, -\NODEDY})]v)     	{$W$ \\ 70};
	 \node (x)	[treenode, fill=black!20] 		at ([shift=({ -\NODEDX, -\NODEDY})]w)     	{$X$ \\ 30};
	 \node (y)	[treenode,fill=red]						at ([shift=({  \NODEDX, -\NODEDY})]x)     	{$Y$ \\ 60};
	 \node (z)	[treenode] 										at ([shift=({ -\NODEDX, -\NODEDY})]y)     	{$Z$ \\ 50};
	 \node (gl) [ground]                      at ([shift=({ -\NODEDX, -\NODEDY})]z)     	{ };
	 \node (gr) [ground]                      at ([shift=({  \NODEDX, -\NODEDY})]z)     	{ };
		
	 \node (sa) [subtree]                     at ([shift=({  \SUBTREEDX, -\SUBTREEDY})]r) {\Large $\alpha$};
	 \node (sb) [subtree]                     at ([shift=({ -\SUBTREEDX, -\SUBTREEDY})]s) {\Large $\beta$};
	 \node (sg) [subtree]                     at ([shift=({  \SUBTREEDX, -\SUBTREEDY})]t) {\Large $\gamma$};
	 \node (sd) [subtree]                     at ([shift=({ -\SUBTREEDX, -\SUBTREEDY})]u) {\Large $\delta$};
	 \node (ss) [subtree]                     at ([shift=({  \SUBTREEDX, -\SUBTREEDY})]v) {\Large $\sigma$};
	 \node (st) [subtree]                     at ([shift=({  \SUBTREEDX, -\SUBTREEDY})]w) {\Large $\tau$};
	 \node (sp) [subtree]                     at ([shift=({ -\SUBTREEDX, -\SUBTREEDY})]x) {\Large $\pi$};
	 \node (sl) [subtree]                     at ([shift=({  \SUBTREEDX, -\SUBTREEDY})]y) {\Large $\lambda$};	
	
	 \node (op) [label={right:{\large $op(50)^{z^{z^z}}$}}] at ([shift=({0.375, 0})]y) {};
	
	 \path[every node/.style={font=\sffamily\small}]
	    (0, 1)  edge[->,very thick]  node {} (r)
		  (r)     edge[->,very thick]  node {} (s)
			(s)     edge[->,very thick]  node {} (t)
			(t)     edge[->,very thick]  node {} (u)
			(u)     edge[->,very thick]  node {} (v)
			(v)     edge[->,very thick]  node {} (w)
			(w)     edge[->,very thick]  node {} (x)
			(x)     edge[->,very thick]  node {} (y)
			(y)     edge[->,very thick]  node {} (z)
			(z)     edge[->]  node {} (gl)
			(z)     edge[->]  node {} (gr)
			(r)     edge[->]  node {} (sa.north)
			(s)     edge[->]  node {} (sb.north)
			(t)     edge[->]  node {} (sg.north)
			(u)     edge[->]  node {} (sd.north)
			(v)     edge[->]  node {} (ss.north)
			(w)     edge[->]  node {} (st.north)
			(x)     edge[->]  node {} (sp.north)
			(y)     edge[->]  node {} (sl.north);		
\end{tikzpicture}
\caption{Operation $op(50)$ is suspended at node $Y$ during its traversal}
\end{figure}
\end{frame}

\begin{frame}[c]{Example}
\begin{figure}[htp]
\begin{tikzpicture}[scale=0.5, transform shape]   
	 \newcommand\NODEDX{1.25}
	 \newcommand\NODEDY{1.25}
	 \newcommand\SUBTREEDX{1.5}
	 \newcommand\SUBTREEDY{0.75}
	
   \node (r)	[treenode] 		                at (0, 0)       		                      	{$R$ \\ 100};
   \node (s)	[treenode, fill=black!20] 		at ([shift=({ -\NODEDX, -\NODEDY})]r)     	{$S$ \\ 10};
	 \node (t)	[treenode] 		                at ([shift=({  \NODEDX, -\NODEDY})]s)    		{$T$ \\ 90};
	 \node (u)	[treenode, fill=black!20] 	  at ([shift=({ -\NODEDX, -\NODEDY})]t)     	{$U$ \\ 20};
	 \node (v)	[treenode] 										at ([shift=({  \NODEDX, -\NODEDY})]u)     	{$V$ \\ 80};
	 \node (w)	[treenode] 										at ([shift=({ -\NODEDX, -\NODEDY})]v)     	{$W$ \\ 70};
	 \node (x)	[treenode, fill=black!20] 		at ([shift=({ -\NODEDX, -\NODEDY})]w)     	{$X$ \\ 30};
	 \node (y)	[treenode, fill=red]					at ([shift=({  \NODEDX, -\NODEDY})]x)     	{$Y$ \\ 60};
	 \node (z)	[treenode] 										at ([shift=({ -\NODEDX, -\NODEDY})]y)     	{$Z$ \\ 50};
	 \node (gl) [ground]                      at ([shift=({ -\NODEDX, -\NODEDY})]z)     	{ };
	 \node (gr) [ground]                      at ([shift=({  \NODEDX, -\NODEDY})]z)     	{ };

	 \node (sa) [subtree]                     at ([shift=({  \SUBTREEDX, -\SUBTREEDY})]r) {\Large $\alpha$};
	 \node (sb) [subtree]                     at ([shift=({ -\SUBTREEDX, -\SUBTREEDY})]s) {\Large $\beta$};
	 \node (sg) [subtree]                     at ([shift=({  \SUBTREEDX, -\SUBTREEDY})]t) {\Large $\gamma$};
	 \node (sd) [subtree]                     at ([shift=({ -\SUBTREEDX, -\SUBTREEDY})]u) {\Large $\delta$};
	 \node (ss) [subtree]                     at ([shift=({  \SUBTREEDX, -\SUBTREEDY})]v) {\Large $\sigma$};
	 \node (st) [subtree]                     at ([shift=({  \SUBTREEDX, -\SUBTREEDY})]w) {\Large $\tau$};
	 %% \node (sp) [subtree]                     at ([shift=({ -\SUBTREEDX, -\SUBTREEDY})]x) {\Large $\pi$};
	 \node (sp) [ground]                    	at ([shift=({ -\NODEDX, -\NODEDY})]x) { };
	 \node (sl) [subtree]                     at ([shift=({  \SUBTREEDX, -\SUBTREEDY})]y) {\Large $\lambda$};	

   \node (op) [label={right:{\large $op(50)^{z^{z^z}}$}}] at ([shift=({0.375, 0})]y) {};
	
	 \path[every node/.style={font=\sffamily\small}]
	    (0, 1)  edge[->,very thick]  node {} (r)
		  (r)     edge[->,very thick]  node {} (s)
			(s)     edge[->,very thick]  node {} (t)
			(t)     edge[->,very thick]  node {} (u)
			(u)     edge[->,very thick]  node {} (v)
			(v)     edge[->,very thick]  node {} (w)
			(w)     edge[->,very thick]  node {} (x)
			(x)     edge[->,very thick]  node {} (y)
			(y)     edge[->,very thick]  node {} (z)
			(z)     edge[->]  node {} (gl)
			(z)     edge[->]  node {} (gr)
			(r)     edge[->]  node {} (sa.north)
			(s)     edge[->]  node {} (sb.north)
			(t)     edge[->]  node {} (sg.north)
			(u)     edge[->]  node {} (sd.north)
			(v)     edge[->]  node {} (ss.north)
			(w)     edge[->]  node {} (st.north)
			(x)     edge[->]  node {} (sp)
			(y)     edge[->]  node {} (sl.north);	
\end{tikzpicture}
\caption{All keys in subtree $\pi$ are deleted one-by-one}
\end{figure}
\end{frame}

\begin{frame}[c]{Example}
\begin{figure}[htp]
\begin{tikzpicture}[scale=0.5, transform shape]
   
	 \newcommand\NODEDX{1.25}
	 \newcommand\NODEDY{1.25}
	 \newcommand\SUBTREEDX{1.5}
	 \newcommand\SUBTREEDY{0.75}
	
   \node (r)	[treenode] 		                at (0, 0)       		                      	{$R$ \\ 100};
   \node (s)	[treenode, fill=black!20] 		at ([shift=({ -\NODEDX, -\NODEDY})]r)     	{$S$ \\ 10};
	 \node (t)	[treenode] 		                at ([shift=({  \NODEDX, -\NODEDY})]s)    		{$T$ \\ 90};
	 \node (u)	[treenode, fill=black!20] 	  at ([shift=({ -\NODEDX, -\NODEDY})]t)     	{$U$ \\ 20};
	 \node (v)	[treenode] 										at ([shift=({  \NODEDX, -\NODEDY})]u)     	{$V$ \\ 80};
	 \node (w)	[treenode] 										at ([shift=({ -\NODEDX, -\NODEDY})]v)     	{$W$ \\ 70};
	 \node (x)	[treenode, fill=black!20, dotted] 		at ([shift=({ -\NODEDX, -\NODEDY})]w)     	{$X$ \\ 30};
	 \node (y)	[treenode] 										at ([shift=({  \NODEDX, -\NODEDY})]x)     	{$Y$ \\ 60};
	 \node (z)	[treenode] 										at ([shift=({ -\NODEDX, -\NODEDY})]y)     	{$Z$ \\ 50};
	 \node (gl) [ground]                      at ([shift=({ -\NODEDX, -\NODEDY})]z)     	{ };
	 \node (gr) [ground]                      at ([shift=({  \NODEDX, -\NODEDY})]z)     	{ };
		
	 \node (sa) [subtree]                     at ([shift=({  \SUBTREEDX, -\SUBTREEDY})]r) {\Large $\alpha$};
	 \node (sb) [subtree]                     at ([shift=({ -\SUBTREEDX, -\SUBTREEDY})]s) {\Large $\beta$};
	 \node (sg) [subtree]                     at ([shift=({  \SUBTREEDX, -\SUBTREEDY})]t) {\Large $\gamma$};
	 \node (sd) [subtree]                     at ([shift=({ -\SUBTREEDX, -\SUBTREEDY})]u) {\Large $\delta$};
	 \node (ss) [subtree]                     at ([shift=({  \SUBTREEDX, -\SUBTREEDY})]v) {\Large $\sigma$};
	 \node (st) [subtree]                     at ([shift=({  \SUBTREEDX, -\SUBTREEDY})]w) {\Large $\tau$};
	 %% \node (sp) [subtree]                     at ([shift=({ -\SUBTREEDX, -\SUBTREEDY})]x) {\Large $\pi$};
	 \node (sp) [ground]                    	at ([shift=({ -\NODEDX, -\NODEDY})]x) { };
	 \node (sl) [subtree]                     at ([shift=({  \SUBTREEDX, -\SUBTREEDY})]y) {\Large $\lambda$};	
	
	 \node (op) [label={right:{\large $op(50)^{z^{z^z}}$}}] at ([shift=({0.375, 0})]y) {};
	
	 \path[every node/.style={font=\sffamily\small}]
	    (0, 1)  edge[->,very thick]  node {} (r)
		  (r)     edge[->,very thick]  node {} (s)
			(s)     edge[->,very thick]  node {} (t)
			(t)     edge[->,very thick]  node {} (u)
			(u)     edge[->,very thick]  node {} (v)
			(v)     edge[->,very thick]  node {} (w)
			%% (w)     edge[->]  node {} (x)
			(w)     edge[->,very thick]  node {} (y)
			(x)     edge[->]  node {} (y)
			(y)     edge[->,very thick]  node {} (z)
			(z)     edge[->]  node {} (gl)
			(z)     edge[->]  node {} (gr)
			(r)     edge[->]  node {} (sa.north)
			(s)     edge[->]  node {} (sb.north)
			(t)     edge[->]  node {} (sg.north)
			(u)     edge[->]  node {} (sd.north)
			(v)     edge[->]  node {} (ss.north)
			(w)     edge[->]  node {} (st.north)
			(x)     edge[->]  node {} (sp)
			(y)     edge[->]  node {} (sl.north);	
\end{tikzpicture}
\caption{Key 30 is deleted (simple delete); node $X$ is removed}
\end{figure}
\end{frame}

\begin{frame}[c]{Example}
\begin{figure}[htp]
\begin{tikzpicture}[scale=0.5, transform shape]
   
	 \newcommand\NODEDX{1.25}
	 \newcommand\NODEDY{1.25}
	 \newcommand\SUBTREEDX{1.5}
	 \newcommand\SUBTREEDY{0.75}
	
   \node (r)	[treenode] 		                at (0, 0)       		                      	{$R$ \\ 100};
   \node (s)	[treenode, fill=black!20] 		at ([shift=({ -\NODEDX, -\NODEDY})]r)     	{$S$ \\ 10};
	 \node (t)	[treenode] 		                at ([shift=({  \NODEDX, -\NODEDY})]s)    		{$T$ \\ 90};
	 \node (u)	[treenode, fill=black!20] 	  at ([shift=({ -\NODEDX, -\NODEDY})]t)     	{$U$ \\ 50};
	 \node (v)	[treenode] 										at ([shift=({  \NODEDX, -\NODEDY})]u)     	{$V$ \\ 80};
	 \node (w)	[treenode] 										at ([shift=({ -\NODEDX, -\NODEDY})]v)     	{$W$ \\ 70};
	 \node (x)	[treenode, fill=black!20, dotted] 		at ([shift=({ -\NODEDX, -\NODEDY})]w)     	{$X$ \\ 30};
	 \node (y)	[treenode] 										at ([shift=({  \NODEDX, -\NODEDY})]x)     	{$Y$ \\ 60};
	 \node (z)	[treenode, dotted] 						at ([shift=({ -\NODEDX, -\NODEDY})]y)     	{$Z$ \\ 50};
	 \node (gl) [ground]                      at ([shift=({ -\NODEDX, -\NODEDY})]z)     	{ };
	 \node (gr) [ground]                      at ([shift=({  \NODEDX, -\NODEDY})]z)     	{ };
		
	 \node (sa) [subtree]                     at ([shift=({  \SUBTREEDX, -\SUBTREEDY})]r) {\Large $\alpha$};
	 \node (sb) [subtree]                     at ([shift=({ -\SUBTREEDX, -\SUBTREEDY})]s) {\Large $\beta$};
	 \node (sg) [subtree]                     at ([shift=({  \SUBTREEDX, -\SUBTREEDY})]t) {\Large $\gamma$};
	 \node (sd) [subtree]                     at ([shift=({ -\SUBTREEDX, -\SUBTREEDY})]u) {\Large $\delta$};
	 \node (ss) [subtree]                     at ([shift=({  \SUBTREEDX, -\SUBTREEDY})]v) {\Large $\sigma$};
	 \node (st) [subtree]                     at ([shift=({  \SUBTREEDX, -\SUBTREEDY})]w) {\Large $\tau$};
	 %% \node (sp) [subtree]                     at ([shift=({ -\SUBTREEDX, -\SUBTREEDY})]x) {\Large $\pi$};
	 \node (sp) [ground]                    	at ([shift=({ -\NODEDX, -\NODEDY})]x) { };
	 \node (sl) [subtree]                     at ([shift=({  \SUBTREEDX, -\SUBTREEDY})]y) {\Large $\lambda$};	
	
	 \node (op) [label={right:{\large $op(50)^{z^{z^z}}$}}] at ([shift=({0.375, 0})]y) {};
	
	 \path[every node/.style={font=\sffamily\small}]
	    (0, 1)  edge[->,very thick]  node {} (r)
		  (r)     edge[->,very thick]  node {} (s)
			(s)     edge[->,very thick]  node {} (t)
			(t)     edge[->,very thick]  node {} (u)
			(u)     edge[->,very thick]  node {} (v)
			(v)     edge[->,very thick]  node {} (w)
			%% (w)     edge[->]  node {} (x)
			(w)     edge[->,very thick]  node {} (y)
			(x)     edge[->]  node {} (y)
			%% (y)     edge[->]  node {} (z)
			(y)     edge[->, very thick]  node {} (gr)
			(z)     edge[->]  node {} (gl)
			(z)     edge[->]  node {} (gr)
			(r)     edge[->]  node {} (sa.north)
			(s)     edge[->]  node {} (sb.north)
			(t)     edge[->]  node {} (sg.north)
			(u)     edge[->]  node {} (sd.north)
			(v)     edge[->]  node {} (ss.north)
			(w)     edge[->]  node {} (st.north)
			(x)     edge[->]  node {} (sp)
			(y)     edge[->]  node {} (sl.north);		
\end{tikzpicture}
\caption{Key 20 is deleted (complex delete); key 20 is replaced with key 50 in node $U$ and node $Z$ is removed}
\end{figure}
\end{frame}

\begin{frame}[c]{Traversal Stack}
\begin{itemize}
\item a stack to keep track of anchor nodes of all nodes in the traversal path
\item reduces tree traversal cost during failures by restarting closer to an operation�s window
\end{itemize}
\end{frame}

\begin{frame}[c]{Traversal Stack}
\begin{figure}[htp]
\begin{tikzpicture}[scale=0.5, transform shape] 
	 \newcommand\NODEDX{1.25}
	 \newcommand\NODEDY{1.25}
	 \newcommand\SUBTREEDX{1.5}
	 \newcommand\SUBTREEDY{0.75}
	
   \node (r)	[treenode, fill=black!20] 		at (0, 0)       		                      	{$R$ \\  -$\infty$};
   \node (s)	[treenode] 										at ([shift=({ \NODEDX, -\NODEDY})]r)     	  {$S$ \\  $\infty$};
	 \node (t)	[treenode] 		                at ([shift=({  -\NODEDX, -\NODEDY})]s)    	{$T$ \\ 90};
	 \node (u)	[treenode, fill=black!20] 	  at ([shift=({ -\NODEDX, -\NODEDY})]t)     	{$U$ \\ 20};
	 \node (v)	[treenode] 										at ([shift=({  \NODEDX, -\NODEDY})]u)     	{$V$ \\ 80};
	 \node (w)	[treenode] 										at ([shift=({ -\NODEDX, -\NODEDY})]v)     	{$W$ \\ 70};
	 \node (x)	[treenode, fill=black!20] 		at ([shift=({ -\NODEDX, -\NODEDY})]w)     	{$X$ \\ 30};
	 \node (y)	[treenode] 										at ([shift=({  \NODEDX, -\NODEDY})]x)     	{$Y$ \\ 60};
	 \node (z)	[treenode] 										at ([shift=({ -\NODEDX, -\NODEDY})]y)     	{$Z$ \\ 50};
	 \node (gl) [ground]                      at ([shift=({ -\NODEDX, -\NODEDY})]z)     	{ };
	 \node (gr) [ground]                      at ([shift=({  \NODEDX, -\NODEDY})]z)     	{ };
		
	 \node (sa) [ground]                      at ([shift=({ -\SUBTREEDX, -\SUBTREEDY})]r) { };
	 \node (sb) [ground]                      at ([shift=({ \SUBTREEDX, -\SUBTREEDY})]s) 	{ };
	 \node (sg) [subtree]                     at ([shift=({  \SUBTREEDX, -\SUBTREEDY})]t) {\Large $\gamma$};
	 \node (sd) [subtree]                     at ([shift=({ -\SUBTREEDX, -\SUBTREEDY})]u) {\Large $\delta$};
	 \node (ss) [subtree]                     at ([shift=({  \SUBTREEDX, -\SUBTREEDY})]v) {\Large $\sigma$};
	 \node (st) [subtree]                     at ([shift=({  \SUBTREEDX, -\SUBTREEDY})]w) {\Large $\tau$};
	 \node (sp) [subtree]                     at ([shift=({ -\SUBTREEDX, -\SUBTREEDY})]x) {\Large $\pi$};
	 \node (sl) [subtree]                     at ([shift=({  \SUBTREEDX, -\SUBTREEDY})]y) {\Large $\lambda$};	
	
	 \node (op) [label={right:{\large $op(50) is here$}}] at ([shift=({0.5, 0})]s) {};
	
	 \path[every node/.style={font=\sffamily\small}]
	    %(0, 1)  edge[->,very thick]  node {} (r)
		  (r)     edge[->,very thick]  node {} (s)
			(s)     edge[->,very thick]  node {} (t)
			(t)     edge[->,very thick]  node {} (u)
			(u)     edge[->,very thick]  node {} (v)
			(v)     edge[->,very thick]  node {} (w)
			(w)     edge[->,very thick]  node {} (x)
			(x)     edge[->,very thick]  node {} (y)
			(y)     edge[->,very thick]  node {} (z)
			(z)     edge[->]  node {} (gl)
			(z)     edge[->]  node {} (gr)
			(r)     edge[->]  node {} (sa)
			(s)     edge[->]  node {} (sb)
			(t)     edge[->]  node {} (sg.north)
			(u)     edge[->]  node {} (sd.north)
			(v)     edge[->]  node {} (ss.north)
			(w)     edge[->]  node {} (st.north)
			(x)     edge[->]  node {} (sp.north)
			(y)     edge[->]  node {} (sl.north);		
\end{tikzpicture}
%\qquad
%\begin{tikzpicture}[scale=1.0, transform shape]
%\node[stack=9]  {
%0,\nodepart{one}Z,left,6
%\nodepart{two}Y,rigt,6
%\nodepart{three}X,left,3
%\nodepart{four}W,left,3
%\nodepart{five}V,right,3
%\nodepart{six}U,left,1
%\nodepart{seven}T,right,1
%\nodepart{eight}S,right,0
%\nodepart{nine}R,right,-1
%};
%\end{tikzpicture}
\qquad
\begin{tikzpicture}[scale=0.5, transform shape]
  \stacktop{} \cellptr{top of stack}
	\separator
	\cell{\texttt{S,R}}        \cellcomL{1} \coordinate () at (currentcell.east);
  \separator
	\cell{\texttt{R,null}}     \cellcomL{0} \coordinate () at (currentcell.east);
  \separator
\end{tikzpicture}
\caption{Operation $op(50)$ starting at R and ending at Z along with the stack}
\end{figure}
\end{frame}
\begin{frame}[c]{Traversal Stack}
\begin{figure}[htp]
\begin{tikzpicture}[scale=0.5, transform shape] 
	 \newcommand\NODEDX{1.25}
	 \newcommand\NODEDY{1.25}
	 \newcommand\SUBTREEDX{1.5}
	 \newcommand\SUBTREEDY{0.75}
	
   \node (r)	[treenode, fill=black!20] 		at (0, 0)       		                      	{$R$ \\  -$\infty$};
   \node (s)	[treenode] 										at ([shift=({ \NODEDX, -\NODEDY})]r)     	  {$S$ \\  $\infty$};
	 \node (t)	[treenode, fill=red]          at ([shift=({  -\NODEDX, -\NODEDY})]s)    	{$T$ \\ 90};
	 \node (u)	[treenode, fill=black!20] 	  at ([shift=({ -\NODEDX, -\NODEDY})]t)     	{$U$ \\ 20};
	 \node (v)	[treenode] 										at ([shift=({  \NODEDX, -\NODEDY})]u)     	{$V$ \\ 80};
	 \node (w)	[treenode] 										at ([shift=({ -\NODEDX, -\NODEDY})]v)     	{$W$ \\ 70};
	 \node (x)	[treenode, fill=black!20] 		at ([shift=({ -\NODEDX, -\NODEDY})]w)     	{$X$ \\ 30};
	 \node (y)	[treenode] 										at ([shift=({  \NODEDX, -\NODEDY})]x)     	{$Y$ \\ 60};
	 \node (z)	[treenode] 										at ([shift=({ -\NODEDX, -\NODEDY})]y)     	{$Z$ \\ 50};
	 \node (gl) [ground]                      at ([shift=({ -\NODEDX, -\NODEDY})]z)     	{ };
	 \node (gr) [ground]                      at ([shift=({  \NODEDX, -\NODEDY})]z)     	{ };
		
	 \node (sa) [ground]                      at ([shift=({ -\SUBTREEDX, -\SUBTREEDY})]r) { };
	 \node (sb) [ground]                      at ([shift=({ \SUBTREEDX, -\SUBTREEDY})]s) 	{ };
	 \node (sg) [subtree]                     at ([shift=({  \SUBTREEDX, -\SUBTREEDY})]t) {\Large $\gamma$};
	 \node (sd) [subtree]                     at ([shift=({ -\SUBTREEDX, -\SUBTREEDY})]u) {\Large $\delta$};
	 \node (ss) [subtree]                     at ([shift=({  \SUBTREEDX, -\SUBTREEDY})]v) {\Large $\sigma$};
	 \node (st) [subtree]                     at ([shift=({  \SUBTREEDX, -\SUBTREEDY})]w) {\Large $\tau$};
	 \node (sp) [subtree]                     at ([shift=({ -\SUBTREEDX, -\SUBTREEDY})]x) {\Large $\pi$};
	 \node (sl) [subtree]                     at ([shift=({  \SUBTREEDX, -\SUBTREEDY})]y) {\Large $\lambda$};	
	
	 \node (op) [label={left:{\large $op(50)$}}] at ([shift=({-0.5, 0})]t) {};
	
	 \path[every node/.style={font=\sffamily\small}]
	    %(0, 1)  edge[->,very thick]  node {} (r)
		  (r)     edge[->,very thick]  node {} (s)
			(s)     edge[->,very thick]  node {} (t)
			(t)     edge[->,very thick]  node {} (u)
			(u)     edge[->,very thick]  node {} (v)
			(v)     edge[->,very thick]  node {} (w)
			(w)     edge[->,very thick]  node {} (x)
			(x)     edge[->,very thick]  node {} (y)
			(y)     edge[->,very thick]  node {} (z)
			(z)     edge[->]  node {} (gl)
			(z)     edge[->]  node {} (gr)
			(r)     edge[->]  node {} (sa)
			(s)     edge[->]  node {} (sb)
			(t)     edge[->]  node {} (sg.north)
			(u)     edge[->]  node {} (sd.north)
			(v)     edge[->]  node {} (ss.north)
			(w)     edge[->]  node {} (st.north)
			(x)     edge[->]  node {} (sp.north)
			(y)     edge[->]  node {} (sl.north);		
\end{tikzpicture}
%\qquad
%\begin{tikzpicture}[scale=1.0, transform shape]
%\node[stack=9]  {
%0,\nodepart{one}Z,left,6
%\nodepart{two}Y,rigt,6
%\nodepart{three}X,left,3
%\nodepart{four}W,left,3
%\nodepart{five}V,right,3
%\nodepart{six}U,left,1
%\nodepart{seven}T,right,1
%\nodepart{eight}S,right,0
%\nodepart{nine}R,right,-1
%};
%\end{tikzpicture}
\qquad
\begin{tikzpicture}[scale=0.5, transform shape]
  \stacktop{} \cellptr{top of stack}
	\separator
	\cell{\Large \texttt{T,R}}        \cellcomL{2} \coordinate () at (currentcell.east);
  \separator
	\cell{\Large \texttt{S,R}}        \cellcomL{1} \coordinate () at (currentcell.east);
  \separator
	\cell{\Large \texttt{R,null}}     \cellcomL{0} \coordinate () at (currentcell.east);
  \separator
\end{tikzpicture}
%\caption{Operation $op(50)$ starting at R and suspended at Y along with the stack}
\end{figure}
\end{frame}


\begin{frame}[c]{Search}
search operations do not restart
\begin{figure}[htp]
\centering
{
	\begin{tikzpicture}[scale=0.5, transform shape]
	\node (p0) [] {search(K)};
	\node (p1) [process, below of=p0, text width=4cm] {Do a binary search for key K in the tree};
	\node (p2) [process, below of=p1, yshift=-1.5cm, text width=4.5cm] {Examine anchor node $A$ of top entry in the stack};
	\node (p3) [decision, below of=p2, yshift=-1.5cm, text width=2cm] {is anchor node marked?};
	\node (p4) [process, right of=p3, xshift=4cm, text width=4.5cm] {pop all entries upto anchor node $A$};
	\node (retT) [process, right of=p1, xshift=4cm, text width=1cm, minimum width=1cm] {return true};
	\node (retF) [process, left of=p2, xshift=-5cm, text width=1cm, minimum width=1cm] {return false};

	\draw [arrow] (p1) -- node[anchor=west] {K not found} (p2);
	\draw [arrow] (p1) -- node[anchor=south] {K found} (retT);
	\draw [arrow] (p2.east) -| node[anchor=north,pos=0.5] {K = A.key}    (retT.south);
	\draw [arrow] (p2.west) -- node (temp) [anchor=north,pos=0.5] {K $<$ A.key}  (retF.east);
	\draw [arrow] (p2) -- node[anchor=east] {K $>$ A.key} (p3);
	\draw [arrow] (p3.west) -| (retF.south) node[below, pos=0.5]{No}  (retF.south);
	\draw [arrow] (p3.east) -- node[anchor=north]{Yes} (p4.west);
	\draw [arrow] (p4.north) -- (p2.south);
	\node (temp1) [above of=temp,xshift=0.5cm,yshift=0.4cm] {(Aold.key $<$ K $<$ Anew.key)};
	\end{tikzpicture}
}
\caption{Sequence of steps in a search operation}
\end{figure}
\end{frame}

\begin{frame}[c]{Search}
\begin{figure}[htp]
\begin{tikzpicture}[scale=0.5, transform shape] 
	 \newcommand\NODEDX{1.25}
	 \newcommand\NODEDY{1.25}
	 \newcommand\SUBTREEDX{1.5}
	 \newcommand\SUBTREEDY{0.75}
	
   \node (r)	[treenode, fill=black!20] 		at (0, 0)       		                      	{$R$ \\  -$\infty$};
   \node (s)	[treenode] 										at ([shift=({ \NODEDX, -\NODEDY})]r)     	  {$S$ \\  $\infty$};
	 \node (t)	[treenode, fill=red]          at ([shift=({  -\NODEDX, -\NODEDY})]s)    	{$T$ \\ 90};
	 \node (u)	[treenode, fill=black!20] 	  at ([shift=({ -\NODEDX, -\NODEDY})]t)     	{$U$ \\ 20};
	 \node (v)	[treenode] 										at ([shift=({  \NODEDX, -\NODEDY})]u)     	{$V$ \\ 80};
	 \node (w)	[treenode] 										at ([shift=({ -\NODEDX, -\NODEDY})]v)     	{$W$ \\ 70};
	 \node (x)	[treenode, fill=black!20] 		at ([shift=({ -\NODEDX, -\NODEDY})]w)     	{$X$ \\ 30};
	 \node (y)	[treenode] 										at ([shift=({  \NODEDX, -\NODEDY})]x)     	{$Y$ \\ 60};
	 \node (z)	[treenode] 										at ([shift=({ -\NODEDX, -\NODEDY})]y)     	{$Z$ \\ 50};
	 \node (gl) [ground]                      at ([shift=({ -\NODEDX, -\NODEDY})]z)     	{ };
	 \node (gr) [ground]                      at ([shift=({  \NODEDX, -\NODEDY})]z)     	{ };
		
	 \node (sa) [ground]                      at ([shift=({ -\SUBTREEDX, -\SUBTREEDY})]r) { };
	 \node (sb) [ground]                      at ([shift=({ \SUBTREEDX, -\SUBTREEDY})]s) 	{ };
	 \node (sg) [subtree]                     at ([shift=({  \SUBTREEDX, -\SUBTREEDY})]t) {\Large $\gamma$};
	 \node (sd) [subtree]                     at ([shift=({ -\SUBTREEDX, -\SUBTREEDY})]u) {\Large $\delta$};
	 \node (ss) [subtree]                     at ([shift=({  \SUBTREEDX, -\SUBTREEDY})]v) {\Large $\sigma$};
	 \node (st) [subtree]                     at ([shift=({  \SUBTREEDX, -\SUBTREEDY})]w) {\Large $\tau$};
	 \node (sp) [subtree]                     at ([shift=({ -\SUBTREEDX, -\SUBTREEDY})]x) {\Large $\pi$};
	 \node (sl) [subtree]                     at ([shift=({  \SUBTREEDX, -\SUBTREEDY})]y) {\Large $\lambda$};	
	
	 \node (op) [label={left:{\large $op(50)$}}] at ([shift=({-0.5, 0})]t) {};
	
	 \path[every node/.style={font=\sffamily\small}]
	    %(0, 1)  edge[->,very thick]  node {} (r)
		  (r)     edge[->,very thick]  node {} (s)
			(s)     edge[->,very thick]  node {} (t)
			(t)     edge[->,very thick]  node {} (u)
			(u)     edge[->,very thick]  node {} (v)
			(v)     edge[->,very thick]  node {} (w)
			(w)     edge[->,very thick]  node {} (x)
			(x)     edge[->,very thick]  node {} (y)
			(y)     edge[->,very thick]  node {} (z)
			(z)     edge[->]  node {} (gl)
			(z)     edge[->]  node {} (gr)
			(r)     edge[->]  node {} (sa)
			(s)     edge[->]  node {} (sb)
			(t)     edge[->]  node {} (sg.north)
			(u)     edge[->]  node {} (sd.north)
			(v)     edge[->]  node {} (ss.north)
			(w)     edge[->]  node {} (st.north)
			(x)     edge[->]  node {} (sp.north)
			(y)     edge[->]  node {} (sl.north);		
\end{tikzpicture}
%\qquad
%\begin{tikzpicture}[scale=1.0, transform shape]
%\node[stack=9]  {
%0,\nodepart{one}Z,left,6
%\nodepart{two}Y,rigt,6
%\nodepart{three}X,left,3
%\nodepart{four}W,left,3
%\nodepart{five}V,right,3
%\nodepart{six}U,left,1
%\nodepart{seven}T,right,1
%\nodepart{eight}S,right,0
%\nodepart{nine}R,right,-1
%};
%\end{tikzpicture}
\qquad
\begin{tikzpicture}[scale=0.5, transform shape]
  \stacktop{} \cellptr{top of stack}
	\separator
	\cell{\Large \texttt{T,R}}        \cellcomL{2} \coordinate () at (currentcell.east);
  \separator
	\cell{\Large \texttt{S,R}}        \cellcomL{1} \coordinate () at (currentcell.east);
  \separator
	\cell{\Large \texttt{R,null}}     \cellcomL{0} \coordinate () at (currentcell.east);
  \separator
\end{tikzpicture}
%\caption{Operation $op(50)$ starting at R and suspended at Y along with the stack}
\end{figure}
\end{frame}
\begin{frame}[c]{Search}
\begin{figure}[htp]
\begin{tikzpicture}[scale=0.33, transform shape]
   
	 \newcommand\NODEDX{1.25}
	 \newcommand\NODEDY{1.25}
	 \newcommand\SUBTREEDX{1.5}
	 \newcommand\SUBTREEDY{0.75}
	
	 \node (r)	[treenode, fill=black!20] 		at (0, 0)       		                      	{$R$ \\  -$\infty$};
   \node (s)	[treenode] 										at ([shift=({ \NODEDX, -\NODEDY})]r)     	  {$S$ \\  $\infty$};
	 \node (t)	[treenode] 		                at ([shift=({  -\NODEDX, -\NODEDY})]s)    	{$T$ \\ 90};
	 \node (u)	[treenode, fill=black!20] 	  at ([shift=({ -\NODEDX, -\NODEDY})]t)     	{$U$ \\ 50};
	 \node (v)	[treenode] 										at ([shift=({  \NODEDX, -\NODEDY})]u)     	{$V$ \\ 80};
	 \node (w)	[treenode] 										at ([shift=({ -\NODEDX, -\NODEDY})]v)     	{$W$ \\ 70};
	 \node (x)	[treenode, fill=black!20, dotted] 		at ([shift=({ -\NODEDX, -\NODEDY})]w)     	{$X$ \\ 30};
	 \node (y)	[treenode] 										at ([shift=({  \NODEDX, -\NODEDY})]x)     	{$Y$ \\ 60};
	 \node (z)	[treenode, dotted] 						at ([shift=({ -\NODEDX, -\NODEDY})]y)     	{$Z$ \\ 50};
	 \node (gl) [ground]                      at ([shift=({ -\NODEDX, -\NODEDY})]z)     	{ };
	 \node (gr) [ground]                      at ([shift=({  \NODEDX, -\NODEDY})]z)     	{ };
		
	 \node (sa) [ground]                      at ([shift=({ -\SUBTREEDX, -\SUBTREEDY})]r) { };
	 \node (sb) [ground]                      at ([shift=({ \SUBTREEDX, -\SUBTREEDY})]s) 	{ };
	 \node (sg) [subtree]                     at ([shift=({  \SUBTREEDX, -\SUBTREEDY})]t) {\Large $\gamma$};
	 \node (sd) [subtree]                     at ([shift=({ -\SUBTREEDX, -\SUBTREEDY})]u) {\Large $\delta$};
	 \node (ss) [subtree]                     at ([shift=({  \SUBTREEDX, -\SUBTREEDY})]v) {\Large $\sigma$};
	 \node (st) [subtree]                     at ([shift=({  \SUBTREEDX, -\SUBTREEDY})]w) {\Large $\tau$};
	 %% \node (sp) [subtree]                     at ([shift=({ -\SUBTREEDX, -\SUBTREEDY})]x) {\Large $\pi$};
	 \node (sp) [ground]                    	at ([shift=({ -\NODEDX, -\NODEDY})]x) { };
	 \node (sl) [subtree]                     at ([shift=({  \SUBTREEDX, -\SUBTREEDY})]y) {\Large $\lambda$};	
	
	 \node (op) [label={right:{\large $search(50)$}}] at ([shift=({0.375, 0})]gr) {};
	
	 \path[every node/.style={font=\sffamily\small}]
	    %(0, 1)  edge[->,very thick]  node {} (r)
		  (r)     edge[->,very thick]  node {} (s)
			(s)     edge[->,very thick]  node {} (t)
			(t)     edge[->,very thick]  node {} (u)
			(u)     edge[->,very thick]  node {} (v)
			(v)     edge[->,very thick]  node {} (w)
			%% (w)     edge[->]  node {} (x)
			(w)     edge[->,very thick]  node {} (y)
			(x)     edge[->]  node {} (y)
			%% (y)     edge[->]  node {} (z)
			(y)     edge[->, very thick]  node {} (gr)
			(z)     edge[->]  node {} (gl)
			(z)     edge[->]  node {} (gr)
			(r)     edge[->]  node {} (sa)
			(s)     edge[->]  node {} (sb)
			(t)     edge[->]  node {} (sg.north)
			(u)     edge[->]  node {} (sd.north)
			(v)     edge[->]  node {} (ss.north)
			(w)     edge[->]  node {} (st.north)
			(x)     edge[->]  node {} (sp)
			(y)     edge[->]  node {} (sl.north);		
\end{tikzpicture}
\quad
\begin{tikzpicture}[scale=0.28, transform shape]
  \stacktop{} \cellptr{top}
	\separator
	\cell{\Large \texttt{Y,X}}        \cellcomL{7} \coordinate () at (currentcell.east);
  \separator
	\cell{\Large \texttt{X,U}}        \cellcomL{6} \coordinate () at (currentcell.east);
  \separator
	\cell{\Large \texttt{W,U}}        \cellcomL{5} \coordinate () at (currentcell.east);
  \separator
	\cell{\Large \texttt{V,U}}        \cellcomL{4} \coordinate () at (currentcell.east);
  \separator
	\cell{\Large \texttt{U,R}}        \cellcomL{3} \coordinate () at (currentcell.east);
  \separator
	\cell{\Large \texttt{T,R}}        \cellcomL{2} \coordinate () at (currentcell.east);
  \separator
	\cell{\Large \texttt{S,R}}        \cellcomL{1} \coordinate () at (currentcell.east);
  \separator
	\cell{\Large \texttt{R,null}}     \cellcomL{0} \coordinate () at (currentcell.east);
  \separator
\end{tikzpicture}
\quad
\begin{tikzpicture}[scale=0.4, transform shape]
	\node (p0) [] {search(K)};
	\node (p1) [process, below of=p0, text width=4cm] {Do a binary search for key K in the tree};
	\node (p2) [process, below of=p1, yshift=-1.5cm, text width=4.5cm,fill=black!20] {Examine anchor node $A$ of top entry in the stack};
	\node (p3) [decision, below of=p2, yshift=-1.5cm, text width=2cm,fill=black!20] {is anchor node marked?};
	\node (p4) [process, right of=p3, xshift=4cm, text width=4.5cm,fill=black!20] {pop all entries upto anchor node $A$};
	\node (retT) [process, right of=p1, xshift=4cm, text width=1cm, minimum width=1cm] {return true};
	\node (retF) [process, left of=p2, xshift=-5cm, text width=1cm, minimum width=1cm] {return false};

	\draw [arrow] (p1) -- node[anchor=west] {K not found} (p2);
	\draw [arrow] (p1) -- node[anchor=south] {K found} (retT);
	\draw [arrow] (p2.east) -| node[anchor=north,pos=0.5] {K = A.key}    (retT.south);
	\draw [arrow] (p2.west) -- node (temp) [anchor=north,pos=0.5] {K $<$ A.key}  (retF.east);
	\draw [arrow] (p2) -- node[anchor=east] {K $>$ A.key} (p3);
	\draw [arrow] (p3.west) -| (retF.south) node[below, pos=0.5]{No}  (retF.south);
	\draw [arrow] (p3.east) -- node[anchor=north]{Yes} (p4.west);
	\draw [arrow] (p4.north) -- (p2.south);
	\node (temp1) [above of=temp,xshift=0.5cm,yshift=0.4cm] {(A.oldKey $<$ K $<$ A.newKey)};
	\end{tikzpicture}
\caption{Key 30 is deleted;key 20 is deleted $\&$ replaced with key 50 in node $U$ and node $Z$ is removed}
\end{figure}
\end{frame}
\begin{frame}[c]{Search}
\begin{figure}[htp]
\begin{tikzpicture}[scale=0.5, transform shape]
   
	 \newcommand\NODEDX{1.25}
	 \newcommand\NODEDY{1.25}
	 \newcommand\SUBTREEDX{1.5}
	 \newcommand\SUBTREEDY{0.75}
	
	 \node (r)	[treenode, fill=black!20] 		at (0, 0)       		                      	{$R$ \\  -$\infty$};
   \node (s)	[treenode] 										at ([shift=({ \NODEDX, -\NODEDY})]r)     	  {$S$ \\  $\infty$};
	 \node (t)	[treenode] 		                at ([shift=({  -\NODEDX, -\NODEDY})]s)    	{$T$ \\ 90};
	 \node (u)	[treenode, fill=black!20] 	  at ([shift=({ -\NODEDX, -\NODEDY})]t)     	{$U$ \\ 50};
	 \node (v)	[treenode] 										at ([shift=({  \NODEDX, -\NODEDY})]u)     	{$V$ \\ 80};
	 \node (w)	[treenode] 										at ([shift=({ -\NODEDX, -\NODEDY})]v)     	{$W$ \\ 70};
	 \node (x)	[treenode, fill=black!20, dotted] 		at ([shift=({ -\NODEDX, -\NODEDY})]w)     	{$X$ \\ 30};
	 \node (y)	[treenode] 										at ([shift=({  \NODEDX, -\NODEDY})]x)     	{$Y$ \\ 60};
	 \node (z)	[treenode, dotted] 						at ([shift=({ -\NODEDX, -\NODEDY})]y)     	{$Z$ \\ 50};
	 \node (gl) [ground]                      at ([shift=({ -\NODEDX, -\NODEDY})]z)     	{ };
	 \node (gr) [ground]                      at ([shift=({  \NODEDX, -\NODEDY})]z)     	{ };
		
	 \node (sa) [ground]                      at ([shift=({ -\SUBTREEDX, -\SUBTREEDY})]r) { };
	 \node (sb) [ground]                      at ([shift=({ \SUBTREEDX, -\SUBTREEDY})]s) 	{ };
	 \node (sg) [subtree]                     at ([shift=({  \SUBTREEDX, -\SUBTREEDY})]t) {\Large $\gamma$};
	 \node (sd) [subtree]                     at ([shift=({ -\SUBTREEDX, -\SUBTREEDY})]u) {\Large $\delta$};
	 \node (ss) [subtree]                     at ([shift=({  \SUBTREEDX, -\SUBTREEDY})]v) {\Large $\sigma$};
	 \node (st) [subtree]                     at ([shift=({  \SUBTREEDX, -\SUBTREEDY})]w) {\Large $\tau$};
	 %% \node (sp) [subtree]                     at ([shift=({ -\SUBTREEDX, -\SUBTREEDY})]x) {\Large $\pi$};
	 \node (sp) [ground]                    	at ([shift=({ -\NODEDX, -\NODEDY})]x) { };
	 \node (sl) [subtree]                     at ([shift=({  \SUBTREEDX, -\SUBTREEDY})]y) {\Large $\lambda$};	
	
	 \node (op) [label={left:{\large $op(50) is here$}}] at ([shift=({-0.5, 0})]w) {};
	
	 \path[every node/.style={font=\sffamily\small}]
	    %(0, 1)  edge[->,very thick]  node {} (r)
		  (r)     edge[->,very thick]  node {} (s)
			(s)     edge[->,very thick]  node {} (t)
			(t)     edge[->,very thick]  node {} (u)
			(u)     edge[->,very thick]  node {} (v)
			(v)     edge[->,very thick]  node {} (w)
			%% (w)     edge[->]  node {} (x)
			(w)     edge[->,very thick]  node {} (y)
			(x)     edge[->]  node {} (y)
			%% (y)     edge[->]  node {} (z)
			(y)     edge[->, very thick]  node {} (gr)
			(z)     edge[->]  node {} (gl)
			(z)     edge[->]  node {} (gr)
			(r)     edge[->]  node {} (sa)
			(s)     edge[->]  node {} (sb)
			(t)     edge[->]  node {} (sg.north)
			(u)     edge[->]  node {} (sd.north)
			(v)     edge[->]  node {} (ss.north)
			(w)     edge[->]  node {} (st.north)
			(x)     edge[->]  node {} (sp)
			(y)     edge[->]  node {} (sl.north);		
\end{tikzpicture}
\qquad
\begin{tikzpicture}[scale=0.5, transform shape]
  \stacktop{} \cellptr{top of stack}
	\separator
	\cell{\texttt{W,U}}        \cellcomL{5} \coordinate () at (currentcell.east);
  \separator
	\cell{\texttt{V,U}}        \cellcomL{4} \coordinate () at (currentcell.east);
  \separator
	\cell{\texttt{U,R}}        \cellcomL{3} \coordinate () at (currentcell.east);
  \separator
	\cell{\texttt{T,R}}        \cellcomL{2} \coordinate () at (currentcell.east);
  \separator
	\cell{\texttt{S,R}}        \cellcomL{1} \coordinate () at (currentcell.east);
  \separator
	\cell{\texttt{R,null}}     \cellcomL{0} \coordinate () at (currentcell.east);
  \separator
\end{tikzpicture}
\caption{Pop upto marked anchor node $X$. Top of stack is now $W$. Examine anchor node $U$}
\end{figure}
\end{frame}

\begin{frame}[c]{Insert}
An insert operation needs to restart only if one of the anchor nodes in the path has become inconsistent
\begin{figure}[htp]
\centering
{
	\begin{tikzpicture}[scale=0.5, transform shape]
	\node (p0) [] {insert(K)};
	\node (p1) [process, below of=p0, text width=4cm] {Do a binary search for key K in the tree};
	\node (p2) [process, below of=p1, yshift=-1.5cm, text width=4.5cm] {Examine anchor node $A$ of top entry in the stack};
	\node (p3) [decision, below of=p2, yshift=-1.5cm, text width=2cm] {is anchor node marked?};
	\node (p4) [process, right of=p3, xshift=4cm, text width=4.5cm] {pop all entries upto anchor node $A$};
	\node (retF) [process, right of=p1, xshift=4cm, text width=1cm, minimum width=1cm] {return false};
	\node (retT) [process, left of=p2, xshift=-6cm, text width=4.5cm, minimum width=1cm] {discard suffix of the path after anchor node and find a restart point};
	\node (p5) [process, left of=p3, xshift=-6cm, text width=4.5cm, minimum width=1cm] {traversal terminates. Terminal node is returned as the injection point};

	\draw [arrow] (p1) -- node[anchor=west] {K not found} (p2);
	\draw [arrow] (p1) -- node[anchor=south] {K found} (retF);
	\draw [arrow] (p2.east) -| node[anchor=north,pos=0.5] {K = A.key}    (retF.south);
	\draw [arrow] (p2.west) -- node[anchor=north,pos=0.5] {K $<$ A.key}  (retT.east);
	\draw [arrow] (p2) -- node[anchor=east] {K $>$ A.key} (p3);
	\draw [arrow] (p3.west)  -- node[anchor=north]{No}  (p5.east);
	\draw [arrow] (p3.east) -- node[anchor=north]{Yes} (p4.west);
	\draw [arrow] (p4.north) -- (p2.south);

	\end{tikzpicture}
}
\caption{Sequence of steps in an insert operation}
\end{figure}
\end{frame}

\begin{frame}[c]{Delete}
A delete operation do not restart except when there is a failure in the execution phase
\begin{figure}[htp]
\centering
{
	\begin{tikzpicture}[scale=0.5, transform shape]
	\node (p0) [] {delete(K)};
	\node (p1) [process, below of=p0, text width=4cm] {Do a binary search for key K in the tree};
	\node (p2) [process, below of=p1, yshift=-1.5cm, text width=4.5cm] {Examine anchor node $A$ of top entry in the stack};
	\node (p3) [decision, below of=p2, yshift=-1.5cm, text width=2cm] {is anchor node marked?};
	\node (p4) [process, right of=p3, xshift=4cm, text width=4.5cm] {pop all entries upto anchor node $A$};
	\node (ex) [process, right of=p1, xshift=6cm, text width=4.5cm, minimum width=1cm] {go to execution phase};
	\node (retF) [process, left of=p2, xshift=-4cm, text width=1cm, minimum width=1cm] {return false};

	\draw [arrow] (p1) -- node[anchor=west] {K not found} (p2);
	\draw [arrow] (p1) -- node[anchor=south] {K found} (ex);
	\draw [arrow] (p2.east) -| node[anchor=north,pos=0.5] {K = A.key}    (ex.south);
	\draw [arrow] (p2.west) -- node[anchor=north,pos=0.5] {K $<$ A.key}  (retF.east);
	\draw [arrow] (p2) -- node[anchor=east] {K $>$ A.key} (p3);
	\draw [arrow] (p3.west)  -| node[anchor=north]{No}  (retF.south);
	\draw [arrow] (p3.east) -- node[anchor=north]{Yes} (p4.west);
	\draw [arrow] (p4.north) -- (p2.south);

	\end{tikzpicture}
}
\caption{Sequence of steps in a delete operation}
\end{figure}
\end{frame}
%\section{Performance on accelerators}
%\label{sec:experiments:micVsSnb}
%\input{c06/micVsSnb}

\chapter{conclusion}
\label{chapter:conclusion}
In this dissertation we presented a blocking and a non-blocking algorithm for concurrent manipulation of a binary search tree in an asynchronous shared memory system that supports search, insert and delete operations. 

Our \emph{lock-based} algorithm is very simple and looks almost identical to a sequential algorithm. In contrast to other lock-based algorithms, it locks edges rather than nodes. This minimizes the contention window of an operation and improves the system throughput. Since the locks are based on edges and as we steal bits from the children edges, the tree node structure is identical to a sequential tree node. This keeps the memory foot print low and reduces the impact of \emph{memory-allocation}. A desirable feature of this algorithm is that its search and insert operations are lock-free; they do not obtain any locks. As indicated by our experiments, our algorithm has the best performance---compared to other concurrent algorithms for a binary search---when the contention is relatively low. Specifically, it achieved the best performance for medium-sized and larger trees with mixed workloads and read-dominated workloads.

Our \emph{lock-free} algorithm combined ideas from two existing lock-free algorithms and is especially \emph{optimized for the conflict-free scenario}. Specifically, when compared to modify operations exiting internal binary search trees, its modify operations 
\begin{enumerate*}[label=(\alph*)]
\item have a smaller contention window, 
\item allocate fewer objects, 
\item execute fewer atomic instructions, and 
\item have a smaller memory footprint. 
\end{enumerate*}
Our experiments indicated that our new lock-free algorithm outperforms other lock-free algorithms in most cases.


We also presented a new approach to recover from such failures more efficiently in a concurrent binary search tree based on internal representation using \emph{local recovery} by restarting the traversal from the ``middle'' of the tree in order to locate an operation's window. Our approach is sufficiently general in the sense that we were able to apply it to a variety of concurrent binary search trees based on both blocking and non-blocking approaches.

We also presented a framework to allow a concurrent algorithm for maintaining an internal BST to recover locally when traversing the tree to locate a key. Our framework is sufficiently general that we were able to apply it to a variety of concurrent binary search trees based on both blocking and non-blocking approaches. we showed by experiments that our local recovery framework improved the performance of  concurrent BST algorithms under non-uniform key distribution (\emph{e.g.}, Zipfian) for many different workloads.

As a future work, we would like to analyze our local recovery algorithm (and possibly refine it if needed) so that, when applied to a \emph{non-blocking} BST, it yields a concurrent BST whose operations have provably low amortized time complexity. Also, we would like to analyze the effect of local recovery for other standard non-uniform distributions like normal and Poisson on real workloads.

We also would like to extend the ideas used in our algorithms and our local recovery technique to other data structures. A simple extension would be to apply them to $k$-ary search trees and then extend it further to $B$-trees. We also plan to explore other data structures like Bloom-filters which are commonly used in big-data applications.

\begin{thesisbib}
\bibliography{Bibliography/citations}
\end{thesisbib}

\begin{vita}
Arunmoezhi Ramachandran was born in Madurai, India and was brought up in Rajapalayam, India. He completed his schooling in 2003. He pursued his Bachelor's degree in Electronics and Communication at College of Engineering Guindy, Chennai. Then he worked at Infosys Technologies Ltd, Chennai as a software engineer from 2007 to 2010. Meanwhile he also pursued his Masters in Software Engineering at Birla Institute of Technology and Science, Pilani. Upon graduating with a Masters degree in 2011, he joined the PhD program in Computer Science at The University of Texas, Dallas, USA.
\end{vita}

\end{document}