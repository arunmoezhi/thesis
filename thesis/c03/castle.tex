\newenvironment{limitscope}{}{}
\begin{limitscope}
%%%%% castle macros - begin
\newcommand{\accesspath}{access-path}
\newcommand{\terminalnode}{terminal node}

\newcommand{\true}{\textsf{true}}
\newcommand{\false}{\textsf{false}}

\newcommand{\CAS}{\textsf{CAS}}

\newcommand{\sNodeOne}{\mathbb{R}}
\newcommand{\sNodeTwo}{\mathbb{S}}
\newcommand{\sKeyOne}{\infty_1}
\newcommand{\sKeyTwo}{\infty_2}

\newcommand{\targetnode}{target node}
\newcommand{\anchornode}{anchor node}

\newcommand{\myparent}{parent}
\newcommand{\myleft}{le\!f\!t}
\newcommand{\myright}{right}

\newcommand{\CASTLE}{\textsc{CASTLE}}
\newcommand{\CITRUS}{\textsc{CITRUS}}
\newcommand{\HJBST}{\textsc{LF-IBST}}
\newcommand{\NMBST}{\textsc{LF-EBST}}

\newcommand{\RemoveChild}{\textsc{RemoveChild}}
\newcommand{\LockAll}{\textsc{LockAll}}
\newcommand{\UnlockAll}{\textsc{UnlockAll}}
\newcommand{\ClearFlags}{\textsc{ClearFlags}}
\newcommand{\FindSmallest}{\textsc{FindSmallest}}

\newcommand{\lFlag}{lFlag}
\newcommand{\mFlag}{mFlag}
\newcommand{\nFlag}{nFlag}

%%%%% castle macros - end

\section{The Lock-Based Algorithm}
\label{sec:algorithm}
We first provide an overview of our algorithm. We then describe the algorithm in more detail and also give its pseudo-code. For ease of exposition, we describe our algorithm assuming no memory reclamation, which can be performed using the well-known technique of hazard pointers~\cite{Mic:2004:TPDS}.

\section{Overview of the Algorithm}
\label{sec:overview}
\begin{figure}[htp]
%\centering
%{
\subcaptionbox{Operation $op(50)$ is suspended at node $Y$ during its traversal. \label{fig:movement|suspended}}[0.45\textwidth]
{
\begin{tikzpicture}[scale=\myfigurescaletwo, transform shape] 
	 \newcommand\NODEDX{1.25}
	 \newcommand\NODEDY{1.25}
	 \newcommand\SUBTREEDX{1.5}
	 \newcommand\SUBTREEDY{0.75}
	
   \node (r)	[treenode] 		                at (0, 0)       		                      	{$R$ \\ 100};
   \node (s)	[treenode, fill=black!20] 		at ([shift=({ -\NODEDX, -\NODEDY})]r)     	{$S$ \\ 10};
	 \node (t)	[treenode] 		                at ([shift=({  \NODEDX, -\NODEDY})]s)    		{$T$ \\ 90};
	 \node (u)	[treenode, fill=black!20] 	  at ([shift=({ -\NODEDX, -\NODEDY})]t)     	{$U$ \\ 20};
	 \node (v)	[treenode] 										at ([shift=({  \NODEDX, -\NODEDY})]u)     	{$V$ \\ 80};
	 \node (w)	[treenode] 										at ([shift=({ -\NODEDX, -\NODEDY})]v)     	{$W$ \\ 70};
	 \node (x)	[treenode, fill=black!20] 		at ([shift=({ -\NODEDX, -\NODEDY})]w)     	{$X$ \\ 30};
	 \node (y)	[treenode] 										at ([shift=({  \NODEDX, -\NODEDY})]x)     	{$Y$ \\ 60};
	 \node (z)	[treenode] 										at ([shift=({ -\NODEDX, -\NODEDY})]y)     	{$Z$ \\ 50};
	 \node (gl) [ground]                      at ([shift=({ -\NODEDX, -\NODEDY})]z)     	{ };
	 \node (gr) [ground]                      at ([shift=({  \NODEDX, -\NODEDY})]z)     	{ };
		
	 \node (sa) [subtree]                     at ([shift=({  \SUBTREEDX, -\SUBTREEDY})]r) {\Large $\alpha$};
	 \node (sb) [subtree]                     at ([shift=({ -\SUBTREEDX, -\SUBTREEDY})]s) {\Large $\beta$};
	 \node (sg) [subtree]                     at ([shift=({  \SUBTREEDX, -\SUBTREEDY})]t) {\Large $\gamma$};
	 \node (sd) [subtree]                     at ([shift=({ -\SUBTREEDX, -\SUBTREEDY})]u) {\Large $\delta$};
	 \node (ss) [subtree]                     at ([shift=({  \SUBTREEDX, -\SUBTREEDY})]v) {\Large $\sigma$};
	 \node (st) [subtree]                     at ([shift=({  \SUBTREEDX, -\SUBTREEDY})]w) {\Large $\tau$};
	 \node (sp) [subtree]                     at ([shift=({ -\SUBTREEDX, -\SUBTREEDY})]x) {\Large $\pi$};
	 \node (sl) [subtree]                     at ([shift=({  \SUBTREEDX, -\SUBTREEDY})]y) {\Large $\lambda$};	
	
	 \node (op) [label={right:{\large $op(40)^{z^{z^z}}$, $op(50)^{z^{z^z}}$}}] at ([shift=({0.375, 0})]y) {};
	
	 \path[every node/.style={font=\sffamily\small}]
	    (0, 1)  edge[->,very thick]  node {} (r)
		  (r)     edge[->,very thick]  node {} (s)
			(s)     edge[->,very thick]  node {} (t)
			(t)     edge[->,very thick]  node {} (u)
			(u)     edge[->,very thick]  node {} (v)
			(v)     edge[->,very thick]  node {} (w)
			(w)     edge[->,very thick]  node {} (x)
			(x)     edge[->,very thick]  node {} (y)
			(y)     edge[->,very thick]  node {} (z)
			(z)     edge[->]  node {} (gl)
			(z)     edge[->]  node {} (gr)
			(r)     edge[->]  node {} (sa.north)
			(s)     edge[->]  node {} (sb.north)
			(t)     edge[->]  node {} (sg.north)
			(u)     edge[->]  node {} (sd.north)
			(v)     edge[->]  node {} (ss.north)
			(w)     edge[->]  node {} (st.north)
			(x)     edge[->]  node {} (sp.north)
			(y)     edge[->]  node {} (sl.north);		
\end{tikzpicture}
}
%%%%%%%%%%%%%%
\qquad
%%%%%%%%%%%%%%
\subcaptionbox{All keys in subtree $\pi$ are deleted one-by-one.\label{fig:movement|subtree|deleted}}[0.45\textwidth]
{

\begin{tikzpicture}[scale=\myfigurescaletwo, transform shape]
   
	 \newcommand\NODEDX{1.25}
	 \newcommand\NODEDY{1.25}
	 \newcommand\SUBTREEDX{1.5}
	 \newcommand\SUBTREEDY{0.75}
	
   \node (r)	[treenode] 		                at (0, 0)       		                      	{$R$ \\ 100};
   \node (s)	[treenode, fill=black!20] 		at ([shift=({ -\NODEDX, -\NODEDY})]r)     	{$S$ \\ 10};
	 \node (t)	[treenode] 		                at ([shift=({  \NODEDX, -\NODEDY})]s)    		{$T$ \\ 90};
	 \node (u)	[treenode, fill=black!20] 	  at ([shift=({ -\NODEDX, -\NODEDY})]t)     	{$U$ \\ 20};
	 \node (v)	[treenode] 										at ([shift=({  \NODEDX, -\NODEDY})]u)     	{$V$ \\ 80};
	 \node (w)	[treenode] 										at ([shift=({ -\NODEDX, -\NODEDY})]v)     	{$W$ \\ 70};
	 \node (x)	[treenode, fill=black!20] 		at ([shift=({ -\NODEDX, -\NODEDY})]w)     	{$X$ \\ 30};
	 \node (y)	[treenode] 										at ([shift=({  \NODEDX, -\NODEDY})]x)     	{$Y$ \\ 60};
	 \node (z)	[treenode] 										at ([shift=({ -\NODEDX, -\NODEDY})]y)     	{$Z$ \\ 50};
	 \node (gl) [ground]                      at ([shift=({ -\NODEDX, -\NODEDY})]z)     	{ };
	 \node (gr) [ground]                      at ([shift=({  \NODEDX, -\NODEDY})]z)     	{ };

	 \node (sa) [subtree]                     at ([shift=({  \SUBTREEDX, -\SUBTREEDY})]r) {\Large $\alpha$};
	 \node (sb) [subtree]                     at ([shift=({ -\SUBTREEDX, -\SUBTREEDY})]s) {\Large $\beta$};
	 \node (sg) [subtree]                     at ([shift=({  \SUBTREEDX, -\SUBTREEDY})]t) {\Large $\gamma$};
	 \node (sd) [subtree]                     at ([shift=({ -\SUBTREEDX, -\SUBTREEDY})]u) {\Large $\delta$};
	 \node (ss) [subtree]                     at ([shift=({  \SUBTREEDX, -\SUBTREEDY})]v) {\Large $\sigma$};
	 \node (st) [subtree]                     at ([shift=({  \SUBTREEDX, -\SUBTREEDY})]w) {\Large $\tau$};
	 %% \node (sp) [subtree]                     at ([shift=({ -\SUBTREEDX, -\SUBTREEDY})]x) {\Large $\pi$};
	 \node (sp) [ground]                    	at ([shift=({ -\NODEDX, -\NODEDY})]x) { };
	 \node (sl) [subtree]                     at ([shift=({  \SUBTREEDX, -\SUBTREEDY})]y) {\Large $\lambda$};	

   \node (op) [label={right:{\large $op(40)^{z^{z^z}}$, $op(50)^{z^{z^z}}$}}] at ([shift=({0.375, 0})]y) {};
	
	 \path[every node/.style={font=\sffamily\small}]
	    (0, 1)  edge[->,very thick]  node {} (r)
		  (r)     edge[->,very thick]  node {} (s)
			(s)     edge[->,very thick]  node {} (t)
			(t)     edge[->,very thick]  node {} (u)
			(u)     edge[->,very thick]  node {} (v)
			(v)     edge[->,very thick]  node {} (w)
			(w)     edge[->,very thick]  node {} (x)
			(x)     edge[->,very thick]  node {} (y)
			(y)     edge[->,very thick]  node {} (z)
			(z)     edge[->]  node {} (gl)
			(z)     edge[->]  node {} (gr)
			(r)     edge[->]  node {} (sa.north)
			(s)     edge[->]  node {} (sb.north)
			(t)     edge[->]  node {} (sg.north)
			(u)     edge[->]  node {} (sd.north)
			(v)     edge[->]  node {} (ss.north)
			(w)     edge[->]  node {} (st.north)
			(x)     edge[->]  node {} (sp)
			(y)     edge[->]  node {} (sl.north);	
\end{tikzpicture}
}
%%%%%%%%%%%%%%
\\
%%%%%%%%%%%%%%
\subcaptionbox{Key 30 is deleted (simple delete); node $X$ is removed. \label{fig:movement|anchor|deleted}}[0.45\textwidth]
{

\begin{tikzpicture}[scale=\myfigurescaletwo, transform shape]
   
	 \newcommand\NODEDX{1.25}
	 \newcommand\NODEDY{1.25}
	 \newcommand\SUBTREEDX{1.5}
	 \newcommand\SUBTREEDY{0.75}
	
   \node (r)	[treenode] 		                at (0, 0)       		                      	{$R$ \\ 100};
   \node (s)	[treenode, fill=black!20] 		at ([shift=({ -\NODEDX, -\NODEDY})]r)     	{$S$ \\ 10};
	 \node (t)	[treenode] 		                at ([shift=({  \NODEDX, -\NODEDY})]s)    		{$T$ \\ 90};
	 \node (u)	[treenode, fill=black!20] 	  at ([shift=({ -\NODEDX, -\NODEDY})]t)     	{$U$ \\ 20};
	 \node (v)	[treenode] 										at ([shift=({  \NODEDX, -\NODEDY})]u)     	{$V$ \\ 80};
	 \node (w)	[treenode] 										at ([shift=({ -\NODEDX, -\NODEDY})]v)     	{$W$ \\ 70};
	 \node (x)	[treenode, fill=black!20, dotted] 		at ([shift=({ -\NODEDX, -\NODEDY})]w)     	{$X$ \\ 30};
	 \node (y)	[treenode] 										at ([shift=({  \NODEDX, -\NODEDY})]x)     	{$Y$ \\ 60};
	 \node (z)	[treenode] 										at ([shift=({ -\NODEDX, -\NODEDY})]y)     	{$Z$ \\ 50};
	 \node (gl) [ground]                      at ([shift=({ -\NODEDX, -\NODEDY})]z)     	{ };
	 \node (gr) [ground]                      at ([shift=({  \NODEDX, -\NODEDY})]z)     	{ };
		
	 \node (sa) [subtree]                     at ([shift=({  \SUBTREEDX, -\SUBTREEDY})]r) {\Large $\alpha$};
	 \node (sb) [subtree]                     at ([shift=({ -\SUBTREEDX, -\SUBTREEDY})]s) {\Large $\beta$};
	 \node (sg) [subtree]                     at ([shift=({  \SUBTREEDX, -\SUBTREEDY})]t) {\Large $\gamma$};
	 \node (sd) [subtree]                     at ([shift=({ -\SUBTREEDX, -\SUBTREEDY})]u) {\Large $\delta$};
	 \node (ss) [subtree]                     at ([shift=({  \SUBTREEDX, -\SUBTREEDY})]v) {\Large $\sigma$};
	 \node (st) [subtree]                     at ([shift=({  \SUBTREEDX, -\SUBTREEDY})]w) {\Large $\tau$};
	 %% \node (sp) [subtree]                     at ([shift=({ -\SUBTREEDX, -\SUBTREEDY})]x) {\Large $\pi$};
	 \node (sp) [ground]                    	at ([shift=({ -\NODEDX, -\NODEDY})]x) { };
	 \node (sl) [subtree]                     at ([shift=({  \SUBTREEDX, -\SUBTREEDY})]y) {\Large $\lambda$};	
	
	 \node (op) [label={right:{\large $op(40)^{z^{z^z}}$, $op(50)^{z^{z^z}}$}}] at ([shift=({0.375, 0})]y) {};
	
	 \path[every node/.style={font=\sffamily\small}]
	    (0, 1)  edge[->,very thick]  node {} (r)
		  (r)     edge[->,very thick]  node {} (s)
			(s)     edge[->,very thick]  node {} (t)
			(t)     edge[->,very thick]  node {} (u)
			(u)     edge[->,very thick]  node {} (v)
			(v)     edge[->,very thick]  node {} (w)
			%% (w)     edge[->]  node {} (x)
			(w)     edge[->,very thick]  node {} (y)
			(x)     edge[->]  node {} (y)
			(y)     edge[->,very thick]  node {} (z)
			(z)     edge[->]  node {} (gl)
			(z)     edge[->]  node {} (gr)
			(r)     edge[->]  node {} (sa.north)
			(s)     edge[->]  node {} (sb.north)
			(t)     edge[->]  node {} (sg.north)
			(u)     edge[->]  node {} (sd.north)
			(v)     edge[->]  node {} (ss.north)
			(w)     edge[->]  node {} (st.north)
			(x)     edge[->]  node {} (sp)
			(y)     edge[->]  node {} (sl.north);	
\end{tikzpicture}
}
%%%%%%%%%%%%%%
\qquad
%%%%%%%%%%%%%%
\subcaptionbox{Key 20 is deleted (complex delete); key 20 is replaced with key 50 in node $U$ and node $Z$ is removed. \label{fig:movement|key|moved}}[0.45\textwidth]
{

\begin{tikzpicture}[scale=\myfigurescaletwo, transform shape]
   
	 \newcommand\NODEDX{1.25}
	 \newcommand\NODEDY{1.25}
	 \newcommand\SUBTREEDX{1.5}
	 \newcommand\SUBTREEDY{0.75}
	
   \node (r)	[treenode] 		                at (0, 0)       		                      	{$R$ \\ 100};
   \node (s)	[treenode, fill=black!20] 		at ([shift=({ -\NODEDX, -\NODEDY})]r)     	{$S$ \\ 10};
	 \node (t)	[treenode] 		                at ([shift=({  \NODEDX, -\NODEDY})]s)    		{$T$ \\ 90};
	 \node (u)	[treenode, fill=black!20] 	  at ([shift=({ -\NODEDX, -\NODEDY})]t)     	{$U$ \\ 50};
	 \node (v)	[treenode] 										at ([shift=({  \NODEDX, -\NODEDY})]u)     	{$V$ \\ 80};
	 \node (w)	[treenode] 										at ([shift=({ -\NODEDX, -\NODEDY})]v)     	{$W$ \\ 70};
	 \node (x)	[treenode, fill=black!20, dotted] 		at ([shift=({ -\NODEDX, -\NODEDY})]w)     	{$X$ \\ 30};
	 \node (y)	[treenode] 										at ([shift=({  \NODEDX, -\NODEDY})]x)     	{$Y$ \\ 60};
	 \node (z)	[treenode, dotted] 						at ([shift=({ -\NODEDX, -\NODEDY})]y)     	{$Z$ \\ 50};
	 \node (gl) [ground]                      at ([shift=({ -\NODEDX, -\NODEDY})]z)     	{ };
	 \node (gr) [ground]                      at ([shift=({  \NODEDX, -\NODEDY})]z)     	{ };
		
	 \node (sa) [subtree]                     at ([shift=({  \SUBTREEDX, -\SUBTREEDY})]r) {\Large $\alpha$};
	 \node (sb) [subtree]                     at ([shift=({ -\SUBTREEDX, -\SUBTREEDY})]s) {\Large $\beta$};
	 \node (sg) [subtree]                     at ([shift=({  \SUBTREEDX, -\SUBTREEDY})]t) {\Large $\gamma$};
	 \node (sd) [subtree]                     at ([shift=({ -\SUBTREEDX, -\SUBTREEDY})]u) {\Large $\delta$};
	 \node (ss) [subtree]                     at ([shift=({  \SUBTREEDX, -\SUBTREEDY})]v) {\Large $\sigma$};
	 \node (st) [subtree]                     at ([shift=({  \SUBTREEDX, -\SUBTREEDY})]w) {\Large $\tau$};
	 %% \node (sp) [subtree]                     at ([shift=({ -\SUBTREEDX, -\SUBTREEDY})]x) {\Large $\pi$};
	 \node (sp) [ground]                    	at ([shift=({ -\NODEDX, -\NODEDY})]x) { };
	 \node (sl) [subtree]                     at ([shift=({  \SUBTREEDX, -\SUBTREEDY})]y) {\Large $\lambda$};	
	
	 \node (op) [label={right:{\large $op(40)^{z^{z^z}}$, $op(50)^{z^{z^z}}$}}] at ([shift=({0.375, 0})]y) {};
	
	 \path[every node/.style={font=\sffamily\small}]
	    (0, 1)  edge[->,very thick]  node {} (r)
		  (r)     edge[->,very thick]  node {} (s)
			(s)     edge[->,very thick]  node {} (t)
			(t)     edge[->,very thick]  node {} (u)
			(u)     edge[->,very thick]  node {} (v)
			(v)     edge[->,very thick]  node {} (w)
			%% (w)     edge[->]  node {} (x)
			(w)     edge[->,very thick]  node {} (y)
			(x)     edge[->]  node {} (y)
			%% (y)     edge[->]  node {} (z)
			(y)     edge[->, very thick]  node {} (gr)
			(z)     edge[->]  node {} (gl)
			(z)     edge[->]  node {} (gr)
			(r)     edge[->]  node {} (sa.north)
			(s)     edge[->]  node {} (sb.north)
			(t)     edge[->]  node {} (sg.north)
			(u)     edge[->]  node {} (sd.north)
			(v)     edge[->]  node {} (ss.north)
			(w)     edge[->]  node {} (st.north)
			(x)     edge[->]  node {} (sp)
			(y)     edge[->]  node {} (sl.north);		
\end{tikzpicture}
}
%}
%%
\caption{An illustration of a key moving up the tree}
%\caption{An illustration of a key moving up the tree while operations $op(40)$ and $op(50)$ are suspended during their traversal. The \accesspath with respect to $op(40)$ (as well as $op(50)$) is shown with thick edges. Shaded nodes denote the \myanchor{} nodes on this path. Nodes with dotted border denote the marked nodes.}
\label{fig:movement}
\end{figure}
As mentioned earlier, every operation on a BST involves first traversing the tree from top to down starting from the root node and following either the left or the right child pointer until either the target key is found or a null pointer is encountered (termination condition). Depending on the outcome of the traversal and the type of the operation, the 
tree may then need to be modified to actually realize the operation. We refer to the period during which the tree is being traversed as \emph{seek phase}. Further, we refer to the period during which the tree is being modified as \emph{\action{} phase}.

During the seek phase, the target key may move from its current location to a new location up the tree. As a result, the traversal may miss the key both at its old location as well as its new location. For an illustration, see \figref{movement}. In the illustration, key 50 has moved up by five nodes. In most concurrent BST algorithms, if it is suspected that the key may have moved up the tree, then the traversal is simply restarted from the root node. Different algorithms use different approaches to detect possible key movement. For example, in~\cite{HowJon:2012:SPAA}, the traversal is restarted if the \emph{last right-turn} node is detected to have undergone some change. For example, in \figref{movement|suspended}, the last right-turn node for the operation $op(50)$ is node $X$. On reaching the terminal node in \figref{movement|key|moved}, after resuming running, $op(50)$ needs to restart since $X$ has since been removed from the tree. 

A \emph{re}-traversal of the tree may also be required if the operation encounters any failure during the \action{} phase. For example, in~\cite{RamMit:2015:PPoPP}, which is a lock-based algorithm, \action{} phase is aborted if, after locking the relevant edges, the validation step fails. This happens if the portion of the tree that lies within the ``operation's window'', which typically consists of a small constant number of nodes, has undergone some change since it was last observed. In that case, the operation moves to the seek phase again. 

In most concurrent BST algorithms, (a single instance of) the execution phase of an operation typically tends to have constant time complexity. The seek phase is where an operation may end up spending most of its time especially if the tree is large. Hence, it is desirable to make the seek phase of an operation more efficient by:
%%
\begin{enumerate*}[label=(\roman*)]
\item reducing the number of restarts due to ``suspected'' key movement, and
\item restarting the traversal from a point ``close'' to the operation's window.
\end{enumerate*}
%%
Note that local recovery may also improve the temporal reference locality of a concurrent BST thereby boosting its cache performance.
This leads to two separate but related questions that any local recovery algorithm needs to address. First, ``If a key is not found, then does the traversal need to restart?''. Second, ``If the traversal needs to be restarted, then from which node should the traversal restart?''

Consider an operation $\alpha$, with the target key denoted by $\mykey(\alpha)$, currently traversing the tree; let $\storedpath(\alpha)$ denote the path taken by $\alpha$ so far. For example, in \figref{movement|suspended}, considering only the subtree shown in the figure, $\storedpath(op(50)) = \ang{R, S, T, U, V, W, X, Y}$. At each non-terminal node in the path (except the last node), $\alpha$ either followed 
the left or the right child pointer. We say that a node in the path is an \emph{\myanchor} node if the operation followed its \emph{right} child pointer; otherwise we say that is a 
\emph{non-\myanchor{}} node. For example, in the path $\storedpath(op(50))$, nodes $S$, $U$ and $X$ are \myanchor{} nodes, whereas nodes $R$, $T$, $V$, $W$ and $Y$ 
are non-\myanchor{} nodes. We assume that the first \myanchor{} node in a traversal path is always a \emph{sentinel} node that is never marked (not shown in the figure). Of course, as an operation is traversing the tree, the tree may undergo changes as a result of which the path taken by the operation may no longer be correct. For example, in \figref{movement|key|moved}, the new \accesspath of $op(50)$ in the subtree, which is obtained from the subtree in \figref{movement|suspended} after applying several delete operations, is now given 
by $\ang{R, S, T, U}$. 

Note that, since in a complex delete operation we assume that the key being deleted is replaced with its successor key, the value of a key stored in a node can only increase. Therefore, the child pointer followed by an operation at a node, if it is still part of the \accesspath, may change (from right to left) for an \myanchor{} node but cannot change for a non-\myanchor{} node. For example, as shown in \figref{movement|suspended}, the node $U$ is an \myanchor{} node for $op(40)$. But due to the changes made to the tree, the key at $U$ has now become 50, as shown in \figref{movement|key|moved}. Hence, the pointer that $op(40)$ now needs to follow at $U$ is left and not right. 

For the remainder of the section, let $\alpha$ be an operation with key 
$\mykey(\alpha)$ traversing the tree and let $\storedpath(\alpha)$ denote the 
path it has traversed so far. To explain our local recovery algorithm, we first define some notions.

\begin{definition}[\myconsistent{} node]
An \myanchor{} node is said to be \emph{\myconsistent} with respect to $\alpha$ 
if its (current) key is less than $\mykey(\alpha)$; otherwise it is said to be \emph{\mynonconsistent} with respect to $\alpha$.
Moreover, it is said to be \emph{\myinconsistent} with respect to $\alpha$
if its key is strictly greater than $\mykey(\alpha)$.
\end{definition}


For example, in \figref{movement|key|moved}, \myanchor{} 
nodes $S$ and $X$ are still \myconsistent{} with respect to $op(40)$ but 
node $U$ has become \mynonconsistent{} with respect to $op(40)$. 


A node is said to be \emph{active} if it is reachable from the root of the tree; otherwise it is said to be \emph{passive}.
For example, in \figref{movement|anchor|deleted}, nodes $R$ and $S$ are active but node $X$ is passive.

\begin{definition}[\mylegal{} node]
\label{def:legal}
An active node $U$ in $\storedpath(\alpha)$ is said to be \emph{\mylegal} with respect to $\alpha$ 
if every \myanchor{} node preceding $U$ in $\storedpath(\alpha)$ is  
either \myconsistent{} with respect to $\alpha$ or passive.
\end{definition}

\begin{lemma}
\label{lem:legal:accesspath}
An active node in $\storedpath(\alpha)$ is \mylegal{} with respect to $\alpha$ \emph{if and only if} it is on the \accesspath of 
$\alpha$ in the tree.
\end{lemma}

We now describe an efficient method to test if a node in $\storedpath(\alpha)$ is \mylegal{}. The method is safe but not sound in the sense that it may misclassify a \mylegal{} node as not \mylegal{} (and thus cause unnecessary restarts), but not vice versa. 


\begin{definition}[\mycritical{} node]
\label{def:critical}
Consider two nodes $U$ and $V$ in $\storedpath(\alpha)$. Then,  
$V$ is said to \emph{\mycritical} with respect to $U$ if the following conditions hold:
%%
\begin{enumerate*}[label=(\roman*)]
\item $V$ precedes $U$ in $\storedpath(\alpha)$,
\item $V$ is an unmarked \myanchor{} node, and 
\item all \myanchor{} nodes between $U$ and $V$ (if any) are marked.
\end{enumerate*}
\end{definition}

For example, in \figref{movement|anchor|deleted}, consider the path \\ $\ang{R, S, T, U, V, W, X, Y}$. In the path, the \mycritical{} node with respect to node $Y$ 
is node $U$ and not node $X$ since $X$ is marked.  


\begin{definition}[\mysafe{} node]
\label{def:safe}
Consider a node $U$ in $\storedpath(\alpha)$ and let $V$ denote its \mycritical{} node. 
Then, $U$ is said to \emph{\mysafe{}} with respect to $\alpha$
if $V$ and all \myanchor{} nodes between $U$ and $V$ (if any) are \myconsistent{} with respect to $\alpha$.
\end{definition}

For example, in \figref{movement|anchor|deleted}, consider the path \\ $\ang{R, S, T, U, V, W, X, Y}$. Node $Y$ is \mysafe{} with respect to $op(40)$. However, in \figref{movement|key|moved}, node $Y$ is no longer \mysafe{} with respect to $op(40)$.

The next lemma relates the notions of \mylegality{} and \mysafety{}.

\begin{lemma}[\mysafety{} $\implies$ \mylegality{}]
\label{lem:safe:legal}
If an active node is \mysafe{} with respect to  $\alpha$, then it is also \mylegal{} with respect to $\alpha$.
\end{lemma}

\begin{lemma}
\label{lem:last|safe:not|present}
Let $U$ denote the last node of the path $\storedpath(\alpha)$ such that:
%%
\begin{enumerate*}[label=(\roman*)]
\item the next pointer that $\alpha$ needs to follow at $U$ is null, and
\item $U$'s key does not match $\mykey(\alpha)$.
\end{enumerate*}
%%
If $U$ is \mysafe{} with respect to $\alpha$, 
then there exists a time during $\alpha$'s traversal when $\mykey(\alpha)$ was not present in the tree.
\end{lemma}

Based on the above discussion, we now describe an efficient procedure to test whether or not a node is \mysafe{}. 

\subsubsection{Procedure to Test for \MySafety{}}
\label{sec:test|safety}
%%
To test whether or not a node $U$ in $\storedpath(\alpha)$ is \mysafe{} with respect to $\alpha$,  we can examine the \myanchor{} nodes (preceding $U$) in 
$\storedpath(\alpha)$ in the reverse order in which they were encountered during the traversal, starting from the one closest to $U$. We keep going as long as we find the \myanchor{} node to be \myconsistent{} and marked. Note that such an \myanchor{} node does not need to be examined again when testing for \mysafety{} in the future since its key value cannot change any more. We stop as soon as we encounter an \myanchor{} node, say $V$, that is either \mynonconsistent{} or unmarked.
%%
If $V$ is an \mynonconsistent{} \myanchor{} node, then we \emph{discard the suffix of $\storedpath(\alpha)$ after $V$}. This is because no node after $V$ in $\storedpath(\alpha)$ can act as a restart point in general.
Moreover, if $\alpha$ is a search or delete operation,  then the next lemma suggests that $\alpha$ 
can complete by announcing that $\mykey(\alpha)$ was not found in the tree. 

\begin{lemma}
\label{lem:inconsistent:not|present}
If an \myanchor{} node in $\storedpath(\alpha)$ is \myinconsistent{} with respect to $\alpha$, 
then there exists a time during $\alpha$'s traversal when $\mykey(\alpha)$ was not present in the tree.
\end{lemma}
%%

Intuitively, if $\mykey(\alpha)$ was present in the tree when $\alpha$ started but was not present in the tree at some point during $\alpha$'s traversal, 
then $\alpha$ can be linearized after the delete operation that removed $\mykey(\alpha)$ from the tree in case $\alpha$ is a search or delete operation.
%%
We next describe how each operation (search, insert and delete) achieves local recovery.

\subsubsection{Working of an Operation}

Assume that $\alpha$'s traversal ends in a terminal node, say $U$, whose key does not match $\mykey(\alpha)$.  As a first step, $\alpha$ invokes the procedure to test whether or not $U$ is a \mysafe{} node. Note that, if the procedure returns false, then it would have truncated the path such that the last node in the truncated path is an \mynonconsistent{} node, say $V$.


\myparagraph{Search or Delete Operation} 
\begin{figure}[htp]
\centering
{
	\begin{tikzpicture}[scale=0.5, transform shape]
	\node (p0) [] {search(K)};
	\node (p1) [process, below of=p0, text width=4cm] {Do a binary search for key K in the tree};
	\node (p2) [process, below of=p1, yshift=-1.5cm, text width=4.5cm] {Examine anchor node $A$ of top entry in the stack};
	\node (p3) [decision, below of=p2, yshift=-1.5cm, text width=2cm] {is anchor node marked?};
	\node (p4) [process, right of=p3, xshift=4cm, text width=4.5cm] {pop all entries upto anchor node $A$};
	\node (retT) [process, right of=p1, xshift=4cm, text width=1cm, minimum width=1cm] {return true};
	\node (retF) [process, left of=p2, xshift=-5cm, text width=1cm, minimum width=1cm] {return false};

	\draw [arrow] (p1) -- node[anchor=west] {K not found} (p2);
	\draw [arrow] (p1) -- node[anchor=south] {K found} (retT);
	\draw [arrow] (p2.east) -| node[anchor=north,pos=0.5] {K = A.key}    (retT.south);
	\draw [arrow] (p2.west) -- node (temp) [anchor=north,pos=0.5] {K $<$ A.key}  (retF.east);
	\draw [arrow] (p2) -- node[anchor=east] {K $>$ A.key} (p3);
	\draw [arrow] (p3.west) -| (retF.south) node[below, pos=0.5]{No}  (retF.south);
	\draw [arrow] (p3.east) -- node[anchor=north]{Yes} (p4.west);
	\draw [arrow] (p4.north) -- (p2.south);
	\node (temp1) [above of=temp,xshift=0.5cm,yshift=0.4cm] {(A.oldKey $<$ K $<$ A.newKey)};
	\pause
	\visible<2>
	{
		\node (p1) [process, below of=p0, text width=4cm,fill=black!20] {Do a binary search for key K in the tree};
	}
	\pause
	\visible<3>
	{
		\node (retT) [process, right of=p1, xshift=4cm, text width=1cm, minimum width=1cm,fill=black!20] {return true};
	}
	\pause
	\visible<4,8>
	{
	\node (p2) [process, below of=p1, yshift=-1.5cm, text width=4.5cm,fill=black!20] {Examine anchor node $A$ of top entry in the stack};
	}
	\pause
	\visible<5>
	{
	\node (p3) [decision, below of=p2, yshift=-1.5cm, text width=2cm,fill=black!20] {is anchor node marked?};
	}
		\pause
	\visible<6>
	{
	\node (retF) [process, left of=p2, xshift=-5cm, text width=1cm, minimum width=1cm,fill=black!20] {return false};
	}
	\pause
	\visible<7>
	{
	\node (p4) [process, right of=p3, xshift=4cm, text width=4.5cm,fill=black!20] {pop all entries upto anchor node $A$};
	}
	\pause
	\end{tikzpicture}
}
\caption{Sequence of steps in a search operation}
\end{figure}
\begin{figure}[htp]
\subcaptionbox{Operation $op(50)$ starting at R and suspended at Y along with the stack\label{fig:searchStack|a}}[0.45\textwidth]
{
\begin{tikzpicture}[scale=0.5, transform shape] 
	 \newcommand\NODEDX{1.25}
	 \newcommand\NODEDY{1.25}
	 \newcommand\SUBTREEDX{1.5}
	 \newcommand\SUBTREEDY{0.75}
	
   \node (r)	[treenode, fill=black!20] 		at (0, 0)       		                      	{$R$ \\  -$\infty$};
   \node (s)	[treenode] 										at ([shift=({ \NODEDX, -\NODEDY})]r)     	  {$S$ \\  $\infty$};
	 \node (t)	[treenode] 		                at ([shift=({  -\NODEDX, -\NODEDY})]s)    	{$T$ \\ 90};
	 \node (u)	[treenode, fill=black!20] 	  at ([shift=({ -\NODEDX, -\NODEDY})]t)     	{$U$ \\ 20};
	 \node (v)	[treenode] 										at ([shift=({  \NODEDX, -\NODEDY})]u)     	{$V$ \\ 80};
	 \node (w)	[treenode] 										at ([shift=({ -\NODEDX, -\NODEDY})]v)     	{$W$ \\ 70};
	 \node (x)	[treenode, fill=black!20] 		at ([shift=({ -\NODEDX, -\NODEDY})]w)     	{$X$ \\ 30};
	 \node (y)	[treenode] 										at ([shift=({  \NODEDX, -\NODEDY})]x)     	{$Y$ \\ 60};
	 \node (z)	[treenode] 										at ([shift=({ -\NODEDX, -\NODEDY})]y)     	{$Z$ \\ 50};
	 \node (gl) [ground]                      at ([shift=({ -\NODEDX, -\NODEDY})]z)     	{ };
	 \node (gr) [ground]                      at ([shift=({  \NODEDX, -\NODEDY})]z)     	{ };
		
	 \node (sa) [ground]                      at ([shift=({ -\SUBTREEDX, -\SUBTREEDY})]r) { };
	 \node (sb) [ground]                      at ([shift=({ \SUBTREEDX, -\SUBTREEDY})]s) 	{ };
	 \node (sg) [subtree]                     at ([shift=({  \SUBTREEDX, -\SUBTREEDY})]t) {\Large $\gamma$};
	 \node (sd) [subtree]                     at ([shift=({ -\SUBTREEDX, -\SUBTREEDY})]u) {\Large $\delta$};
	 \node (ss) [subtree]                     at ([shift=({  \SUBTREEDX, -\SUBTREEDY})]v) {\Large $\sigma$};
	 \node (st) [subtree]                     at ([shift=({  \SUBTREEDX, -\SUBTREEDY})]w) {\Large $\tau$};
	 \node (sp) [subtree]                     at ([shift=({ -\SUBTREEDX, -\SUBTREEDY})]x) {\Large $\pi$};
	 \node (sl) [subtree]                     at ([shift=({  \SUBTREEDX, -\SUBTREEDY})]y) {\Large $\lambda$};	
	
	 \node (op) [label={right:{\large $op(50) is here$}}] at ([shift=({0.5, 0})]y) {};
	
	 \path[every node/.style={font=\sffamily\small}]
	    %(0, 1)  edge[->,very thick]  node {} (r)
		  (r)     edge[->,very thick]  node {} (s)
			(s)     edge[->,very thick]  node {} (t)
			(t)     edge[->,very thick]  node {} (u)
			(u)     edge[->,very thick]  node {} (v)
			(v)     edge[->,very thick]  node {} (w)
			(w)     edge[->,very thick]  node {} (x)
			(x)     edge[->,very thick]  node {} (y)
			(y)     edge[->,very thick]  node {} (z)
			(z)     edge[->]  node {} (gl)
			(z)     edge[->]  node {} (gr)
			(r)     edge[->]  node {} (sa)
			(s)     edge[->]  node {} (sb)
			(t)     edge[->]  node {} (sg.north)
			(u)     edge[->]  node {} (sd.north)
			(v)     edge[->]  node {} (ss.north)
			(w)     edge[->]  node {} (st.north)
			(x)     edge[->]  node {} (sp.north)
			(y)     edge[->]  node {} (sl.north);		
\end{tikzpicture}
\quad
\begin{tikzpicture}[scale=0.45, transform shape]
  \stacktop{} \cellptr{top}
	\separator
	\cell{\Large \texttt{Y,X}}        \cellcomL{7} \coordinate () at (currentcell.east);
  \separator
	\cell{\Large \texttt{X,U}}        \cellcomL{6} \coordinate () at (currentcell.east);
  \separator
	\cell{\Large \texttt{W,U}}        \cellcomL{5} \coordinate () at (currentcell.east);
  \separator
	\cell{\Large \texttt{V,U}}        \cellcomL{4} \coordinate () at (currentcell.east);
  \separator
	\cell{\Large \texttt{U,R}}        \cellcomL{3} \coordinate () at (currentcell.east);
  \separator
	\cell{\Large \texttt{T,R}}        \cellcomL{2} \coordinate () at (currentcell.east);
  \separator
	\cell{\Large \texttt{S,R}}        \cellcomL{1} \coordinate () at (currentcell.east);
  \separator
	\cell{\Large \texttt{R,null}}     \cellcomL{0} \coordinate () at (currentcell.east);
  \separator
\end{tikzpicture}
}
\subcaptionbox{Key 30 is deleted;key 20 is deleted $\&$ replaced with key 50 in node $U$ and node $Z$ is removed\label{fig:searchStack|b}}[0.45\textwidth]
{
\begin{tikzpicture}[scale=0.5, transform shape]
   
	 \newcommand\NODEDX{1.25}
	 \newcommand\NODEDY{1.25}
	 \newcommand\SUBTREEDX{1.5}
	 \newcommand\SUBTREEDY{0.75}
	
	 \node (r)	[treenode, fill=black!20] 		at (0, 0)       		                      	{$R$ \\  -$\infty$};
   \node (s)	[treenode] 										at ([shift=({ \NODEDX, -\NODEDY})]r)     	  {$S$ \\  $\infty$};
	 \node (t)	[treenode] 		                at ([shift=({  -\NODEDX, -\NODEDY})]s)    	{$T$ \\ 90};
	 \node (u)	[treenode, fill=black!20] 	  at ([shift=({ -\NODEDX, -\NODEDY})]t)     	{$U$ \\ 50};
	 \node (v)	[treenode] 										at ([shift=({  \NODEDX, -\NODEDY})]u)     	{$V$ \\ 80};
	 \node (w)	[treenode] 										at ([shift=({ -\NODEDX, -\NODEDY})]v)     	{$W$ \\ 70};
	 \node (x)	[treenode, fill=black!20, dotted] 		at ([shift=({ -\NODEDX, -\NODEDY})]w)     	{$X$ \\ 30};
	 \node (y)	[treenode] 										at ([shift=({  \NODEDX, -\NODEDY})]x)     	{$Y$ \\ 60};
	 \node (z)	[treenode, dotted] 						at ([shift=({ -\NODEDX, -\NODEDY})]y)     	{$Z$ \\ 50};
	 \node (gl) [ground]                      at ([shift=({ -\NODEDX, -\NODEDY})]z)     	{ };
	 \node (gr) [ground]                      at ([shift=({  \NODEDX, -\NODEDY})]z)     	{ };
		
	 \node (sa) [ground]                      at ([shift=({ -\SUBTREEDX, -\SUBTREEDY})]r) { };
	 \node (sb) [ground]                      at ([shift=({ \SUBTREEDX, -\SUBTREEDY})]s) 	{ };
	 \node (sg) [subtree]                     at ([shift=({  \SUBTREEDX, -\SUBTREEDY})]t) {\Large $\gamma$};
	 \node (sd) [subtree]                     at ([shift=({ -\SUBTREEDX, -\SUBTREEDY})]u) {\Large $\delta$};
	 \node (ss) [subtree]                     at ([shift=({  \SUBTREEDX, -\SUBTREEDY})]v) {\Large $\sigma$};
	 \node (st) [subtree]                     at ([shift=({  \SUBTREEDX, -\SUBTREEDY})]w) {\Large $\tau$};
	 %% \node (sp) [subtree]                     at ([shift=({ -\SUBTREEDX, -\SUBTREEDY})]x) {\Large $\pi$};
	 \node (sp) [ground]                    	at ([shift=({ -\NODEDX, -\NODEDY})]x) { };
	 \node (sl) [subtree]                     at ([shift=({  \SUBTREEDX, -\SUBTREEDY})]y) {\Large $\lambda$};	
	
	 \node (op) [label={right:{\large $op(50) is here$}}] at ([shift=({0.375, 0})]gr) {};
	
	 \path[every node/.style={font=\sffamily\small}]
	    %(0, 1)  edge[->,very thick]  node {} (r)
		  (r)     edge[->,very thick]  node {} (s)
			(s)     edge[->,very thick]  node {} (t)
			(t)     edge[->,very thick]  node {} (u)
			(u)     edge[->,very thick]  node {} (v)
			(v)     edge[->,very thick]  node {} (w)
			%% (w)     edge[->]  node {} (x)
			(w)     edge[->,very thick]  node {} (y)
			(x)     edge[->]  node {} (y)
			%% (y)     edge[->]  node {} (z)
			(y)     edge[->, very thick]  node {} (gr)
			(z)     edge[->]  node {} (gl)
			(z)     edge[->]  node {} (gr)
			(r)     edge[->]  node {} (sa)
			(s)     edge[->]  node {} (sb)
			(t)     edge[->]  node {} (sg.north)
			(u)     edge[->]  node {} (sd.north)
			(v)     edge[->]  node {} (ss.north)
			(w)     edge[->]  node {} (st.north)
			(x)     edge[->]  node {} (sp)
			(y)     edge[->]  node {} (sl.north);		
\end{tikzpicture}
\quad
\begin{tikzpicture}[scale=0.45, transform shape]
  \stacktop{} \cellptr{top}
	\separator
	\cell{\Large \texttt{Y,X}}        \cellcomL{7} \coordinate () at (currentcell.east);
  \separator
	\cell{\Large \texttt{X,U}}        \cellcomL{6} \coordinate () at (currentcell.east);
  \separator
	\cell{\Large \texttt{W,U}}        \cellcomL{5} \coordinate () at (currentcell.east);
  \separator
	\cell{\Large \texttt{V,U}}        \cellcomL{4} \coordinate () at (currentcell.east);
  \separator
	\cell{\Large \texttt{U,R}}        \cellcomL{3} \coordinate () at (currentcell.east);
  \separator
	\cell{\Large \texttt{T,R}}        \cellcomL{2} \coordinate () at (currentcell.east);
  \separator
	\cell{\Large \texttt{S,R}}        \cellcomL{1} \coordinate () at (currentcell.east);
  \separator
	\cell{\Large \texttt{R,null}}     \cellcomL{0} \coordinate () at (currentcell.east);
  \separator
\end{tikzpicture}
}
\subcaptionbox{Pop upto marked anchor node $X$. Top of stack is now $W$. Examine anchor node $U$\label{fig:searchStack|c}}[0.45\textwidth]
{
\begin{tikzpicture}[scale=0.5, transform shape]
   
	 \newcommand\NODEDX{1.25}
	 \newcommand\NODEDY{1.25}
	 \newcommand\SUBTREEDX{1.5}
	 \newcommand\SUBTREEDY{0.75}
	
	 \node (r)	[treenode, fill=black!20] 		at (0, 0)       		                      	{$R$ \\  -$\infty$};
   \node (s)	[treenode] 										at ([shift=({ \NODEDX, -\NODEDY})]r)     	  {$S$ \\  $\infty$};
	 \node (t)	[treenode] 		                at ([shift=({  -\NODEDX, -\NODEDY})]s)    	{$T$ \\ 90};
	 \node (u)	[treenode, fill=black!20] 	  at ([shift=({ -\NODEDX, -\NODEDY})]t)     	{$U$ \\ 50};
	 \node (v)	[treenode] 										at ([shift=({  \NODEDX, -\NODEDY})]u)     	{$V$ \\ 80};
	 \node (w)	[treenode] 										at ([shift=({ -\NODEDX, -\NODEDY})]v)     	{$W$ \\ 70};
	 \node (x)	[treenode, fill=black!20, dotted] 		at ([shift=({ -\NODEDX, -\NODEDY})]w)     	{$X$ \\ 30};
	 \node (y)	[treenode] 										at ([shift=({  \NODEDX, -\NODEDY})]x)     	{$Y$ \\ 60};
	 \node (z)	[treenode, dotted] 						at ([shift=({ -\NODEDX, -\NODEDY})]y)     	{$Z$ \\ 50};
	 \node (gl) [ground]                      at ([shift=({ -\NODEDX, -\NODEDY})]z)     	{ };
	 \node (gr) [ground]                      at ([shift=({  \NODEDX, -\NODEDY})]z)     	{ };
		
	 \node (sa) [ground]                      at ([shift=({ -\SUBTREEDX, -\SUBTREEDY})]r) { };
	 \node (sb) [ground]                      at ([shift=({ \SUBTREEDX, -\SUBTREEDY})]s) 	{ };
	 \node (sg) [subtree]                     at ([shift=({  \SUBTREEDX, -\SUBTREEDY})]t) {\Large $\gamma$};
	 \node (sd) [subtree]                     at ([shift=({ -\SUBTREEDX, -\SUBTREEDY})]u) {\Large $\delta$};
	 \node (ss) [subtree]                     at ([shift=({  \SUBTREEDX, -\SUBTREEDY})]v) {\Large $\sigma$};
	 \node (st) [subtree]                     at ([shift=({  \SUBTREEDX, -\SUBTREEDY})]w) {\Large $\tau$};
	 %% \node (sp) [subtree]                     at ([shift=({ -\SUBTREEDX, -\SUBTREEDY})]x) {\Large $\pi$};
	 \node (sp) [ground]                    	at ([shift=({ -\NODEDX, -\NODEDY})]x) { };
	 \node (sl) [subtree]                     at ([shift=({  \SUBTREEDX, -\SUBTREEDY})]y) {\Large $\lambda$};	
	
	 \node (op) [label={left:{\large $op(50) is here$}}] at ([shift=({-0.5, 0})]w) {};
	
	 \path[every node/.style={font=\sffamily\small}]
	    %(0, 1)  edge[->,very thick]  node {} (r)
		  (r)     edge[->,very thick]  node {} (s)
			(s)     edge[->,very thick]  node {} (t)
			(t)     edge[->,very thick]  node {} (u)
			(u)     edge[->,very thick]  node {} (v)
			(v)     edge[->,very thick]  node {} (w)
			%% (w)     edge[->]  node {} (x)
			(w)     edge[->,very thick]  node {} (y)
			(x)     edge[->]  node {} (y)
			%% (y)     edge[->]  node {} (z)
			(y)     edge[->, very thick]  node {} (gr)
			(z)     edge[->]  node {} (gl)
			(z)     edge[->]  node {} (gr)
			(r)     edge[->]  node {} (sa)
			(s)     edge[->]  node {} (sb)
			(t)     edge[->]  node {} (sg.north)
			(u)     edge[->]  node {} (sd.north)
			(v)     edge[->]  node {} (ss.north)
			(w)     edge[->]  node {} (st.north)
			(x)     edge[->]  node {} (sp)
			(y)     edge[->]  node {} (sl.north);		
\end{tikzpicture}
\quad
\begin{tikzpicture}[scale=0.45, transform shape]
  \stacktop{} \cellptr{top}
	\separator
	\cell{\Large \texttt{W,U}}        \cellcomL{5} \coordinate () at (currentcell.east);
  \separator
	\cell{\Large \texttt{V,U}}        \cellcomL{4} \coordinate () at (currentcell.east);
  \separator
	\cell{\Large \texttt{U,R}}        \cellcomL{3} \coordinate () at (currentcell.east);
  \separator
	\cell{\Large \texttt{T,R}}        \cellcomL{2} \coordinate () at (currentcell.east);
  \separator
	\cell{\Large \texttt{S,R}}        \cellcomL{1} \coordinate () at (currentcell.east);
  \separator
	\cell{\Large \texttt{R,null}}     \cellcomL{0} \coordinate () at (currentcell.east);
  \separator
\end{tikzpicture}
}
\caption{An illustration of local recovery for a search operation}
\label{fig:searchStack}
\end{figure}
\begin{figure}
\centering
{
	\begin{tikzpicture}[scale=0.9,transform shape]
	\node (p0) [] {delete(K)};
	\node (p1) [process, below of=p0, text width=4cm] {Do a binary search for key K in the tree};
	\node (p2) [process, below of=p1, yshift=-1.5cm, text width=4.5cm] {Examine anchor node $A$ of top entry in the stack};
	\node (p3) [decision, below of=p2, yshift=-1.5cm, text width=2cm] {is anchor node marked?};
	\node (p4) [process, right of=p3, xshift=4cm, text width=4.5cm] {pop all entries upto anchor node $A$};
	\node (ex) [process, right of=p1, xshift=6cm, text width=4.5cm, minimum width=1cm] {go to execution phase};
	\node (retF) [process, left of=p2, xshift=-4.5cm, text width=1.2cm, minimum width=1cm] {return false};

	\draw [arrow] (p1) -- node[anchor=west] {K not found} (p2);
	\draw [arrow] (p1) -- node[anchor=south] {K found} (ex);
	\draw [arrow] (p2.east) -| node[anchor=north,pos=0.5] {K = A.key}    (ex.south);
	\draw [arrow] (p2.west) -- node[anchor=north,pos=0.5] {K $<$ A.key}  (retF.east);
	\draw [arrow] (p2) -- node[anchor=east] {K $>$ A.key} (p3);
	\draw [arrow] (p3.west)  -| node[anchor=north]{No}  (retF.south);
	\draw [arrow] (p3.east) -- node[anchor=north]{Yes} (p4.west);
	\draw [arrow] (p4.north) -- (p2.south);

	\end{tikzpicture}
}
\caption{Sequence of steps in a delete operation}
\label{fig:localRecoveryDelete}
\end{figure}
If the procedure returns true, then $\alpha$ announces that the key was not found and terminates (\lemref{last|safe:not|present}). 
On the other hand, if the procedure returns false, then $\alpha$ compares $V$'s key with $\mykey(\alpha)$. If $V$ is found to be \myinconsistent{}, then  again $\alpha$ announces that the key was not found and terminates (\lemref{inconsistent:not|present}).  Otherwise, it implies that $V$'s key matches $\mykey(\alpha)$. In this case, $\alpha$ either announces that the key was found and terminates (search operation) or moves to the \action{} phase to try to remove the key from the tree (delete operation).
Note that a search operation does not need to restart. A delete operation does not need to restart due to key movements; it only needs to restart if a failure occurs in the \action{} phase in which case it attempts to find a suitable node in the truncated path from where to restart the traversal (explained later).

\Figref{localRecoverySearch} gives an overview of the steps involved in a search operation. \Figref{searchStack} shows an example of how a stack is used to do local recovery. Each entry in the stack has the node encountered in the traversal along with its anchor node. \Figref{localRecoveryDelete} gives an overview of the steps involved in a delete operation.



\paragraph{Insert Operation:} 
\begin{figure}[htp]
\centering
{
	\begin{tikzpicture}
	\node (p0) [] {insert(K)};
	\node (p1) [process, below of=p0, text width=4cm] {Do a binary search for key K in the tree};
	\node (p2) [process, below of=p1, yshift=-1.5cm, text width=4.5cm] {Examine anchor node $A$ of top entry in the stack};
	\node (p3) [decision, below of=p2, yshift=-1.5cm, text width=2cm] {is anchor node marked?};
	\node (p4) [process, right of=p3, xshift=4cm, text width=4.5cm] {pop all entries upto anchor node $A$};
	\node (retF) [process, right of=p1, xshift=4cm, text width=1cm, minimum width=1cm] {return false};
	\node (retT) [process, left of=p2, xshift=-6cm, text width=4.5cm, minimum width=1cm] {discard suffix of the path after anchor node and find a restart point};
	\node (p5) [process, left of=p3, xshift=-6cm, text width=4.5cm, minimum width=1cm] {traversal terminates. Terminal node is returned as the injection point};

	\draw [arrow] (p1) -- node[anchor=west] {K not found} (p2);
	\draw [arrow] (p1) -- node[anchor=south] {K found} (retF);
	\draw [arrow] (p2.east) -| node[anchor=north,pos=0.5] {K = A.key}    (retF.south);
	\draw [arrow] (p2.west) -- node[anchor=north,pos=0.5] {K $<$ A.key}  (retT.east);
	\draw [arrow] (p2) -- node[anchor=east] {K $>$ A.key} (p3);
	\draw [arrow] (p3.west)  -- node[anchor=north]{No}  (p5.east);
	\draw [arrow] (p3.east) -- node[anchor=north]{Yes} (p4.west);
	\draw [arrow] (p4.north) -- (p2.south);

	\end{tikzpicture}
}
\caption{Sequence of steps in an insert operation}
\label{fig:localRecoveryInsert}
\end{figure}
If the procedure returns true, then $\alpha$ moves to the \action{} phase to try to add $\mykey(\alpha)$ to the tree (\lemref{last|safe:not|present}).
In this case, the new node will be attached as a child node of $U$.
On the other hand, if the procedure returns false, then $\alpha$ compares $V$'s key with $\mykey(\alpha)$.
If $V$'s key matches $\mykey(\alpha)$, then $\alpha$ announces that the key was found and terminates.
Otherwise, it implies that $V$ is  a \myinconsistent{} node.
In that case, $\alpha$ attempts to find a suitable node in the truncated path from where to restart the traversal (explained later).
\Figref{localRecoveryInsert} gives an overview of the steps involved in an insert operation.

\subsubsection{Procedure to Find a Restart Point}
\label{sec:find|restart|point}

To find a restart point, $\alpha$ first rolls back to an unmarked node on the path, say $U$, and invokes the procedure to test whether or not $U$ is a \mysafe{} node. If the procedure returns true, then $U$ is returned as a valid restart point (\lemsref{legal:accesspath}{safe:legal}).
Otherwise, as mentioned earlier, the procedure would have truncated the path such that the last node in the truncated path is an \mynonconsistent{} node, say $V$. 
Next, $\alpha$ compares $V$'s key with $\mykey(\alpha)$. There are two possible cases depending on the outcome of the comparison.

\myparagraph{Case 1 ($V$'s key matches $\mykey(\alpha)$)} In this case, $\alpha$ either moves to the \action{} phase (if $\alpha$ is a delete operation) or announces that the key was found and terminates (otherwise).

\myparagraph{Case 2 ($V$'s key is greater than $\mykey(\alpha)$} In this case, $\alpha$ either  recursively invokes the procedure to find a restart point (if $\alpha$ is an insert operation) or announces that the key was not found and terminates (otherwise).


\section{Details of the Algorithm}
\label{sec:description}


\begin{limitscope}

%% To limit the scope of the macros defined below

%% macros for pseudocode

\newcommand{\child}{child}
\newcommand{\node}{node}
\newcommand{\parent}{parent}

\newcommand{\mainSeekRecord}{seekTargetKey}
\newcommand{\successorSeekRecord}{seekSuccessorKey}


\newcommand{\targetStack}{targetStack}
\newcommand{\successorStack}{successorStack}


\newcommand{\successorStackInUse}{successorStackInUse}
\newcommand{\targetNode}{targetNode}




\newcommand{\key}{key}

\newcommand{\done}{done}
\newcommand{\result}{result}
\newcommand{\status}{status}
\newcommand{\restart}{restart}





\newcommand{\cKey}{key}
\newcommand{\nKey}{key}
\newcommand{\cNode}{current}
\newcommand{\pNode}{parent}
\newcommand{\nMarked}{marked}



\newcommand{\which}{which}
\newcommand{\address}{address}

\newcommand{\anchor}{anchor}

\newcommand{\stack}{stack}
\newcommand{\sTop}{top}
\newcommand{\sBottom}{bottom}
\newcommand{\current}{current}


\newcommand{\admissible}{admissible}
\newcommand{\critical}{critical}
\newcommand{\reference}{re\!f\!erence}

%% \newcommand{\OptReturn}[1][]{\Return #1\;}
\newcommand{\OptReturn}[1][]{}

\newcommand{\injectionPoint}{injectionPoint}



\newcommand{\Search}{\textsc{Search}}
\newcommand{\Insert}{\textsc{Insert}}
\newcommand{\Delete}{\textsc{Delete}}


\newcommand{\Inject}{\textsc{Inject}}



\remove{
\newcommand{\SeekForSuccessor}{\textsc{SeekForSuccessor}}
\newcommand{\NeedSuccessorKey}{\textsc{NeedSuccessorKey}}
\newcommand{\GetChild}{\textsc{GetChild}}
\newcommand{\Move}{\textsc{Move}}
\newcommand{\GetAddress}{\textsc{GetAddress}}
\newcommand{\IsNull}{\textsc{IsNull}}
\newcommand{\PopulateSeekRecord}{\textsc{PopulateSeekRecord}}
}



\newcommand{\mline}[1]{\DontPrintSemicolon #1 \PrintSemicolon}


\newcommand{\LEFT}{\textsf{LEFT}}
\newcommand{\RIGHT}{\textsf{RIGHT}}


\newcommand{\rarrow}{\!\rightarrow\!}
\newcommand{\type}{type}


\newcommand{\SEARCH}{\textsf{SEARCH}}
\newcommand{\INSERT}{\textsf{INSERT}}
\newcommand{\DELETE}{\textsf{DELETE}}

\newcommand{\STOPFOUND}{\textsf{FOUND}}
\newcommand{\STOPNOTFOUND}{\textsf{NOT\_FOUND}}
\newcommand{\DONOTKNOW}{\textsf{CONTINUE}}

\newcommand{\TARGETSTACK}{\textsf{TARGET\_STACK}}
\newcommand{\SUCCESSORSTACK}{\textsf{SUCCESSOR\_STACK}}

%%%%%%%%%%%%%%%%%%%%%%%%%%%%%%%%%%%%%%%%%%%%%%%%%%%%%%%%%%%%%%%%%%%%%%%%%%%%%%%%%%%%

\newcommand{\DefineKeyWords}{
%%
\SetKw{Boolean}{boolean}
\SetKw{Integer}{integer}
\SetKw{LAnd}{~and~}
\SetKw{LOr}{~or~}
\SetKw{LNot}{not}
\SetKw{Struct}{struct}
\SetKw{Null}{null}
\SetKw{True}{true}
\SetKw{False}{false}
\SetKw{Break}{break}
\SetKw{Continue}{continue}
\SetKw{Enum}{enum}
\SetKw{Word}{word}
%%
}

%%%%%%%%%%%%%%%%%%%%%%%%%%%%%%%%%%%%%%%%%%%%%%%%%%%%%%%%%%%%%%%%%%%%%%%%%%%%%%%%%%%%%

%% Data structures used by the local recovery algorithm

%%%%%%%%%%%%%%%%%%%%%%%%%%%%%%%%%%%%%%%%%%%%%%%%%%%%%%%%%%%%%%%%%%%%%%%%%%%%%%%%%%%%%

\begin{algorithm}[tb]
\caption{Data Structures Used} 
\label{algo:local-data|structures}
%%
\DefineKeyWords
%%
\tcp{Used to store information about a node visited during tree traversal}
\DontPrintSemicolon
\Struct StackEntry \{\;
\PrintSemicolon
%%
\label{lin:local-data|structures:begin}
\label{lin:local-stack|entry:begin}
\Indp 
   NodePtr $\node$\;
	 \Enum Direction $\which$\;
   \Integer $\anchor$\;
\Indm 
\}\;
\label{lin:local-stack|entry:end}

\BlankLine

\tcp{Used to store the path from the root node to the current node in the tree}
\DontPrintSemicolon
\Struct \TraversalRecord{} \{\;
\PrintSemicolon
%%
\label{lin:local-traversal|record:begin}
\Indp 
   StackEntry[~] $\stack$\;
	 \Integer $\sTop$\;
	 %% \Integer $\sBottom$\;
\Indm 
\}\;
\label{lin:local-traversal|record:end}

\BlankLine

\tcp{Used to store information about the operation currently in progress}
\DontPrintSemicolon
\Struct \OpRecord{} \{\;
\PrintSemicolon
%%
\label{lin:local-op|record:begin}
\Indp 
   \Enum Type $\type$\;
	 Key $\key$\;
	 %% \TraversalRecord{} $\targetStack$, $\successorStack$\;
	 \TraversalRecord{} $\targetStack$\;
	 %%\TraversalRecord{} $\successorStack$\;
	 %% \Boolean $\successorStackInUse$\;
	 %% NodePtr $\targetNode$\;
	 NodePtr $\injectionPoint$\;
	 \BlankLine
	 \tcp{algorithm-specific fields}
\Indm
\}\;
\label{lin:local-op|record:end}

\BlankLine

\tcp{Used to store the outcome of a tree traversal}
\DontPrintSemicolon
\Struct \SeekRecord \{\;
\PrintSemicolon
%%
\label{lin:local-seek|record:begin}
\Indp 
   %% \TraversalRecord{}Ptr $\traversalRecord$\;
	 \tcp{algorithm-specific fields (\emph{e.g.}, target node and its parent)}
\Indm 
\}\;
\label{lin:local-seek|record:end}

\remove{
   \BlankLine
%%
   \tcp{Local records used when executing an operation}
   \OpRecord{}Ptr $\opRecord$\;
   SeekRecordPtr $\mainSeekRecord$\;
   SeekRecordPtr $\successorSeekRecord$\; 
}
\label{lin:local-data|structures:end}
%%
\end{algorithm}


%%%%%%%%%%%%%%%%%%%%%%%%%%%%%%%%%%%%%%%%%%%%%%%%%%%%%%%%%%%%%%%%%%%%%%%%%%%%%%%%%%%%%

%% Functions used for manipulating traversal stack

%%%%%%%%%%%%%%%%%%%%%%%%%%%%%%%%%%%%%%%%%%%%%%%%%%%%%%%%%%%%%%%%%%%%%%%%%%%%%%%%%%%%%

\begin{algorithm}[tb]
\caption{Functions for Manipulating Traversal Stack} 
\label{algo:local-stack|functions}
%%
\DefineKeyWords
%%
\tcp{Returns the number of elements in the stack}
\DontPrintSemicolon
\Integer \Size( $\traversalRecord$ )\;
\PrintSemicolon
\label{lin:local-stack|begin}
\label{lin:local-size:begin}
\Begin
{
   
   \Return $\traversalRecord \rarrow \sTop + 1$\;
	 \label{lin:local-size:end}
}
%%
\BlankLine
%%
\tcp{Returns the topmost node in the stack}
\DontPrintSemicolon
NodePtr \GetTop( $\traversalRecord$ )\;
\PrintSemicolon
\label{lin:local-get|top:begin}
\Begin
{
   
   $\curly{ \stack, \sTop }$ := $\traversalRecord$\;
	 \label{lin:local-stack|retrieve}
	 \Return $\stack[\sTop] \rarrow \node$\;
	 \label{lin:local-get|top:end}
}
%%
\BlankLine
%%
\tcp{Returns the second topmost node in the stack}
\DontPrintSemicolon
NodePtr \GetSecondToTop( $\traversalRecord$ )\;
\label{lin:local-get|second|to|top:begin}
\PrintSemicolon
\Begin
{
   
   $\curly{ \stack, \sTop }$ := $\traversalRecord$\;
	 \Return $\stack[\sTop-1] \rarrow \node$\;
	 \label{lin:local-get|second|to|top:end}
}
%%
\BlankLine
%%
\tcp{Adds the given node to the stack along with its \myanchor{} node}
\DontPrintSemicolon
\AddToTop(  $\traversalRecord$, $\node$, $\which$ )\;
\PrintSemicolon
\label{lin:local-add|to|top:begin}
\Begin
{
%%
   
   $\curly{ \stack, \sTop }$ := $\traversalRecord$\;
   	
	 \tcp{find the \myanchor{} node}
   \lIf{$\which$ = \RIGHT}
	 {
	    $\anchor$ := $\sTop$
	 }
	 \lElse
	 {
	    $\anchor$ := $\stack[\sTop] \rarrow \anchor$
	 }
   
	 \tcp{push the node into the stack}
   $\stack[\sTop + 1]$ := $\curly{ \node, \which, \anchor }$\;
	 $\traversalRecord \rarrow \sTop$ := $\sTop + 1$\;
	 \OptReturn
	 \label{lin:local-add|to|top:end}
	
%%
}
%%
\BlankLine
%%
\tcp{Removes the topmost node from the stack}
\DontPrintSemicolon
\RemoveFromTop ( $\traversalRecord$ )\;
\PrintSemicolon
\label{lin:local-remove|from|top:begin}
\Begin
{
%%
   
   $\curly{ \stack, \sTop }$ := $\traversalRecord$\;
	
	 \tcp{update the \myanchor{} node of the penultimate entry if needed}
	 $\anchor$ := $\stack[\sTop - 1] \rarrow \anchor$\;
	 \If{$\stack[\sTop] \rarrow \anchor$ $<$ $\stack[\anchor] \rarrow \anchor$}
	 {
	    $\stack[\anchor] \rarrow \anchor$ := $\stack[\sTop] \rarrow \anchor$\;  
	 }
	
	 \tcp{pop the node from the stack}
	 $\traversalRecord \rarrow \sTop$ := $\sTop - 1$\;
	 \OptReturn
	 \label{lin:local-remove|from|top:end}
	 	 
%%
}
%%
\BlankLine
%%
\tcp{Pops the stack until a given entry}
\DontPrintSemicolon
\RemoveUntilCritical( $\traversalRecord$,  $index$ )\;
\PrintSemicolon
\label{lin:local-remove|until|critical:begin}
\Begin
{
  
   $\traversalRecord \rarrow \sTop$ := $index$\;
	 \OptReturn
	 \label{lin:local-remove|until|critical:end}

}
\end{algorithm}

\begin{algorithm}
\caption{Functions for Manipulating Traversal Stack (Continued)} 
\label{algo:local-local-stack|functions|2}
\DefineKeyWords
\tcp{Remember the \mycritical{} node (to avoid locating it again)}
\DontPrintSemicolon
\RememberCritical( $\traversalRecord$,  $\critical$ )\;
\PrintSemicolon
\label{lin:local-remember|critical:begin}
\Begin
{
  
   %% $\curly{ \stack, \sTop }$ := $\traversalRecord$\;  
	 $\curly{ \stack, \sTop }$ := $\traversalRecord$\;
   	
	 $\anchor$ := $\stack[\sTop] \rarrow \anchor$\;
	 \If{$\critical$ $<$ $\stack[\anchor] \rarrow \anchor$}
	 {
	   $\stack[\anchor] \rarrow \anchor$ := $\critical$\;
	 }
	 \OptReturn
	 \label{lin:local-remember|critical:end}
}
%%
\BlankLine
%%
\tcp{Returns a given entry in the stack}
\DontPrintSemicolon
\{ NodePtr, \Enum Direction, \Integer \}  \GetFullEntry( $\traversalRecord$,  $index$ )\;
\PrintSemicolon
\label{lin:local-get|full|entry:begin}
\Begin
{
   
   $\curly{ \stack, \sTop }$ := $\traversalRecord$\;
	 
	
	 \remove{
	
	    \tcp{find the location of the entry in the stack}
			
	    \lIf{$entry$ = $\top$}
	    {
	       $index$ := $\sTop$
	    }
	    \lElse
	    {
	       $index$ := $entry$
	    }
	
	}
	
	\lIf{$index$ = $\top$}
	{
	   \Return $\stack[\sTop]$
	}
	\lElse
	{
	   \Return $\stack[index]$
	}
	

	\label{lin:local-get|full|entry:end}
}
%%
\BlankLine
%%
\tcp{initializes the traversal stack} 
\DontPrintSemicolon
\InitializeTraversalRecord( $\traversalRecord$, $\type$ )\;
\PrintSemicolon
\label{lin:local-initialize|traversal|record:begin}
\Begin
{
%%
  \tcp{initialize the stack using sentinel nodes}
	\tcp{sentinel nodes are never removed from the stack}
	\tcp{a sentinel node is always a safe starting point for the traversal}
	
	\remove{
		
  \uIf{$\type$ = \TARGETSTACK}
	{
		
	   %% $\traversalRecord \rarrow \stack[0]$ := $\ang{\sNodeOne, -1}$\;
     %% $\traversalRecord \rarrow \stack[1]$ := $\ang{\sNodeTwo, 0 }$\;
	   %% $\traversalRecord \rarrow \sTop$ := 1\;
	   %% $\traversalRecord \rarrow \anchor$ := 0\;
	   \tcp{initialize the stack using sentinel nodes}
	   \tcp{sentinel nodes are never removed from the stack}
	   \tcp{a sentinel node is always a safe starting point for the traversal}
	}
	\lElse
	{
	   $\traversalRecord \rarrow \sTop$ := -1
	}
	
	}
	
	\OptReturn
	\label{lin:local-initialize|traversal|record:end}

%%
}
\end{algorithm}


\begin{comment}


%%%%%%%%%%%%%%%%%%%%%%%%%%%%%%%%%%%%%%%%%%%%%%%%%%%%%%%%%%%%%%%%%%%%%%%%%%%%%%%%%%%%%

%% Seek for search operation

%%%%%%%%%%%%%%%%%%%%%%%%%%%%%%%%%%%%%%%%%%%%%%%%%%%%%%%%%%%%%%%%%%%%%%%%%%%%%%%%%%%%%




\begin{algorithm}[tb]
\caption{Seek Function for Target Key (Search Operation)} 
\label{algo:local-seek:search}
%%
\DefineKeyWords
%%
\tcp{Traverses the tree starting from the root until either the key is found or a null pointer is encountered}
%% \tcp{also populates the traversal stack}
\DontPrintSemicolon
\Boolean \TraverseTree( $\opRecord$, $\seekRecord$ )\;
\PrintSemicolon
\label{lin:local-traverse|tree:begin}
\Begin
{
%%
   $\traversalRecord$ := $\opRecord \rarrow \targetStack$\;
	
	 \tcp{initialize the stack and the variables used in the traversal}
	 \InitializeTraversalRecord( $\traversalRecord$, \TARGETSTACK{} )\;
	 \label{lin:local-traverse|tree:initialize}
	 %% \tcp{initialize the variables used in the traversal}
   $\cNode$ := \GetTop( $\traversalRecord$ )\;
	 \label{lin:local-traverse|tree:start}
	 \BlankLine
	 \tcp{traverse the tree (starting from $\cNode$)}
	 \While{\True}
	 {
			\label{lin:local-traverse|tree:while:begin}
	    $\cKey$ := \GetKey( $\cNode$ )\;
			\label{lin:local-traverse|tree:while:first}
		  $\which$ := $\opRecord \rarrow \key < \cKey$ ? \LEFT{} : \RIGHT{}\;
			\label{lin:local-traverse|tree:select}
		  \tcp{read the next address to de-reference}
		  %% $\ang{ \ast, \ast, \nFlag, \address}$ := $\cNode \rarrow \child[\which]$\;  
			$\reference$ := \GetChild( $\cNode$, $\which$ )\;
			
			\BlankLine
			
		  \lIf{$\opRecord \rarrow \key$ = $\cKey$}
			{ 
			   \Return \True 
				 \label{lin:local-traverse|tree:match}
			}	     
			\lIf{\IsNull( $\reference$ )}
			{ 
			   \Return \False
				 \label{lin:local-traverse|tree:null}
			}	  
				
			\BlankLine
			
		  \tcp{traverse the next edge}
		  %% $\pNode$ := $\cNode$;
			$\address$ := \GetAddress( $\reference$ )\;
			$\cNode$ := $\address$\;
			\tcp{push the next node to be visited into the stack}
			\AddToTop( $\traversalRecord$, $\address$, $\which$ )\;
			\label{lin:local-traverse|tree:stack}
			\label{lin:local-traverse|tree:while:end}
			      
	  }	
		
		\OptReturn[\False]
		\label{lin:local-traverse|tree:end}
%%
}
%%
\BlankLine
%%
\tcp{Checks if the key being searched for has moved up in the path}
\DontPrintSemicolon
\Boolean \ExamineStack( $\opRecord$, $\seekRecord$ )\;
\PrintSemicolon
\label{lin:local-examine|stack:begin}
\Begin
{
%%
  
   $\result$ := \False\;
	 $\traversalRecord$ := $\opRecord \rarrow \targetStack$\;
	 
	 \BlankLine
	
   \tcp{start with the \myanchor{} closest to the topmost node in the stack}
	 $\curly{ \ast, \ast, \critical }$ := \GetFullEntry( $\traversalRecord$, $\top$ )\;
	 \label{lin:local-examine|stack:start}	
			
	 \BlankLine
			
			
	 \While{\True}
	 {
	    \label{lin:local-examine|stack:while:begin}
	    \tcp{retrieve the node and its closest \myanchor{} node from the stack}
	    $\curly{ \node, \ast, \anchor }$ := \GetFullEntry( $\traversalRecord$, $\critical$ )\;
			\tcp{read the attributes of the node}					
		  $\nMarked$ := \IsMarked( $\node$ )\; 
			$\nKey$ := \GetKey( $\node$ )\;
					
			\uIf{$\opRecord \rarrow \key$ = $\nKey$}
			{  
			   \label{lin:local-examine|stack:while:found:begin}
			   \tcp{the key stored in the node matches the one being searched for}
			   $\result$ := \True\;
				 \Break\;
				 \label{lin:local-examine|stack:while:found:end}
			} \uElseIf{($\opRecord \rarrow \key$ $<$ $\nKey$) \LOr \LNot($\nMarked$)}
			{
			   \label{lin:local-examine|stack:while:not|found:begin}
			   \tcp{the target key did not exist continuously in the tree}
			   \Break\;
				 \label{lin:local-examine|stack:while:not|found:end}
			} \Else(\tcp*[h]{examine the preceding \myanchor{} node})
			{			
			   \label{lin:local-examine|stack:while:continue:begin}
			   %% \tcp{examine the preceding \myanchor{} node}
			   $\critical$ := $\anchor$\;
				 \label{lin:local-examine|stack:while:continue:end}
			}
		  \label{lin:local-examine|stack:while:end} 
   }
	
	 %% \BlankLine
	 
	 %% \tcp{return the outcome}
	 %% \PopulateSeekRecord( $\seekRecord$, $\traversalRecord$ )\;
	 \Return $\result$\;
	 \label{lin:local-examine|stack:end}	
%%
}
%%
\BlankLine
%%
\tcp{Looks for a given key in the tree (invoked by a search operation)}
\DontPrintSemicolon
\Boolean \SeekForSearch( $\opRecord$, $\seekRecord$ )\;
\PrintSemicolon
\label{lin:local-seek|search:begin}
\Begin
{
%%
   
   
   \tcp{traverse the tree from top to down}
	 
	 $\result$ := \TraverseTree( $\opRecord$, $\seekRecord$ )\;
	 \label{lin:local-seek|search:traverse|tree}	
	 \If{\LNot($\result$)}	
	 {
	     \tcp{check if the key has moved up in the path}
	     $\result$ := \ExamineStack( $\opRecord$, $\seekRecord$ )\;
			 \label{lin:local-seek|search:examine|stack}
	 }
	
	 \tcp{return the outcome}
	 %% \tcp{return the outcome}
	 \PopulateSeekRecord( $\seekRecord$, $\traversalRecord$ )\;
	 \Return $\result$\;
   \label{lin:local-seek|search:end}
%%
}
%%
\end{algorithm}

\end{comment}

%%%%%%%%%%%%%%%%%%%%%%%%%%%%%%%%%%%%%%%%%%%%%%%%%%%%%%%%%%%%%%%%%%%%%%%%%%%%%%%%%%%%%

%% Functions used to achieve local recovery

%%%%%%%%%%%%%%%%%%%%%%%%%%%%%%%%%%%%%%%%%%%%%%%%%%%%%%%%%%%%%%%%%%%%%%%%%%%%%%%%%%%%%

\begin{algorithm}[tb]
\caption{Functions used to Achieve Local Recovery} 
\label{algo:local-local:recovery}
%%
\DefineKeyWords
%%
\tcp{Determines if the last node in the path is \mysafe}
\DontPrintSemicolon
\Boolean \FindAdmissible( $\opRecord$, $\traversalRecord$ )\;
\PrintSemicolon
\label{lin:local-test|safety:begin}
\Begin
{
%% 
   
   \tcp{examine the \myanchor{} nodes in the path one-by-one starting from the closest one}
	
	 $\curly{ \ast, \ast, \critical }$ := \GetFullEntry( $\traversalRecord$, $\top$ )\;
	 \While{\True}
	 {
	    \label{lin:local-test|safety:while:begin}
	    \tcp{retrieve the node and its \myanchor{} from the stack}
	    $\curly{ \node, \ast, \anchor }$ := \GetFullEntry( $\traversalRecord$, $\critical$ )\;
		  \tcp{read the attributes of the node}
			$\nMarked$ := \IsMarked( $\node$ )\; 
			$\nKey$ := \GetKey( $\node$ )\;
			
			\BlankLine
			
			\uIf(\tcp*[f]{the \myanchor{} node is still \myconsistent{}}){$\opRecord \rarrow \key$ $>$ $\nKey$}
		  {
			   \label{lin:local-test|safety:while:consistent:begin}
				
				 \uIf(\tcp*[f]{the last node is \mysafe{}}){\LNot($\nMarked$)}
				 {  
				    \RememberCritical( $\traversalRecord$, $\critical$ )\;
				    \Return \True\;
				 }
				 \Else(\tcp*[f]{the \myanchor{} node is \myinadmissible{}. discard the suffix and return})
				 {
				    %\uIf{\IsGreen(~)} 
						%{
						%   \tcp{examine the previous \myanchor{} node}
				    %   $\critical$ := $\anchor$\;
						%}
						%\Else
						%{
						   \RemoveUntilCritical( $\traversalRecord$, $\critical$ )\;
							 \Return \False\;
						%}
						
				 }
				 \label{lin:local-test|safety:while:consistent:end}
				
			}
			\Else(\tcp*[f]{the \myanchor{} node is \mynonconsistent{}. discard the suffix and return})
			{
			    \label{lin:local-test|safety:while:nonconsistent:begin}
			    \RemoveUntilCritical( $\traversalRecord$, $\critical$ )\;
					\Return \False\;
					\label{lin:local-test|safety:while:nonconsistent:end}
			}
				
			
			\label{lin:local-test|safety:while:end} 
	 }
	 
	 \OptReturn[\False]
	 \label{lin:local-test|safety:end}
	  
%%
}
%%
\BlankLine
%%
\tcp{Find a suitable node in the path from where to restart}
\DontPrintSemicolon
\Boolean \FindStartPoint( $\opRecord$, $\traversalRecord$ )\;
\PrintSemicolon
\label{lin:local-find|start|point:begin}
\Begin
{
%%
   
	 \While{\True}
   {
	    \label{lin:local-find|start|point:while:begin}
	    \tcp{backtrack until an unmarked node}
			$\cNode$ := \GetTop( $\traversalRecord$ )\;
			\label{lin:local-find|start|point:while:backtrack:begin}
    		
	    \While{\IsMarked( $\cNode$ )}
			{
			  
			   \RemoveFromTop( $\traversalRecord$ )\;
				 $\cNode$ := \GetTop( $\traversalRecord$ )\;
         
			}
			
			\BlankLine

      \tcp{check if the algorithm needs a clean parent node}
			\If{\NeedCleanParentNode( $\opRecord$, $\cNode$ )}
			{ 
			   \label{lin:local-find|start|point:while:clean:begin}
				 $\pNode$ := \GetSecondToTop( $\traversalRecord$ )\; 
				 \If{\LNot(\IsClean( $\pNode$ ))}
				 {
				    \tcp{need to backtrack even further}
						
											
				    \RemoveFromTop( $\traversalRecord$ )\;
						\Continue\;
						\label{lin:local-find|start|point:while:clean:end}
						\label{lin:local-find|start|point:while:backtrack:end}
				 }
			}
			
			
			\BlankLine
			
			\tcp{check if the last node in the path is a suitable restart point}
		
	    $\result$ := \FindAdmissible( $\opRecord$, $\traversalRecord$ )\;
	    \label{lin:local-find|start|point:while:test|safety}
			
			\lIf{$\result$}
			{
			   \Return \True
			}
			   
			\tcp{the path has been truncated and its last node is \myinadmissible{}}
		  $\status$ := \ExamineTop( $\opRecord$ )\;
			
			\lIf{$\status$ $\in$ \{ \STOPFOUND{}, \STOPNOTFOUND{} \}}
			{
			   \Return \False
			}
			
			\label{lin:local-find|start|point:while:end}
	 }
	
	 \OptReturn[\False]
	 \label{lin:local-find|start|point:end}
%%
}
\end{algorithm}
\begin{algorithm}[tb]
\caption{Functions used to Achieve Local Recovery  (Continued)} 
\label{algo:local-local:recovery|2}
\DefineKeyWords
\tcp{Invoked after a node in the path was deemed to be not \mysafe{}. In this case, the path would have been truncated such that its last node is \myinadmissible{}}
\tcp{Returns one of the following three values: \STOPFOUND{}, \STOPNOTFOUND{} or \DONOTKNOW{}}
\DontPrintSemicolon
\Enum Outcome \ExamineTop( $\opRecord$ )\;
\PrintSemicolon
\label{lin:local-examine|top:begin}
\Begin
{
   $\traversalRecord$ := $\opRecord \rarrow \targetStack$\;
	 \tcp{retrieve the topmost node from the stack which must be \myinadmissible{}}
	 $\cNode$ := \GetTop( $\traversalRecord$ )\;
	 $\cKey$ := \GetKey( $\cNode$ )\;
	 \BlankLine
	 \uIf{$\opRecord \rarrow \key$ $>$ $\cKey$} 
	 {   
	     \tcp{the last node is \myconsistent{}}
	     %% \tcp{can only happen for a non-green algorithm}
			 \Return \DONOTKNOW\;
			 \label{lin:local-examine|top:continue|1}
	 } 
	 \uElseIf{$\opRecord \rarrow \key$ $<$ $\cKey$}
	 {  
	    \label{lin:local-examine|top:inconsistent}
	    \tcp{the last node is \myinconsistent{}}
	    \lIf{$\opRecord \rarrow \type$ = \INSERT}
			{
			   \Return \DONOTKNOW
				 \label{lin:local-examine|top:continue|2}
			}
			\lElse
			{
			   \Return \STOPNOTFOUND
			}
	
	 }
	 \Else(\tcp*[h]{the last node contains the matching key})
	 {
	    \label{lin:local-examine|top:matching}
	    
	    \Return \STOPFOUND\;
	 }
	
	 \OptReturn[\DONOTKNOW]
	 \label{lin:local-examine|top:end}
}
%%
\end{algorithm}






%%%%%%%%%%%%%%%%%%%%%%%%%%%%%%%%%%%%%%%%%%%%%%%%%%%%%%%%%%%%%%%%%%%%%%%%%%%%%%%%%%%%%

%% Seek for target key

%%%%%%%%%%%%%%%%%%%%%%%%%%%%%%%%%%%%%%%%%%%%%%%%%%%%%%%%%%%%%%%%%%%%%%%%%%%%%%%%%%%%%




\begin{algorithm}[tb]
\caption{Seek Function for Target Key} 
\label{algo:local-seek}
%%
\DefineKeyWords
%%
\tcp{Looks for a given key in the tree (invoked by every operation)}
\DontPrintSemicolon
\Boolean \SeekForTarget( $\opRecord$, $\seekRecord$ )\;
\PrintSemicolon
\label{lin:local-seek:begin}
\Begin
{
%%
   
   $\traversalRecord$ := $\opRecord \rarrow \targetStack$\;
	 $\status$ := \DONOTKNOW\;
	 
	 \BlankLine
	
	 
	 \While{$\status$ = \DONOTKNOW}
	 { 
	    \label{lin:local-seek:while:begin}
	    \tcp{find a suitable restart point in the path}
      $\result$ := \FindStartPoint( $\opRecord$, $\traversalRecord$ )\;
	    \label{lin:local-seek:while:find|start|point} 
			
			\If(\tcp*[f]{examine the last node in the path}){\LNot($\result$)} 
			{
			   $\status$ := \ExamineTop( $\opRecord$ )\;
				 \Continue\;
				    
			}
			
	    \tcp{traverse the tree starting from the topmost node in the stack}
      $\cNode$ := \GetTop( $\traversalRecord$ )\;
			\label{lin:local-seek:while:traversal:begin}
	    \While{\True}
	    {
			  \label{lin:local-seek:while:traversal:first}
	       $\cKey$ := \GetKey( $\cNode$ )\;
		     $\which$ := $\opRecord \rarrow \key < \cKey$ ? \LEFT{} : \RIGHT{}\;
				 \label{lin:local-seek:while:traversal:select}
		    
			   \tcp{read the next address to de-reference}
		     $\reference$ := \GetChild( $\cNode$, $\which$ )\;
			   		
		     \BlankLine
				
		     \If{($\opRecord \rarrow \key$ = $\cKey$) \LOr \IsNull( $\reference$ )}
			   {
				    \label{lin:local-seek:while:traversal:stop:begin}
						
						\lIf{\IsNull( $\reference$ )}
						{
						   $\cKey$ := \GetKey( $\cNode$ )
						}
						
			      \tcp{either stop or backtrack \& restart }
												
						\uIf{$\opRecord \rarrow \key$ $\not=$ $\cKey$}
				    {
						   \tcp{if an insert operation, store the injection point}
							 \If{$\opRecord \rarrow \type$ = \INSERT}
							 {
							    $\opRecord \rarrow \injectionPoint$ :=  \GetAddress( $\reference$ )\;
									\label{lin:local-seek:while:traversal:store|injection}
							 }
							
							 \tcp{test if the terminal node is \mysafe} 		
						   $\result$ := \FindAdmissible( $\opRecord$, $\traversalRecord$ )\;
							 \label{lin:local-seek:while:traversal:test|safety}
							
					     \uIf(\tcp*[f]{terminal node is a \mysafe{} node}){$\result$}
						   {  
						      $\status$ := \STOPNOTFOUND\;
									\label{lin:local-seek:while:traversal:safe}
						   }
						   \Else(\tcp*[f]{examine the last node in the path})
							 {
							    $\status$ := \ExamineTop( $\opRecord$ )\;
									\label{lin:local-seek:while:traversal:not|safe}
						   }
						} 
						\Else(\tcp*[f]{terminal node contains the matching key})
						{
						   $\status$ := \STOPFOUND\;
							 \label{lin:local-seek:while:traversal:match}
						}
											
					  
						\Break; \tcp*[f]{terminate the current traversal}
						\label{lin:local-seek:while:traversal:stop:end}
						 
			   }
			  				
				 \BlankLine
					
			   $\address$ := \GetAddress( $\reference$ ); \tcp*[f]{traverse the next edge}
				
				 
				 \If{$\opRecord \rarrow \type$ $\in$ \{ \INSERT{}, \DELETE{} \}}
				 {
				    
				    $\restart$ := \Move( $\cNode$, $\address$, $\which$ )\;
				    \label{lin:local-seek:while:traversal:move}
				    \If(\tcp*[f]{the algorithm wants to restart the traversal}){$\restart$}
				    {
						   \Break\;
				    }
				 }
				 
				
				 
				
				 \AddToTop( $\traversalRecord$, $\address$, $\which$ ); \tcp*[f]{push the node visited into the stack}
			   \label{lin:local-seek:while:traversal:push}
			   \label{lin:local-seek:while:traversal:end}
	    } %% inner while loop
			
			
			
	    \label{lin:local-seek:while:end}	
	 }	 %% outer while loop
		
	 \BlankLine
		
	 \tcp{return the outcome}

	 \PopulateSeekRecord( $\seekRecord$, $\opRecord$ )\;
	 \label{lin:local-seek:populate}
	 \Return ($\status$ = \STOPFOUND \  ? \  \True : \False)\;
   \label{lin:local-seek:end}
%%
}
\end{algorithm}





\remove{



%%%%%%%%%%%%%%%%%%%%%%%%%%%%%%%%%%%%%%%%%%%%%%%%%%%%%%%%%%%%%%%%%%%%%%%%%%%%%%%%%%%%%

%% Seek for successor key

%%%%%%%%%%%%%%%%%%%%%%%%%%%%%%%%%%%%%%%%%%%%%%%%%%%%%%%%%%%%%%%%%%%%%%%%%%%%%%%%%%%%%



\begin{algorithm}[tb]
\caption{Seek Function for Successor Key} 
\label{algo:local-seek:successor}
%%
\DefineKeyWords
%%
\tcp{Looks for the next largest key with respect to a given key (invoked by a complex delete operation)}
\DontPrintSemicolon
\Boolean \SeekForSuccessor( $\opRecord$, $\seekRecord$ )\;
\PrintSemicolon
\label{lin:local-seek|successor:begin}
\Begin
{
%%
   
   \tcp{the stack used in locating the successor key is initialized before this function is invoked}	
	 $\traversalRecord$ := $\opRecord \rarrow \successorStack$\;
	 \While{\True}
	 {
	
	    \label{lin:local-seek|successor:while:begin}
		  \tcp{backtrack until either an unmarked node or the stack becomes empty}
			
	    
			\While{(\Size( $\traversalRecord$ ) $>$ 1)}      
			{
			    \label{lin:local-seek|successor:while:backtrack:begin}
			    $\cNode$ := \GetTop( $\traversalRecord$ )\;
					\lIf{\LNot(\IsMarked( $\cNode$ ))}
					{
					   \Break
					} 
					\lElse
					{
					   \RemoveFromTop( $\traversalRecord$ )
					}
					\label{lin:local-seek|successor:while:backtrack:end}
			}
	
	    \BlankLine
			
			\tcp{backtrack further if a clean parent is needed but the parent is not clean}
			\If{(\Size( $\traversalRecord$ ) $\geq$ 2)}
			{
			   \label{lin:local-seek|successor:while:clean:begin}
			   \If{\NeedCleanParentNode( $\opRecord$, $\cNode$ )}
		     {
				    
			      \tcp{the parent node should be a clean node}
				    $\pNode$ := \GetSecondToTop( $\traversalRecord$ )\;
				    \If{\LNot(\IsClean( $\pNode$ ))}
				    {
				       \RemoveFromTop( $\traversalRecord$ )\;
					     \Continue\;
							 \label{lin:local-seek|successor:while:clean:end}
				    }
			   }
				
			}
			
			\BlankLine
			
	    \tcp{check if the successor key is still needed}
	    $\reference$ := \NeedSuccessorKey( $\opRecord$ )\;
			\label{lin:local-seek|successor:while:need|successor}
			\If{\IsNull( $\reference$ )}{ 
			   \tcp{successor key no longer required}
			   \Return \false\;
			}
			
			\BlankLine
			
			$\cNode$ := \GetTop( $\traversalRecord$ )\;
			\uIf{(\Size( $\traversalRecord$ ) = 1)}
			{
			   \label{lin:local-seek|successor:while:traversal:if:begin}
			   %% $\address$ := \GetAddress( $\reference$ )\;
				 \tcp{visit the node pointed to by the reference returned by \NeedSuccessorKey{} function}
			   $\which$ := \RIGHT\;
				 \label{lin:local-seek|successor:while:traversal:if:end}
			}
			\Else
			{
			   \label{lin:local-seek|successor:while:traversal:else:begin}
			   \tcp{follow the left child node of the top node, if it exists}
		 	   $\reference$ := \GetChild( $\cNode$, \LEFT{} )\;
			   %% \lIf{\IsNull( $\reference$ )}{ \Break }
				 %% $\address$ := \GetAddress( $\reference$ )\;
			   $\which$ := \LEFT\;
				 \label{lin:local-seek|successor:while:traversal:else:end}
			}
			
			\Repeat{\True}
	    {
			   \label{lin:local-seek|successor:while:traversal:begin}
			   \tcp{stop if reference is null}
				 \lIf{\IsNull( $\reference$ )}{ \Break }
				
				 \tcp{obtain the address of the node}
				 $\address$ := \GetAddress( $\reference$ )\;	
				
				 \BlankLine
				
				 \tcp{traverse the edge}
				 $\restart$ := \Move( $\cNode$, $\address$, $\which$ )\;
				 \label{lin:local-seek|successor:while:traversal:move}
			   \If{$\restart$}
			   {
			      \tcp{the algorithm wants to restart the traversal}
				    \Break\;
						\label{lin:local-seek|successor:while:traversal:restart}
			   }  
				 
			   \tcp{push the node visited into the stack}
				 \AddToTop( $\traversalRecord$, $\address$, $\which$ )\;
				 \label{lin:local-seek|successor:while:traversal:stack}
				 \label{lin:local-seek|successor:while:traversal:advance:begin}
			   $\cNode$ := $\address$\;
				 \tcp{determine the next node to be visited}
			   $\reference$ := \GetChild( $\cNode$, \LEFT{} )\;
			   $\which$ := \LEFT{}\;
				 \label{lin:local-seek|successor:while:traversal:advance:end}
				 \label{lin:local-seek|successor:while:traversal:end}
			}
			\label{lin:local-seek|successor:while:end}
	 }
	
	
	 \BlankLine
	
	 \tcp{return the outcome}
	 \PopulateSeekRecord( $\seekRecord$, $\opRecord$ )\;
	 \Return \True;
	 \label{lin:local-seek|successor:end}
%%
}
%%
\end{algorithm}

}


\end{limitscope}

A pseudo-code of the local recovery algorithm is given in \pseudosref{local-data|structures}{local-seek:successor}.
The pseudo-code only shows the seek phase of an algorithm and not its \action{} phase since the \action{} phase is algorithm-specific. 

The local recovery algorithm assumes that the original algorithm supports the following functions:
\begin{enumerate*}[label=(\alph*)]
\item \GetKey(~), \IsMarked(~) and \GetChild(~) return the various attributes of a tree node,
\item \IsNull(~) returns true if a reference is null and false otherwise,
\item \GetAddress(~) returns the node address stored in a reference, if non-null,
\item \Move(~) enables the original algorithm to move along an edge, which may invoke helping and cause traversal to restart as in~\cite{HowJon:2012:SPAA},
\item \NeedCleanParentNode(~) returns true if the operation needs the parent node to be clean and have no operation in progress (needed for a delete operation since it needs to modify a child pointer at the parent node), and
\item \PopulateSeekRecord(~) copies the relevant information from the traversal state required by the algorithm into a seek record (\emph{e.g.}, addresses of the terminal node and its parent).
\end{enumerate*}

\Pseudoref{local-data|structures} shows the data structures used by the local recovery algorithm.
Note that all the data structures shown in \pseudoref{local-data|structures} are \emph{local} to a process not shared among processes.
A process uses three main data structures, namely \TraversalRecord{}, \OpRecord{} and \SeekRecord{}. 
A \TraversalRecord{} (\linesref{local-traversal|record:begin}{local-traversal|record:end}) is essentially a stack used to store the nodes visited during tree traversal when looking for a key.
Each entry in a traversal stack (\linesref{local-stack|entry:begin}{local-stack|entry:end}) stores the address of the node, the location of its closest \myanchor{} node (within the stack) and whether the node is a left or right child of its parent.
An \OpRecord{} (\linesref{local-op|record:begin}{local-op|record:end}) stores information about the operation such as type and key as well two stacks: one used when looking for the target key (all operations) and one used when looking for the successor key (only complex delete operations).
Finally, a \SeekRecord{} (\linesref{local-seek|record:begin}{local-seek|record:end}) is used to return the outcome of a tree traversal to the original algorithm.
Its fields are algorithm-specific. 
%%
For example, for \CASTLE{}, \SeekRecord{} contains three fields: 
\begin{enumerate*}[label=(\alph*)]
\item two addresses, namely those of the target node and its parent, and
\item the contents of the injection point where an insert operation needs to attach the new node. 
\end{enumerate*}

\Pseudoref{local-stack|functions} shows the functions used to manipulate a traversal stack. 
The function \Size{} (\linesref{local-size:begin}{local-size:end}) returns the number of entries in the stack. 
The functions \GetTop{} (\linesref{local-get|top:begin}{local-get|top:end}) and \GetSecondToTop{} (\linesref{local-get|second|to|top:begin}{local-get|second|to|top:end}) return the address of the node stored in the topmost entry and the entry below it, respectively. 
The function \AddToTop{} (\linesref{local-add|to|top:begin}{local-add|to|top:end}) adds an entry to the top of the stack while \RemoveFromTop{} (\linesref{local-remove|from|top:begin}{local-remove|from|top:end}) removes an entry from the top of the stack. 
The function \RemoveUntilCritical{} (\linesref{local-remove|until|critical:begin}{local-remove|until|critical:end}) removes the entries from the top of the stack until a given point.
The function \RememberCritical{} (\linesref{local-remember|critical:begin}{local-remember|critical:end}) updates the \myanchor{} field of the \myanchor{} node of the topmost entry in the stack.
The function \GetFullEntry{} (\linesref{local-get|full|entry:begin}{local-get|full|entry:end} returns all the three fields of a given entry in the stack (may not be the topmost entry). 
The function \InitializeTraversalRecord{} (\linesref{local-initialize|traversal|record:begin}{local-initialize|traversal|record:end}) initializes a traversal stack using sentinel nodes.

\Pseudosref[ \& ]{local-local:recovery}{local-seek} show the functions used to look for a key in the tree.
The function \SeekForTarget{} (\linesref{local-seek:begin}{local-seek:end}) first backtracks to a \mysafe{} node in the stack (\lineref{local-seek:while:find|start|point}).
Initially, the start point of the traversal will be a sentinel node, which is always a \mysafe{} node.
The function then traverses the tree from top to down by following either the left or the right child pointer (\lineref{local-seek:while:traversal:select}) until it either finds the key or encounters a null pointer (\linesref{local-seek:while:traversal:stop:begin}{local-seek:while:traversal:stop:end}).
In case the terminal node of the traversal does not contain the matching key, the function tests whether or not the terminal node is \mysafe{} by invoking  \FindAdmissible{} function (\lineref{local-seek:while:traversal:test|safety}).
If the node is found to be \mysafe{}, then the traversal is ended (\lineref{local-seek:while:traversal:safe}.
Otherwise, the traversal path would have been truncated in which case the function determines if the traversal can still stop by invoking \ExamineTop{} function that looks at the last node in the path (\lineref{local-seek:while:traversal:not|safe}). 
As the traversal moves down the tree, the function also populates the traversal stack (\lineref{local-seek:while:traversal:move}).
The function \FindAdmissible{} (\linesref{local-test|safety:begin}{local-test|safety:end}) implements the method described in \secref{test|safety}.
The function \FindStartPoint{} (\linesref{local-find|start|point:begin}{local-find|start|point:end}) implements the method described in \secref{find|restart|point}.
The function \ExamineTop{} (\linesref{local-examine|top:begin}{local-examine|top:end}) examines the topmost node in the stack. If the node is found to be a \myinconsistent{} node, then the traversal can stop for a search and delete operation (\lineref{local-examine|top:inconsistent}).
If the node's key matches the operation's key, then also the traversal can stop (\lineref{local-examine|top:matching}).
Otherwise, the traversal needs to be restarted (\linesref[ \& ]{local-examine|top:continue|1}{local-examine|top:continue|2}).

\Pseudoref{local-seek:successor} shows the function \SeekForSuccessor{} used to locate the successor key by a complex delete operation (\linesref{local-seek|successor:begin}{local-seek|successor:end}). The function first backtracks to an unmarked node with a clean parent if required (\linesref{local-seek|successor:while:backtrack:begin}{local-seek|successor:while:clean:end}). It then checks whether or not the successor key is still needed by invoking \NeedSuccessorKey{} function (\lineref{local-seek|successor:while:need|successor}). The function \NeedSuccessorKey{} returns a reference, which is null if the successor key is no longer needed and contains the address of the target node's right child otherwise. This address is used as a traversal point if the stack only contains a single entry (the node whose key needs to be replaced). If the successor key is still needed, then the function repeatedly follows the left child pointer until it encounters a null pointer (\linesref{local-seek|successor:while:traversal:begin}{local-seek|successor:while:traversal:end}). While moving down the tree, the function also populates the traversal stack (\lineref{local-seek|successor:while:traversal:stack}).
\begin{limitscope}

%% To limit the scope of the macros defined below

%% macros for pseudocode

\newcommand{\child}{child}
\newcommand{\node}{node}
\newcommand{\parent}{parent}

\newcommand{\mainSeekRecord}{seekTargetKey}
\newcommand{\successorSeekRecord}{seekSuccessorKey}


\newcommand{\targetStack}{targetStack}
\newcommand{\successorStack}{successorStack}


\newcommand{\successorStackInUse}{successorStackInUse}
\newcommand{\targetNode}{targetNode}




\newcommand{\key}{key}

\newcommand{\done}{done}
\newcommand{\result}{result}
\newcommand{\status}{status}
\newcommand{\restart}{restart}





\newcommand{\cKey}{key}
\newcommand{\nKey}{key}
\newcommand{\cNode}{current}
\newcommand{\pNode}{parent}
\newcommand{\nMarked}{marked}



\newcommand{\which}{which}
\newcommand{\address}{address}

\newcommand{\anchor}{anchor}

\newcommand{\stack}{stack}
\newcommand{\sTop}{top}
\newcommand{\sBottom}{bottom}
\newcommand{\current}{current}

\remove{
\newcommand{\traversalRecord}{state}
\newcommand{\TraversalRecord}{State}
\newcommand{\opRecord}{opRecord}
\newcommand{\OpRecord}{OpRecord}
\newcommand{\seekRecord}{seekRecord}
\newcommand{\SeekRecord}{SeekRecord}
}

\newcommand{\admissible}{admissible}
\newcommand{\critical}{critical}
\newcommand{\reference}{re\!f\!erence}

%% \newcommand{\OptReturn}[1][]{\Return #1\;}
\newcommand{\OptReturn}[1][]{}

\newcommand{\injectionPoint}{injectionPoint}



\newcommand{\Search}{\textsc{Search}}
\newcommand{\Insert}{\textsc{Insert}}
\newcommand{\Delete}{\textsc{Delete}}
\newcommand{\Seek}{\textsc{Seek}}

\newcommand{\Inject}{\textsc{Inject}}


%%
\newcommand{\WFSeekForSearchBOSize}{\textsc{WFSeekForSearchBasedOnSize}}
\newcommand{\WFSeekForSearchBOHeight}{\textsc{WFSeekForSearchBasedOnHeight}}
%%
\newcommand{\WFTraverseTreeBOCount}{\textsc{TraverseBasedOnCount}}
\newcommand{\WFTraverseTreeBOTimeStamp}{\textsc{TraverseBasedOnTimeStamp}}
%%
\remove{
\newcommand{\SeekForSuccessor}{\textsc{SeekForSuccessor}}
\newcommand{\NeedSuccessorKey}{\textsc{NeedSuccessorKey}}
\newcommand{\GetChild}{\textsc{GetChild}}
\newcommand{\Move}{\textsc{Move}}
\newcommand{\GetAddress}{\textsc{GetAddress}}
\newcommand{\IsNull}{\textsc{IsNull}}
\newcommand{\PopulateSeekRecord}{\textsc{PopulateSeekRecord}}
}



\newcommand{\mline}[1]{\DontPrintSemicolon #1 \PrintSemicolon}


\newcommand{\LEFT}{\textsf{LEFT}}
\newcommand{\RIGHT}{\textsf{RIGHT}}


\newcommand{\rarrow}{\!\rightarrow\!}
\newcommand{\type}{type}
\newcommand{\limit}{limit}


\newcommand{\SEARCH}{\textsf{SEARCH}}
\newcommand{\INSERT}{\textsf{INSERT}}
\newcommand{\DELETE}{\textsf{DELETE}}

\newcommand{\STOPFOUND}{\textsf{FOUND}}
\newcommand{\STOPNOTFOUND}{\textsf{NOT\_FOUND}}
\newcommand{\ADMISSIBLE}{\textsf{SAFE}}
\newcommand{\INADMISSIBLE}{\textsf{NOT\_SAFE}}

\newcommand{\TARGETSTACK}{\textsf{TARGET\_STACK}}
\newcommand{\SUCCESSORSTACK}{\textsf{SUCCESSOR\_STACK}}

%%%%%%%%%%%%%%%%%%%%%%%%%%%%%%%%%%%%%%%%%%%%%%%%%%%%%%%%%%%%%%%%%%%%%%%%%%%%%%%%%%%%

\newcommand{\DefineKeyWords}{
%%
\SetKw{Boolean}{boolean}
\SetKw{Integer}{integer}
\SetKw{LAnd}{~and~}
\SetKw{LOr}{~or~}
\SetKw{LNot}{not}
\SetKw{Struct}{struct}
\SetKw{Null}{null}
\SetKw{True}{true}
\SetKw{False}{false}
\SetKw{Break}{break}
\SetKw{Continue}{continue}
\SetKw{Enum}{enum}
\SetKw{Word}{word}
%%
}

%%%%%%%%%%%%%%%%%%%%%%%%%%%%%%%%%%%%%%%%%%%%%%%%%%%%%%%%%%%%%%%%%%%%%%%%%%%%%%%%%%%%%

%% Seek for successor key

%%%%%%%%%%%%%%%%%%%%%%%%%%%%%%%%%%%%%%%%%%%%%%%%%%%%%%%%%%%%%%%%%%%%%%%%%%%%%%%%%%%%%



\begin{algorithm}[tbh]
\caption{Seek Function for Successor Key} 
\label{algo:seek:successor}
%%
\DefineKeyWords
%%
\tcp{Looks for the next largest key with respect to a given key}
\DontPrintSemicolon
\Boolean \SeekForSuccessor( $\opRecord$, $\seekRecord$ )\;
\PrintSemicolon
\label{lin:local-seek|successor:begin}
\Begin
{
%%
   
   \tcp{the stack used in locating the successor key is initialized}
	 $\traversalRecord$ := $\opRecord \rarrow \successorStack$\;
	 \While{\True}
	 {
	
	    \label{lin:local-seek|successor:while:begin}
		  \tcp{backtrack until either an unmarked node or the stack becomes empty}
			
	    
			\While{(\Size( $\traversalRecord$ ) $>$ 1)}      
			{
			    \label{lin:local-seek|successor:while:backtrack:begin}
			    $\cNode$ := \GetTop( $\traversalRecord$ )\;
					\lIf{\LNot(\IsMarked( $\cNode$ ))}
					{
					   \Break
					} 
					\lElse
					{
					   \RemoveFromTop( $\traversalRecord$ )
					}
					\label{lin:local-seek|successor:while:backtrack:end}
			}
	
	    \BlankLine
			
			\tcp{backtrack further if a clean parent is needed but the parent is not clean}
			\If{(\Size( $\traversalRecord$ ) $\geq$ 2)}
			{
			   \label{lin:local-seek|successor:while:clean:begin}
			   \If{\NeedCleanParentNode( $\opRecord$, $\cNode$ )}
		     {
				    
			      \tcp{the parent node should be a clean node}
				    $\pNode$ := \GetSecondToTop( $\traversalRecord$ )\;
				    \If{\LNot(\IsClean( $\pNode$ ))}
				    {
				       \RemoveFromTop( $\traversalRecord$ )\;
					     \Continue\;
							 \label{lin:local-seek|successor:while:clean:end}
				    }
			   }
				
			}
			
			%\BlankLine
			
	    \tcp{check if the successor key is still needed}
	    $\reference$ := \NeedSuccessorKey( $\opRecord$ )\;
			\label{lin:local-seek|successor:while:need|successor}
			\If(\tcp*[f]{successor key no longer required}){\IsNull( $\reference$ )}{ 
			   \Return \false\;
			}
					
			$\cNode$ := \GetTop( $\traversalRecord$ )\;
			\uIf{(\Size( $\traversalRecord$ ) = 1)}
			{
			   \label{lin:local-seek|successor:while:traversal:if:begin}
			   %% $\address$ := \GetAddress( $\reference$ )\;
				 \tcp{visit the node pointed to by the reference returned by \NeedSuccessorKey{} function}
			   $\which$ := \RIGHT\;
				 \label{lin:local-seek|successor:while:traversal:if:end}
			}
			\Else(\tcp*[f]{follow the left child node of the top node, if it exists})
			{
			   \label{lin:local-seek|successor:while:traversal:else:begin}
		 	   $\reference$ := \GetChild( $\cNode$, \LEFT{} )\;
			   %% \lIf{\IsNull( $\reference$ )}{ \Break }
				 %% $\address$ := \GetAddress( $\reference$ )\;
			   $\which$ := \LEFT\;
				 \label{lin:local-seek|successor:while:traversal:else:end}
			}
			%\BlankLine
			\Repeat(\tcp*[f]{stop if reference is null}){\True}
	    {
			   \label{lin:local-seek|successor:while:traversal:begin}
				 \lIf{\IsNull( $\reference$ )}{ \Break }
				
				 \tcp{obtain the address of the node}
				 $\address$ := \GetAddress( $\reference$ )\;	
				
				 \tcp{traverse the edge}
				 $\restart$ := \Move( $\cNode$, $\address$, $\which$ )\;
				 \label{lin:local-seek|successor:while:traversal:move}
			   \If(\tcp*[f]{the algorithm wants to restart the traversal}){$\restart$}
			   {
				    \Break\;
						\label{lin:local-seek|successor:while:traversal:restart}
			   }  
				 
			   \tcp{push the node visited into the stack}
				 \AddToTop( $\traversalRecord$, $\address$, $\which$ )\;
				 \label{lin:local-seek|successor:while:traversal:stack}
				 \label{lin:local-seek|successor:while:traversal:advance:begin}
			   $\cNode$ := $\address$\;
				 \tcp{determine the next node to be visited}
			   $\reference$ := \GetChild( $\cNode$, \LEFT{} )\;
			   $\which$ := \LEFT{}\;
				 \label{lin:local-seek|successor:while:traversal:advance:end}
				 \label{lin:local-seek|successor:while:traversal:end}
			}
			\label{lin:local-seek|successor:while:end}
	 }
	
	 \tcp{return the outcome}
	 \PopulateSeekRecord( $\seekRecord$, $\opRecord$ )\;
	 \Return \True;
	 \label{lin:local-seek|successor:end}
%%
}
%%
\end{algorithm}
\end{limitscope}

\subsection{Formal Description}

We refer to our algorithm as \CASTLE{} (\underline{C}oncurrent \underline{A}lgorithm for Binary \underline{S}earch \underline{T}ree by \underline{L}ocking \underline{E}dges). 

\begin{limitscope}

%% To limit the scope of the macros defined below

%% macros for pseudocode
\newcommand{\leftChild}{le\!f\!t}
\newcommand{\rightChild}{right}
\newcommand{\child}{child}
\newcommand{\canReplace}{readyToReplace}
\newcommand{\markAndKey}{mKey}

\newcommand{\node}{node}
\newcommand{\parent}{parent}

\newcommand{\terminalEdge}{lastEdge}
\newcommand{\targetEdge}{targetEdge}
\newcommand{\parentTargetEdge}{pTargetEdge}
\newcommand{\successorEdge}{successorEdge}
\newcommand{\parentSuccessorEdge}{pSuccessorEdge}
\newcommand{\injectionEdge}{injectionEdge}
\newcommand{\penultimateEdge}{pLastEdge}

\newcommand{\targetKey}{targetKey}
\newcommand{\currentKey}{currentKey}

\newcommand{\newNode}{newNode}
\newcommand{\reference}{re\!f\!erence}
\newcommand{\state}{state}

\newcommand{\StateRecord}{StateRecord}
\newcommand{\AnchorRecord}{AnchorRecord}

\newcommand{\mline}[1]{\DontPrintSemicolon #1 \PrintSemicolon}

\newcommand{\prev}{prev}
\newcommand{\curr}{curr}

\newcommand{\prevSeekRecord}{pSeekRecord}
\newcommand{\prevAnchorRecord}{pAnchorRecord}
%% \newcommand{\currSeekRecord}{cSeekRecord}
\newcommand{\anchorRecord}{anchorRecord}

\newcommand{\oldContents}{oldValue}
\newcommand{\newContents}{newValue}

\newcommand{\INJECTION}{\textsf{INJECTION}}
\newcommand{\DISCOVERY}{\textsf{DISCOVERY}}
\newcommand{\CLEANUP}{\textsf{CLEANUP}}
\newcommand{\FINISHED}{\textsf{FINISHED}}

\newcommand{\DELETEFLAG}{\textsf{DELETE\_FLAG}}
\newcommand{\PROMOTEFLAG}{\textsf{PROMOTE\_FLAG}}
\newcommand{\INTENTFLAG}{\textsf{INTENT\_FLAG}}
\newcommand{\flag}{f\!lag}

\newcommand{\COMPLEX}{\textsf{COMPLEX}}
\newcommand{\SIMPLE}{\textsf{SIMPLE}}

\newcommand{\LEFT}{\textsf{LEFT}}
\newcommand{\RIGHT}{\textsf{RIGHT}}

\newcommand{\targetSeekRecord}{targetRecord}
\newcommand{\successorSeekRecord}{successorRecord}

\newcommand{\dFlag}{d}
\newcommand{\iFlag}{i}
\newcommand{\pFlag}{p}
\newcommand{\nFlag}{n}
\newcommand{\mFlag}{m}
\newcommand{\lNFlag}{lN}
\newcommand{\rNFlag}{rN}

\newcommand{\rarrow}{\!\rightarrow\!}


%%%%%%%%%%%%%%%%%%%%%%%%%%%%%%%%%%%%%%%%%%%%%%%%%%%%%%%%%%%%%%%%%%%%%%%%%%%%%%%%%%%%

\newcommand{\DefineKeyWords}{
%%
\SetKw{Boolean}{Boolean}
\SetKw{LAnd}{~and~}
\SetKw{LOr}{~or~}
\SetKw{LNot}{not}
\SetKw{Struct}{struct}
\SetKw{Null}{null}
\SetKw{True}{true}
\SetKw{False}{false}
\SetKw{Break}{break}
\SetKw{Continue}{continue}
\SetKw{Enum}{enum}
%%
}

%%%%%%%%%%%%%%%%%%%%%%%%%%%%%%%%%%%%%%%%%%%%%%%%%%%%%%%%%%%%%%%%%%%%%%%%%%%%%%%%%%%%%

%% DATA STRUCTURES


\begin{algorithm}[htp]
%%
\DefineKeyWords
%%

%% define data structures used in the algorithm

\DontPrintSemicolon
\Struct Node \{\;
\label{ln:icdcn-node|begin}
\PrintSemicolon
\Indp 
   $\{ \Boolean, \text{Key} \}$ $\markAndKey$\;
   $\{ \Boolean, \Boolean, \Boolean, \Boolean, \text{NodePtr} \}$ $\child[2]$\;
   \Boolean $\canReplace$\;
\Indm
\}\;
\label{ln:icdcn-node|end}
%%
\BlankLine

\DontPrintSemicolon
\Struct Edge \{\;
\label{ln:icdcn-edge|begin}
\PrintSemicolon
\Indp 
   %% NodePtr $\parent$\;
   %% NodePtr $\child$\;
	 NodePtr $\parent$, $\child$\;
   \Enum $which$ \{ \LEFT{}, \RIGHT{} \}\;
\Indm
\}\;
\label{ln:icdcn-edge|end}
%%
\BlankLine

\DontPrintSemicolon
\Struct SeekRecord \{\;
\PrintSemicolon
\Indp 
%%
   %% Edge $\terminalEdge$\;
%%
   %% Edge $\penultimateEdge$\;
%%
   %% Edge $\injectionEdge$\;
   Edge $\terminalEdge$, $\penultimateEdge$, $\injectionEdge$\;
\Indm
\}\;
%%
\BlankLine


\BlankLine
\DontPrintSemicolon
\Struct \AnchorRecord{} \{\;
\PrintSemicolon
\Indp 
   NodePtr $\node$\;
   Key $key$\;
\Indm
\}\;
%%

\BlankLine
\DontPrintSemicolon
\Struct \StateRecord{} \{\;
\PrintSemicolon
\Indp 
%%
   %% int $depth$\;
   %% Edge $\targetEdge$\;
	 %% Edge $\parentTargetEdge$\;
	 Edge $\targetEdge$, $\parentTargetEdge$\;
%%
   %% Key $\targetKey$\;
	 %% Key $\currentKey$\;
	 Key $\targetKey$, $\currentKey$\;
   \Enum $mode$ \{ \INJECTION{}, \DISCOVERY{}, \CLEANUP{} \}\;
   \Enum $type$ \{ \SIMPLE{}, \COMPLEX{} \} \;
%%
   \tcp{the next field stores pointer to a seek record; it is used for finding the successor if the delete operation is complex}
   SeekRecordPtr $\successorSeekRecord$\; 
\Indm
\}\;
%%
\BlankLine
\tcp{object to store information about the tree traversal when looking for a given key (used by the seek function)}
SeekRecordPtr $\targetSeekRecord$ := new seek record\;
\tcp{object to store information about process' own delete operation}
\StateRecord{Ptr} $myState$ := new state\;


\caption{Data Structures Used}
\label{algo:icdcn-data|structures}
\end{algorithm}



\begin{algorithm}[htp]
%%
\DefineKeyWords


%% SEEK


%%
%% traverses the tree from the root node to a leaf node looking for a given key
%%
\DontPrintSemicolon
\Seek( $key$, $seekRecord$ )\;
\PrintSemicolon
\Begin
{
   $\prevAnchorRecord$ := $\curly{ \snodetwo{}, \skey{1} }$\;
   \While{\True}
   {
	    \tcp{initialize all variables used in traversal}
		  $\penultimateEdge$ := $\curly{ \snodeone, \snodetwo, \RIGHT }$; \qquad
			$\terminalEdge$ := $\curly{ \snodetwo, \snodethree, \RIGHT }$\;
			$\curr$ := $\snodethree$; \qquad
			$\anchorRecord$ := $\curly{ \snodetwo{}, \skey{1} }$\;
			\BlankLine
			\While{\True}
			{
			    \tcp{read the key stored in the current node}
			    $\ang{ \ast, cKey }$ := $\curr \rarrow \markAndKey$\;
				  \tcp{find the next edge to follow}
					$which$ := $key < cKey$ ? \LEFT : \RIGHT\;
				  $\ang{ \nFlag, \ast, \dFlag, \pFlag, next }$ := $\curr \rarrow \child[which]$\;
					\tcp{check for the completion of the traversal}
				  \If{$key = cKey$ \LOr $\nFlag$}
				  {
				     \tcp{either key found or no next edge to follow; stop the traversal}
						 $seekRecord \rarrow \penultimateEdge$ := $\penultimateEdge$\;
						 $seekRecord \rarrow \terminalEdge$ := $\terminalEdge$\;
						 $seekRecord \rarrow \injectionEdge$ := $\curly{ \curr, next, which }$\;
						 \BlankLine
						 \uIf(\tcp*[h]{keys match}){$key = cKey$}
						 { 
						    \Return\;
						 }
						 \lElse { \Break }
				  }
				  \BlankLine			   
				  \If{$which$ = \RIGHT}
				  {
				     \tcp{next edge to be traversed is a right edge; keep track of the current node and its key}
						 $\anchorRecord$ := $\ang{ \curr, cKey }$\;
				  }	
				  \BlankLine
				  \tcp{traverse the next edge}
					$\penultimateEdge$ := $\terminalEdge$; \qquad
					$\terminalEdge$ := $\curly{ \curr, next, which }$; \qquad
				  $\curr$ := $next$\; 
		  }
		  \tcp{key was not found; check if can stop}
		  $\ang{ \ast, \ast, \dFlag, \pFlag, \ast }$ := $\anchorRecord.\node \rarrow \child[\RIGHT]$\;			
			\uIf{\LNot($\dFlag$) \LAnd \LNot($\pFlag$)}
			{
			   \tcp{anchor node is still part of the tree; check if anchor node's key has changed}
				 $\ang{ \ast, aKey }$ := $\anchorRecord.\node \rarrow \markAndKey$\;
				 \lIf{$\anchorRecord.key$ = $aKey$}
			   {  
				    \Return
				 } 
			}	
			\Else
			{ 
			   \tcp{check if the anchor record (the node and its key) matches that of the previous traversal}
			   \If{$\prevAnchorRecord = \anchorRecord$}
			   {
				    \tcp{return the results of the previous traversal}
					  $seekRecord$ := $\prevSeekRecord$\;
				    \Return\;
		     }
			}
			\tcp{store the results of the traversal and restart}
			$\prevSeekRecord$ := $seekRecord$; \qquad
			$\prevAnchorRecord$ := $\anchorRecord$;					
   }
} 
%% End of seek function
\caption{Seek Function}
\label{algo:icdcn-seek}
\end{algorithm}


%% SEARCH
\begin{algorithm}[htp]
%%
\DefineKeyWords
\DontPrintSemicolon
\Boolean \Search( $key$ )\;
\PrintSemicolon
\Begin
{
   \Seek( $key$, $mySeekRecord$ )\;
	 \BlankLine
	 %% $\node$ := $mySeekRecord \rarrow \node$\;
   $\node$ := $mySeekRecord \rarrow \terminalEdge.\child$\;
   $\ang{ \ast , nKey }$ := $\node \rarrow \markAndKey$\;
	 \BlankLine
   \lIf{nKey = key}{\Return \True}
   \lElse{\Return \False}
}
\caption{Search Operation}
\label{algo:icdcn-search}
\end{algorithm}

%% INSERT
\begin{algorithm}[htp]
%%
\DefineKeyWords
\DontPrintSemicolon
\Boolean \Insert( $key$ )\;
\PrintSemicolon
\Begin
{
   \While{\True}
	 {
      \Seek( $key$, $\targetSeekRecord$ )\;
			\BlankLine
			$\targetEdge$ := $\targetSeekRecord \rarrow \terminalEdge$\;
			$\node$ := $\targetEdge.\child$\;
			$\ang{ \ast , nKey }$ := $\node \rarrow \markAndKey$\; 
			\lIf{$key = nKey$}{\Return \False}
			\BlankLine
			\tcp{create a new node and initialize its fields}
			$\newNode$ := create a new node\;
			$\newNode \rarrow \markAndKey$ := $\ang{ 0_m, key }$\;
			$\newNode \rarrow \child[\LEFT]$ := $\ang{ 1_n, 0_i, 0_d, 0_p, \Null }$\;
			$\newNode \rarrow \child[\RIGHT]$ := $\ang{ 1_n, 0_i, 0_d, 0_p, \Null }$\;
			$\newNode \rarrow \canReplace$ := \False\;
			\BlankLine
			$which$ := $\targetSeekRecord \rarrow \injectionEdge.which$\;
			$address$ := $\targetSeekRecord \rarrow \injectionEdge.\child$\;
			$result$ := \CAS($\node \rarrow \child[which]$, $\ang{ 1_n, 0_i, 0_d, 0_p, address }$, $\ang{ 0_n, 0_i, 0_d, 0_p, \newNode }$)\;
			\lIf{$result$}{\Return \True}
			\BlankLine	
			\tcp{help if needed}
		  $\ang{ \ast, \ast, \dFlag, \pFlag, \ast }$ := $\node \rarrow \child[which]$\;
			\lIf{$\dFlag$}
			{
			   \HelpTargetNode( $\targetEdge$ )
			} 
			\lElseIf{$p$}
			{  
			   \HelpSuccessorNode( $\targetEdge$ )
			}
	}
}
\caption{Insert Operation}
\label{algo:icdcn-insert}
\end{algorithm}

%% DELETE
\begin{algorithm}[htp]
\DefineKeyWords
\DontPrintSemicolon
\Boolean \Delete( $key$ )\;
\PrintSemicolon
\Begin
{
   \tcp{initialize the state record}
 	 $myState \rarrow \targetKey$ := $key$; $\qquad$
	 $myState \rarrow \currentKey$ := $key$\;
	 $myState \rarrow mode$ := \INJECTION\;
	 \BlankLine
   \While{\True}
	 {
      \Seek( $myState \rarrow \currentKey$, $\targetSeekRecord$ )\;
			$\targetEdge$ := $\targetSeekRecord \rarrow \terminalEdge$; $\qquad$
			$\parentTargetEdge$ := $\targetSeekRecord \rarrow \penultimateEdge$\;
			$\ang{ \ast , nKey }$ := $\targetEdge.\child \rarrow \markAndKey$\; 
			\BlankLine
			\If{$myState \rarrow \currentKey \neq nKey$}
			{
			   \tcp{the key does not exist in the tree}
			   \lIf{$myState \rarrow mode$ = \INJECTION}{\Return \False}
				 \lElse{\Return \True}
			}
 	    \BlankLine		   	
			\tcp{perform appropriate action depending on the mode}
	    \If{$myState \rarrow mode$ = \INJECTION}
		  {
				 \tcp{store a reference to the target edge}
   	     $myState \rarrow \targetEdge$ := $\targetEdge$\;
   	     $myState \rarrow \parentTargetEdge$ := $\parentTargetEdge$\;
				 \tcp{attempt to inject the operation at the node}
				 %% $result$ := \Inject( $myState$ )\;
				 \Inject( $myState$ )\;					 								 
			}
			\BlankLine
			\tcp{mode would have changed if injection was successful}
				 
			\If{$myState \rarrow mode \neq$ \INJECTION}
			{
				 \tcp{check if the target node found by the seek function matches the one stored in the state record}			
			   %%\If{$\left(\text{\parbox[c]{1.75in}{$myState \rarrow \targetEdge.\child$ $\neq$  \\ \mbox{}\hfill$\targetEdge.\child$}}\right)$}
				 \lIf{$myState \rarrow \targetEdge.\child$ $\neq$ $\targetEdge.\child$}
				 {
				    \Return \True
				 }						
				 \tcp{update the target edge information using the most recent seek}
				 $myState \rarrow \targetEdge$ := $\targetEdge$\; 			 				
		  }				
			\BlankLine							
			\If{$myState \rarrow mode$ = \DISCOVERY}
			{
				 \tcp{complex delete operation; locate the successor node and mark its child edges with promote flag} 
			   \FindAndMarkSuccessor( $myState$ )\;			 
			}			
			\If{$myState \rarrow mode$ = \DISCOVERY}
			{
				 \tcp{complex delete operation; promote the successor node's key and remove the successor node}
		     \RemoveSuccessor( $myState$ )\;						
			}			
			\BlankLine				
			\If{$myState \rarrow mode$ = \CLEANUP}
			{
			   \tcp{either remove the target node (simple delete) or replace it with a new node with all fields unmarked  (complex delete)}
			   $result$ := \Cleanup( $myState$ )\;
				 \lIf{$result$}{\Return \True}
				 \Else{
				    $\ang{ \ast, nKey }$ := $\targetEdge.\child \rarrow \markAndKey$\;
						$myState \rarrow \currentKey$ := $nKey$\;
				 }					
		  }
	 }	
   %% \Return\;
}
\caption{Delete Operation}
\label{algo:icdcn-delete}
\end{algorithm}

%% INJECT
\begin{algorithm}[htp]
%%
\DefineKeyWords
\DontPrintSemicolon
\Inject( $\state$ )\;
\PrintSemicolon
\Begin
{
   $\targetEdge$ := $\state \rarrow \targetEdge$\;
	 \tcp{try to set the intent flag on the target edge}
	 \tcp{retrieve attributes of the target edge}
	 $\parent$ := $\targetEdge.\parent$\;
	 $\node$ := $\targetEdge.\child$\;
	 $which$ := $\targetEdge.which$\;
	 \BlankLine
	 \mline{$result$ := \CAS( \parbox[t]{2.075in}{$\parent \rarrow \child[which]$, \\ $\ang{ 0_n, 0_i, 0_d, 0_p, \node }$,  $\ang{ 0_n, 1_i, 0_d, 0_p, \node }$ );}\;}
	 \If{\LNot($result$)}
	 {
	    \tcp{unable to set the intent flag; help if needed}
			$\ang{ \ast, \iFlag, \dFlag, \pFlag, address }$ := $\parent \rarrow \child[which]$\;
			\lIf{$\iFlag$}
			{
			   \HelpTargetNode( $\targetEdge$ )
			} 
			\uElseIf{$\dFlag$}
			{
			   \HelpTargetNode( $\state \rarrow \parentTargetEdge$ )\;
			} 
			\ElseIf{$\pFlag$}
			{
			   \HelpSuccessorNode( $\state \rarrow \parentTargetEdge$ )\;
			}

      \Return;					
	 }

   \BlankLine
	 \tcp{mark the left edge for deletion}

	 $result$ := \MarkChildEdge( $\state$, \LEFT{} )\;
	 
	 \lIf{\LNot($result$)}
	 {
	    \Return
	 } 
	 \tcp{mark the right edge for deletion; cannot fail}
	 \MarkChildEdge( $\state$, \RIGHT{} )\;
	   
	 \BlankLine
	 \tcp{initialize the type and mode of the operation}
	 \InitializeTypeAndUpdateMode( $\state$ );	
}

\caption{Injecting a Deletion Operation}
\label{algo:icdcn-inject}
\end{algorithm}









%% FINDANDMARKSUCCESSOR


\begin{algorithm}[htp]
%%
\DefineKeyWords

\DontPrintSemicolon
\FindAndMarkSuccessor( $\state$ )\;
\PrintSemicolon
\Begin
{
   \tcp{retrieve the addresses from the state record}
   $\node := \state \rarrow \targetEdge.\child$\;
	 $seekRecord$ := $\state \rarrow \successorSeekRecord$\; 
   
	 \BlankLine
   \While{\True}
	 {
	 
	    \tcp{read the mark flag of the key in the target node}  
	    $\ang{ \mFlag, \ast}$ := $\node \rarrow \markAndKey$\; 
	    

	  	\tcp{find the node with the smallest key in the right subtree}
	    $result$ := \FindSmallest( $\state$ )\;
			
						
			\BlankLine
			\If{$\mFlag$ \LOr \LNot($result$)} 
			{
			   \tcp{successor node had already been selected \emph{before} the traversal or the right subtree is empty}
				 \Break\;
			}
			
				
			\tcp{retrieve the information from the seek record}
			$\successorEdge$ := $seekRecord \rarrow \terminalEdge$\;
			%% $\ang{ \nFlag, \ast, \ast, \ast, \leftChild}$ :=  $\successorEdge.\child \rarrow \child[\LEFT]$\;
			%% \lIf{\LNot($\nFlag$)}{ \Continue }
			$\leftChild$ := $seekRecord \rarrow \injectionEdge.\child$\;
			
			\BlankLine
			\tcp{read the mark flag of the key under deletion}
      $\ang{ \mFlag, \ast}$ := $\node \rarrow \markAndKey$\;
			
			\If(\tcp*[h]{successor node has already been selected}){$\mFlag$}
			{
			   %% \tcp{successor node has already been selected}
			   %% \lIf{$p$}{ \Break }
				 %% \lElse{ \Continue }
				 \Continue\;
				 
			}
			
			
		


          
			\tcp{try to set the promote flag on the left edge}
			\mline{$result$ := \CAS( \parbox[t]{1.875in}{$\successorEdge.\child \rarrow \child[\LEFT]$, \\ 
			                                             $\ang{ 1_n, 0_i, 0_d, 0_p, \leftChild }$, \\ $\ang{ 1_n, 0_i, 0_d, 1_p, \node }$ );}\;}
			
			\lIf{$result$}{\Break}
			
			\BlankLine
			\tcp{attempt to mark the edge failed; recover from the failure and retry if needed}
			%% $\ang{ n, \ast, d, p, \ast }$ := $\successorEdge.\child \rarrow \child[\LEFT]$\;
			$\ang{ \nFlag, \ast, \dFlag, \ast, \ast }$ := $\successorEdge.\child \rarrow \child[\LEFT]$\;
			
			
      %% \lIf{$p$}
      %% {
      %%    \Break
      %% }   

      %% \If{\LNot($n$)}
			%% { 
			%%   \tcp{the node found has since gained a left child}
			%%   \Continue\;
			%% }

			\If{$\nFlag$ \LAnd $\dFlag$}
			{
			    \tcp{the node found is undergoing deletion; need to help}
					
								
					%% \mline{\HelpTargetNode( \parbox[t]{1.5in}{$\successorEdge$, \\ $\state \rarrow depth + 1$ );}\;}
		      \HelpTargetNode( $\successorEdge$ )\;
       } 
	 }	
   \BlankLine
   \tcp{update the operation mode}
	 \UpdateMode( $\state$ );
}

\caption{Locating the Successor Node}
\label{algo:icdcn-findandmark}
\end{algorithm}




%% REMOVESUCCESSOR
\begin{algorithm}[htp]
%%
\DefineKeyWords
\DontPrintSemicolon
\RemoveSuccessor( $\state$ )\;
\PrintSemicolon
\Begin
{
   \tcp{retrieve addresses from the state record}
   $\node$ := $\state \rarrow \targetEdge.\child$\;
   $seekRecord$ := $\state \rarrow \successorSeekRecord$\;
   \tcp{extract information about the successor node}
	 %% \tcp{assumes that the state's seek record contains valid information}
   $\successorEdge$ := $seekRecord \rarrow \terminalEdge$\;
	 \BlankLine
	 \tcp{ascertain that seek record for successor node contains valid information}
	 $\ang{ \ast, \ast, \ast, \pFlag, address }$ := $\successorEdge.\child \rarrow \child[\LEFT]$\;
	 \If{\LNot($\pFlag$) \LOr ($address$ $\neq$ $\node$)}
	 {
	    $\node \rarrow \canReplace$ := \True\;
			\UpdateMode( $\state$ )\;
	    \Return\;
	 }
   \BlankLine
   \tcp{mark the right edge for promotion if unmarked}
   \MarkChildEdge( $\state$, \RIGHT{} )\; 
   \BlankLine
   \tcp{promote the key}
   $\node \rarrow \markAndKey$ := $\ang{ 1_m, \successorEdge.\child \rarrow \markAndKey }$\;
   \While{\True}
   {
      \tcp{check if the successor is the right child of the target node itself}
	    \uIf{$\successorEdge.\parent$ = $\node$}
	    {
	       \tcp{need to modify the right edge of target node whose delete flag is set}
				 $dFlag$ := 1; \qquad
			   $which$ := \RIGHT\;
	    }
	    \Else
	    {
			   $dFlag$ := 0; \qquad
			   $which$ := \LEFT\;
	    }
      $\ang{ \ast, \iFlag, \ast, \ast, \ast }$ := $\successorEdge.\parent \rarrow \child[which]$\;			
      \BlankLine			
	    $\ang{ \nFlag, \ast, \ast, \ast, \rightChild }$ := $\successorEdge.\child \rarrow \child[\RIGHT]$\;	
	    $\oldContents$ := $\ang{ 0_n, \iFlag, dFlag, 0_p, \successorEdge.\child }$\;	    
			\uIf(\tcp*[f]{only set the null flag; do not change the address}){$\nFlag$}
	    {				
				 %\mline{\parbox[t]{1.75in}{$\newContents$ := \\ \mbox{} \qquad $\ang{ 1_n, 0_i, dFlag, 0_p,  \successorEdge.\child }$;}}
				 $\newContents$ := $\ang{ 1_n, 0_i, dFlag, 0_p,  \successorEdge.\child }$\;
	    }
	    \Else(\tcp*[f]{switch the pointer to point to the grand child})
	    {	 				 
				 $\newContents$ := $\ang{ 0_n, 0_i, dFlag, 0_p, \rightChild }$ \;		 
	    }	 
      \remove{ \lIf{$result$}{\Break} }			
			%\mline{$result$ := \CAS( \parbox[t]{1.77in}{$\successorEdge.\parent \rarrow \child[which]$, \\ $\oldContents$, $\newContents$ );}\;}
			$result$ := \CAS($\successorEdge.\parent \rarrow \child[which]$,$\oldContents$, $\newContents$)\;
			\lIf{$result$ \LOr $dFlag$}{ \Break }
	    %\BlankLine			
			$\ang{ \ast, \ast, \dFlag, \ast, \ast }$ := $\successorEdge.\parent \rarrow \child[which]$\;
			$\penultimateEdge$ := $seekRecord \rarrow \penultimateEdge$\;
			\If{$\dFlag$ \LAnd ($\penultimateEdge.\parent$ $\neq$ \Null)}
			{
			   %% \mline{\HelpTargetNode( \parbox[t]{1.25in}{$\penultimateEdge$, \\ $\state \rarrow depth + 1$ );}\;}
				 \HelpTargetNode( $\penultimateEdge$ )\;
			}			
      \BlankLine			
 	    $result$ := \FindSmallest( $\state$ )\;
			$\terminalEdge$ := $seekRecord \rarrow \terminalEdge$\;
	    %\If{$\left(\text{\parbox[c]{1.875in}{\LNot($result$) \LOr \\ $\terminalEdge.\child$ $\neq$ $\successorEdge.\child$}}\right)$}
			\If{\LNot($result$) \LOr $\terminalEdge.\child$ $\neq$ $\successorEdge.\child$}
			{
			   \Break;
				 \tcp*[f]{the successor node has already been removed}
			} 
			\lElse
			{
			   $\successorEdge$ := $seekRecord \rarrow \terminalEdge$
			}
   }
   \BlankLine
	 $\node \rarrow \canReplace$ := \True\;
   \UpdateMode( $\state$ )\;	
}
\caption{Removing the Successor Node}
\label{algo:icdcn-remove}
\end{algorithm}



%% CLEANUP

\begin{algorithm}[htp]
%%
\DefineKeyWords



\DontPrintSemicolon
\Boolean \Cleanup( $\state$ )\;
\PrintSemicolon
\Begin
{
   %% \tcp{retrieve the addresses from the state record}
   %% $\node$ := $\state \rarrow \node$\;
	 %% $\parent$ := $\state \rarrow \parent$\;
	
	 %% \BlankLine
	
	 %% \tcp{determine which edge of the parent needs to be switched} 
	 %% $\ang{ \ast, pKey }$ := $\parent \rarrow \markAndKey$\;
	 %% $\ang{ \ast, nKey }$ := $\node \rarrow \markAndKey$\;
	 %% $pWhich$ := $nKey < pKey$ ? \LEFT : \RIGHT\;
	 $\ang{\parent, \node, pWhich}$ := $\state \rarrow \targetEdge$\;
	 
	
	 \BlankLine
	 
	 \uIf{$\state \rarrow type$ = \COMPLEX}
	 {
	  	   
	    \tcp{replace the node with a new copy in which all fields are unmarked} 
			$\ang{ \ast, nKey }$ := $\node \rarrow \markAndKey$\;
			$newNode \rarrow \markAndKey$ := $\ang{ 0_m, nKey }$\;		
			\BlankLine
			\tcp{initialize left and right child pointers}
  		$\ang{ \ast, \ast, \ast, \ast, \leftChild }$ := $\node \rarrow \child[\LEFT]$\;
			$\newNode \rarrow \child[\LEFT]$  := $\ang{ 0_n, 0_i, 0_d, 0_p, \leftChild }$\;
			$\ang{ \nFlag, \ast, \ast, \ast, \rightChild }$ := $\node \rarrow \child[\RIGHT]$\;
			\uIf{$\nFlag$}
			{
			  $\newNode \rarrow \child[\RIGHT]$  := $\ang{ 1_n, 0_i, 0_d, 0_p, \Null }$\;
			}
			\lElse
			{
			  $\newNode \rarrow \child[\RIGHT]$  := $\ang{ 0_n, 0_i, 0_d, 0_p, \rightChild }$
			}
			\BlankLine
			\tcp{initialize the arguments of \CAS{} instruction}
			$\oldContents$ := $\ang{ 0_n, 1_i, 0_d, 0_p, \node }$\;
			$\newContents$ := $\ang{ 0_n, 0_i, 0_d, 0_p, \newNode }$\;
			
			%% \tcp{switch the edge at the parent}
			%% \mline{$result$ := \CAS( \parbox[t]{1.875in}{$\parent \rarrow \child[pWhich]$, \\ $\ang{ 0_n, 1_i, 0_d, 0_p, \node }$, $\ang{ 0_n, 0_i, 0_d, 0_p, \newNode }$ );}\;}
			 
	
	 }
	 \Else(\tcp*[h]{remove the node})
	 {
	   			
	    %% \tcp{remove the node}
			
			\tcp{determine to which grand child will the edge at the parent be switched}
			\uIf{$\node \rarrow \child[\LEFT]$ = $\ang{ 1_n, \ast, \ast, \ast, \ast }$}
			{
		     $nWhich$ := \RIGHT\;
			}
			\lElse{$nWhich$ := \LEFT}
			
			\BlankLine
			\tcp{initialize the arguments of the \CAS{} instruction}
			$\oldContents$ := $\ang{ 0_n, 1_i, 0_d, 0_p, \node }$\;
			$\ang{ \nFlag, \ast, \ast, \ast, address }$ := $\node \rarrow \child[nWhich]$\; 
  		\uIf(\tcp*[h]{set the null flag only}){$\nFlag$}
			{
			   $\newContents$ := $\ang{ 1_n, 0_i, 0_d, 0_p, \node }$\;
			   %% \tcp{set the null flag only; do not change the address}
			   %% \mline{$result$ := \CAS( \parbox[t]{1.25in}{$\parent \rarrow \child[pWhich]$, \\ $\ang{ 0_n, 1_i, 0_d, 0_p, \node }$, \\ $\ang{ 1_n, 0_i, 0_d, 0_p, \node }$ );}\;}
			}
			\Else(\tcp*[h]{change the pointer to the grand child})
			{
			   $\newContents$ := $\ang{ 0_n, 0_i, 0_d, 0_p, address }$ \;
				 %% \mline{$result$ := \CAS( \parbox[t]{1.25in}{$\parent \rarrow \child[pWhich]$, \\ $\ang{ 0_n, 1_i, 0_d, 0_p, \node }$, \\ $\ang{ 0_n, 0_i, 0_d, 0_p, address }$ );}\;}
			}
			
			
			
	 }
	
	  \BlankLine
		\mline{$result$ := \CAS( \parbox[t]{1.75in}{$\parent \rarrow \child[pWhich]$, \\ $\oldContents$, $\newContents$ );}\;}
		\Return $result$\;
		



}

\caption{Cleaning Up the Tree}
\label{algo:icdcn-cleanup}
\end{algorithm}





\begin{algorithm}[htp]
%%
\DefineKeyWords
\DontPrintSemicolon
\Boolean \MarkChildEdge( $\state$, $which$ )\;
\PrintSemicolon
\Begin
{

   \uIf{$\state \rarrow mode$ = \INJECTION}
	 {
	    $edge$ := $\state \rarrow \targetEdge$\; 
	    $\flag$ := \DELETEFLAG\;
	 }
	 \Else
	 {
	    $edge$ := $( \state \rarrow \successorSeekRecord ) \rarrow \terminalEdge$\; 
	    $\flag$ := \PROMOTEFLAG\;
	 }
	 
	 
   $\node$ := $edge.\child$\;
	
   \BlankLine
  
	 \While{\True}
	 {
	    $\ang{\nFlag, \iFlag, \dFlag, \pFlag, address}$ := $\node \rarrow \child[which]$\;
			
			\uIf{$\iFlag$}
			{
			   $helpeeEdge$ := $\curly{ \node, address, which }$\;
				 %% \HelpTargetNode( $helpeeEdge$, $\state \rarrow depth + 1$ )\;
				 \HelpTargetNode( $helpeeEdge$ )\;
				 \Continue\;
			}
			\uElseIf{$\dFlag$}
			{
			   \uIf{$\flag$ = \PROMOTEFLAG}
				 {
				    %% \HelpTargetNode( $edge$, $\state \rarrow depth + 1$  )\;
						\HelpTargetNode( $edge$ )\;
						\Return \False\;
				 } 
				 \lElse
				 {
				    \Return \True
				 }
			}
			\ElseIf{$\pFlag$}
			{
			   \uIf{$\flag$ = \DELETEFLAG}
				 {
				    %% \HelpSuccessorNode( $edge$, $\state \rarrow depth + 1$  )\;
						\HelpSuccessorNode( $edge$ )\;
						\Return \False\;
				 } 
				 \lElse
				 {
				    \Return \True
				 }
			}
			
			$\oldContents$ := $\ang{ \nFlag, 0_i, 0_d, 0_p, address }$\;
			$\newContents$ := $\oldContents \: | \: \flag$\;
			\mline{$result$ := \CAS( \parbox[t]{1.5in}{$\node \rarrow \child[which]$, $\oldContents$, \\ $\newContents$ );}\;}
			
			\lIf{$result$}{ \Break }
			
			
	 }

   \Return \True\;
}
\caption{Mark Child Edge}
\label{algo:icdcn-markChildEdge}


%\remove{

\end{algorithm}

%% FINDSMALLEST

\begin{algorithm}[htp]
%%
\DefineKeyWords
%}

\BlankLine

\DontPrintSemicolon
\Boolean \FindSmallest( $\state$ )\;
\PrintSemicolon
\Begin
{
   \tcp{find the node with the smallest key in the subtree rooted at the right child of the target node}
	 $\node$ := $\state \rarrow \targetEdge.\child$\;
	 $seekRecord$ := $\state \rarrow seekRecord$\;
	 $\ang{ \nFlag, \ast, \ast, \ast, \rightChild }$ := $\node \rarrow \child[\RIGHT]$\;
	 \If(\tcp*[h]{the right subtree is empty}){$\nFlag$}
	 {
	    %% \tcp{the right subtree is empty}
			\Return \False\;
	 }
	
	 \BlankLine	
		
	 \tcp{initialize the variables used in the traversal}
	 
	
	 %% $\ang{ \ast, \ast, \ast, \ast, \rightChild }$ := $\node \rarrow \child[RIGHT]$\;
	 $\terminalEdge$ := $\ang{ \node, \rightChild, \RIGHT }$\;
	 $\penultimateEdge$ := $\ang{ \node, \rightChild, \RIGHT }$\;
		 
	 %% \BlankLine
	 	
	 \While{\True}
	 {
	    $\curr$ := $\terminalEdge.\child$\;
      $\ang{ \nFlag, \ast, \ast, \ast, \leftChild }$ := $\curr \rarrow \child[\LEFT]$\;			
			\If(\tcp*[h]{reached the node with the smallest key}){$\nFlag$}	
			{
			   $\injectionEdge$ := $\ang{\curr, \leftChild, \LEFT}$\;
			   \Break\;
			}				
			\BlankLine			
			\tcp{traverse the next edge}			
			$\penultimateEdge$ := $\terminalEdge$\;
	    $\terminalEdge$ := $\ang{ \curr, \leftChild, \LEFT }$\;			
	 }	
	 \BlankLine
	 \tcp{initialize seek record and return}
	 $seekRecord \rarrow \terminalEdge$ := $\terminalEdge$\;
	 $seekRecord \rarrow \penultimateEdge$ := $\penultimateEdge$\;
   $seekRecord \rarrow \injectionEdge$ := $\injectionEdge$\;
	 \Return \True\;	
}
\caption{Find Smallest}
\label{algo:icdcn-findSmallest}
\end{algorithm}


\begin{algorithm}[htp]
%%
\DefineKeyWords


\DontPrintSemicolon
\InitializeTypeAndUpdateMode( $\state$ )\;
\PrintSemicolon
\Begin
{

   \tcp{retrieve the target node's address from the state record}
   $\node$ := $\state \rarrow \targetEdge.\child$\;
	 
	
	 \BlankLine
	 %% $\canReplace$ := $\node \rarrow \canReplace$\;
	 $\ang{ \lNFlag, \ast, \ast, \ast, \ast }$ := $\node \rarrow \child[\LEFT]$\;
	 $\ang{ \rNFlag, \ast, \ast, \ast, \ast }$ := $\node \rarrow \child[\RIGHT]$\;
	
	 \uIf{$\lNFlag$ \LOr $\rNFlag$}
	 {
	    \tcp{one of the child pointers is null}
	    $\ang{\mFlag, \ast }$ := $\node \rarrow \markAndKey$\;
	    \lIf{$\mFlag$}
	    {
	      $\state \rarrow type$ := \COMPLEX
	      %% $\node \rarrow \canReplace$ := \True\;
	    }
	    \lElse
	    {
	      $\state \rarrow type$ := \SIMPLE
	     }
	 }
	 \Else(\tcp*[h]{both child pointers are non-null})
	 {
	    %% \tcp{both child pointers are non-null}
	    $\state \rarrow type$ := \COMPLEX\;
	 }
	
	 \UpdateMode( $\state$ )\;
	
	 %% \Return\;

}

\remove{

\end{algorithm}


%% UPDATEMODE

\begin{algorithm}[htp]
%%
\DefineKeyWords

}

\BlankLine

\DontPrintSemicolon
\UpdateMode( $\state$ )\;
\PrintSemicolon
\Begin
{
	
	 \tcp{update the operation mode}

	 \BlankLine
	 \uIf(\tcp*[h]{simple delete}){$\state \rarrow type$ = \SIMPLE}
	 {
	    %% \tcp{simple delete}	
			$\state \rarrow mode$ := \CLEANUP\;
	 }
	 \Else(\tcp*[h]{complex delete})
	 {
	  	%% \tcp{complex delete}	

      $\node$ := $\state \rarrow \targetEdge.\child$\;
			\uIf{$\node \rarrow \canReplace$}
			{
			   $\state \rarrow mode$ := \CLEANUP\;
			}
			\lElse{$\state \rarrow mode$ := \DISCOVERY}
	 }
	
	 %% \Return\;
}

\caption{Helper Routines}
\label{algo:icdcn-helper|2}
\end{algorithm}

%% HELP

\begin{algorithm}[htp]
%%
\DefineKeyWords




\DontPrintSemicolon
%% \HelpTargetNode( $helpeeEdge$, $depth$ )\;
\HelpTargetNode( $helpeeEdge$ )\;
\PrintSemicolon
\Begin
{
   %% \lIf{$depth$ = number of processes}{ \Return }
	 %% \BlankLine		
	 \tcp{intent flag must be set on the edge}
	 \tcp{obtain new state record and initialize it}
	 $\state \rarrow \targetEdge$ := $helpeeEdge$\;
	 %% $\state \rarrow depth$ := $depth$\;
	 $\state \rarrow mode$ := \INJECTION\;
	 \BlankLine	
	 \tcp{mark the left and right edges if unmarked}
	 $result$ := \MarkChildEdge( $\state$, \LEFT{} )\;
	 \lIf{\LNot($result$)}{ 
	    %% \tcp{promote flag must have been set on the left edge}
			%% \HelpSuccessorNode( $helpeeEdge$, $depth + 1$ )\;
	    \Return
	 }
	 \MarkChildEdge( $\state$, \RIGHT{} )\;
	 
	 \InitializeTypeAndUpdateMode( $\state$ )\;
	
			
	 
	 \BlankLine
	
	 \tcp{perform the remaining steps of a delete operation}
   \If{$\state \rarrow mode$ = \DISCOVERY}
	 {
			%% \tcp{complex delete operation; locate the successor node and mark its child edges with promote flag}		
	    \FindAndMarkSuccessor( $\state$ )\;
	 						
	 }
			
	 \BlankLine
			
	 \If{$\state \rarrow mode$ = \DISCOVERY}
	 {
						
			%% \tcp{complex delete operation; promote the successor node's key and remove the successor node}
	    \RemoveSuccessor( $\state$ )\;
		   						
	 }
				
	 \BlankLine	
				
	 \lIf{$\state \rarrow mode$ = \CLEANUP}
	 {
	    %% \tcp{either remove the target node (simple delete) or replace it with a new node with unmarked edges (complex delete)}
	    \Cleanup( $\state$ )
	 }
	
	 %% \Return\;
}

\remove{

\end{algorithm}	
	


\begin{algorithm}[htp]
%%
\DefineKeyWords

}

\BlankLine

\DontPrintSemicolon
%% \HelpSuccessorNode( $helpeeEdge$, $depth$ )\;
\HelpSuccessorNode( $helpeeEdge$ )\;
\PrintSemicolon
\Begin
{
   %% \lIf{$depth$ = number of processes}{ \Return }
	 %% \BlankLine
   \tcp{retrieve the address of the successor node}
   $\parent$ := $helpeeEdge.\parent$\;
	 $\node$ := $helpeeEdge.\child$\;
	 
	 \tcp{promote flat must be set on the successor node's left edge}
	 \tcp{retrieve the address of the target node}
	 $\ang{ \ast, \ast, \ast, \ast, \leftChild }$ := $\node \rarrow \child[\LEFT]$\;
	 \BlankLine	
	 \tcp{obtain new state record and initialize it}
	 $\state \rarrow \targetEdge$ := $\curly{ \Null, \leftChild, \_ }$\;
	 %% $\state \rarrow depth$ := $depth$\;
	 $\state \rarrow mode$ := \DISCOVERY\;
	 $seekRecord$ := $\state \rarrow \successorSeekRecord$\;
	 \tcp{initialize the seek record in the state record}
	 $seekRecord \rarrow \terminalEdge$ := $helpeeEdge$\;
	 $seekRecord \rarrow \penultimateEdge$ := $\curly{ \Null, \parent, \_ }$\;
   \tcp{promote the successor node's key and remove the successor node}
	 \RemoveSuccessor( $\state$ )\;
	 \tcp{no need to perform the cleanup}
	
	
	 %% \Return\;

}


\caption{Helping Conflicting Delete Operations}
\label{algo:icdcn-helping}
\end{algorithm}
\end{limitscope}

A pseudo-code of our algorithm is given in \algosref{data|structures}{helper}.
Different data structures used in our algorithm are shown in \algoref{data|structures}. Besides tree node, we use three additional records:
\begin{enumerate*}[label=(\alph*)]
\item  \emph{seek record:} to store the outcome of a tree traversal both when looking for the target key and the successor key, 
\item \emph{anchor record:} to store information about the \anchornode{} during the seek phase, and
\item  \emph{lock record:} to store information about a tree edge that needs to be locked. 
\end{enumerate*}

The pseudo-code for the seek function is shown in \algoref{seek}. The 
pseudo-codes for search, insert and delete operations are shown  in 
\algoref{search}, \algoref{insert} and \algoref{delete}, respectively. 
\Algoref{lock:unlock} contains the pseudo-code for locking and unlocking a 
set of tree edges, as specified in an array. Finally, \algoref{helper} contains the pseudo-codes for 
three helper functions used by a delete operation, namely:
\begin{enumerate*}[label=(\alph*)]
\item \ClearFlags{}: to clear lock and mark flags from a child field, 
\item \FindSmallest{}: to locate the smallest key in a subtree, and
\item \RemoveChild{}: to remove a given child of a node.
\end{enumerate*}

In the pseudo-code, to improve clarity, we sometimes use subscripts $l$, $m$ and $n$ to denote lock, mark and null flags, respectively.  
\section{Correctness Proof}
It is convenient to treat insert and delete operations that do not change the tree as search operations. We call a tree node \emph{active} if it is reachable from the root of the 
tree. We call a tree node  \emph{passive} if it was active earlier but is not active any more. Note that, before an active node is made passive by a delete operation, both its 
children edges are \emph{marked}. Also, a \CAS{} instruction performed on an edge (by either an insert operation or a delete operation as part of locking) is successful only if the edge is unmarked. As a result, clearly, if an insert operation completes successfully, then  its target node was active when its edge was modified to make the new node (containing the target key) a part of the tree. Likewise, if a delete operation completes successfully, then all the nodes involved in the operation (up to three nodes) were active when their edges were locked.

\subsection*{All Executions are Linearizable}

We show that an arbitrary execution of our algorithm is linearizable by specifying the \emph{linearization point} of each operation. Note that the linearization point of an operation is the point during its execution at which the operation appeared to have taken effect. Our algorithm supports three types of operations: search, insert and delete. We now specify the linearization point of each operation.

\begin{enumerate}[leftmargin=*]
\item \emph{Insert operation:} The operation is linearized at the point at which it performed the successful \CAS{} instruction that resulted in its target key becoming part of the tree.					
\item \emph{Delete operation:} There are two cases depending on whether the delete operation is simple or complex. If the operation is simple delete, then the operation is linearized at the point at which a successful write step was performed at the parent of the target node that resulted in the target node becoming passive. Otherwise, it is linearized at the point at which the original key of the target node was replaced with its successor key.   
\item \emph{Search operation:} There are two cases depending on whether the target node was active when the operation read the key stored in the node. If the target node was not active, then the operation is linearized at the point at which the target node became passive. Otherwise, it is linearized at the point at which the read step was performed.
\end{enumerate}

It can be easily verified that, for any execution of the algorithm, the sequence of operations obtained by ordering operations based on their linearization points is legal, \emph{i.e.}, all operations in the sequence satisfy their specification. 

Thus we have:
\begin{theorem}
Every execution of our algorithm is linearizable.
\end{theorem}
			 
\subsection*{All Executions are Deadlock-Free}	
We say that the system is in a \emph{quiescent state} if no modify operation completes hereafter. We say that the system is in a \emph{potent state} if it has one or more pending modify operations. Note that quiescence is a \emph{stable property}; once the system is in a quiescent state, it stays in a quiescent state. We show that our algorithm is deadlock-free by proving that a potent state is necessarily non-quiescent. 


Note that, in a quiescent state, no edges in the tree can be marked. This is because a delete operation marks edges only after it has successfully obtained all the locks, after which it is guaranteed to complete. This also implies that the tree cannot undergo any changes now because that would imply eventual completion of a modify operation. Thus, once a system has reached a quiescent state, all modify operation currently pending repeatedly alternate between seek and execution phases. We say that the system is in a \emph{strongly-quiescent state} if all pending modify operations started their most recent seek phase \emph{after} the system became quiescent. Note that, like quiescence, strong quiescence is also a stable property. Now, once the system has reached a strongly quiescent state, the following can be easily verified. First, for a given modify operation, every traversal of the tree in the seek phase returns the same target node. Second, for a given delete operation, the set of edges it needs to lock remains the same. 


Now, assume that the system eventually reaches a state that is both potent and quiescent. Clearly, from this state, the system will eventually reach a state that is potent and strongly-quiescent. Note that a delete operation in our algorithm locks edges in a \emph{top-down}, \emph{left-right} manner. As a result, there cannot be a ``cycle'' involving delete operations. If a delete operation continues to fail in the execution phase, then it is necessarily because it tried to acquire lock on an already locked edge. (Recall that the set of edges does not change any more and there are no marked edges in the tree.) We can construct a chain of operations such that each operation in the chain tried to lock an edge already locked by the next operation in the chain. Clearly, the length of the chain is bounded. This implies that the last operation in the chain is guaranteed to obtain all the locks and will eventually complete. This contradicts the fact that the system is in a quiescent state. 


Thus, we have:
\begin{theorem}
Every execution of our algorithm is deadlock-free.
\end{theorem}

\end{limitscope}