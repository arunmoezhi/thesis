
\begin{limitscope}

%% To limit the scope of the macros defined below

%% macros for pseudocode

\newcommand{\child}{child}
\newcommand{\node}{node}
\newcommand{\parent}{parent}
\newcommand{\nullAddress}{nullAddress}

\newcommand{\mainSeekRecord}{seekTargetKey}
\newcommand{\successorSeekRecord}{seekSuccessorKey}
\newcommand{\seekRecord}{seekRecord}

\newcommand{\key}{key}

\newcommand{\done}{done}
\newcommand{\result}{result}
\newcommand{\offset}{o\!f\!f\!set}

\newcommand{\leftChild}{lField}
\newcommand{\rightChild}{rField}

\newcommand{\field}{\!f\!ield}
\newcommand{\newField}{newField}

\newcommand{\cKey}{cKey}
\newcommand{\pKey}{pKey}
\newcommand{\cNode}{cNode}
\newcommand{\pNode}{pNode}
\newcommand{\aNode}{aNode}
\newcommand{\sNode}{sNode}
\newcommand{\sParent}{sParent}


\newcommand{\match}{match}

\newcommand{\which}{which}
\newcommand{\address}{address}

\newcommand{\AnchorRecord}{AnchorRecord}

\newcommand{\pSeekRecord}{pSeekRecord}
\newcommand{\pAnchorRecord}{pAnchorRecord}
%% \newcommand{\currSeekRecord}{cSeekRecord}
\newcommand{\anchorRecord}{anchorRecord}


\newcommand{\newNode}{newNode}


%%%%%\newcommand{\nFlag}{nFlag}
%%%%%\newcommand{\mFlag}{mFlag}
%%%%%\newcommand{\lFlag}{lFlag}

\newcommand{\lockArray}{lockArray}
\newcommand{\size}{size}
\newcommand{\contents}{addressSeen}
%% \newcommand{\oldContents}{oldValue}
\newcommand{\newContents}{lockedAddress}









\newcommand{\Search}{\textsc{Search}}
\newcommand{\Insert}{\textsc{Insert}}
\newcommand{\Delete}{\textsc{Delete}}
\newcommand{\Seek}{\textsc{Seek}}
%%%%\newcommand{\FindSmallest}{\textsc{FindSmallest}}
\newcommand{\Inject}{\textsc{Inject}}
%%%%\newcommand{\RemoveChild}{\textsc{RemoveChild}}
%%%%\newcommand{\LockAll}{\textsc{LockAll}}
%%%%\newcommand{\UnlockAll}{\textsc{UnlockAll}}
%%%%\newcommand{\ClearFlags}{\textsc{ClearFlags}}

\newcommand{\mline}[1]{\DontPrintSemicolon #1 \PrintSemicolon}


\newcommand{\LEFT}{\textsf{LEFT}}
\newcommand{\RIGHT}{\textsf{RIGHT}}


\newcommand{\rarrow}{\!\rightarrow\!}





%%%%%%%%%%%%%%%%%%%%%%%%%%%%%%%%%%%%%%%%%%%%%%%%%%%%%%%%%%%%%%%%%%%%%%%%%%%%%%%%%%%%

\newcommand{\DefineKeyWords}{
%%
\SetKw{Boolean}{boolean}
\SetKw{LAnd}{~and~}
\SetKw{LOr}{~or~}
\SetKw{LNot}{not}
\SetKw{Struct}{struct}
\SetKw{Null}{null}
\SetKw{True}{true}
\SetKw{False}{false}
\SetKw{Break}{break}
\SetKw{Continue}{continue}
\SetKw{Enum}{enum}
\SetKw{Word}{word}
%%
}

%%%%%%%%%%%%%%%%%%%%%%%%%%%%%%%%%%%%%%%%%%%%%%%%%%%%%%%%%%%%%%%%%%%%%%%%%%%%%%%%%%%%%




\begin{algorithm}[t]
\caption{Data Structures Used} 
\label{algo:data|structures}
%%
\DefineKeyWords
%%
\tcp{a tree node}
\DontPrintSemicolon
\Struct Node \{\;
\label{ln:node|begin}
\PrintSemicolon
%%
\Indp 
   Key $\key$\; 
   \{ \Boolean, \Boolean, \Boolean, NodePtr \} $\child[2]$\; 
	 \tcp{ each $\child$ field contains four subfields: $\lFlag$, $\mFlag$, $\nFlag$ and $\address$ }
\Indm 
%%
\}\;

\BlankLine

\tcp{used to store the results of a tree traversal}
\DontPrintSemicolon
\Struct SeekRecord \{\;
\label{ln:seek|record|begin}
\PrintSemicolon
%%
\Indp 
   NodePtr $\node$\;
   NodePtr $\parent$\;
   NodePtr $\nullAddress$\;
\Indm 
\}\;

\BlankLine

\tcp{used to store information about an anchor node}
\DontPrintSemicolon
\Struct AnchorRecord \{\;
\label{ln:anchor|record|begin}
\PrintSemicolon
%%
\Indp 
   NodePtr $\node$\;
   Key $\key$\;
\Indm 
\}\;

\BlankLine

\tcp{used to store information about an edge to lock}
\DontPrintSemicolon
\Struct LockRecord \{\;
\label{ln:lock|record|begin}
\PrintSemicolon
\Indp
   NodePtr $\node$\;
	 \Enum \{ \LEFT{}, \RIGHT{} \} $\which$\;
	 \{ \Boolean, NodePtr \} $\contents$\; 
	 \tcp{ $\contents$ contains two subfields: $\nFlag$ and $\address$ }
\Indm
\}\;

\BlankLine
\tcp{local seek record used when looking for a node}
SeekRecordPtr $\mainSeekRecord$, $\successorSeekRecord$\;
%% SeekRecordPtr $\successorSeekRecord$\;
\tcp{local array used to store the set of edges to lock}
LockRecord $\lockArray[4]$\;
\label{lin:structures|end}
%%
\end{algorithm}

\begin{algorithm}[htp]
\caption{Seek Function} 
\label{algo:seek}
%%
\DefineKeyWords
%%
\DontPrintSemicolon
\Seek( $\key$, $\seekRecord$ )\;
\PrintSemicolon
\Begin
{
   \While{\True}
   {
	    \label{lin:seek|begin}
			\tcp{initialize the variables used in the traversal}
	    $\pNode$ := $\sNodeOne$; \qquad
			$\cNode$ := $\sNodeTwo$\;
			$\address$ := $\sNodeTwo \rarrow \child[\RIGHT].\address$\;
			$\anchorRecord$ := $\curly{ \sNodeOne, \sKeyOne }$\;
			\While{\True}
	    {
		     \tcp{reached terminal node; read the key stored in the current node}
		     $\cKey$ := $\cNode \rarrow \key$\;
		     \If{$\key$ = $\cKey$}
				 {
				    $\seekRecord$ := $\curly{ \cNode, \pNode, \address }$\;
				    \Return\;
				 }
		     $\which$ := $\key < \cKey$ ? $\LEFT$ : $\RIGHT$\;
		     \tcp{read the next address to dereference along with mark and null flags}
		     $\ang{ \ast, \ast, \nFlag, \address}$ := $\cNode \rarrow \child[\which]$\; 
		     \If(\tcp*[f]{the null flag is set; reached terminal node}){$\nFlag$}
		     {		      
						$\aNode$ := $\anchorRecord \rarrow \node$\;    					
					  \uIf{$\aNode \rarrow \child[\RIGHT].\mFlag$}
						{
						   \tcp{the anchor node is marked; it may no longer be part of the tree}
							
						   \uIf{$\anchorRecord$ = $\pAnchorRecord$}
							 {
							    \tcp{the anchor record of the current traversal matches that of the previous traversal}
							    $\seekRecord$ := $\pSeekRecord$\;
									\Return\;
							 }
							 \lElse{ \Break }
						} 
					  \Else(\tcp*[f]{the anchor node is definitely part of the tree})
						{
							 \uIf(\tcp*[f]{seek can terminate now}){$\aNode \rarrow \key$ $<$ $\key$}
			         {
							    $\seekRecord$ := $\curly{ \cNode, \pNode, \address }$\;
									\Return\;
							 } 
							 \lElse{ \Break }
			      } 
		     }
				 \BlankLine
				 \tcp{update the anchor record if needed}
		     \If{$\which$ = $\RIGHT$}
		     {
			      \tcp{the next edge to be traversed is a right edge; keep track of current node and its key}
			      $\anchorRecord$ := $\curly{ \cNode, \cKey }$\;
		     }
				 \BlankLine
		     \tcp{traverse the next edge}
		     $\pNode$ := $\cNode$; \qquad
		     $\cNode$ : =$\address$\; 		
				 \label{lin:seek|end}
	    } 
	 }
}
\end{algorithm}

\begin{algorithm}[htp]
\caption{Search Operation} 
\label{algo:search}
%%
\DefineKeyWords
%%
\DontPrintSemicolon
\Boolean \Search( $\key$ )\;
\PrintSemicolon
%%
\Begin
{
   \Seek( $\key$, $\mainSeekRecord$ )\;
	 \label{lin:search|begin}
	 $\node$ := $\mainSeekRecord \rarrow \node$\;
   \uIf{$\node \rarrow \key$ = $\key$}
	 { 
	    
	    \Return \True; \tcp*[f]{key found}
	 }
   \Else
	 { 
	    
	    \Return \False; \tcp*[f]{key not found}
			\label{lin:search|end}
	 }
}
\end{algorithm}




\begin{algorithm}[htp]
\caption{Insert Operation} 
\label{algo:insert}
%%
\DefineKeyWords
%%
\DontPrintSemicolon
\Boolean \Insert( $\key$ )\;
\PrintSemicolon
%%
\Begin
{
   \While{\True}
	 {
      \Seek( $\key$, $\mainSeekRecord$ )\;
			\label{lin:insert|begin}
			\BlankLine
	    $\node$ := $\mainSeekRecord \rarrow \node$\;
			\uIf{$\node \rarrow \key$ = $\key$}
	    { 
	       
	       \Return \False; \tcp*[f]{key found}
	    }
      \Else
			{ 
	       \tcp{key not found; add the key to the tree}
	       
			   $\newNode$ := create a new node\;
				 \tcp{initialize its fields}
				 $\newNode \rarrow \key$ := $\key$\;
				 $\newNode \rarrow \child[\LEFT]$ := $\ang{ 0_l, 0_m, 1_n, \Null }$\;
         $\newNode \rarrow \child[\RIGHT]$ := $\ang{ 0_l, 0_m, 1_n, \Null }$\;
				 \BlankLine
				 \tcp{determine which child field (left or right) needs to be modified}
				 $\which$ := $\key < \node \rarrow \key$  ? $\LEFT$ : $\RIGHT$\;
				 \tcp{fetch the address observed by the seek function in that field}
			   $\address$ := $\mainSeekRecord \rarrow \nullAddress$\;
				 \BlankLine
			   \mline{$\result$ := \CAS( \parbox[t]{1.825in}{$\node \rarrow \child[\which]$, \\ $\ang{ 0_l, 0_m, 1_n, \address }$, \\ $\ang{ 0_l, 0_m, 0_n, \newNode }$ );}\;}
			
			   \If{$result$}
				 {
				    \tcp{new key successfully added to the tree}
				    \Return \True\;
						\label{lin:insert|end}
				 }
				 
				 %% \tcp{restart from the seek phase}
				
					
	    }
   }
}
%%
\end{algorithm}



\begin{algorithm}[htp]
\caption{Delete Operation} 
\label{algo:delete}
%%
\DefineKeyWords
%%
\DontPrintSemicolon
\Boolean \Delete( $\key$ )\;
\PrintSemicolon
%%
\Begin
{
   \While{\True}
	 {
      \Seek( $\key$, $\mainSeekRecord$ )\;
			\label{lin:delete|begin}
	    $\node$ := $\mainSeekRecord \rarrow \node$\;
			\uIf(\tcp*[f]{key not found}){$\node \rarrow \key$ $\neq$ $\key$}
	    { 
	       \Return \False;
	    }
      \Else(\tcp*[f]{key found; read contents of target node's children fields})
			{ 
			   $\leftChild$ := \ClearFlags( $\node \rarrow \child[\LEFT]$ )\;
				 $\rightChild$ := \ClearFlags( $\node \rarrow \child[\RIGHT]$ )\;
				
				 \uIf(\tcp*[f]simple delete operation){$\leftChild.\nFlag$ ~\LOr~ $\rightChild.\nFlag$}
				 {
					  $\parent$ := $\mainSeekRecord \rarrow \parent$\;
						\lIf{$\key < \parent \rarrow \key$}{ $\which$ := \LEFT }
						\lElse{ $\which$ := \RIGHT }
					  $\lockArray[0]$ := $\curly{ \parent, \which, \ang{0, \node} }$\;
					  $\lockArray[1]$ := $\curly{ \node, \LEFT, \leftChild }$\;
					  $\lockArray[2]$ := $\curly{ \node, \RIGHT, \rightChild }$\;
						$\result$ := \LockAll( $\lockArray$, 3 )\;
						\If(\tcp*[f]{all locks acquired; perform the operation}){$result$}
						{
						   \uIf(\tcp*[f]{key still matches; remove the node}){$\node \rarrow \key$ = $\key$}
							 {
						      \RemoveChild( $\parent$, $\which$ );
								  $\match$ := \True\;	
							 } 
							 \lElse{ $\match$ := \False }
							
							 \UnlockAll( $\lockArray$, 3 )\;
							 \Return $\match$;
						}
				 }
				 \Else(\tcp*[f]{ complex delete operation; locate the successor node})
				 {
						\FindSmallest($\node$, $\rightChild.\address$, $\successorSeekRecord$)\;
					  $\sNode$ := $\successorSeekRecord \rarrow \node$;
						$\sParent$ := $\successorSeekRecord \rarrow \parent$\;
						\tcp{determine the edges to be locked}
					  $\lockArray[0]$ := $\curly{ \node, \RIGHT, \rightChild }$\;
					  \uIf{$\node$ $\neq$ $\sParent$}
						{
						   \tcp{successor node is not the right child of target node}
						   $\lockArray[1]$ := $\curly{ \sParent, \LEFT, \ang{ 0, \sNode } }$;
						   $\size$ := 4;   
						}
						\lElse { $\size$ := 3 }
						$\leftChild$ := \ClearFlags( $\sNode \rarrow \child[\LEFT]$ )\;
						$\rightChild$ := \ClearFlags( $\sNode \rarrow \child[\RIGHT]$ )\;
						$\lockArray[\size - 2]$ := $\curly{ \sNode, \LEFT, \leftChild }$\;
					  $\lockArray[\size - 1]$ := $\curly{ \sNode, \RIGHT, \rightChild }$\;
						$\result$ := \LockAll( $\lockArray$, $\size$ )\;
						\If(\tcp*[f]{all locks acquired; perform the operation}){$result$}
						{
						   \uIf{$\node \rarrow \key$ = $\key$}
							 {
							     \tcp{key still matches; copy key in successor node to target node}
									 $\node \rarrow \key$ := $\sNode \rarrow \key$\;
						       \RemoveChild( $\sParent$, \LEFT{} );
									 $\match$ := \True\;
							 }		
							 \lElse{ $\match$ := \False }
							 \UnlockAll( $\lockArray$, $\size$ )\;
							 \Return $\match$\;
							 \label{lin:delete|end}
						}
				 } 
	    }
   }
}
%%
\end{algorithm}

\begin{algorithm}[htp]
\caption{Lock and Unlock Functions} 
\label{algo:lock:unlock}
%%
\DefineKeyWords
%%
\DontPrintSemicolon
\Boolean \LockAll( $\lockArray$, $\size$ )\;
\PrintSemicolon
%%
\Begin
{
   \For{$i$ $\leftarrow$ $0$ \KwTo $\size-1$}
	 {
			\label{lin:lock|begin}
	    \tcp{acquire lock for the $i$-th entry}
			$\node$ := $\lockArray[i].\node$\;
			$\which$ := $\lockArray[i].\which$\;
			$\newContents$ := $\lockArray[i].\contents$\;
			$\newContents.\lFlag$ := \True\;
			\tcp{set the lock flag in the child edge}
			$\result$ := \CAS($\node \rarrow \child[\which]$, $\lockArray[i].\contents$, $\newContents$)\;
			\If{\LNot($\result$)}
			{
			   \tcp{release all the locks acquired so far}
			   \UnlockAll( $\lockArray$, $i - 1$ )\;
				 \Return \False;
			}
	 }
	 \Return \True;
	 \label{lin:lock|end}
}


\BlankLine

%%
\DontPrintSemicolon
\UnlockAll( $\lockArray$, $\size$ )\;
\PrintSemicolon
%%
\Begin
{
    \For{$i$ $\leftarrow$ $\size-1$ \KwTo $0$}
		{
				\label{lin:unlock|begin}
		   $\node$ := $\lockArray[i].\node$\;
			 $\which$ := $\lockArray[i].\which$\;
			 \tcp{clear the lock flag in the child edge}
			 $\node \rarrow \child[\which].\lFlag$ := \False\;
			 \label{lin:unlock|end}
		}	
}				
				
\end{algorithm}

\begin{algorithm}[htp]
\caption{Helper Functions used by Delete Operation} 
\label{algo:helper}
%%
\DefineKeyWords
%%
\DontPrintSemicolon
\Word \ClearFlags( \Word $\field$ )\;
\PrintSemicolon
%%
\Begin
{
		\label{lin:clearFlags|begin}
   $\newField$ := $\field$ with lock and mark flags cleared\;
	 \Return $\newField$\;
	 \label{lin:clearFlags|end}
}
%%
\BlankLine
%%
\DontPrintSemicolon
\FindSmallest( $\parent$, $\node$, $\seekRecord$ )\;
\PrintSemicolon
%%
\Begin
{
		\label{lin:findSmallest|begin}
   \tcp{initialize the variables used in the traversal}
   $\pNode$ := $\parent$; \qquad
	 $\cNode$ := $\node$\;
	
	 \BlankLine
	
	 \While{\True}
	 {
	    $\ang{ \ast, \ast, \nFlag, \address }$ := $\cNode \rarrow \child[\LEFT]$\;
			
			\uIf{\LNot($\nFlag$)}
			{
			   \tcp{visit the next node}
			   $\pNode$ := $\cNode$; \qquad
				 $\cNode$ := $\address$\;
			}
			\Else
			{  
			   \tcp{reached the successor node}
			   $\seekRecord$ := $\curly{ \cNode, \pNode, \address }$\;
			   \Break\;
				 \label{lin:findSmallest|end}
			}
	 }
	 
	 
	 %% \Return\;
	
}
%%
\BlankLine
%%
\DontPrintSemicolon
\RemoveChild( $\parent$, $\which$ )\;
%%
hu
\PrintSemicolon
\Begin
{
	\label{lin:removeChild|begin}
   \tcp{determine the address of the child to be removed}
   $\node$ := $\parent \rarrow \child[\which]$\;
	 \BlankLine
   \tcp{mark both the children edges of the node to be removed}
	 $\node \rarrow \child[\LEFT].\mFlag$ := \True\;
	 $\node \rarrow \child[\RIGHT].\mFlag$ := \True\;
   \BlankLine
	 \tcp{determine whether both the child pointers of the node to be removed are null}
 	 %\uIf{\Big(\parbox[c]{1.5in}{$\node \rarrow \child[\LEFT].\nFlag$ ~\LAnd~ \\ $\node \rarrow \child[\RIGHT].\nFlag$}\Big)}
	 \uIf{$\node \rarrow \child[\LEFT].\nFlag$ ~\LAnd~ $\node \rarrow \child[\RIGHT].\nFlag$}
	 {
	    \tcp{set the null flag only}
						  
		  $\parent \rarrow \child[\which].\nFlag$ := \True\;
							
	 } 
	 \Else
	 {
	    \tcp{switch the pointer at the parent to point to its appropriate grandchild}
	    \uIf{$\node \rarrow \child[\LEFT].\nFlag$}
		  {
			   $\address$ := $\node \rarrow \child[\RIGHT].\address$\;
			}
			\lElse
			{
			   $\address$ := $\node \rarrow \child[\LEFT].\address$
			}
							
			$\parent \rarrow \child[\which].\address$ := $\address$\;
			\label{lin:removeChild|end}
	 }					
}
\end{algorithm}

\end{limitscope}