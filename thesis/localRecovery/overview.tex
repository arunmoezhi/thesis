\section{Overview of the Algorithm}
\label{sec:overview}
\begin{figure}[htp]
%\centering
%{
\subcaptionbox{Operation $op(50)$ is suspended at node $Y$ during its traversal. \label{fig:movement|suspended}}[0.45\textwidth]
{
\begin{tikzpicture}[scale=\myfigurescaletwo, transform shape] 
	 \newcommand\NODEDX{1.25}
	 \newcommand\NODEDY{1.25}
	 \newcommand\SUBTREEDX{1.5}
	 \newcommand\SUBTREEDY{0.75}
	
   \node (r)	[treenode] 		                at (0, 0)       		                      	{$R$ \\ 100};
   \node (s)	[treenode, fill=black!20] 		at ([shift=({ -\NODEDX, -\NODEDY})]r)     	{$S$ \\ 10};
	 \node (t)	[treenode] 		                at ([shift=({  \NODEDX, -\NODEDY})]s)    		{$T$ \\ 90};
	 \node (u)	[treenode, fill=black!20] 	  at ([shift=({ -\NODEDX, -\NODEDY})]t)     	{$U$ \\ 20};
	 \node (v)	[treenode] 										at ([shift=({  \NODEDX, -\NODEDY})]u)     	{$V$ \\ 80};
	 \node (w)	[treenode] 										at ([shift=({ -\NODEDX, -\NODEDY})]v)     	{$W$ \\ 70};
	 \node (x)	[treenode, fill=black!20] 		at ([shift=({ -\NODEDX, -\NODEDY})]w)     	{$X$ \\ 30};
	 \node (y)	[treenode] 										at ([shift=({  \NODEDX, -\NODEDY})]x)     	{$Y$ \\ 60};
	 \node (z)	[treenode] 										at ([shift=({ -\NODEDX, -\NODEDY})]y)     	{$Z$ \\ 50};
	 \node (gl) [ground]                      at ([shift=({ -\NODEDX, -\NODEDY})]z)     	{ };
	 \node (gr) [ground]                      at ([shift=({  \NODEDX, -\NODEDY})]z)     	{ };
		
	 \node (sa) [subtree]                     at ([shift=({  \SUBTREEDX, -\SUBTREEDY})]r) {\Large $\alpha$};
	 \node (sb) [subtree]                     at ([shift=({ -\SUBTREEDX, -\SUBTREEDY})]s) {\Large $\beta$};
	 \node (sg) [subtree]                     at ([shift=({  \SUBTREEDX, -\SUBTREEDY})]t) {\Large $\gamma$};
	 \node (sd) [subtree]                     at ([shift=({ -\SUBTREEDX, -\SUBTREEDY})]u) {\Large $\delta$};
	 \node (ss) [subtree]                     at ([shift=({  \SUBTREEDX, -\SUBTREEDY})]v) {\Large $\sigma$};
	 \node (st) [subtree]                     at ([shift=({  \SUBTREEDX, -\SUBTREEDY})]w) {\Large $\tau$};
	 \node (sp) [subtree]                     at ([shift=({ -\SUBTREEDX, -\SUBTREEDY})]x) {\Large $\pi$};
	 \node (sl) [subtree]                     at ([shift=({  \SUBTREEDX, -\SUBTREEDY})]y) {\Large $\lambda$};	
	
	 \node (op) [label={right:{\large $op(40)^{z^{z^z}}$, $op(50)^{z^{z^z}}$}}] at ([shift=({0.375, 0})]y) {};
	
	 \path[every node/.style={font=\sffamily\small}]
	    (0, 1)  edge[->,very thick]  node {} (r)
		  (r)     edge[->,very thick]  node {} (s)
			(s)     edge[->,very thick]  node {} (t)
			(t)     edge[->,very thick]  node {} (u)
			(u)     edge[->,very thick]  node {} (v)
			(v)     edge[->,very thick]  node {} (w)
			(w)     edge[->,very thick]  node {} (x)
			(x)     edge[->,very thick]  node {} (y)
			(y)     edge[->,very thick]  node {} (z)
			(z)     edge[->]  node {} (gl)
			(z)     edge[->]  node {} (gr)
			(r)     edge[->]  node {} (sa.north)
			(s)     edge[->]  node {} (sb.north)
			(t)     edge[->]  node {} (sg.north)
			(u)     edge[->]  node {} (sd.north)
			(v)     edge[->]  node {} (ss.north)
			(w)     edge[->]  node {} (st.north)
			(x)     edge[->]  node {} (sp.north)
			(y)     edge[->]  node {} (sl.north);		
\end{tikzpicture}
}
%%%%%%%%%%%%%%
\qquad
%%%%%%%%%%%%%%
\subcaptionbox{All keys in subtree $\pi$ are deleted one-by-one.\label{fig:movement|subtree|deleted}}[0.45\textwidth]
{

\begin{tikzpicture}[scale=\myfigurescaletwo, transform shape]
   
	 \newcommand\NODEDX{1.25}
	 \newcommand\NODEDY{1.25}
	 \newcommand\SUBTREEDX{1.5}
	 \newcommand\SUBTREEDY{0.75}
	
   \node (r)	[treenode] 		                at (0, 0)       		                      	{$R$ \\ 100};
   \node (s)	[treenode, fill=black!20] 		at ([shift=({ -\NODEDX, -\NODEDY})]r)     	{$S$ \\ 10};
	 \node (t)	[treenode] 		                at ([shift=({  \NODEDX, -\NODEDY})]s)    		{$T$ \\ 90};
	 \node (u)	[treenode, fill=black!20] 	  at ([shift=({ -\NODEDX, -\NODEDY})]t)     	{$U$ \\ 20};
	 \node (v)	[treenode] 										at ([shift=({  \NODEDX, -\NODEDY})]u)     	{$V$ \\ 80};
	 \node (w)	[treenode] 										at ([shift=({ -\NODEDX, -\NODEDY})]v)     	{$W$ \\ 70};
	 \node (x)	[treenode, fill=black!20] 		at ([shift=({ -\NODEDX, -\NODEDY})]w)     	{$X$ \\ 30};
	 \node (y)	[treenode] 										at ([shift=({  \NODEDX, -\NODEDY})]x)     	{$Y$ \\ 60};
	 \node (z)	[treenode] 										at ([shift=({ -\NODEDX, -\NODEDY})]y)     	{$Z$ \\ 50};
	 \node (gl) [ground]                      at ([shift=({ -\NODEDX, -\NODEDY})]z)     	{ };
	 \node (gr) [ground]                      at ([shift=({  \NODEDX, -\NODEDY})]z)     	{ };

	 \node (sa) [subtree]                     at ([shift=({  \SUBTREEDX, -\SUBTREEDY})]r) {\Large $\alpha$};
	 \node (sb) [subtree]                     at ([shift=({ -\SUBTREEDX, -\SUBTREEDY})]s) {\Large $\beta$};
	 \node (sg) [subtree]                     at ([shift=({  \SUBTREEDX, -\SUBTREEDY})]t) {\Large $\gamma$};
	 \node (sd) [subtree]                     at ([shift=({ -\SUBTREEDX, -\SUBTREEDY})]u) {\Large $\delta$};
	 \node (ss) [subtree]                     at ([shift=({  \SUBTREEDX, -\SUBTREEDY})]v) {\Large $\sigma$};
	 \node (st) [subtree]                     at ([shift=({  \SUBTREEDX, -\SUBTREEDY})]w) {\Large $\tau$};
	 %% \node (sp) [subtree]                     at ([shift=({ -\SUBTREEDX, -\SUBTREEDY})]x) {\Large $\pi$};
	 \node (sp) [ground]                    	at ([shift=({ -\NODEDX, -\NODEDY})]x) { };
	 \node (sl) [subtree]                     at ([shift=({  \SUBTREEDX, -\SUBTREEDY})]y) {\Large $\lambda$};	

   \node (op) [label={right:{\large $op(40)^{z^{z^z}}$, $op(50)^{z^{z^z}}$}}] at ([shift=({0.375, 0})]y) {};
	
	 \path[every node/.style={font=\sffamily\small}]
	    (0, 1)  edge[->,very thick]  node {} (r)
		  (r)     edge[->,very thick]  node {} (s)
			(s)     edge[->,very thick]  node {} (t)
			(t)     edge[->,very thick]  node {} (u)
			(u)     edge[->,very thick]  node {} (v)
			(v)     edge[->,very thick]  node {} (w)
			(w)     edge[->,very thick]  node {} (x)
			(x)     edge[->,very thick]  node {} (y)
			(y)     edge[->,very thick]  node {} (z)
			(z)     edge[->]  node {} (gl)
			(z)     edge[->]  node {} (gr)
			(r)     edge[->]  node {} (sa.north)
			(s)     edge[->]  node {} (sb.north)
			(t)     edge[->]  node {} (sg.north)
			(u)     edge[->]  node {} (sd.north)
			(v)     edge[->]  node {} (ss.north)
			(w)     edge[->]  node {} (st.north)
			(x)     edge[->]  node {} (sp)
			(y)     edge[->]  node {} (sl.north);	
\end{tikzpicture}
}
%%%%%%%%%%%%%%
\\
%%%%%%%%%%%%%%
\subcaptionbox{Key 30 is deleted (simple delete); node $X$ is removed. \label{fig:movement|anchor|deleted}}[0.45\textwidth]
{

\begin{tikzpicture}[scale=\myfigurescaletwo, transform shape]
   
	 \newcommand\NODEDX{1.25}
	 \newcommand\NODEDY{1.25}
	 \newcommand\SUBTREEDX{1.5}
	 \newcommand\SUBTREEDY{0.75}
	
   \node (r)	[treenode] 		                at (0, 0)       		                      	{$R$ \\ 100};
   \node (s)	[treenode, fill=black!20] 		at ([shift=({ -\NODEDX, -\NODEDY})]r)     	{$S$ \\ 10};
	 \node (t)	[treenode] 		                at ([shift=({  \NODEDX, -\NODEDY})]s)    		{$T$ \\ 90};
	 \node (u)	[treenode, fill=black!20] 	  at ([shift=({ -\NODEDX, -\NODEDY})]t)     	{$U$ \\ 20};
	 \node (v)	[treenode] 										at ([shift=({  \NODEDX, -\NODEDY})]u)     	{$V$ \\ 80};
	 \node (w)	[treenode] 										at ([shift=({ -\NODEDX, -\NODEDY})]v)     	{$W$ \\ 70};
	 \node (x)	[treenode, fill=black!20, dotted] 		at ([shift=({ -\NODEDX, -\NODEDY})]w)     	{$X$ \\ 30};
	 \node (y)	[treenode] 										at ([shift=({  \NODEDX, -\NODEDY})]x)     	{$Y$ \\ 60};
	 \node (z)	[treenode] 										at ([shift=({ -\NODEDX, -\NODEDY})]y)     	{$Z$ \\ 50};
	 \node (gl) [ground]                      at ([shift=({ -\NODEDX, -\NODEDY})]z)     	{ };
	 \node (gr) [ground]                      at ([shift=({  \NODEDX, -\NODEDY})]z)     	{ };
		
	 \node (sa) [subtree]                     at ([shift=({  \SUBTREEDX, -\SUBTREEDY})]r) {\Large $\alpha$};
	 \node (sb) [subtree]                     at ([shift=({ -\SUBTREEDX, -\SUBTREEDY})]s) {\Large $\beta$};
	 \node (sg) [subtree]                     at ([shift=({  \SUBTREEDX, -\SUBTREEDY})]t) {\Large $\gamma$};
	 \node (sd) [subtree]                     at ([shift=({ -\SUBTREEDX, -\SUBTREEDY})]u) {\Large $\delta$};
	 \node (ss) [subtree]                     at ([shift=({  \SUBTREEDX, -\SUBTREEDY})]v) {\Large $\sigma$};
	 \node (st) [subtree]                     at ([shift=({  \SUBTREEDX, -\SUBTREEDY})]w) {\Large $\tau$};
	 %% \node (sp) [subtree]                     at ([shift=({ -\SUBTREEDX, -\SUBTREEDY})]x) {\Large $\pi$};
	 \node (sp) [ground]                    	at ([shift=({ -\NODEDX, -\NODEDY})]x) { };
	 \node (sl) [subtree]                     at ([shift=({  \SUBTREEDX, -\SUBTREEDY})]y) {\Large $\lambda$};	
	
	 \node (op) [label={right:{\large $op(40)^{z^{z^z}}$, $op(50)^{z^{z^z}}$}}] at ([shift=({0.375, 0})]y) {};
	
	 \path[every node/.style={font=\sffamily\small}]
	    (0, 1)  edge[->,very thick]  node {} (r)
		  (r)     edge[->,very thick]  node {} (s)
			(s)     edge[->,very thick]  node {} (t)
			(t)     edge[->,very thick]  node {} (u)
			(u)     edge[->,very thick]  node {} (v)
			(v)     edge[->,very thick]  node {} (w)
			%% (w)     edge[->]  node {} (x)
			(w)     edge[->,very thick]  node {} (y)
			(x)     edge[->]  node {} (y)
			(y)     edge[->,very thick]  node {} (z)
			(z)     edge[->]  node {} (gl)
			(z)     edge[->]  node {} (gr)
			(r)     edge[->]  node {} (sa.north)
			(s)     edge[->]  node {} (sb.north)
			(t)     edge[->]  node {} (sg.north)
			(u)     edge[->]  node {} (sd.north)
			(v)     edge[->]  node {} (ss.north)
			(w)     edge[->]  node {} (st.north)
			(x)     edge[->]  node {} (sp)
			(y)     edge[->]  node {} (sl.north);	
\end{tikzpicture}
}
%%%%%%%%%%%%%%
\qquad
%%%%%%%%%%%%%%
\subcaptionbox{Key 20 is deleted (complex delete); key 20 is replaced with key 50 in node $U$ and node $Z$ is removed. \label{fig:movement|key|moved}}[0.45\textwidth]
{

\begin{tikzpicture}[scale=\myfigurescaletwo, transform shape]
   
	 \newcommand\NODEDX{1.25}
	 \newcommand\NODEDY{1.25}
	 \newcommand\SUBTREEDX{1.5}
	 \newcommand\SUBTREEDY{0.75}
	
   \node (r)	[treenode] 		                at (0, 0)       		                      	{$R$ \\ 100};
   \node (s)	[treenode, fill=black!20] 		at ([shift=({ -\NODEDX, -\NODEDY})]r)     	{$S$ \\ 10};
	 \node (t)	[treenode] 		                at ([shift=({  \NODEDX, -\NODEDY})]s)    		{$T$ \\ 90};
	 \node (u)	[treenode, fill=black!20] 	  at ([shift=({ -\NODEDX, -\NODEDY})]t)     	{$U$ \\ 50};
	 \node (v)	[treenode] 										at ([shift=({  \NODEDX, -\NODEDY})]u)     	{$V$ \\ 80};
	 \node (w)	[treenode] 										at ([shift=({ -\NODEDX, -\NODEDY})]v)     	{$W$ \\ 70};
	 \node (x)	[treenode, fill=black!20, dotted] 		at ([shift=({ -\NODEDX, -\NODEDY})]w)     	{$X$ \\ 30};
	 \node (y)	[treenode] 										at ([shift=({  \NODEDX, -\NODEDY})]x)     	{$Y$ \\ 60};
	 \node (z)	[treenode, dotted] 						at ([shift=({ -\NODEDX, -\NODEDY})]y)     	{$Z$ \\ 50};
	 \node (gl) [ground]                      at ([shift=({ -\NODEDX, -\NODEDY})]z)     	{ };
	 \node (gr) [ground]                      at ([shift=({  \NODEDX, -\NODEDY})]z)     	{ };
		
	 \node (sa) [subtree]                     at ([shift=({  \SUBTREEDX, -\SUBTREEDY})]r) {\Large $\alpha$};
	 \node (sb) [subtree]                     at ([shift=({ -\SUBTREEDX, -\SUBTREEDY})]s) {\Large $\beta$};
	 \node (sg) [subtree]                     at ([shift=({  \SUBTREEDX, -\SUBTREEDY})]t) {\Large $\gamma$};
	 \node (sd) [subtree]                     at ([shift=({ -\SUBTREEDX, -\SUBTREEDY})]u) {\Large $\delta$};
	 \node (ss) [subtree]                     at ([shift=({  \SUBTREEDX, -\SUBTREEDY})]v) {\Large $\sigma$};
	 \node (st) [subtree]                     at ([shift=({  \SUBTREEDX, -\SUBTREEDY})]w) {\Large $\tau$};
	 %% \node (sp) [subtree]                     at ([shift=({ -\SUBTREEDX, -\SUBTREEDY})]x) {\Large $\pi$};
	 \node (sp) [ground]                    	at ([shift=({ -\NODEDX, -\NODEDY})]x) { };
	 \node (sl) [subtree]                     at ([shift=({  \SUBTREEDX, -\SUBTREEDY})]y) {\Large $\lambda$};	
	
	 \node (op) [label={right:{\large $op(40)^{z^{z^z}}$, $op(50)^{z^{z^z}}$}}] at ([shift=({0.375, 0})]y) {};
	
	 \path[every node/.style={font=\sffamily\small}]
	    (0, 1)  edge[->,very thick]  node {} (r)
		  (r)     edge[->,very thick]  node {} (s)
			(s)     edge[->,very thick]  node {} (t)
			(t)     edge[->,very thick]  node {} (u)
			(u)     edge[->,very thick]  node {} (v)
			(v)     edge[->,very thick]  node {} (w)
			%% (w)     edge[->]  node {} (x)
			(w)     edge[->,very thick]  node {} (y)
			(x)     edge[->]  node {} (y)
			%% (y)     edge[->]  node {} (z)
			(y)     edge[->, very thick]  node {} (gr)
			(z)     edge[->]  node {} (gl)
			(z)     edge[->]  node {} (gr)
			(r)     edge[->]  node {} (sa.north)
			(s)     edge[->]  node {} (sb.north)
			(t)     edge[->]  node {} (sg.north)
			(u)     edge[->]  node {} (sd.north)
			(v)     edge[->]  node {} (ss.north)
			(w)     edge[->]  node {} (st.north)
			(x)     edge[->]  node {} (sp)
			(y)     edge[->]  node {} (sl.north);		
\end{tikzpicture}
}
%}
%%
\caption{An illustration of a key moving up the tree}
%\caption{An illustration of a key moving up the tree while operations $op(40)$ and $op(50)$ are suspended during their traversal. The \accesspath with respect to $op(40)$ (as well as $op(50)$) is shown with thick edges. Shaded nodes denote the \myanchor{} nodes on this path. Nodes with dotted border denote the marked nodes.}
\label{fig:movement}
\end{figure}
As mentioned earlier, every operation on a BST involves first traversing the tree from top to down starting from the root node and following either the left or the right child pointer until either the target key is found or a null pointer is encountered (termination condition). Depending on the outcome of the traversal and the type of the operation, the 
tree may then need to be modified to actually realize the operation. We refer to the period during which the tree is being traversed as \emph{seek phase}. Further, we refer to the period during which the tree is being modified as \emph{execution phase}.

During the seek phase, the target key may move from its current location to a new location up the tree. As a result, the traversal may miss the key both at its old location as well as its new location. For an illustration, see \figref{movement}. In the illustration, key 50 has moved up by five nodes. In most concurrent BST algorithms, if it is suspected that the key may have moved up the tree, then the traversal is simply restarted from the root node. Different algorithms use different approaches to detect possible key movement. For example, in~\cite{HowJon:2012:SPAA}, the traversal is restarted if the \emph{last right-turn} node is detected to have undergone some change. For example, in \figref{movement|suspended}, the last right-turn node for the operation $op(50)$ is node $X$. On reaching the terminal node in \figref{movement|key|moved}, after resuming running, $op(50)$ needs to restart since $X$ has since been removed from the tree. 

A \emph{re}-traversal of the tree may also be required if the operation encounters any failure during the execution phase. For example, in~\cite{RamMit:2015:PPoPP}, which is a lock-based algorithm, execution phase is aborted if, after locking the relevant edges, the validation step fails. This happens if the portion of the tree that lies within the ``operation's window'', which typically consists of a small constant number of nodes, has undergone some change since it was last observed. In that case, the operation 
moves to the seek phase again. 

In most concurrent BST algorithms, (a single instance of) the execution phase of an operation typically tends to have constant time complexity. The seek phase is where an operation may end up spending most of its time especially if the tree is large. Hence, it is desirable to make the seek phase of an operation more efficient by:
%%
\begin{enumerate*}[label=(\roman*)]
\item reducing the number of restarts due to ``suspected'' key movement, and
\item restarting the traversal from a point ``close'' to the operation's window.
\end{enumerate*}
%%
This leads to two separate but related questions that any local recovery algorithm needs to address. First, ``If a key is not found, then does the traversal need to restart?''. Second, ``If the traversal needs to be restarted, then from which node should the traversal restart?''

Consider an operation $\alpha$ currently traversing the tree; let $\storedpath(\alpha)$ denote the path taken by $\alpha$ so far. For example, in \figref{movement|suspended}, considering only the subtree shown in the figure, $\storedpath(op(50)) = \ang{R, S, T, U, V, W, X, Y}$. At each node in the path (except the last node), $\alpha$ either followed 
the left or the right child pointer. We say that a node in the path is an \emph{\myanchor} node if the operation followed its \emph{right} child pointer; otherwise we say that is a 
\emph{non-\myanchor{}} node. For example, in the path $\storedpath(op(50))$, nodes $S$, $U$ and $X$ are \myanchor{} nodes, whereas nodes $R$, $T$, $V$, $W$ and $Y$ 
are non-\myanchor{} nodes. We assume that the first \myanchor{} node in a traversal path is always a \emph{sentinel} node that is never marked (not shown in the figure). Of course, as an operation is traversing the tree, the tree may undergo changes as a result of which the path taken by the operation may no longer be correct. For example, in \figref{movement|key|moved}, the new \accesspath of $op(50)$ in the subtree, which is obtained from the subtree in \figref{movement|suspended} after applying several delete operations, is now given 
by $\ang{R, S, T, U}$. 

Note that, since in a complex delete operation we assume that the key being deleted is replaced with its successor key, the value of a key stored in a node can only increase. Therefore, the child pointer followed by an operation at a node, if it is still part of the \accesspath, may change (from right to left) for an \myanchor{} node but cannot change for a non-\myanchor{} node. For example, as shown in \figref{movement|suspended}, the node $U$ is an \myanchor{} node for $op(40)$. But due to the changes made to the tree, the key at $U$ has now become 50, as shown in \figref{movement|key|moved}. Hence, the pointer that $op(40)$ now needs to follow at $U$ is left and not right. We say that an \myanchor
{} node is \emph{\myconsistent} with respect to an operation if its key is still less than the operation's key; otherwise, we say that it is \emph{\myinconsistent}. For example, in \figref{movement|key|moved}, \myanchor{} nodes $S$ and $X$ are still \myconsistent{} with respect to $op(40)$ but node $U$ has become \myinconsistent{} with respect to $op(40)$. 

Clearly, in case an operation needs to restart, \emph{no node} in the path \emph{after an \myinconsistent{} \myanchor{} node} in general can serve as a restart point since the path taken and the path that needs to be taken may now diverge. This implies that, to find a restart point, a local recovery algorithm should locate the \emph{shallowest} \myinconsistent{} \myanchor{} node in the path (with the least depth) and discard the suffix of the path after such a node. Moreover, a restart point has to be a node that is still a part of the tree. 

This leads to the following approach to find a restart point for an operation $\alpha$ when needed. Find a node $C$ in the path taken so far by $\alpha$ such that the following two conditions hold. First, $C$ is not marked. Second, every \myanchor{} node in the path preceding $C$ is \myconsistent{} with respect to $\alpha$. To check for the second condition, it is not necessary to examine every \myanchor{} node in the path preceding $C$ as stated in the following lemma. 

\begin{lemma}
\label{lem:consistent}
Consider an operation $\alpha$ and let $\storedpath(\alpha)$ denote the path 
taken by $\alpha$ when traversing the tree. 
Let $A$ be an \myanchor{} node in $\storedpath(\alpha)$ and let 
$\sigma = A_0, A_1, \ldots, A_k$, where $A_k = A$, 
denote the sequence of \myanchor{} nodes in $\storedpath(\alpha)$ up 
to and including $A$. Then, if $A$ is unmarked and \myconsistent{} with 
respect to $\alpha$, then, for every $i$ with $0 \leq i \leq k$, $A_i$ is 
also \myconsistent{} with respect to $\alpha$. Moreover, the \accesspath of $\alpha$ in the current tree includes $A$.
\end{lemma}

We say that an \myanchor{} node is \emph{\mycritical{}} with respect to a node $C$ in the path if it is the \emph{closest} preceding \myanchor{} node to $C$ that is also unmarked. For example, in \figref{movement|anchor|deleted}, the \mycritical{} \myanchor{} node with respect to node $Y$ is node $U$ since node $X$ is marked. Using the above lemma, the second condition can now be replaced with the following: the \mycritical{} \myanchor{} node with respect to $C$ in the path, say $A$, as well as every \myanchor{} node in the path that lies between $A$ and $C$, which will be marked, is \myconsistent{} with respect to $\alpha$.

The next lemma states a useful  property about an \myinconsistent{} \myanchor{} node.

\begin{lemma}
\label{lem:inconsistent}
Consider an operation $\alpha$ with target key $k$ and let $\storedpath(\alpha)$ denote the path 
taken by $\alpha$ when traversing the tree. Let $A$ be an \myanchor{} node in $\storedpath(\alpha)$. Assume that $A$ is now \myinconsistent{} with respect to $\alpha$. Then, at the time $A$ became \myinconsistent{}, the tree did not contain $k$.
\end{lemma}

To see why the above lemma holds, let $k_{old}$ ($k_{new}$) be the key stored in $A$ just before (after) $A$ became \myinconsistent{}. Clearly, $k_{old} < k < k_{new}$. Let $t$ denote the time just after which $k_{old}$ was replaced with $k_{new}$ in $A$. Note that $k_{new}$ must be the next smallest key in the right subtree of $A$ at time $t$. This implies that the right subtree of $A$ did not contain $k$ at time $t$. Further, from \lemref{consistent}, $A$ is on the \accesspath of $\alpha$ at time $t$. Hence, we can conclude that the tree does not contain $k$ at time $t$. Note that, if a key is not present in the tree at some point while a search/delete operation is in progress, it is acceptable for the operation to say that key was not found. In this case, the operation will be linearized after the delete operation that removed the key from the tree.

When an operation fails to find the target key after traversing the tree from top to bottom, it examines the path it took to check whether or not the key has moved up the tree and/or a re-traversal is required. To that end, it examines the \myanchor{} nodes in the reverse order in which they were visited, starting from the one closest to the terminal node. We now discuss the behaviour of each operation one-by-one.


\paragraph{Search Operation:} 
\begin{figure}[htp]
\centering
{
	\begin{tikzpicture}[scale=0.5, transform shape]
	\node (p0) [] {search(K)};
	\node (p1) [process, below of=p0, text width=4cm] {Do a binary search for key K in the tree};
	\node (p2) [process, below of=p1, yshift=-1.5cm, text width=4.5cm] {Examine anchor node $A$ of top entry in the stack};
	\node (p3) [decision, below of=p2, yshift=-1.5cm, text width=2cm] {is anchor node marked?};
	\node (p4) [process, right of=p3, xshift=4cm, text width=4.5cm] {pop all entries upto anchor node $A$};
	\node (retT) [process, right of=p1, xshift=4cm, text width=1cm, minimum width=1cm] {return true};
	\node (retF) [process, left of=p2, xshift=-5cm, text width=1cm, minimum width=1cm] {return false};

	\draw [arrow] (p1) -- node[anchor=west] {K not found} (p2);
	\draw [arrow] (p1) -- node[anchor=south] {K found} (retT);
	\draw [arrow] (p2.east) -| node[anchor=north,pos=0.5] {K = A.key}    (retT.south);
	\draw [arrow] (p2.west) -- node (temp) [anchor=north,pos=0.5] {K $<$ A.key}  (retF.east);
	\draw [arrow] (p2) -- node[anchor=east] {K $>$ A.key} (p3);
	\draw [arrow] (p3.west) -| (retF.south) node[below, pos=0.5]{No}  (retF.south);
	\draw [arrow] (p3.east) -- node[anchor=north]{Yes} (p4.west);
	\draw [arrow] (p4.north) -- (p2.south);
	\node (temp1) [above of=temp,xshift=0.5cm,yshift=0.4cm] {(A.oldKey $<$ K $<$ A.newKey)};
	\pause
	\visible<2>
	{
		\node (p1) [process, below of=p0, text width=4cm,fill=black!20] {Do a binary search for key K in the tree};
	}
	\pause
	\visible<3>
	{
		\node (retT) [process, right of=p1, xshift=4cm, text width=1cm, minimum width=1cm,fill=black!20] {return true};
	}
	\pause
	\visible<4,8>
	{
	\node (p2) [process, below of=p1, yshift=-1.5cm, text width=4.5cm,fill=black!20] {Examine anchor node $A$ of top entry in the stack};
	}
	\pause
	\visible<5>
	{
	\node (p3) [decision, below of=p2, yshift=-1.5cm, text width=2cm,fill=black!20] {is anchor node marked?};
	}
		\pause
	\visible<6>
	{
	\node (retF) [process, left of=p2, xshift=-5cm, text width=1cm, minimum width=1cm,fill=black!20] {return false};
	}
	\pause
	\visible<7>
	{
	\node (p4) [process, right of=p3, xshift=4cm, text width=4.5cm,fill=black!20] {pop all entries upto anchor node $A$};
	}
	\pause
	\end{tikzpicture}
}
\caption{Sequence of steps in a search operation}
\end{figure}
\begin{figure}[htp]
\subcaptionbox{Operation $op(50)$ starting at R and suspended at Y along with the stack\label{fig:searchStack|a}}[0.45\textwidth]
{
\begin{tikzpicture}[scale=0.5, transform shape] 
	 \newcommand\NODEDX{1.25}
	 \newcommand\NODEDY{1.25}
	 \newcommand\SUBTREEDX{1.5}
	 \newcommand\SUBTREEDY{0.75}
	
   \node (r)	[treenode, fill=black!20] 		at (0, 0)       		                      	{$R$ \\  -$\infty$};
   \node (s)	[treenode] 										at ([shift=({ \NODEDX, -\NODEDY})]r)     	  {$S$ \\  $\infty$};
	 \node (t)	[treenode] 		                at ([shift=({  -\NODEDX, -\NODEDY})]s)    	{$T$ \\ 90};
	 \node (u)	[treenode, fill=black!20] 	  at ([shift=({ -\NODEDX, -\NODEDY})]t)     	{$U$ \\ 20};
	 \node (v)	[treenode] 										at ([shift=({  \NODEDX, -\NODEDY})]u)     	{$V$ \\ 80};
	 \node (w)	[treenode] 										at ([shift=({ -\NODEDX, -\NODEDY})]v)     	{$W$ \\ 70};
	 \node (x)	[treenode, fill=black!20] 		at ([shift=({ -\NODEDX, -\NODEDY})]w)     	{$X$ \\ 30};
	 \node (y)	[treenode] 										at ([shift=({  \NODEDX, -\NODEDY})]x)     	{$Y$ \\ 60};
	 \node (z)	[treenode] 										at ([shift=({ -\NODEDX, -\NODEDY})]y)     	{$Z$ \\ 50};
	 \node (gl) [ground]                      at ([shift=({ -\NODEDX, -\NODEDY})]z)     	{ };
	 \node (gr) [ground]                      at ([shift=({  \NODEDX, -\NODEDY})]z)     	{ };
		
	 \node (sa) [ground]                      at ([shift=({ -\SUBTREEDX, -\SUBTREEDY})]r) { };
	 \node (sb) [ground]                      at ([shift=({ \SUBTREEDX, -\SUBTREEDY})]s) 	{ };
	 \node (sg) [subtree]                     at ([shift=({  \SUBTREEDX, -\SUBTREEDY})]t) {\Large $\gamma$};
	 \node (sd) [subtree]                     at ([shift=({ -\SUBTREEDX, -\SUBTREEDY})]u) {\Large $\delta$};
	 \node (ss) [subtree]                     at ([shift=({  \SUBTREEDX, -\SUBTREEDY})]v) {\Large $\sigma$};
	 \node (st) [subtree]                     at ([shift=({  \SUBTREEDX, -\SUBTREEDY})]w) {\Large $\tau$};
	 \node (sp) [subtree]                     at ([shift=({ -\SUBTREEDX, -\SUBTREEDY})]x) {\Large $\pi$};
	 \node (sl) [subtree]                     at ([shift=({  \SUBTREEDX, -\SUBTREEDY})]y) {\Large $\lambda$};	
	
	 \node (op) [label={right:{\large $op(50) is here$}}] at ([shift=({0.5, 0})]y) {};
	
	 \path[every node/.style={font=\sffamily\small}]
	    %(0, 1)  edge[->,very thick]  node {} (r)
		  (r)     edge[->,very thick]  node {} (s)
			(s)     edge[->,very thick]  node {} (t)
			(t)     edge[->,very thick]  node {} (u)
			(u)     edge[->,very thick]  node {} (v)
			(v)     edge[->,very thick]  node {} (w)
			(w)     edge[->,very thick]  node {} (x)
			(x)     edge[->,very thick]  node {} (y)
			(y)     edge[->,very thick]  node {} (z)
			(z)     edge[->]  node {} (gl)
			(z)     edge[->]  node {} (gr)
			(r)     edge[->]  node {} (sa)
			(s)     edge[->]  node {} (sb)
			(t)     edge[->]  node {} (sg.north)
			(u)     edge[->]  node {} (sd.north)
			(v)     edge[->]  node {} (ss.north)
			(w)     edge[->]  node {} (st.north)
			(x)     edge[->]  node {} (sp.north)
			(y)     edge[->]  node {} (sl.north);		
\end{tikzpicture}
\quad
\begin{tikzpicture}[scale=0.45, transform shape]
  \stacktop{} \cellptr{top}
	\separator
	\cell{\Large \texttt{Y,X}}        \cellcomL{7} \coordinate () at (currentcell.east);
  \separator
	\cell{\Large \texttt{X,U}}        \cellcomL{6} \coordinate () at (currentcell.east);
  \separator
	\cell{\Large \texttt{W,U}}        \cellcomL{5} \coordinate () at (currentcell.east);
  \separator
	\cell{\Large \texttt{V,U}}        \cellcomL{4} \coordinate () at (currentcell.east);
  \separator
	\cell{\Large \texttt{U,R}}        \cellcomL{3} \coordinate () at (currentcell.east);
  \separator
	\cell{\Large \texttt{T,R}}        \cellcomL{2} \coordinate () at (currentcell.east);
  \separator
	\cell{\Large \texttt{S,R}}        \cellcomL{1} \coordinate () at (currentcell.east);
  \separator
	\cell{\Large \texttt{R,null}}     \cellcomL{0} \coordinate () at (currentcell.east);
  \separator
\end{tikzpicture}
}
\subcaptionbox{Key 30 is deleted;key 20 is deleted $\&$ replaced with key 50 in node $U$ and node $Z$ is removed\label{fig:searchStack|b}}[0.45\textwidth]
{
\begin{tikzpicture}[scale=0.5, transform shape]
   
	 \newcommand\NODEDX{1.25}
	 \newcommand\NODEDY{1.25}
	 \newcommand\SUBTREEDX{1.5}
	 \newcommand\SUBTREEDY{0.75}
	
	 \node (r)	[treenode, fill=black!20] 		at (0, 0)       		                      	{$R$ \\  -$\infty$};
   \node (s)	[treenode] 										at ([shift=({ \NODEDX, -\NODEDY})]r)     	  {$S$ \\  $\infty$};
	 \node (t)	[treenode] 		                at ([shift=({  -\NODEDX, -\NODEDY})]s)    	{$T$ \\ 90};
	 \node (u)	[treenode, fill=black!20] 	  at ([shift=({ -\NODEDX, -\NODEDY})]t)     	{$U$ \\ 50};
	 \node (v)	[treenode] 										at ([shift=({  \NODEDX, -\NODEDY})]u)     	{$V$ \\ 80};
	 \node (w)	[treenode] 										at ([shift=({ -\NODEDX, -\NODEDY})]v)     	{$W$ \\ 70};
	 \node (x)	[treenode, fill=black!20, dotted] 		at ([shift=({ -\NODEDX, -\NODEDY})]w)     	{$X$ \\ 30};
	 \node (y)	[treenode] 										at ([shift=({  \NODEDX, -\NODEDY})]x)     	{$Y$ \\ 60};
	 \node (z)	[treenode, dotted] 						at ([shift=({ -\NODEDX, -\NODEDY})]y)     	{$Z$ \\ 50};
	 \node (gl) [ground]                      at ([shift=({ -\NODEDX, -\NODEDY})]z)     	{ };
	 \node (gr) [ground]                      at ([shift=({  \NODEDX, -\NODEDY})]z)     	{ };
		
	 \node (sa) [ground]                      at ([shift=({ -\SUBTREEDX, -\SUBTREEDY})]r) { };
	 \node (sb) [ground]                      at ([shift=({ \SUBTREEDX, -\SUBTREEDY})]s) 	{ };
	 \node (sg) [subtree]                     at ([shift=({  \SUBTREEDX, -\SUBTREEDY})]t) {\Large $\gamma$};
	 \node (sd) [subtree]                     at ([shift=({ -\SUBTREEDX, -\SUBTREEDY})]u) {\Large $\delta$};
	 \node (ss) [subtree]                     at ([shift=({  \SUBTREEDX, -\SUBTREEDY})]v) {\Large $\sigma$};
	 \node (st) [subtree]                     at ([shift=({  \SUBTREEDX, -\SUBTREEDY})]w) {\Large $\tau$};
	 %% \node (sp) [subtree]                     at ([shift=({ -\SUBTREEDX, -\SUBTREEDY})]x) {\Large $\pi$};
	 \node (sp) [ground]                    	at ([shift=({ -\NODEDX, -\NODEDY})]x) { };
	 \node (sl) [subtree]                     at ([shift=({  \SUBTREEDX, -\SUBTREEDY})]y) {\Large $\lambda$};	
	
	 \node (op) [label={right:{\large $op(50) is here$}}] at ([shift=({0.375, 0})]gr) {};
	
	 \path[every node/.style={font=\sffamily\small}]
	    %(0, 1)  edge[->,very thick]  node {} (r)
		  (r)     edge[->,very thick]  node {} (s)
			(s)     edge[->,very thick]  node {} (t)
			(t)     edge[->,very thick]  node {} (u)
			(u)     edge[->,very thick]  node {} (v)
			(v)     edge[->,very thick]  node {} (w)
			%% (w)     edge[->]  node {} (x)
			(w)     edge[->,very thick]  node {} (y)
			(x)     edge[->]  node {} (y)
			%% (y)     edge[->]  node {} (z)
			(y)     edge[->, very thick]  node {} (gr)
			(z)     edge[->]  node {} (gl)
			(z)     edge[->]  node {} (gr)
			(r)     edge[->]  node {} (sa)
			(s)     edge[->]  node {} (sb)
			(t)     edge[->]  node {} (sg.north)
			(u)     edge[->]  node {} (sd.north)
			(v)     edge[->]  node {} (ss.north)
			(w)     edge[->]  node {} (st.north)
			(x)     edge[->]  node {} (sp)
			(y)     edge[->]  node {} (sl.north);		
\end{tikzpicture}
\quad
\begin{tikzpicture}[scale=0.45, transform shape]
  \stacktop{} \cellptr{top}
	\separator
	\cell{\Large \texttt{Y,X}}        \cellcomL{7} \coordinate () at (currentcell.east);
  \separator
	\cell{\Large \texttt{X,U}}        \cellcomL{6} \coordinate () at (currentcell.east);
  \separator
	\cell{\Large \texttt{W,U}}        \cellcomL{5} \coordinate () at (currentcell.east);
  \separator
	\cell{\Large \texttt{V,U}}        \cellcomL{4} \coordinate () at (currentcell.east);
  \separator
	\cell{\Large \texttt{U,R}}        \cellcomL{3} \coordinate () at (currentcell.east);
  \separator
	\cell{\Large \texttt{T,R}}        \cellcomL{2} \coordinate () at (currentcell.east);
  \separator
	\cell{\Large \texttt{S,R}}        \cellcomL{1} \coordinate () at (currentcell.east);
  \separator
	\cell{\Large \texttt{R,null}}     \cellcomL{0} \coordinate () at (currentcell.east);
  \separator
\end{tikzpicture}
}
\subcaptionbox{Pop upto marked anchor node $X$. Top of stack is now $W$. Examine anchor node $U$\label{fig:searchStack|c}}[0.45\textwidth]
{
\begin{tikzpicture}[scale=0.5, transform shape]
   
	 \newcommand\NODEDX{1.25}
	 \newcommand\NODEDY{1.25}
	 \newcommand\SUBTREEDX{1.5}
	 \newcommand\SUBTREEDY{0.75}
	
	 \node (r)	[treenode, fill=black!20] 		at (0, 0)       		                      	{$R$ \\  -$\infty$};
   \node (s)	[treenode] 										at ([shift=({ \NODEDX, -\NODEDY})]r)     	  {$S$ \\  $\infty$};
	 \node (t)	[treenode] 		                at ([shift=({  -\NODEDX, -\NODEDY})]s)    	{$T$ \\ 90};
	 \node (u)	[treenode, fill=black!20] 	  at ([shift=({ -\NODEDX, -\NODEDY})]t)     	{$U$ \\ 50};
	 \node (v)	[treenode] 										at ([shift=({  \NODEDX, -\NODEDY})]u)     	{$V$ \\ 80};
	 \node (w)	[treenode] 										at ([shift=({ -\NODEDX, -\NODEDY})]v)     	{$W$ \\ 70};
	 \node (x)	[treenode, fill=black!20, dotted] 		at ([shift=({ -\NODEDX, -\NODEDY})]w)     	{$X$ \\ 30};
	 \node (y)	[treenode] 										at ([shift=({  \NODEDX, -\NODEDY})]x)     	{$Y$ \\ 60};
	 \node (z)	[treenode, dotted] 						at ([shift=({ -\NODEDX, -\NODEDY})]y)     	{$Z$ \\ 50};
	 \node (gl) [ground]                      at ([shift=({ -\NODEDX, -\NODEDY})]z)     	{ };
	 \node (gr) [ground]                      at ([shift=({  \NODEDX, -\NODEDY})]z)     	{ };
		
	 \node (sa) [ground]                      at ([shift=({ -\SUBTREEDX, -\SUBTREEDY})]r) { };
	 \node (sb) [ground]                      at ([shift=({ \SUBTREEDX, -\SUBTREEDY})]s) 	{ };
	 \node (sg) [subtree]                     at ([shift=({  \SUBTREEDX, -\SUBTREEDY})]t) {\Large $\gamma$};
	 \node (sd) [subtree]                     at ([shift=({ -\SUBTREEDX, -\SUBTREEDY})]u) {\Large $\delta$};
	 \node (ss) [subtree]                     at ([shift=({  \SUBTREEDX, -\SUBTREEDY})]v) {\Large $\sigma$};
	 \node (st) [subtree]                     at ([shift=({  \SUBTREEDX, -\SUBTREEDY})]w) {\Large $\tau$};
	 %% \node (sp) [subtree]                     at ([shift=({ -\SUBTREEDX, -\SUBTREEDY})]x) {\Large $\pi$};
	 \node (sp) [ground]                    	at ([shift=({ -\NODEDX, -\NODEDY})]x) { };
	 \node (sl) [subtree]                     at ([shift=({  \SUBTREEDX, -\SUBTREEDY})]y) {\Large $\lambda$};	
	
	 \node (op) [label={left:{\large $op(50) is here$}}] at ([shift=({-0.5, 0})]w) {};
	
	 \path[every node/.style={font=\sffamily\small}]
	    %(0, 1)  edge[->,very thick]  node {} (r)
		  (r)     edge[->,very thick]  node {} (s)
			(s)     edge[->,very thick]  node {} (t)
			(t)     edge[->,very thick]  node {} (u)
			(u)     edge[->,very thick]  node {} (v)
			(v)     edge[->,very thick]  node {} (w)
			%% (w)     edge[->]  node {} (x)
			(w)     edge[->,very thick]  node {} (y)
			(x)     edge[->]  node {} (y)
			%% (y)     edge[->]  node {} (z)
			(y)     edge[->, very thick]  node {} (gr)
			(z)     edge[->]  node {} (gl)
			(z)     edge[->]  node {} (gr)
			(r)     edge[->]  node {} (sa)
			(s)     edge[->]  node {} (sb)
			(t)     edge[->]  node {} (sg.north)
			(u)     edge[->]  node {} (sd.north)
			(v)     edge[->]  node {} (ss.north)
			(w)     edge[->]  node {} (st.north)
			(x)     edge[->]  node {} (sp)
			(y)     edge[->]  node {} (sl.north);		
\end{tikzpicture}
\quad
\begin{tikzpicture}[scale=0.45, transform shape]
  \stacktop{} \cellptr{top}
	\separator
	\cell{\Large \texttt{W,U}}        \cellcomL{5} \coordinate () at (currentcell.east);
  \separator
	\cell{\Large \texttt{V,U}}        \cellcomL{4} \coordinate () at (currentcell.east);
  \separator
	\cell{\Large \texttt{U,R}}        \cellcomL{3} \coordinate () at (currentcell.east);
  \separator
	\cell{\Large \texttt{T,R}}        \cellcomL{2} \coordinate () at (currentcell.east);
  \separator
	\cell{\Large \texttt{S,R}}        \cellcomL{1} \coordinate () at (currentcell.east);
  \separator
	\cell{\Large \texttt{R,null}}     \cellcomL{0} \coordinate () at (currentcell.east);
  \separator
\end{tikzpicture}
}
\caption{An illustration of local recovery for a search operation}
\label{fig:searchStack}
\end{figure}
A search operation \emph{does not} need to restart. When it examines an \myanchor{} node as mentioned above, there are three possibilities. First, if the \myanchor{} node's key matches the target key, then the key has been found and the operation terminates. Second, if the \myanchor{} node's key is greater than the target key (the \myanchor{} node has become \myinconsistent{}), then the operation concludes that the key is not  present in the tree and terminates. Finally, if the \myanchor{} node's key is smaller than the target key (the \myanchor{} node is still \myconsistent{}), then the operation terminates if the node is not marked; otherwise it moves to the preceding \myanchor{} node and repeats the comparison. \Figref{localRecoverySearch} gives an overview of the steps involved in a search operation. \Figref{searchStack} shows an example of how a stack is used to do local recovery. Each entry in the stack has the node encountered in the traversal along with its anchor node.



\paragraph{Insert Operation:} 
\begin{figure}[htp]
\centering
{
	\begin{tikzpicture}
	\node (p0) [] {insert(K)};
	\node (p1) [process, below of=p0, text width=4cm] {Do a binary search for key K in the tree};
	\node (p2) [process, below of=p1, yshift=-1.5cm, text width=4.5cm] {Examine anchor node $A$ of top entry in the stack};
	\node (p3) [decision, below of=p2, yshift=-1.5cm, text width=2cm] {is anchor node marked?};
	\node (p4) [process, right of=p3, xshift=4cm, text width=4.5cm] {pop all entries upto anchor node $A$};
	\node (retF) [process, right of=p1, xshift=4cm, text width=1cm, minimum width=1cm] {return false};
	\node (retT) [process, left of=p2, xshift=-6cm, text width=4.5cm, minimum width=1cm] {discard suffix of the path after anchor node and find a restart point};
	\node (p5) [process, left of=p3, xshift=-6cm, text width=4.5cm, minimum width=1cm] {traversal terminates. Terminal node is returned as the injection point};

	\draw [arrow] (p1) -- node[anchor=west] {K not found} (p2);
	\draw [arrow] (p1) -- node[anchor=south] {K found} (retF);
	\draw [arrow] (p2.east) -| node[anchor=north,pos=0.5] {K = A.key}    (retF.south);
	\draw [arrow] (p2.west) -- node[anchor=north,pos=0.5] {K $<$ A.key}  (retT.east);
	\draw [arrow] (p2) -- node[anchor=east] {K $>$ A.key} (p3);
	\draw [arrow] (p3.west)  -- node[anchor=north]{No}  (p5.east);
	\draw [arrow] (p3.east) -- node[anchor=north]{Yes} (p4.west);
	\draw [arrow] (p4.north) -- (p2.south);

	\end{tikzpicture}
}
\caption{Sequence of steps in an insert operation}
\label{fig:localRecoveryInsert}
\end{figure}
An insert operation needs to restart only if one of the \myanchor{} nodes in the path has become \myinconsistent{}. When it examines an \myanchor{} node as mentioned above, there are three possibilities. First, if the \myanchor{} node's key matches the target key, then the key has been found and the operation terminates. Second, if the \myanchor{} node's key is greater than the target key (the \myanchor{} node has become \myinconsistent{}), then it discards the suffix of the path after the \myanchor{} node and restarts the traversal from a restart point. Finally, if the \myanchor{} node's key is smaller than the target key (the \myanchor{} node is still \myconsistent{}), then the traversal terminates if the node is not marked (the terminal node is returned  as the injection point); otherwise it moves to the preceding \myanchor{} node and repeats the comparison. \Figref{localRecoveryInsert} gives an overview of the steps involved in an insert operation.

\paragraph{Delete Operation:} 
\begin{figure}
\centering
{
	\begin{tikzpicture}[scale=0.9,transform shape]
	\node (p0) [] {delete(K)};
	\node (p1) [process, below of=p0, text width=4cm] {Do a binary search for key K in the tree};
	\node (p2) [process, below of=p1, yshift=-1.5cm, text width=4.5cm] {Examine anchor node $A$ of top entry in the stack};
	\node (p3) [decision, below of=p2, yshift=-1.5cm, text width=2cm] {is anchor node marked?};
	\node (p4) [process, right of=p3, xshift=4cm, text width=4.5cm] {pop all entries upto anchor node $A$};
	\node (ex) [process, right of=p1, xshift=6cm, text width=4.5cm, minimum width=1cm] {go to execution phase};
	\node (retF) [process, left of=p2, xshift=-4.5cm, text width=1.2cm, minimum width=1cm] {return false};

	\draw [arrow] (p1) -- node[anchor=west] {K not found} (p2);
	\draw [arrow] (p1) -- node[anchor=south] {K found} (ex);
	\draw [arrow] (p2.east) -| node[anchor=north,pos=0.5] {K = A.key}    (ex.south);
	\draw [arrow] (p2.west) -- node[anchor=north,pos=0.5] {K $<$ A.key}  (retF.east);
	\draw [arrow] (p2) -- node[anchor=east] {K $>$ A.key} (p3);
	\draw [arrow] (p3.west)  -| node[anchor=north]{No}  (retF.south);
	\draw [arrow] (p3.east) -- node[anchor=north]{Yes} (p4.west);
	\draw [arrow] (p4.north) -- (p2.south);

	\end{tikzpicture}
}
\caption{Sequence of steps in a delete operation}
\label{fig:localRecoveryDelete}
\end{figure}
A delete operation also \emph{does not} need to restart except when there is a failure in the execution phase. When it examines an \myanchor{} node as mentioned above, there are three possibilities. First, if the \myanchor{} node's key matches the target key, then the key has been found and the operation moves to the execution phase. Second, if the \myanchor{} node's key is greater than the target key (the \myanchor{} node has become \myinconsistent{}), then the operation concludes that the key is not present in the tree and terminates. Finally, if the \myanchor{} node's key is smaller than the target key (the \myanchor{} node is still \myconsistent{}), then the traversal terminates if the node is not marked; otherwise it moves to the preceding \myanchor{} node and repeats the comparison. \Figref{localRecoveryDelete} gives an overview of the steps involved in a delete operation.