\begin{frame}[c]{Lock Based BST[PPoPP'15 Poster]}
Contributions
\begin{itemize}
\item combine edge-based locking with internal representation of BST
\item optimistic tree traversal 
\end{itemize}
\end{frame}

\begin{frame}[c]{Lock Based BST[PPoPP'15 Poster]}
\begin{itemize}
\item common workloads have more searches than updates
\begin{itemize}
\item design is optimized for searches
\item search operations are oblivious to locks
\end{itemize}
\pause
\item Any real life workload will have more inserts than deletes
\begin{itemize}
\item insert operations do not obtain any locks
\item performs only one atomic operation
\end{itemize}
\pause
\item removal of a node in a concurrent BST is challenging
\begin{itemize}
\item delete operations uses locks
\item locks can be obtained on nodes or edges
\item locking edges instead of nodes increases concurrency
\end{itemize}
\end{itemize}
\end{frame}

\begin{frame}{Lock Based BST - Challenges in search}
\begin{tikzpicture}[scale=0.55, transform shape,mylabel/.style={thin, draw=black, align=center, minimum width=0.3cm, minimum height=0.3cm,fill=white}]
		\newcommand\XA{0}
		\newcommand\YA{0}
		\node (x)		[treenode] 									at (-6,0)       							{$U$ \\ 50};
		\node (a)		[subtree] 									at (-7.5,-1)  								{\Large $\alpha$};
		\node (y0)	[treenode] 									at (-4,-1.5) 									{$V$ \\ 100};
		\node (y1)	[treenode]									at (-5.5,-3)									{$W$ \\ 60}; 
		\node (b)		[subtree] 									at (-3,-2.5)    							{\Large $\beta$};
		\node (y2)	[treenode]									at (-7,-4.5)									{$X$ \\ 55}; 
		\node (g)		[subtree] 									at (-4,-4)   									{\Large $\gamma$};
		\node (gnd)	[ground] 										at (-8.5,-5.75)								{}; 
		\node (o)		[ground] 										at (-5.5,-5.5)  							{};
		\node (x1) 	[] 													at (-8.5,0) 									{\large lastRightTurn node};
		\draw[->] (-6, 1.5) --  (x);
		\draw[->] (x) -- (y0);
		\draw[->] (x) -- (a.north);
		\draw[->] (y0) -- (y1);
		\draw[->] (y0) -- (b.north);
		\draw[->] (y1) -- (y2);
		\draw[->] (y1) -- (g.north);
		\draw[->] (y2) -- (gnd);
		\draw[->] (y2) -- (o);
		%% legend
		\node [thin, draw=black, align=center, minimum width=5cm, minimum height=1.5cm] at (0,-7) {\Large Thread $A$ - search(55) \\ \Large Thread $B$ - delete(50)};	
		\pause
		\node (y1)	[treenode, fill=red]		at (-5.5,-3)											{$W$ \\ 60}; 
		\node (y1l) [rectangle,align=center,minimum size=1cm] at (-8.5,-3) 		{\large Thread $A$ \\stalled here};
		\pause
		\node (y2l) [rectangle,align=center,minimum size=1cm] at(-9.5,-4.5) 	{\large key is promoted \\by Thread $B$};
		\pause
		\path[every node/.style={font=\sffamily\small}]
		(-3.25, 0) edge[->,semithick, double] node [above, outer sep=3pt] 		{\large \texttt delete 50} (-0.75, 0);

		\node (ix)	[treenode] 									at (1,0)       								{$U$ \\ 55};
		\node (ia)	[subtree] 									at (-0.5,-1)  								{\Large $\alpha$};
		\node (iy0)	[treenode] 									at (3,-1.5) 									{$V$ \\ 100};
		\node (iy1)	[treenode, fill=red]									at (1.5,-3)					{$W$ \\ 60}; 
		\node (ib)	[subtree] 									at (4,-2.5)    								{\Large $\beta$};
		\node (io)	[ground]										at (0,-4)											{}; 
		\node (ignd)[subtree] 									at (3,-4)   									{\Large $\gamma$};
    
    \draw[->] (1, 1.5) --  (ix);
    \draw[->] (ix) --  (iy0);
    \draw[->] (ix) --  (ia.north);
    \draw[->] (iy0) --  (iy1);
    \draw[->] (iy0) --  (ib.north);
    \draw[->] (iy1) --  (io);
    \draw[->] (iy1) --  (ignd.north);
		\pause
		\node (iy1)	[treenode, fill=green]									at (1.5,-3)					{$W$ \\ 60};
		\node (y12) [rectangle,align=center,minimum size=1cm] at (0,-3) 		{\large Thread $A$ \\wakes up};
		\pause
		\node (io)	[ground,color=red]										at (0,-4)											{}; 
		\node (y13) [rectangle,align=center,minimum size=1cm] at (0,-5) 		{\large key not found};
	\end{tikzpicture}
\visible<6>
{
\\Keep track of last right turn node and its key. If search terminates at a NULL node, check if the current key in the last right turn node has changed. If yes restart the operation from root.
}
\end{frame}

\ifdefined\LONG
\begin{frame}[c]{Lock Based BST - Delete}
pseudocode for delete
\begin{algorithm}[H]
locate the node to delete\;
\uIf{simple delete}
{
lock the edge $\langle$parent,node$\rangle$\;
lock the children edges\;
make the parent point to the non-null child using a simple write instruction\;
release all locks\;
}
\Else(// complex delete)
{
lock the edge $\langle$node,rightChild$\rangle$\;
find the successor\;
lock the edge $\langle$successorParent,successor$\rangle$\;
lock the children edges of successor\;
promote key\;
remove successor by a making successorParent point to non-null child of successor\;
release all locks\;
}
\end{algorithm}
\end{frame}
\fi

\begin{frame}[c]{Lock Based BST - Simple Delete}
\begin{tikzpicture}[scale=0.6, transform shape,mylabel/.style={thin, draw=black, align=center, minimum width=0.3cm, minimum height=0.3cm,fill=white}]
		\newcommand\XA{-3}
		\newcommand\YA{0}
		\node (x)		[treenode] 																at (\XA,\YA)       					{$V$ \\ 100};
		\node (y)		[treenode, fill=black!20] 								at (\XA-1.5,\YA-1.5) 				{$U$ \\ 50};
		\node (a)		[subtree] 																at (\XA+2,\YA-1)      			{\Large $\alpha$};
		\node (b)		[subtree] 																at (\XA-3,\YA-2.5)    			{\Large $\beta$};
		\node (gnd)	[ground] 																	at (\XA+0.5,\YA-2.5)				{}; 
		\node (xl) 	[rectangle,align=center,minimum size=1cm] at (\XA-2,\YA) 							{\large targetnode's \\ \large parent};
		\node (yl) 	[] 																				at (\XA-3.25,\YA-1.5) 			{\large targetnode};
		
 		\draw[->] (x) -- (y);
 		\draw[->] (y) -- (b.north);
 		\draw[->] (y) -- (gnd);
		\draw[->] (\XA,\YA+1) -- (x);
		\draw[->] (x) -- (a.north);
		\pause
		\draw[->] (x) -- node[mylabel] {$\boldsymbol{l}$} (y);
		\pause
		\draw[->] (y) -- node[mylabel] {$\boldsymbol{l}$} (b.north);
		\pause
		\draw[->] (y) -- node[mylabel] {$\boldsymbol{l}$} (gnd);
    \pause
		\path[every node/.style={font=\sffamily\small}]
		(\XA+1.5,\YA) edge[->,semithick, double] 							node [above, outer sep=3pt] {\large \texttt delete 50} (\XA+4,\YA);
		
		\node (ix)	[treenode] 																at (\XA+6,\YA) 							{$V$ \\ 100};
		\node (ib)	[subtree] 																at (\XA+4.5,\YA-1)					{\Large $\beta$};
		\node (ia)	[subtree] 																at (\XA+7.5,\YA-1) 					{\Large $\alpha$};

		\draw[->] (\XA+6,\YA+1) -- (ix);
		\draw[->] (ix) -- (ib.north);
		\draw[->] (ix) -- (ia.north);
	\end{tikzpicture}
\end{frame}

\begin{frame}[c]{Lock Based BST - Complex Delete}
\begin{tikzpicture}[scale=0.55, transform shape,mylabel/.style={thin, draw=black, align=center, minimum width=0.3cm, minimum height=0.3cm,fill=white}]
		\newcommand\XA{0}
		\newcommand\YA{0}
		\node (x)		[treenode, fill=black!20] 	at (-6,0)       							{$U$ \\ 50};
		\node (a)		[subtree] 									at (-7.5,-1)  								{\Large $\alpha$};
		\node (y0)	[treenode] 									at (-4,-1.5) 									{$V$ \\ 100};
		\node (y1)	[treenode]									at (-5.5,-3)									{$W$ \\ 60}; 
		\node (b)		[subtree] 									at (-3,-2.5)    							{\Large $\beta$};
		\node (y2)	[treenode]									at (-7,-4.5)									{$X$ \\ 55}; 
		\node (g)		[subtree] 									at (-4,-4)   									{\Large $\gamma$};
		\node (gnd)	[ground] 										at (-8.5,-5.75)									{}; 
		\node (o)		[subtree] 									at (-5.5,-5.5)  							{\Large $\omega$};
		\node (x1) 	[] 													at (-7.75,0) 									{\large targetnode};
		\node (y1l) [rectangle,align=center,minimum size=1cm] at (-7.5,-3) 		{\large successor node's \\ \large parent};
		\node (y2l) [rectangle,align=center,minimum size=1cm] at(-8.5,-4.5) 	{\large successor \\ \large node};
		
		\draw[->] (-6, 1.5) --  (x);
		\draw[->] (x) -- (y0);
		\draw[->] (x) -- (a.north);
		\draw[->] (y0) -- (y1);
		\draw[->] (y0) -- (b.north);
		\draw[->] (y1) -- (y2);
		\draw[->] (y1) -- (g.north);
		\draw[->] (y2) -- (gnd);
		\draw[->] (y2) -- (o.north);
    \pause
    \draw[->] (x) -- node[mylabel] {$\boldsymbol{l}$} (y0);
    \pause
    \draw[->] (y1) -- node[mylabel] {$\boldsymbol{l}$} (y2);
    \pause
    \draw[->] (y2) -- node[mylabel] {$\boldsymbol{l}$} (gnd);
    \pause
  	\draw[->] (y2) -- node[mylabel] {$\boldsymbol{l}$} (o.north);
    \pause
		\path[every node/.style={font=\sffamily\small}]
		(-3.25, 0) edge[->,semithick, double] node [above, outer sep=3pt] 		{\large \texttt delete 50} (-0.75, 0);

		\node (ix)	[treenode] 									at (1,0)       								{$U$ \\ 55};
		\node (ia)	[subtree] 									at (-0.5,-1)  								{\Large $\alpha$};
		\node (iy0)	[treenode] 									at (3,-1.5) 									{$V$ \\ 100};
		\node (iy1)	[treenode]									at (1.5,-3)										{$W$ \\ 60}; 
		\node (ib)	[subtree] 									at (4,-2.5)    								{\Large $\beta$};
		\node (io)	[subtree]										at (0,-4)											{\Large $\omega$}; 
		\node (ignd)[subtree] 									at (3,-4)   									{\Large $\gamma$};
    
    \draw[->] (1, 1.5) --  (ix);
    \draw[->] (ix) --  (iy0);
    \draw[->] (ix) --  (ia.north);
    \draw[->] (iy0) --  (iy1);
    \draw[->] (iy0) --  (ib.north);
    \draw[->] (iy1) --  (io.north);
    \draw[->] (iy1) --  (ignd.north);
	\end{tikzpicture}
\end{frame}

\begin{frame}{Lock Based BST - More challenges in search}
A scenario in which the last right turn node is removed
\begin{figure}
\centering
{
	\begin{tikzpicture}[scale=0.7, transform shape]
		\newcommand\XA{-5}
		\newcommand\YA{0}
		\node[] at (-4.5,-3.5) {(a)};
		\node (x)		[treenode] 									at (\XA,\YA)       						{$W$ \\ 5};
		\node (S1)	[subtree,scale=0.5] 				at ([shift=({-1,-0.5})]x)  		{};
		\node (y0)	[treenode] 									at ([shift=({1.5,-1})]x) 			{$X$ \\ 6};
		\node (gnd)	[ground]										at ([shift=({-1.5,-1})]y0)		{}; 
		\node (y1)	[treenode,fill=red] 				at ([shift=({1.5,-1})]y0)   	{$Y$ \\ 14};
		\node (y2)	[treenode] 									at ([shift=({-1.5,-1})]y1)  	{$Z$ \\ 13};
		%\node (S2)	[subtree,scale=0.5]					at ([shift=({0.5,-1})]y1)			{}; 

		\path[every node/.style={font=\sffamily\large}]
		(\XA,\YA+1) 	edge[->,thick] 		node 					{} (x)
		(x) 		edge[->]								node 					{} (S1.north)
		(x) 		edge[->]								node 					{} (y0)
		(y0) 		edge[->,dotted,thick]		node 					{} (gnd)
		(y0) 		edge[->,dotted,thick]		node 					{} (y1)
		(y1) 		edge[->]								node 					{} (y2);
		%(y1) 		edge[->]							node 					{} (S2.north);
		\pause
		\renewcommand\XA{0}
		\node[] at (0.5,-3.5) {(b)};
		\node (ix)	[treenode] 										at (\XA,\YA)       						{$W$ \\ 5};
		\node (iS1)	[subtree,scale=0.5] 					at ([shift=({-1,-0.5})]ix)  	{};
		\node (iy1)	[treenode,fill=red] 					at ([shift=({1.5,-1})]ix)   	{$Y$ \\ 14};
		\node (iy2)	[treenode] 										at ([shift=({-1,-1.5})]iy1)  	{$Z$ \\ 13};
		%\node (iS2)	[subtree,scale=0.5]					at ([shift=({0.5,-1.0})]iy1)	{}; 

		\path[every node/.style={font=\sffamily\large}]
		(\XA,\YA+1) 	edge[->,thick] 		node 					{} (ix)
		(ix) 		edge[->]								node 					{} (iS1.north)
		(ix) 		edge[->]								node 					{} (iy1)
		(iy1) 	edge[->]								node 					{} (iy2);
		%(iy1) 	edge[->]								node 					{} (iS2.north);
		\pause
		\renewcommand\XA{3.8}
		\node[] at (3.5,-3.5) {(c)};
		\node (iix)		[treenode] 									at (\XA,\YA)       							{$W$ \\ 13};
		\node (iiS1)	[subtree,scale=0.5] 				at ([shift=({-1,-0.5})]iix)  		{};
		\node (iiy1)	[treenode,fill=green] 			at ([shift=({1.25,-1.25})]iix)  {$Y$ \\ 14};
		\node (iignd)	[ground]										at ([shift=({-1.5,-1})]iiy1)		{}; 
		
		\path[every node/.style={font=\sffamily\large}]
		(iix) 		edge[->]								node 					{} (iiS1.north)
		(iix) 		edge[->]								node 					{} (iiy1)
		(iiy1) 		edge[->]								node 					{} (iignd);	
	\end{tikzpicture}
}
\end{figure}

\begin{itemize}
\item<1> \footnotesize Search(13) gets stalled at $Y$ in (a). Its last right turn node is $X$
\item<2> \footnotesize Delete(6) removes $X$ from the tree in (b). The key stored in $X$ is still 6
\item<3> \footnotesize Delete(5) results in 13 moving up the tree from $Z$ to $W$ in (c). When search(13) wakes up, it will miss 13 as the key in the last right turn node has not changed
\end{itemize}


\begin{itemize}
\item<4> \footnotesize In the first traversal search(13) saw the node $X$
\item<4> \footnotesize In the second traversal there are two cases
\begin{itemize}
\item<4> \tiny case1, search(13) did not find $X$ - save the traversal and restart
\item<4> \tiny case2, search(13) did find $X$ - use the results of previous traversal
\end{itemize}
\end{itemize}

\end{frame}