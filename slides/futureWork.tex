\begin{frame}{Future Work}
\begin{itemize}
\item analyze our local recovery algorithm (amortized time complexity)
\item develop concurrent K-ary BST which can improve spatial locality
\item work on other data structures like tries, bloom filters, etc.
\item evaluate using real workloads.
\end{itemize}
\end{frame}

\ifdefined\LONG
\begin{frame}{Future Work - K-ary BST}
\begin{itemize}
\item ideas from Lock Based BST can be extended to external K-ary BST
\item updates are relatively easier to handle as they obtain locks
\item inserts might result in  node splits
\item searches are hard if we need to maintain their lock-free property
\item extend it further to $B$-trees
\end{itemize}
\end{frame}
\fi

\begin{comment}
\begin{frame}{Future Work - Local Recovery}
\begin{itemize}
\item currently upon failure, an operation restarts from the root
\item Ellen et.al[PODC'14] have shown that local recovery can be done for external BST
\item Local recovery on an internal BST is hard due to key movements
\item We are currently working on extending our algorithms to enable local recovery
\end{itemize}
\end{frame}

\begin{frame}{Future Work - other data structures}
\begin{itemize}
\item Tries are extensively used in text processing
\item Tree like structure. So our ideas $can$ $possibly$ be applied
\end{itemize}
\end{frame}
\end{comment}

\begin{frame}[c]
\centering
\Huge Thank you
\end{frame}