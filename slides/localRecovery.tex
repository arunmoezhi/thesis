\begin{frame}[c]{Local recovery[PPoPP'16 Poster]}
Overview
\begin{itemize}
\item a general technique for local recovery for concurrent BSTs
\item reduces tree traversal cost during failures by restarting closer to an operation�s window
\end{itemize}
Motivation
\begin{itemize}
\item in most concurrent BSTs, execution phase of an operation have constant time complexity
\item seek phase is where an operation may end up spending most of its time (esp for large trees)
\item this technique reduces the seek time
\end{itemize}
\end{frame}

\begin{frame}[c]{Example}
\begin{figure}[htp]
\begin{tikzpicture}[scale=0.5, transform shape] 
	 \newcommand\NODEDX{1.25}
	 \newcommand\NODEDY{1.25}
	 \newcommand\SUBTREEDX{1.5}
	 \newcommand\SUBTREEDY{0.75}
	
   \node (r)	[treenode] 		                at (0, 0)       		                      	{$R$ \\ 100};
   \node (s)	[treenode, fill=black!20] 		at ([shift=({ -\NODEDX, -\NODEDY})]r)     	{$S$ \\ 10};
	 \node (t)	[treenode] 		                at ([shift=({  \NODEDX, -\NODEDY})]s)    		{$T$ \\ 90};
	 \node (u)	[treenode, fill=black!20] 	  at ([shift=({ -\NODEDX, -\NODEDY})]t)     	{$U$ \\ 20};
	 \node (v)	[treenode] 										at ([shift=({  \NODEDX, -\NODEDY})]u)     	{$V$ \\ 80};
	 \node (w)	[treenode] 										at ([shift=({ -\NODEDX, -\NODEDY})]v)     	{$W$ \\ 70};
	 \node (x)	[treenode, fill=black!20] 		at ([shift=({ -\NODEDX, -\NODEDY})]w)     	{$X$ \\ 30};
	 \node (y)	[treenode,fill=red]						at ([shift=({  \NODEDX, -\NODEDY})]x)     	{$Y$ \\ 60};
	 \node (z)	[treenode] 										at ([shift=({ -\NODEDX, -\NODEDY})]y)     	{$Z$ \\ 50};
	 \node (gl) [ground]                      at ([shift=({ -\NODEDX, -\NODEDY})]z)     	{ };
	 \node (gr) [ground]                      at ([shift=({  \NODEDX, -\NODEDY})]z)     	{ };
		
	 \node (sa) [subtree]                     at ([shift=({  \SUBTREEDX, -\SUBTREEDY})]r) {\Large $\alpha$};
	 \node (sb) [subtree]                     at ([shift=({ -\SUBTREEDX, -\SUBTREEDY})]s) {\Large $\beta$};
	 \node (sg) [subtree]                     at ([shift=({  \SUBTREEDX, -\SUBTREEDY})]t) {\Large $\gamma$};
	 \node (sd) [subtree]                     at ([shift=({ -\SUBTREEDX, -\SUBTREEDY})]u) {\Large $\delta$};
	 \node (ss) [subtree]                     at ([shift=({  \SUBTREEDX, -\SUBTREEDY})]v) {\Large $\sigma$};
	 \node (st) [subtree]                     at ([shift=({  \SUBTREEDX, -\SUBTREEDY})]w) {\Large $\tau$};
	 \node (sp) [subtree]                     at ([shift=({ -\SUBTREEDX, -\SUBTREEDY})]x) {\Large $\pi$};
	 \node (sl) [subtree]                     at ([shift=({  \SUBTREEDX, -\SUBTREEDY})]y) {\Large $\lambda$};	
	
	 \node (op) [label={right:{\large $op(50)^{z^{z^z}}$}}] at ([shift=({0.375, 0})]y) {};
	
	 \path[every node/.style={font=\sffamily\small}]
	    (0, 1)  edge[->,very thick]  node {} (r)
		  (r)     edge[->,very thick]  node {} (s)
			(s)     edge[->,very thick]  node {} (t)
			(t)     edge[->,very thick]  node {} (u)
			(u)     edge[->,very thick]  node {} (v)
			(v)     edge[->,very thick]  node {} (w)
			(w)     edge[->,very thick]  node {} (x)
			(x)     edge[->,very thick]  node {} (y)
			(y)     edge[->,very thick]  node {} (z)
			(z)     edge[->]  node {} (gl)
			(z)     edge[->]  node {} (gr)
			(r)     edge[->]  node {} (sa.north)
			(s)     edge[->]  node {} (sb.north)
			(t)     edge[->]  node {} (sg.north)
			(u)     edge[->]  node {} (sd.north)
			(v)     edge[->]  node {} (ss.north)
			(w)     edge[->]  node {} (st.north)
			(x)     edge[->]  node {} (sp.north)
			(y)     edge[->]  node {} (sl.north);		
\end{tikzpicture}
\caption{Operation $op(50)$ is suspended at node $Y$ during its traversal}
\end{figure}
\end{frame}

\begin{frame}[c]{Example}
\begin{figure}[htp]
\begin{tikzpicture}[scale=0.5, transform shape]   
	 \newcommand\NODEDX{1.25}
	 \newcommand\NODEDY{1.25}
	 \newcommand\SUBTREEDX{1.5}
	 \newcommand\SUBTREEDY{0.75}
	
   \node (r)	[treenode] 		                at (0, 0)       		                      	{$R$ \\ 100};
   \node (s)	[treenode, fill=black!20] 		at ([shift=({ -\NODEDX, -\NODEDY})]r)     	{$S$ \\ 10};
	 \node (t)	[treenode] 		                at ([shift=({  \NODEDX, -\NODEDY})]s)    		{$T$ \\ 90};
	 \node (u)	[treenode, fill=black!20] 	  at ([shift=({ -\NODEDX, -\NODEDY})]t)     	{$U$ \\ 20};
	 \node (v)	[treenode] 										at ([shift=({  \NODEDX, -\NODEDY})]u)     	{$V$ \\ 80};
	 \node (w)	[treenode] 										at ([shift=({ -\NODEDX, -\NODEDY})]v)     	{$W$ \\ 70};
	 \node (x)	[treenode, fill=black!20] 		at ([shift=({ -\NODEDX, -\NODEDY})]w)     	{$X$ \\ 30};
	 \node (y)	[treenode, fill=red]					at ([shift=({  \NODEDX, -\NODEDY})]x)     	{$Y$ \\ 60};
	 \node (z)	[treenode] 										at ([shift=({ -\NODEDX, -\NODEDY})]y)     	{$Z$ \\ 50};
	 \node (gl) [ground]                      at ([shift=({ -\NODEDX, -\NODEDY})]z)     	{ };
	 \node (gr) [ground]                      at ([shift=({  \NODEDX, -\NODEDY})]z)     	{ };

	 \node (sa) [subtree]                     at ([shift=({  \SUBTREEDX, -\SUBTREEDY})]r) {\Large $\alpha$};
	 \node (sb) [subtree]                     at ([shift=({ -\SUBTREEDX, -\SUBTREEDY})]s) {\Large $\beta$};
	 \node (sg) [subtree]                     at ([shift=({  \SUBTREEDX, -\SUBTREEDY})]t) {\Large $\gamma$};
	 \node (sd) [subtree]                     at ([shift=({ -\SUBTREEDX, -\SUBTREEDY})]u) {\Large $\delta$};
	 \node (ss) [subtree]                     at ([shift=({  \SUBTREEDX, -\SUBTREEDY})]v) {\Large $\sigma$};
	 \node (st) [subtree]                     at ([shift=({  \SUBTREEDX, -\SUBTREEDY})]w) {\Large $\tau$};
	 %% \node (sp) [subtree]                     at ([shift=({ -\SUBTREEDX, -\SUBTREEDY})]x) {\Large $\pi$};
	 \node (sp) [ground]                    	at ([shift=({ -\NODEDX, -\NODEDY})]x) { };
	 \node (sl) [subtree]                     at ([shift=({  \SUBTREEDX, -\SUBTREEDY})]y) {\Large $\lambda$};	

   \node (op) [label={right:{\large $op(50)^{z^{z^z}}$}}] at ([shift=({0.375, 0})]y) {};
	
	 \path[every node/.style={font=\sffamily\small}]
	    (0, 1)  edge[->,very thick]  node {} (r)
		  (r)     edge[->,very thick]  node {} (s)
			(s)     edge[->,very thick]  node {} (t)
			(t)     edge[->,very thick]  node {} (u)
			(u)     edge[->,very thick]  node {} (v)
			(v)     edge[->,very thick]  node {} (w)
			(w)     edge[->,very thick]  node {} (x)
			(x)     edge[->,very thick]  node {} (y)
			(y)     edge[->,very thick]  node {} (z)
			(z)     edge[->]  node {} (gl)
			(z)     edge[->]  node {} (gr)
			(r)     edge[->]  node {} (sa.north)
			(s)     edge[->]  node {} (sb.north)
			(t)     edge[->]  node {} (sg.north)
			(u)     edge[->]  node {} (sd.north)
			(v)     edge[->]  node {} (ss.north)
			(w)     edge[->]  node {} (st.north)
			(x)     edge[->]  node {} (sp)
			(y)     edge[->]  node {} (sl.north);	
\end{tikzpicture}
\caption{All keys in subtree $\pi$ are deleted one-by-one}
\end{figure}
\end{frame}

\begin{frame}[c]{Example}
\begin{figure}[htp]
\begin{tikzpicture}[scale=0.5, transform shape]
   
	 \newcommand\NODEDX{1.25}
	 \newcommand\NODEDY{1.25}
	 \newcommand\SUBTREEDX{1.5}
	 \newcommand\SUBTREEDY{0.75}
	
   \node (r)	[treenode] 		                at (0, 0)       		                      	{$R$ \\ 100};
   \node (s)	[treenode, fill=black!20] 		at ([shift=({ -\NODEDX, -\NODEDY})]r)     	{$S$ \\ 10};
	 \node (t)	[treenode] 		                at ([shift=({  \NODEDX, -\NODEDY})]s)    		{$T$ \\ 90};
	 \node (u)	[treenode, fill=black!20] 	  at ([shift=({ -\NODEDX, -\NODEDY})]t)     	{$U$ \\ 20};
	 \node (v)	[treenode] 										at ([shift=({  \NODEDX, -\NODEDY})]u)     	{$V$ \\ 80};
	 \node (w)	[treenode] 										at ([shift=({ -\NODEDX, -\NODEDY})]v)     	{$W$ \\ 70};
	 \node (x)	[treenode, fill=black!20, dotted] 		at ([shift=({ -\NODEDX, -\NODEDY})]w)     	{$X$ \\ 30};
	 \node (y)	[treenode] 										at ([shift=({  \NODEDX, -\NODEDY})]x)     	{$Y$ \\ 60};
	 \node (z)	[treenode] 										at ([shift=({ -\NODEDX, -\NODEDY})]y)     	{$Z$ \\ 50};
	 \node (gl) [ground]                      at ([shift=({ -\NODEDX, -\NODEDY})]z)     	{ };
	 \node (gr) [ground]                      at ([shift=({  \NODEDX, -\NODEDY})]z)     	{ };
		
	 \node (sa) [subtree]                     at ([shift=({  \SUBTREEDX, -\SUBTREEDY})]r) {\Large $\alpha$};
	 \node (sb) [subtree]                     at ([shift=({ -\SUBTREEDX, -\SUBTREEDY})]s) {\Large $\beta$};
	 \node (sg) [subtree]                     at ([shift=({  \SUBTREEDX, -\SUBTREEDY})]t) {\Large $\gamma$};
	 \node (sd) [subtree]                     at ([shift=({ -\SUBTREEDX, -\SUBTREEDY})]u) {\Large $\delta$};
	 \node (ss) [subtree]                     at ([shift=({  \SUBTREEDX, -\SUBTREEDY})]v) {\Large $\sigma$};
	 \node (st) [subtree]                     at ([shift=({  \SUBTREEDX, -\SUBTREEDY})]w) {\Large $\tau$};
	 %% \node (sp) [subtree]                     at ([shift=({ -\SUBTREEDX, -\SUBTREEDY})]x) {\Large $\pi$};
	 \node (sp) [ground]                    	at ([shift=({ -\NODEDX, -\NODEDY})]x) { };
	 \node (sl) [subtree]                     at ([shift=({  \SUBTREEDX, -\SUBTREEDY})]y) {\Large $\lambda$};	
	
	 \node (op) [label={right:{\large $op(50)^{z^{z^z}}$}}] at ([shift=({0.375, 0})]y) {};
	
	 \path[every node/.style={font=\sffamily\small}]
	    (0, 1)  edge[->,very thick]  node {} (r)
		  (r)     edge[->,very thick]  node {} (s)
			(s)     edge[->,very thick]  node {} (t)
			(t)     edge[->,very thick]  node {} (u)
			(u)     edge[->,very thick]  node {} (v)
			(v)     edge[->,very thick]  node {} (w)
			%% (w)     edge[->]  node {} (x)
			(w)     edge[->,very thick]  node {} (y)
			(x)     edge[->]  node {} (y)
			(y)     edge[->,very thick]  node {} (z)
			(z)     edge[->]  node {} (gl)
			(z)     edge[->]  node {} (gr)
			(r)     edge[->]  node {} (sa.north)
			(s)     edge[->]  node {} (sb.north)
			(t)     edge[->]  node {} (sg.north)
			(u)     edge[->]  node {} (sd.north)
			(v)     edge[->]  node {} (ss.north)
			(w)     edge[->]  node {} (st.north)
			(x)     edge[->]  node {} (sp)
			(y)     edge[->]  node {} (sl.north);	
\end{tikzpicture}
\caption{Key 30 is deleted (simple delete); node $X$ is removed}
\end{figure}
\end{frame}

\begin{frame}[c]{Example}
\begin{figure}[htp]
\begin{tikzpicture}[scale=0.5, transform shape]
   
	 \newcommand\NODEDX{1.25}
	 \newcommand\NODEDY{1.25}
	 \newcommand\SUBTREEDX{1.5}
	 \newcommand\SUBTREEDY{0.75}
	
   \node (r)	[treenode] 		                at (0, 0)       		                      	{$R$ \\ 100};
   \node (s)	[treenode, fill=black!20] 		at ([shift=({ -\NODEDX, -\NODEDY})]r)     	{$S$ \\ 10};
	 \node (t)	[treenode] 		                at ([shift=({  \NODEDX, -\NODEDY})]s)    		{$T$ \\ 90};
	 \node (u)	[treenode, fill=black!20] 	  at ([shift=({ -\NODEDX, -\NODEDY})]t)     	{$U$ \\ 50};
	 \node (v)	[treenode] 										at ([shift=({  \NODEDX, -\NODEDY})]u)     	{$V$ \\ 80};
	 \node (w)	[treenode] 										at ([shift=({ -\NODEDX, -\NODEDY})]v)     	{$W$ \\ 70};
	 \node (x)	[treenode, fill=black!20, dotted] 		at ([shift=({ -\NODEDX, -\NODEDY})]w)     	{$X$ \\ 30};
	 \node (y)	[treenode] 										at ([shift=({  \NODEDX, -\NODEDY})]x)     	{$Y$ \\ 60};
	 \node (z)	[treenode, dotted] 						at ([shift=({ -\NODEDX, -\NODEDY})]y)     	{$Z$ \\ 50};
	 \node (gl) [ground]                      at ([shift=({ -\NODEDX, -\NODEDY})]z)     	{ };
	 \node (gr) [ground]                      at ([shift=({  \NODEDX, -\NODEDY})]z)     	{ };
		
	 \node (sa) [subtree]                     at ([shift=({  \SUBTREEDX, -\SUBTREEDY})]r) {\Large $\alpha$};
	 \node (sb) [subtree]                     at ([shift=({ -\SUBTREEDX, -\SUBTREEDY})]s) {\Large $\beta$};
	 \node (sg) [subtree]                     at ([shift=({  \SUBTREEDX, -\SUBTREEDY})]t) {\Large $\gamma$};
	 \node (sd) [subtree]                     at ([shift=({ -\SUBTREEDX, -\SUBTREEDY})]u) {\Large $\delta$};
	 \node (ss) [subtree]                     at ([shift=({  \SUBTREEDX, -\SUBTREEDY})]v) {\Large $\sigma$};
	 \node (st) [subtree]                     at ([shift=({  \SUBTREEDX, -\SUBTREEDY})]w) {\Large $\tau$};
	 %% \node (sp) [subtree]                     at ([shift=({ -\SUBTREEDX, -\SUBTREEDY})]x) {\Large $\pi$};
	 \node (sp) [ground]                    	at ([shift=({ -\NODEDX, -\NODEDY})]x) { };
	 \node (sl) [subtree]                     at ([shift=({  \SUBTREEDX, -\SUBTREEDY})]y) {\Large $\lambda$};	
	
	 \node (op) [label={right:{\large $op(50)^{z^{z^z}}$}}] at ([shift=({0.375, 0})]y) {};
	
	 \path[every node/.style={font=\sffamily\small}]
	    (0, 1)  edge[->,very thick]  node {} (r)
		  (r)     edge[->,very thick]  node {} (s)
			(s)     edge[->,very thick]  node {} (t)
			(t)     edge[->,very thick]  node {} (u)
			(u)     edge[->,very thick]  node {} (v)
			(v)     edge[->,very thick]  node {} (w)
			%% (w)     edge[->]  node {} (x)
			(w)     edge[->,very thick]  node {} (y)
			(x)     edge[->]  node {} (y)
			%% (y)     edge[->]  node {} (z)
			(y)     edge[->, very thick]  node {} (gr)
			(z)     edge[->]  node {} (gl)
			(z)     edge[->]  node {} (gr)
			(r)     edge[->]  node {} (sa.north)
			(s)     edge[->]  node {} (sb.north)
			(t)     edge[->]  node {} (sg.north)
			(u)     edge[->]  node {} (sd.north)
			(v)     edge[->]  node {} (ss.north)
			(w)     edge[->]  node {} (st.north)
			(x)     edge[->]  node {} (sp)
			(y)     edge[->]  node {} (sl.north);		
\end{tikzpicture}
\caption{Key 20 is deleted (complex delete); key 20 is replaced with key 50 in node $U$ and node $Z$ is removed}
\end{figure}
\end{frame}

\begin{frame}[c]{Traversal Stack}
\begin{itemize}
\item a stack to keep track of anchor nodes of all nodes in the traversal path
\item reduces tree traversal cost during failures by restarting closer to an operation�s window
\end{itemize}
\end{frame}

\begin{frame}[c]{Traversal Stack}
\begin{figure}[htp]
\begin{tikzpicture}[scale=0.5, transform shape] 
	 \newcommand\NODEDX{1.25}
	 \newcommand\NODEDY{1.25}
	 \newcommand\SUBTREEDX{1.5}
	 \newcommand\SUBTREEDY{0.75}
	
   \node (r)	[treenode, fill=black!20] 		at (0, 0)       		                      	{$R$ \\  -$\infty$};
   \node (s)	[treenode] 										at ([shift=({ \NODEDX, -\NODEDY})]r)     	  {$S$ \\  $\infty$};
	 \node (t)	[treenode] 		                at ([shift=({  -\NODEDX, -\NODEDY})]s)    	{$T$ \\ 90};
	 \node (u)	[treenode, fill=black!20] 	  at ([shift=({ -\NODEDX, -\NODEDY})]t)     	{$U$ \\ 20};
	 \node (v)	[treenode] 										at ([shift=({  \NODEDX, -\NODEDY})]u)     	{$V$ \\ 80};
	 \node (w)	[treenode] 										at ([shift=({ -\NODEDX, -\NODEDY})]v)     	{$W$ \\ 70};
	 \node (x)	[treenode, fill=black!20] 		at ([shift=({ -\NODEDX, -\NODEDY})]w)     	{$X$ \\ 30};
	 \node (y)	[treenode] 										at ([shift=({  \NODEDX, -\NODEDY})]x)     	{$Y$ \\ 60};
	 \node (z)	[treenode] 										at ([shift=({ -\NODEDX, -\NODEDY})]y)     	{$Z$ \\ 50};
	 \node (gl) [ground]                      at ([shift=({ -\NODEDX, -\NODEDY})]z)     	{ };
	 \node (gr) [ground]                      at ([shift=({  \NODEDX, -\NODEDY})]z)     	{ };
		
	 \node (sa) [ground]                      at ([shift=({ -\SUBTREEDX, -\SUBTREEDY})]r) { };
	 \node (sb) [ground]                      at ([shift=({ \SUBTREEDX, -\SUBTREEDY})]s) 	{ };
	 \node (sg) [subtree]                     at ([shift=({  \SUBTREEDX, -\SUBTREEDY})]t) {\Large $\gamma$};
	 \node (sd) [subtree]                     at ([shift=({ -\SUBTREEDX, -\SUBTREEDY})]u) {\Large $\delta$};
	 \node (ss) [subtree]                     at ([shift=({  \SUBTREEDX, -\SUBTREEDY})]v) {\Large $\sigma$};
	 \node (st) [subtree]                     at ([shift=({  \SUBTREEDX, -\SUBTREEDY})]w) {\Large $\tau$};
	 \node (sp) [subtree]                     at ([shift=({ -\SUBTREEDX, -\SUBTREEDY})]x) {\Large $\pi$};
	 \node (sl) [subtree]                     at ([shift=({  \SUBTREEDX, -\SUBTREEDY})]y) {\Large $\lambda$};	
	
	 \node (op) [label={right:{\large $op(50) is here$}}] at ([shift=({0.5, 0})]s) {};
	
	 \path[every node/.style={font=\sffamily\small}]
	    %(0, 1)  edge[->,very thick]  node {} (r)
		  (r)     edge[->,very thick]  node {} (s)
			(s)     edge[->,very thick]  node {} (t)
			(t)     edge[->,very thick]  node {} (u)
			(u)     edge[->,very thick]  node {} (v)
			(v)     edge[->,very thick]  node {} (w)
			(w)     edge[->,very thick]  node {} (x)
			(x)     edge[->,very thick]  node {} (y)
			(y)     edge[->,very thick]  node {} (z)
			(z)     edge[->]  node {} (gl)
			(z)     edge[->]  node {} (gr)
			(r)     edge[->]  node {} (sa)
			(s)     edge[->]  node {} (sb)
			(t)     edge[->]  node {} (sg.north)
			(u)     edge[->]  node {} (sd.north)
			(v)     edge[->]  node {} (ss.north)
			(w)     edge[->]  node {} (st.north)
			(x)     edge[->]  node {} (sp.north)
			(y)     edge[->]  node {} (sl.north);		
\end{tikzpicture}
%\qquad
%\begin{tikzpicture}[scale=1.0, transform shape]
%\node[stack=9]  {
%0,\nodepart{one}Z,left,6
%\nodepart{two}Y,rigt,6
%\nodepart{three}X,left,3
%\nodepart{four}W,left,3
%\nodepart{five}V,right,3
%\nodepart{six}U,left,1
%\nodepart{seven}T,right,1
%\nodepart{eight}S,right,0
%\nodepart{nine}R,right,-1
%};
%\end{tikzpicture}
\qquad
\begin{tikzpicture}[scale=0.5, transform shape]
  \stacktop{} \cellptr{top of stack}
	\separator
	\cell{\texttt{S,R}}        \cellcomL{1} \coordinate () at (currentcell.east);
  \separator
	\cell{\texttt{R,null}}     \cellcomL{0} \coordinate () at (currentcell.east);
  \separator
\end{tikzpicture}
\caption{Operation $op(50)$ starting at R and ending at Z along with the stack}
\end{figure}
\end{frame}
\begin{frame}[c]{Traversal Stack}
\begin{figure}[htp]
\begin{tikzpicture}[scale=0.5, transform shape] 
	 \newcommand\NODEDX{1.25}
	 \newcommand\NODEDY{1.25}
	 \newcommand\SUBTREEDX{1.5}
	 \newcommand\SUBTREEDY{0.75}
	
   \node (r)	[treenode, fill=black!20] 		at (0, 0)       		                      	{$R$ \\  -$\infty$};
   \node (s)	[treenode] 										at ([shift=({ \NODEDX, -\NODEDY})]r)     	  {$S$ \\  $\infty$};
	 \node (t)	[treenode, fill=red]          at ([shift=({  -\NODEDX, -\NODEDY})]s)    	{$T$ \\ 90};
	 \node (u)	[treenode, fill=black!20] 	  at ([shift=({ -\NODEDX, -\NODEDY})]t)     	{$U$ \\ 20};
	 \node (v)	[treenode] 										at ([shift=({  \NODEDX, -\NODEDY})]u)     	{$V$ \\ 80};
	 \node (w)	[treenode] 										at ([shift=({ -\NODEDX, -\NODEDY})]v)     	{$W$ \\ 70};
	 \node (x)	[treenode, fill=black!20] 		at ([shift=({ -\NODEDX, -\NODEDY})]w)     	{$X$ \\ 30};
	 \node (y)	[treenode] 										at ([shift=({  \NODEDX, -\NODEDY})]x)     	{$Y$ \\ 60};
	 \node (z)	[treenode] 										at ([shift=({ -\NODEDX, -\NODEDY})]y)     	{$Z$ \\ 50};
	 \node (gl) [ground]                      at ([shift=({ -\NODEDX, -\NODEDY})]z)     	{ };
	 \node (gr) [ground]                      at ([shift=({  \NODEDX, -\NODEDY})]z)     	{ };
		
	 \node (sa) [ground]                      at ([shift=({ -\SUBTREEDX, -\SUBTREEDY})]r) { };
	 \node (sb) [ground]                      at ([shift=({ \SUBTREEDX, -\SUBTREEDY})]s) 	{ };
	 \node (sg) [subtree]                     at ([shift=({  \SUBTREEDX, -\SUBTREEDY})]t) {\Large $\gamma$};
	 \node (sd) [subtree]                     at ([shift=({ -\SUBTREEDX, -\SUBTREEDY})]u) {\Large $\delta$};
	 \node (ss) [subtree]                     at ([shift=({  \SUBTREEDX, -\SUBTREEDY})]v) {\Large $\sigma$};
	 \node (st) [subtree]                     at ([shift=({  \SUBTREEDX, -\SUBTREEDY})]w) {\Large $\tau$};
	 \node (sp) [subtree]                     at ([shift=({ -\SUBTREEDX, -\SUBTREEDY})]x) {\Large $\pi$};
	 \node (sl) [subtree]                     at ([shift=({  \SUBTREEDX, -\SUBTREEDY})]y) {\Large $\lambda$};	
	
	 \node (op) [label={left:{\large $op(50)$}}] at ([shift=({-0.5, 0})]t) {};
	
	 \path[every node/.style={font=\sffamily\small}]
	    %(0, 1)  edge[->,very thick]  node {} (r)
		  (r)     edge[->,very thick]  node {} (s)
			(s)     edge[->,very thick]  node {} (t)
			(t)     edge[->,very thick]  node {} (u)
			(u)     edge[->,very thick]  node {} (v)
			(v)     edge[->,very thick]  node {} (w)
			(w)     edge[->,very thick]  node {} (x)
			(x)     edge[->,very thick]  node {} (y)
			(y)     edge[->,very thick]  node {} (z)
			(z)     edge[->]  node {} (gl)
			(z)     edge[->]  node {} (gr)
			(r)     edge[->]  node {} (sa)
			(s)     edge[->]  node {} (sb)
			(t)     edge[->]  node {} (sg.north)
			(u)     edge[->]  node {} (sd.north)
			(v)     edge[->]  node {} (ss.north)
			(w)     edge[->]  node {} (st.north)
			(x)     edge[->]  node {} (sp.north)
			(y)     edge[->]  node {} (sl.north);		
\end{tikzpicture}
%\qquad
%\begin{tikzpicture}[scale=1.0, transform shape]
%\node[stack=9]  {
%0,\nodepart{one}Z,left,6
%\nodepart{two}Y,rigt,6
%\nodepart{three}X,left,3
%\nodepart{four}W,left,3
%\nodepart{five}V,right,3
%\nodepart{six}U,left,1
%\nodepart{seven}T,right,1
%\nodepart{eight}S,right,0
%\nodepart{nine}R,right,-1
%};
%\end{tikzpicture}
\qquad
\begin{tikzpicture}[scale=0.5, transform shape]
  \stacktop{} \cellptr{top of stack}
	\separator
	\cell{\Large \texttt{T,R}}        \cellcomL{2} \coordinate () at (currentcell.east);
  \separator
	\cell{\Large \texttt{S,R}}        \cellcomL{1} \coordinate () at (currentcell.east);
  \separator
	\cell{\Large \texttt{R,null}}     \cellcomL{0} \coordinate () at (currentcell.east);
  \separator
\end{tikzpicture}
%\caption{Operation $op(50)$ starting at R and suspended at Y along with the stack}
\end{figure}
\end{frame}


\begin{frame}[c]{Search}
search operations do not restart
\begin{figure}[htp]
\centering
{
	\begin{tikzpicture}[scale=0.5, transform shape]
	\node (p0) [] {search(K)};
	\node (p1) [process, below of=p0, text width=4cm] {Do a binary search for key K in the tree};
	\node (p2) [process, below of=p1, yshift=-1.5cm, text width=4.5cm] {Examine anchor node $A$ of top entry in the stack};
	\node (p3) [decision, below of=p2, yshift=-1.5cm, text width=2cm] {is anchor node marked?};
	\node (p4) [process, right of=p3, xshift=4cm, text width=4.5cm] {pop all entries upto anchor node $A$};
	\node (retT) [process, right of=p1, xshift=4cm, text width=1cm, minimum width=1cm] {return true};
	\node (retF) [process, left of=p2, xshift=-5cm, text width=1cm, minimum width=1cm] {return false};

	\draw [arrow] (p1) -- node[anchor=west] {K not found} (p2);
	\draw [arrow] (p1) -- node[anchor=south] {K found} (retT);
	\draw [arrow] (p2.east) -| node[anchor=north,pos=0.5] {K = A.key}    (retT.south);
	\draw [arrow] (p2.west) -- node (temp) [anchor=north,pos=0.5] {K $<$ A.key}  (retF.east);
	\draw [arrow] (p2) -- node[anchor=east] {K $>$ A.key} (p3);
	\draw [arrow] (p3.west) -| (retF.south) node[below, pos=0.5]{No}  (retF.south);
	\draw [arrow] (p3.east) -- node[anchor=north]{Yes} (p4.west);
	\draw [arrow] (p4.north) -- (p2.south);
	\node (temp1) [above of=temp,xshift=0.5cm,yshift=0.4cm] {(Aold.key $<$ K $<$ Anew.key)};
	\end{tikzpicture}
}
\caption{Sequence of steps in a search operation}
\end{figure}
\end{frame}

\begin{frame}[c]{Search}
\begin{figure}[htp]
\begin{tikzpicture}[scale=0.5, transform shape] 
	 \newcommand\NODEDX{1.25}
	 \newcommand\NODEDY{1.25}
	 \newcommand\SUBTREEDX{1.5}
	 \newcommand\SUBTREEDY{0.75}
	
   \node (r)	[treenode, fill=black!20] 		at (0, 0)       		                      	{$R$ \\  -$\infty$};
   \node (s)	[treenode] 										at ([shift=({ \NODEDX, -\NODEDY})]r)     	  {$S$ \\  $\infty$};
	 \node (t)	[treenode, fill=red]          at ([shift=({  -\NODEDX, -\NODEDY})]s)    	{$T$ \\ 90};
	 \node (u)	[treenode, fill=black!20] 	  at ([shift=({ -\NODEDX, -\NODEDY})]t)     	{$U$ \\ 20};
	 \node (v)	[treenode] 										at ([shift=({  \NODEDX, -\NODEDY})]u)     	{$V$ \\ 80};
	 \node (w)	[treenode] 										at ([shift=({ -\NODEDX, -\NODEDY})]v)     	{$W$ \\ 70};
	 \node (x)	[treenode, fill=black!20] 		at ([shift=({ -\NODEDX, -\NODEDY})]w)     	{$X$ \\ 30};
	 \node (y)	[treenode] 										at ([shift=({  \NODEDX, -\NODEDY})]x)     	{$Y$ \\ 60};
	 \node (z)	[treenode] 										at ([shift=({ -\NODEDX, -\NODEDY})]y)     	{$Z$ \\ 50};
	 \node (gl) [ground]                      at ([shift=({ -\NODEDX, -\NODEDY})]z)     	{ };
	 \node (gr) [ground]                      at ([shift=({  \NODEDX, -\NODEDY})]z)     	{ };
		
	 \node (sa) [ground]                      at ([shift=({ -\SUBTREEDX, -\SUBTREEDY})]r) { };
	 \node (sb) [ground]                      at ([shift=({ \SUBTREEDX, -\SUBTREEDY})]s) 	{ };
	 \node (sg) [subtree]                     at ([shift=({  \SUBTREEDX, -\SUBTREEDY})]t) {\Large $\gamma$};
	 \node (sd) [subtree]                     at ([shift=({ -\SUBTREEDX, -\SUBTREEDY})]u) {\Large $\delta$};
	 \node (ss) [subtree]                     at ([shift=({  \SUBTREEDX, -\SUBTREEDY})]v) {\Large $\sigma$};
	 \node (st) [subtree]                     at ([shift=({  \SUBTREEDX, -\SUBTREEDY})]w) {\Large $\tau$};
	 \node (sp) [subtree]                     at ([shift=({ -\SUBTREEDX, -\SUBTREEDY})]x) {\Large $\pi$};
	 \node (sl) [subtree]                     at ([shift=({  \SUBTREEDX, -\SUBTREEDY})]y) {\Large $\lambda$};	
	
	 \node (op) [label={left:{\large $op(50)$}}] at ([shift=({-0.5, 0})]t) {};
	
	 \path[every node/.style={font=\sffamily\small}]
	    %(0, 1)  edge[->,very thick]  node {} (r)
		  (r)     edge[->,very thick]  node {} (s)
			(s)     edge[->,very thick]  node {} (t)
			(t)     edge[->,very thick]  node {} (u)
			(u)     edge[->,very thick]  node {} (v)
			(v)     edge[->,very thick]  node {} (w)
			(w)     edge[->,very thick]  node {} (x)
			(x)     edge[->,very thick]  node {} (y)
			(y)     edge[->,very thick]  node {} (z)
			(z)     edge[->]  node {} (gl)
			(z)     edge[->]  node {} (gr)
			(r)     edge[->]  node {} (sa)
			(s)     edge[->]  node {} (sb)
			(t)     edge[->]  node {} (sg.north)
			(u)     edge[->]  node {} (sd.north)
			(v)     edge[->]  node {} (ss.north)
			(w)     edge[->]  node {} (st.north)
			(x)     edge[->]  node {} (sp.north)
			(y)     edge[->]  node {} (sl.north);		
\end{tikzpicture}
%\qquad
%\begin{tikzpicture}[scale=1.0, transform shape]
%\node[stack=9]  {
%0,\nodepart{one}Z,left,6
%\nodepart{two}Y,rigt,6
%\nodepart{three}X,left,3
%\nodepart{four}W,left,3
%\nodepart{five}V,right,3
%\nodepart{six}U,left,1
%\nodepart{seven}T,right,1
%\nodepart{eight}S,right,0
%\nodepart{nine}R,right,-1
%};
%\end{tikzpicture}
\qquad
\begin{tikzpicture}[scale=0.5, transform shape]
  \stacktop{} \cellptr{top of stack}
	\separator
	\cell{\Large \texttt{T,R}}        \cellcomL{2} \coordinate () at (currentcell.east);
  \separator
	\cell{\Large \texttt{S,R}}        \cellcomL{1} \coordinate () at (currentcell.east);
  \separator
	\cell{\Large \texttt{R,null}}     \cellcomL{0} \coordinate () at (currentcell.east);
  \separator
\end{tikzpicture}
%\caption{Operation $op(50)$ starting at R and suspended at Y along with the stack}
\end{figure}
\end{frame}
\begin{frame}[c]{Search}
\begin{figure}[htp]
\begin{tikzpicture}[scale=0.33, transform shape]
   
	 \newcommand\NODEDX{1.25}
	 \newcommand\NODEDY{1.25}
	 \newcommand\SUBTREEDX{1.5}
	 \newcommand\SUBTREEDY{0.75}
	
	 \node (r)	[treenode, fill=black!20] 		at (0, 0)       		                      	{$R$ \\  -$\infty$};
   \node (s)	[treenode] 										at ([shift=({ \NODEDX, -\NODEDY})]r)     	  {$S$ \\  $\infty$};
	 \node (t)	[treenode] 		                at ([shift=({  -\NODEDX, -\NODEDY})]s)    	{$T$ \\ 90};
	 \node (u)	[treenode, fill=black!20] 	  at ([shift=({ -\NODEDX, -\NODEDY})]t)     	{$U$ \\ 50};
	 \node (v)	[treenode] 										at ([shift=({  \NODEDX, -\NODEDY})]u)     	{$V$ \\ 80};
	 \node (w)	[treenode] 										at ([shift=({ -\NODEDX, -\NODEDY})]v)     	{$W$ \\ 70};
	 \node (x)	[treenode, fill=black!20, dotted] 		at ([shift=({ -\NODEDX, -\NODEDY})]w)     	{$X$ \\ 30};
	 \node (y)	[treenode] 										at ([shift=({  \NODEDX, -\NODEDY})]x)     	{$Y$ \\ 60};
	 \node (z)	[treenode, dotted] 						at ([shift=({ -\NODEDX, -\NODEDY})]y)     	{$Z$ \\ 50};
	 \node (gl) [ground]                      at ([shift=({ -\NODEDX, -\NODEDY})]z)     	{ };
	 \node (gr) [ground]                      at ([shift=({  \NODEDX, -\NODEDY})]z)     	{ };
		
	 \node (sa) [ground]                      at ([shift=({ -\SUBTREEDX, -\SUBTREEDY})]r) { };
	 \node (sb) [ground]                      at ([shift=({ \SUBTREEDX, -\SUBTREEDY})]s) 	{ };
	 \node (sg) [subtree]                     at ([shift=({  \SUBTREEDX, -\SUBTREEDY})]t) {\Large $\gamma$};
	 \node (sd) [subtree]                     at ([shift=({ -\SUBTREEDX, -\SUBTREEDY})]u) {\Large $\delta$};
	 \node (ss) [subtree]                     at ([shift=({  \SUBTREEDX, -\SUBTREEDY})]v) {\Large $\sigma$};
	 \node (st) [subtree]                     at ([shift=({  \SUBTREEDX, -\SUBTREEDY})]w) {\Large $\tau$};
	 %% \node (sp) [subtree]                     at ([shift=({ -\SUBTREEDX, -\SUBTREEDY})]x) {\Large $\pi$};
	 \node (sp) [ground]                    	at ([shift=({ -\NODEDX, -\NODEDY})]x) { };
	 \node (sl) [subtree]                     at ([shift=({  \SUBTREEDX, -\SUBTREEDY})]y) {\Large $\lambda$};	
	
	 \node (op) [label={right:{\large $search(50)$}}] at ([shift=({0.375, 0})]gr) {};
	
	 \path[every node/.style={font=\sffamily\small}]
	    %(0, 1)  edge[->,very thick]  node {} (r)
		  (r)     edge[->,very thick]  node {} (s)
			(s)     edge[->,very thick]  node {} (t)
			(t)     edge[->,very thick]  node {} (u)
			(u)     edge[->,very thick]  node {} (v)
			(v)     edge[->,very thick]  node {} (w)
			%% (w)     edge[->]  node {} (x)
			(w)     edge[->,very thick]  node {} (y)
			(x)     edge[->]  node {} (y)
			%% (y)     edge[->]  node {} (z)
			(y)     edge[->, very thick]  node {} (gr)
			(z)     edge[->]  node {} (gl)
			(z)     edge[->]  node {} (gr)
			(r)     edge[->]  node {} (sa)
			(s)     edge[->]  node {} (sb)
			(t)     edge[->]  node {} (sg.north)
			(u)     edge[->]  node {} (sd.north)
			(v)     edge[->]  node {} (ss.north)
			(w)     edge[->]  node {} (st.north)
			(x)     edge[->]  node {} (sp)
			(y)     edge[->]  node {} (sl.north);		
\end{tikzpicture}
\quad
\begin{tikzpicture}[scale=0.28, transform shape]
  \stacktop{} \cellptr{top}
	\separator
	\cell{\Large \texttt{Y,X}}        \cellcomL{7} \coordinate () at (currentcell.east);
  \separator
	\cell{\Large \texttt{X,U}}        \cellcomL{6} \coordinate () at (currentcell.east);
  \separator
	\cell{\Large \texttt{W,U}}        \cellcomL{5} \coordinate () at (currentcell.east);
  \separator
	\cell{\Large \texttt{V,U}}        \cellcomL{4} \coordinate () at (currentcell.east);
  \separator
	\cell{\Large \texttt{U,R}}        \cellcomL{3} \coordinate () at (currentcell.east);
  \separator
	\cell{\Large \texttt{T,R}}        \cellcomL{2} \coordinate () at (currentcell.east);
  \separator
	\cell{\Large \texttt{S,R}}        \cellcomL{1} \coordinate () at (currentcell.east);
  \separator
	\cell{\Large \texttt{R,null}}     \cellcomL{0} \coordinate () at (currentcell.east);
  \separator
\end{tikzpicture}
\quad
\begin{tikzpicture}[scale=0.4, transform shape]
	\node (p0) [] {search(K)};
	\node (p1) [process, below of=p0, text width=4cm] {Do a binary search for key K in the tree};
	\node (p2) [process, below of=p1, yshift=-1.5cm, text width=4.5cm,fill=black!20] {Examine anchor node $A$ of top entry in the stack};
	\node (p3) [decision, below of=p2, yshift=-1.5cm, text width=2cm,fill=black!20] {is anchor node marked?};
	\node (p4) [process, right of=p3, xshift=4cm, text width=4.5cm,fill=black!20] {pop all entries upto anchor node $A$};
	\node (retT) [process, right of=p1, xshift=4cm, text width=1cm, minimum width=1cm] {return true};
	\node (retF) [process, left of=p2, xshift=-5cm, text width=1cm, minimum width=1cm] {return false};

	\draw [arrow] (p1) -- node[anchor=west] {K not found} (p2);
	\draw [arrow] (p1) -- node[anchor=south] {K found} (retT);
	\draw [arrow] (p2.east) -| node[anchor=north,pos=0.5] {K = A.key}    (retT.south);
	\draw [arrow] (p2.west) -- node (temp) [anchor=north,pos=0.5] {K $<$ A.key}  (retF.east);
	\draw [arrow] (p2) -- node[anchor=east] {K $>$ A.key} (p3);
	\draw [arrow] (p3.west) -| (retF.south) node[below, pos=0.5]{No}  (retF.south);
	\draw [arrow] (p3.east) -- node[anchor=north]{Yes} (p4.west);
	\draw [arrow] (p4.north) -- (p2.south);
	\node (temp1) [above of=temp,xshift=0.5cm,yshift=0.4cm] {(A.oldKey $<$ K $<$ A.newKey)};
	\end{tikzpicture}
\caption{Key 30 is deleted;key 20 is deleted $\&$ replaced with key 50 in node $U$ and node $Z$ is removed}
\end{figure}
\end{frame}
\begin{frame}[c]{Search}
\begin{figure}[htp]
\begin{tikzpicture}[scale=0.5, transform shape]
   
	 \newcommand\NODEDX{1.25}
	 \newcommand\NODEDY{1.25}
	 \newcommand\SUBTREEDX{1.5}
	 \newcommand\SUBTREEDY{0.75}
	
	 \node (r)	[treenode, fill=black!20] 		at (0, 0)       		                      	{$R$ \\  -$\infty$};
   \node (s)	[treenode] 										at ([shift=({ \NODEDX, -\NODEDY})]r)     	  {$S$ \\  $\infty$};
	 \node (t)	[treenode] 		                at ([shift=({  -\NODEDX, -\NODEDY})]s)    	{$T$ \\ 90};
	 \node (u)	[treenode, fill=black!20] 	  at ([shift=({ -\NODEDX, -\NODEDY})]t)     	{$U$ \\ 50};
	 \node (v)	[treenode] 										at ([shift=({  \NODEDX, -\NODEDY})]u)     	{$V$ \\ 80};
	 \node (w)	[treenode] 										at ([shift=({ -\NODEDX, -\NODEDY})]v)     	{$W$ \\ 70};
	 \node (x)	[treenode, fill=black!20, dotted] 		at ([shift=({ -\NODEDX, -\NODEDY})]w)     	{$X$ \\ 30};
	 \node (y)	[treenode] 										at ([shift=({  \NODEDX, -\NODEDY})]x)     	{$Y$ \\ 60};
	 \node (z)	[treenode, dotted] 						at ([shift=({ -\NODEDX, -\NODEDY})]y)     	{$Z$ \\ 50};
	 \node (gl) [ground]                      at ([shift=({ -\NODEDX, -\NODEDY})]z)     	{ };
	 \node (gr) [ground]                      at ([shift=({  \NODEDX, -\NODEDY})]z)     	{ };
		
	 \node (sa) [ground]                      at ([shift=({ -\SUBTREEDX, -\SUBTREEDY})]r) { };
	 \node (sb) [ground]                      at ([shift=({ \SUBTREEDX, -\SUBTREEDY})]s) 	{ };
	 \node (sg) [subtree]                     at ([shift=({  \SUBTREEDX, -\SUBTREEDY})]t) {\Large $\gamma$};
	 \node (sd) [subtree]                     at ([shift=({ -\SUBTREEDX, -\SUBTREEDY})]u) {\Large $\delta$};
	 \node (ss) [subtree]                     at ([shift=({  \SUBTREEDX, -\SUBTREEDY})]v) {\Large $\sigma$};
	 \node (st) [subtree]                     at ([shift=({  \SUBTREEDX, -\SUBTREEDY})]w) {\Large $\tau$};
	 %% \node (sp) [subtree]                     at ([shift=({ -\SUBTREEDX, -\SUBTREEDY})]x) {\Large $\pi$};
	 \node (sp) [ground]                    	at ([shift=({ -\NODEDX, -\NODEDY})]x) { };
	 \node (sl) [subtree]                     at ([shift=({  \SUBTREEDX, -\SUBTREEDY})]y) {\Large $\lambda$};	
	
	 \node (op) [label={left:{\large $op(50) is here$}}] at ([shift=({-0.5, 0})]w) {};
	
	 \path[every node/.style={font=\sffamily\small}]
	    %(0, 1)  edge[->,very thick]  node {} (r)
		  (r)     edge[->,very thick]  node {} (s)
			(s)     edge[->,very thick]  node {} (t)
			(t)     edge[->,very thick]  node {} (u)
			(u)     edge[->,very thick]  node {} (v)
			(v)     edge[->,very thick]  node {} (w)
			%% (w)     edge[->]  node {} (x)
			(w)     edge[->,very thick]  node {} (y)
			(x)     edge[->]  node {} (y)
			%% (y)     edge[->]  node {} (z)
			(y)     edge[->, very thick]  node {} (gr)
			(z)     edge[->]  node {} (gl)
			(z)     edge[->]  node {} (gr)
			(r)     edge[->]  node {} (sa)
			(s)     edge[->]  node {} (sb)
			(t)     edge[->]  node {} (sg.north)
			(u)     edge[->]  node {} (sd.north)
			(v)     edge[->]  node {} (ss.north)
			(w)     edge[->]  node {} (st.north)
			(x)     edge[->]  node {} (sp)
			(y)     edge[->]  node {} (sl.north);		
\end{tikzpicture}
\qquad
\begin{tikzpicture}[scale=0.5, transform shape]
  \stacktop{} \cellptr{top of stack}
	\separator
	\cell{\texttt{W,U}}        \cellcomL{5} \coordinate () at (currentcell.east);
  \separator
	\cell{\texttt{V,U}}        \cellcomL{4} \coordinate () at (currentcell.east);
  \separator
	\cell{\texttt{U,R}}        \cellcomL{3} \coordinate () at (currentcell.east);
  \separator
	\cell{\texttt{T,R}}        \cellcomL{2} \coordinate () at (currentcell.east);
  \separator
	\cell{\texttt{S,R}}        \cellcomL{1} \coordinate () at (currentcell.east);
  \separator
	\cell{\texttt{R,null}}     \cellcomL{0} \coordinate () at (currentcell.east);
  \separator
\end{tikzpicture}
\caption{Pop upto marked anchor node $X$. Top of stack is now $W$. Examine anchor node $U$}
\end{figure}
\end{frame}

\begin{frame}[c]{Insert}
An insert operation needs to restart only if one of the anchor nodes in the path has become inconsistent
\begin{figure}[htp]
\centering
{
	\begin{tikzpicture}[scale=0.5, transform shape]
	\node (p0) [] {insert(K)};
	\node (p1) [process, below of=p0, text width=4cm] {Do a binary search for key K in the tree};
	\node (p2) [process, below of=p1, yshift=-1.5cm, text width=4.5cm] {Examine anchor node $A$ of top entry in the stack};
	\node (p3) [decision, below of=p2, yshift=-1.5cm, text width=2cm] {is anchor node marked?};
	\node (p4) [process, right of=p3, xshift=4cm, text width=4.5cm] {pop all entries upto anchor node $A$};
	\node (retF) [process, right of=p1, xshift=4cm, text width=1cm, minimum width=1cm] {return false};
	\node (retT) [process, left of=p2, xshift=-6cm, text width=4.5cm, minimum width=1cm] {discard suffix of the path after anchor node and find a restart point};
	\node (p5) [process, left of=p3, xshift=-6cm, text width=4.5cm, minimum width=1cm] {traversal terminates. Terminal node is returned as the injection point};

	\draw [arrow] (p1) -- node[anchor=west] {K not found} (p2);
	\draw [arrow] (p1) -- node[anchor=south] {K found} (retF);
	\draw [arrow] (p2.east) -| node[anchor=north,pos=0.5] {K = A.key}    (retF.south);
	\draw [arrow] (p2.west) -- node[anchor=north,pos=0.5] {K $<$ A.key}  (retT.east);
	\draw [arrow] (p2) -- node[anchor=east] {K $>$ A.key} (p3);
	\draw [arrow] (p3.west)  -- node[anchor=north]{No}  (p5.east);
	\draw [arrow] (p3.east) -- node[anchor=north]{Yes} (p4.west);
	\draw [arrow] (p4.north) -- (p2.south);

	\end{tikzpicture}
}
\caption{Sequence of steps in an insert operation}
\end{figure}
\end{frame}

\begin{frame}[c]{Delete}
A delete operation do not restart except when there is a failure in the execution phase
\begin{figure}[htp]
\centering
{
	\begin{tikzpicture}[scale=0.5, transform shape]
	\node (p0) [] {delete(K)};
	\node (p1) [process, below of=p0, text width=4cm] {Do a binary search for key K in the tree};
	\node (p2) [process, below of=p1, yshift=-1.5cm, text width=4.5cm] {Examine anchor node $A$ of top entry in the stack};
	\node (p3) [decision, below of=p2, yshift=-1.5cm, text width=2cm] {is anchor node marked?};
	\node (p4) [process, right of=p3, xshift=4cm, text width=4.5cm] {pop all entries upto anchor node $A$};
	\node (ex) [process, right of=p1, xshift=6cm, text width=4.5cm, minimum width=1cm] {go to execution phase};
	\node (retF) [process, left of=p2, xshift=-4cm, text width=1cm, minimum width=1cm] {return false};

	\draw [arrow] (p1) -- node[anchor=west] {K not found} (p2);
	\draw [arrow] (p1) -- node[anchor=south] {K found} (ex);
	\draw [arrow] (p2.east) -| node[anchor=north,pos=0.5] {K = A.key}    (ex.south);
	\draw [arrow] (p2.west) -- node[anchor=north,pos=0.5] {K $<$ A.key}  (retF.east);
	\draw [arrow] (p2) -- node[anchor=east] {K $>$ A.key} (p3);
	\draw [arrow] (p3.west)  -| node[anchor=north]{No}  (retF.south);
	\draw [arrow] (p3.east) -- node[anchor=north]{Yes} (p4.west);
	\draw [arrow] (p4.north) -- (p2.south);

	\end{tikzpicture}
}
\caption{Sequence of steps in a delete operation}
\end{figure}
\end{frame}